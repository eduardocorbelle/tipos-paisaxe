%\documentclass{article}
\usepackage[english,spanish,galician]{babel}
\usepackage[T1]{fontenc}
\usepackage[utf8x]{inputenc}
\usepackage{default}
\usepackage{array}
\usepackage{chngpage}
\usepackage{url}
\usepackage{booktabs}
\usepackage{animate}


\title{Cartografía de unidades da paisaxe}
\subtitle{Proposta preliminar}
\author{Eduardo Corbelle Rico} 
\institute{Laboratorio do territorio\\Universidade de Santiago de Compostela}
\date{2 de xullo de 2015}

\titlegraphic{
\includegraphics[height=1cm]{/home/edujose/Traballo/Recursos/logo_usc/logo_ux.pdf}%
}




\graphicspath{{"/home/edujose/Traballo/Proxectos/2015_04_CatalogoPaisaxe/Reports/Report1/Figures/"}}


\usetheme{Frankfurt} % Frankfurt
\usefonttheme{professionalfonts}
\usecolortheme{USC2} % masterra 
\useinnertheme{circles}
\setbeamercovered{transparent}
\setbeamertemplate{navigation symbols}{}
\setcounter{tocdepth}{1}


\begin{document}

%\mode<presentation>
%----------- páxina de título ----------------------------------------------%
\begin{frame}[plain]
 \vspace{2cm}
 \titlepage
\end{frame}

%%----------- Contidos ----------------------------------------------%
%\begin{frame}{Contidos}
%  \tableofcontents%[pausesections]
%\end{frame}
%
%\mode*
%\maketitle
%\tableofcontents



%----------- diapositiva ---------------------------------------------------%
\begin{frame}
\frametitle{Obxectivos}
 \begin{itemize}
  \item Propoñer unha clasificación de \alert{tipos de paisaxe}
  \item Cartografar a localización de cada tipo (\alert{unidades})
 \end{itemize}
 
\pause
 \begin{block}{Compoñentes da clasificación}
  \begin{itemize}
   \item Clases de forma do relevo
   \item Clases de uso/cuberta do solo
   \item Clases bioclimáticas
  \end{itemize}
 \end{block} 
\end{frame}


%----------- diapositiva ---------------------------------------------------%
\begin{frame}
\frametitle{Clases de forma do relevo}
  \begin{enumerate}
   \item Chaira
   \item Val
   \item Canón
   \item Relevo ondulado
   \item Montaña
  \end{enumerate}
\end{frame}



%----------- diapositiva ---------------------------------------------------%
\begin{frame}
\frametitle{Clases de uso/cuberta do solo}
  \begin{enumerate}
   \item Mato seco e rochedo
   \item Brezais húmidos e turbeiras
   \item Bosques naturais
   \item Bosques transformados
   \item Agrícola e gandeiro extensivo
   \item Agrícola e gandeiro intensivo
   \item Residencial diseminado
   \item Urbano e industrial
  \end{enumerate}
\end{frame}



%----------- diapositiva ---------------------------------------------------%
\begin{frame}
\frametitle{Clases bioclimáticas}
  \begin{enumerate}
   \item Subhiperoceánico
   \item Semihiperoceánico
   \item Euoceánico
   \item Semicontinental
  \end{enumerate}
\end{frame}

%----------- diapositiva ---------------------------------------------------%
\begin{frame}
\frametitle{Tipos de paisaxe}
 \begin{block}{Exemplo}
 Chaira\\
 Agrícola e gandeiro intensivo\\
 Clima semihiperoceánico
 \end{block}
 
 \pause
 \begin{block}{Número (potencial) de tipos}
 5 clases forma $\times$ 8 clases uso $\times$ 4 clases clima $ = $ 160 tipos

 \bigskip
 Tipo $\neq$ Unidade
 \end{block}
\end{frame}


%----------- diapositiva ---------------------------------------------------%
\begin{frame}
\frametitle{Principais fontes de información}
 \begin{itemize}
  \item MDT25 (IGN)
  \item SIOSE 2011 (IET)
  \item Mapa de hábitats (Laboratorio de botánica, USC)
  \item Rodríguez Guitián \& Ramil Rego, RR.RR., 2007
 \end{itemize}
\end{frame}


%----------- diapositiva ---------------------------------------------------%
\begin{frame}
\frametitle{Escala de traballo}
Grande área paisaxística
  \begin{itemize}
   \item Comarca paisaxística
    \begin{itemize}
     \item \alert{\textbf{Unidade da paisaxe}}
      \begin{itemize}
       \item Elemento da paisaxe
      \end{itemize}
    \end{itemize}
  \end{itemize}
\end{frame}


%----------- diapositiva ---------------------------------------------------%
\begin{frame}
\frametitle{Criterios de elección de clases}
\begin{itemize}
 \item Relevancia para a xestión
 \item Posibilidade de identificación automática (patrón)
\end{itemize}
\end{frame}

%----------- diapositiva ---------------------------------------------------%
\begin{frame}
\frametitle{Patróns: Exemplos}
\includegraphics<1>[width=\textwidth]{EscenaA1}
\includegraphics<2>[width=\textwidth]{EscenaA2}
\end{frame}

%----------- diapositiva ---------------------------------------------------%
\begin{frame}
\frametitle{Patróns: Exemplos}
\includegraphics<1>[width=\textwidth]{EscenaB1}
\includegraphics<2>[width=\textwidth]{EscenaB2}
\end{frame}

%----------- diapositiva ---------------------------------------------------%
\begin{frame}
\frametitle{Patróns: Exemplos}
\includegraphics<1>[width=\textwidth]{EscenaC1}
\includegraphics<2>[width=\textwidth]{EscenaC2}
\end{frame}







\end{document}

