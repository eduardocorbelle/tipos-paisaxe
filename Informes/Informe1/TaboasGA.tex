% latex table generated in R 3.1.1 by xtable 1.7-4 package
% Fri Oct  2 19:41:22 2015
\begin{table}[p]
\centering
\caption{Grandes Áreas paisaxísticas e código asignado} 
\label{xtaboa0}
\begin{tabular}{rr}
  \hline
 & Código \\ 
  \hline
Golfo Ártabro &   1 \\ 
  A Mariña - Baixo Eo &   2 \\ 
  Costa Sur - Baixo Miño &   3 \\ 
  Ribeiras Encaixadas do Miño e do Sil &   4 \\ 
  Serras Orientais &   5 \\ 
  Chairas e Fosas Luguesas &   6 \\ 
  Galicia Central &   7 \\ 
  Chairas, Fosas e Serras Ourensás &   8 \\ 
  Serras Surorientais &   9 \\ 
  Galicia Setentrional &  10 \\ 
  Chairas e Fosas Occidentais &  11 \\ 
  Rías Baixas &  12 \\ 
   \hline
\end{tabular}
\end{table}
% latex table generated in R 3.1.1 by xtable 1.7-4 package
% Fri Oct  2 19:41:22 2015
\begin{table}[p]
\centering
\caption{Grandes unidades morfolóxicas por Grandes Áreas paisaxísticas (datos en km²)} 
\label{xtaboa1}
\begin{tabular}{rrrrrrrrrrrrr}
  \hline
 & 1 & 2 & 3 & 4 & 5 & 6 & 7 & 8 & 9 & 10 & 11 & 12 \\ 
  \hline
Canons & 36 & 0 & 0 & 318 & 41 & 6 & 0 & 0 & 82 & 0 & 0 & 0 \\ 
  Chairas e vales interiores & 0 & 1 & 171 & 1316 & 75 & 3217 & 231 & 1137 & 3 & 0 & 0 & 0 \\ 
  Litoral Cantabro-Atlantico & 535 & 348 & 261 & 0 & 0 & 0 & 0 & 0 & 0 & 365 & 533 & 1022 \\ 
  Serras & 139 & 152 & 310 & 837 & 1794 & 1308 & 1308 & 1705 & 2116 & 635 & 0 & 459 \\ 
  Vales sublitorais & 577 & 424 & 392 & 1 & 584 & 28 & 3548 & 0 & 1 & 628 & 1543 & 1219 \\ 
   \hline
\end{tabular}
\end{table}
% latex table generated in R 3.1.1 by xtable 1.7-4 package
% Fri Oct  2 19:41:22 2015
\begin{table}[p]
\centering
\caption{Grandes unidades morfolóxicas por Grandes Áreas paisaxísticas (datos en porcentaxe)} 
\label{xtaboa1p}
\begin{tabular}{rrrrrrrrrrrrr}
  \hline
 & 1 & 2 & 3 & 4 & 5 & 6 & 7 & 8 & 9 & 10 & 11 & 12 \\ 
  \hline
Canons & 2.8 & 0.0 & 0.0 & 12.9 & 1.7 & 0.1 & 0.0 & 0.0 & 3.7 & 0.0 & 0.0 & 0.0 \\ 
  Chairas e vales interiores & 0.0 & 0.1 & 14.5 & 53.3 & 3.0 & 70.6 & 4.5 & 40.0 & 0.1 & 0.0 & 0.0 & 0.0 \\ 
  Litoral Cantabro-Atlantico & 41.5 & 37.7 & 22.1 & 0.0 & 0.0 & 0.0 & 0.0 & 0.0 & 0.0 & 22.4 & 25.7 & 37.9 \\ 
  Serras & 10.8 & 16.4 & 26.2 & 33.9 & 71.9 & 28.7 & 25.4 & 59.9 & 96.1 & 39.0 & 0.0 & 17.0 \\ 
  Vales sublitorais & 44.7 & 45.9 & 33.2 & 0.0 & 23.4 & 0.6 & 68.9 & 0.0 & 0.0 & 38.6 & 74.3 & 45.1 \\ 
   \hline
\end{tabular}
\end{table}
% latex table generated in R 3.1.1 by xtable 1.7-4 package
% Fri Oct  2 19:41:22 2015
\begin{table}[p]
\centering
\caption{Clases de cuberta por Grandes Áreas paisaxísticas (datos en km²)} 
\label{xtaboa2}
\begin{tabular}{rrrrrrrrrrrrr}
  \hline
 & 1 & 2 & 3 & 4 & 5 & 6 & 7 & 8 & 9 & 10 & 11 & 12 \\ 
  \hline
Agrosistema extensivo & 68 & 25 & 62 & 622 & 963 & 1813 & 1356 & 882 & 576 & 135 & 128 & 135 \\ 
  Agrosistema intensivo (mosaico agroforestal) & 474 & 310 & 127 & 246 & 152 & 1253 & 1946 & 92 & 31 & 470 & 831 & 460 \\ 
  Agrosistema intensivo (plantacion forestal) & 205 & 381 & 248 & 211 & 204 & 250 & 319 & 177 & 140 & 440 & 384 & 553 \\ 
  Agrosistema intensivo (superficie de cultivo) & 13 & 24 & 2 & 67 & 21 & 474 & 231 & 327 & 57 & 18 & 153 & 22 \\ 
  Bosque & 48 & 20 & 25 & 281 & 427 & 165 & 133 & 247 & 166 & 53 & 0 & 39 \\ 
  Conxunto Historico & 1 & 1 & 3 & 1 & 0 & 1 & 1 & 0 & 0 & 0 & 3 & 1 \\ 
  Extractivo & 0 & 0 & 6 & 2 & 4 & 0 & 7 & 3 & 22 & 12 & 1 & 0 \\ 
  Matogueira e rochedo & 99 & 50 & 319 & 789 & 713 & 321 & 774 & 1005 & 1195 & 213 & 388 & 702 \\ 
  Rururbano (diseminado) & 305 & 61 & 346 & 113 & 7 & 172 & 346 & 62 & 12 & 64 & 155 & 649 \\ 
  Turbeira & 26 & 26 & 1 & 0 & 2 & 86 & 6 & 19 & 0 & 207 & 10 & 4 \\ 
  Urbano & 48 & 24 & 23 & 23 & 1 & 25 & 33 & 4 & 0 & 9 & 14 & 75 \\ 
  Viñedo & 0 & 0 & 18 & 115 & 0 & 0 & 1 & 24 & 2 & 0 & 0 & 34 \\ 
   \hline
\end{tabular}
\end{table}
% latex table generated in R 3.1.1 by xtable 1.7-4 package
% Fri Oct  2 19:41:22 2015
\begin{table}[p]
\centering
\caption{Clases de cuberta por Grandes Áreas paisaxísticas (datos en porcentaxe)} 
\label{xtaboa2p}
\begin{tabular}{rrrrrrrrrrrrr}
  \hline
 & 1 & 2 & 3 & 4 & 5 & 6 & 7 & 8 & 9 & 10 & 11 & 12 \\ 
  \hline
Agrosistema extensivo & 5.3 & 2.7 & 5.2 & 25.2 & 38.6 & 39.8 & 26.3 & 31.0 & 26.2 & 8.3 & 6.2 & 5.0 \\ 
  Agrosistema intensivo (mosaico agroforestal) & 36.7 & 33.6 & 10.8 & 10.0 & 6.1 & 27.5 & 37.8 & 3.2 & 1.4 & 28.8 & 40.0 & 17.0 \\ 
  Agrosistema intensivo (plantacion forestal) & 15.9 & 41.2 & 20.9 & 8.5 & 8.2 & 5.5 & 6.2 & 6.2 & 6.4 & 27.0 & 18.5 & 20.5 \\ 
  Agrosistema intensivo (superficie de cultivo) & 1.0 & 2.6 & 0.1 & 2.7 & 0.9 & 10.4 & 4.5 & 11.5 & 2.6 & 1.1 & 7.4 & 0.8 \\ 
  Bosque & 3.7 & 2.2 & 2.1 & 11.4 & 17.1 & 3.6 & 2.6 & 8.7 & 7.6 & 3.3 & 0.0 & 1.4 \\ 
  Conxunto Historico & 0.1 & 0.1 & 0.2 & 0.0 & 0.0 & 0.0 & 0.0 & 0.0 & 0.0 & 0.0 & 0.1 & 0.0 \\ 
  Extractivo & 0.0 & 0.0 & 0.5 & 0.1 & 0.1 & 0.0 & 0.1 & 0.1 & 1.0 & 0.7 & 0.0 & 0.0 \\ 
  Matogueira e rochedo & 7.7 & 5.5 & 27.0 & 31.9 & 28.6 & 7.0 & 15.0 & 35.3 & 54.3 & 13.1 & 18.7 & 26.0 \\ 
  Rururbano (diseminado) & 23.6 & 6.6 & 29.2 & 4.6 & 0.3 & 3.8 & 6.7 & 2.2 & 0.5 & 4.0 & 7.5 & 24.0 \\ 
  Turbeira & 2.0 & 2.8 & 0.1 & 0.0 & 0.1 & 1.9 & 0.1 & 0.7 & 0.0 & 12.7 & 0.5 & 0.2 \\ 
  Urbano & 3.7 & 2.6 & 1.9 & 0.9 & 0.0 & 0.6 & 0.6 & 0.1 & 0.0 & 0.6 & 0.7 & 2.8 \\ 
  Viñedo & 0.0 & 0.0 & 1.5 & 4.7 & 0.0 & 0.0 & 0.0 & 0.9 & 0.1 & 0.0 & 0.0 & 1.3 \\ 
   \hline
\end{tabular}
\end{table}
% latex table generated in R 3.1.1 by xtable 1.7-4 package
% Fri Oct  2 19:41:22 2015
\begin{table}[p]
\centering
\caption{Termotipos por Grandes Áreas paisaxísticas (datos en km²)} 
\label{xtaboa3}
\begin{tabular}{rrrrrrrrrrrrr}
  \hline
 & 1 & 2 & 3 & 4 & 5 & 6 & 7 & 8 & 9 & 10 & 11 & 12 \\ 
  \hline
Mesomediterráneo & 0 & 0 & 0 & 333 & 15 & 0 & 0 & 0 & 34 & 0 & 0 & 0 \\ 
  Mesotemperado inferior & 417 & 474 & 204 & 1075 & 348 & 1193 & 2932 & 1611 & 378 & 696 & 960 & 659 \\ 
  Mesotemperado superior & 247 & 184 & 87 & 344 & 1036 & 3148 & 1398 & 555 & 578 & 530 & 228 & 207 \\ 
  Supra e orotemperado & 4 & 2 & 54 & 124 & 1053 & 175 & 181 & 416 & 1202 & 107 & 0 & 71 \\ 
  Termotemperado & 610 & 262 & 822 & 595 & 34 & 43 & 642 & 243 & 0 & 285 & 867 & 1718 \\ 
   \hline
\end{tabular}
\end{table}
% latex table generated in R 3.1.1 by xtable 1.7-4 package
% Fri Oct  2 19:41:22 2015
\begin{table}[p]
\centering
\caption{Termotipos por Grandes Áreas paisaxísticas (datos en porcentaxe)} 
\label{xtaboa3p}
\begin{tabular}{rrrrrrrrrrrrr}
  \hline
 & 1 & 2 & 3 & 4 & 5 & 6 & 7 & 8 & 9 & 10 & 11 & 12 \\ 
  \hline
Mesomediterráneo & 0.0 & 0.0 & 0.0 & 13.5 & 0.6 & 0.0 & 0.0 & 0.0 & 1.5 & 0.0 & 0.0 & 0.0 \\ 
  Mesotemperado inferior & 32.3 & 51.2 & 17.2 & 43.5 & 14.0 & 26.2 & 56.9 & 56.6 & 17.2 & 42.7 & 46.2 & 24.4 \\ 
  Mesotemperado superior & 19.2 & 19.9 & 7.4 & 13.9 & 41.5 & 69.1 & 27.1 & 19.5 & 26.3 & 32.5 & 11.0 & 7.7 \\ 
  Supra e orotemperado & 0.3 & 0.3 & 4.5 & 5.0 & 42.2 & 3.8 & 3.5 & 14.6 & 54.6 & 6.6 & 0.0 & 2.6 \\ 
  Termotemperado & 47.3 & 28.3 & 69.5 & 24.1 & 1.3 & 0.9 & 12.5 & 8.5 & 0.0 & 17.5 & 41.8 & 63.6 \\ 
   \hline
\end{tabular}
\end{table}
