% latex table generated in R 3.2.5 by xtable 1.8-0 package
% Fri May  6 19:53:39 2016
\begin{table}[p]
\centering
\caption{Principais tipos de paisaxe,  Golfo Ártabro ( 1 )} 
\label{Tipos 1}
\begin{tabular}{rlrr}
  \hline
 & Tipo & Área (km²) & Porcentaxe \\ 
  \hline
1 & Vales sublitorais ; Agrosistema intensivo (mosaico agroforestal) ; Mesotemperado inferior & 280.65 & 21.70 \\ 
  2 & Litoral Cantabro-Atlantico ; Agrosistema intensivo (mosaico agroforestal) ; Termotemperado & 240.53 & 18.60 \\ 
  3 & Vales sublitorais ; Agrosistema intensivo (plantacion forestal) ; Mesotemperado inferior & 170.52 & 13.20 \\ 
  4 & Litoral Cantabro-Atlantico ; Rururbano (diseminado) ; Termotemperado & 90.61 & 7.00 \\ 
  5 & Litoral Cantabro-Atlantico ; Agrosistema intensivo (plantacion forestal) ; Termotemperado & 74.73 & 5.80 \\ 
  6 & Litoral Cantabro-Atlantico ; Urbano ; Termotemperado & 63.65 & 4.90 \\ 
  7 & Serras ; Matogueira e rochedo ; Mesotemperado superior & 39.46 & 3.10 \\ 
  8 & Serras ; Agrosistema intensivo (mosaico agroforestal) ; Mesotemperado superior & 31.44 & 2.40 \\ 
  9 & Canons ; Bosque ; Mesotemperado inferior & 20.14 & 1.60 \\ 
  10 & Vales sublitorais ; Agrosistema intensivo (mosaico agroforestal) ; Termotemperado & 20.33 & 1.60 \\ 
  11 & Serras ; Turbeira ; Mesotemperado superior & 19.74 & 1.50 \\ 
  12 & Vales sublitorais ; Agrosistema intensivo (mosaico agroforestal) ; Mesotemperado superior & 18.96 & 1.50 \\ 
  13 & Litoral Cantabro-Atlantico ; Matogueira e rochedo ; Termotemperado & 15.40 & 1.20 \\ 
  14 & Vales sublitorais ; Matogueira e rochedo ; Mesotemperado superior & 15.11 & 1.20 \\ 
  15 & Litoral Cantabro-Atlantico ; Agrosistema extensivo ; Termotemperado & 13.12 & 1.00 \\ 
  16 & Litoral Cantabro-Atlantico ; Agrosistema intensivo (superficie de cultivo) ; Termotemperado & 13.22 & 1.00 \\ 
  17 & Total & 1127.61 & 87.30 \\ 
   \hline
\end{tabular}
\end{table}
% latex table generated in R 3.2.5 by xtable 1.8-0 package
% Fri May  6 19:53:39 2016
\begin{table}[p]
\centering
\caption{Principais tipos de paisaxe,  A Mariña - Baixo Eo ( 2 )} 
\label{Tipos 2}
\begin{tabular}{rlrr}
  \hline
 & Tipo & Área (km²) & Porcentaxe \\ 
  \hline
1 & Vales sublitorais ; Agrosistema intensivo (plantacion forestal) ; Mesotemperado inferior & 218.95 & 23.70 \\ 
  2 & Litoral Cantabro-Atlantico ; Agrosistema intensivo (plantacion forestal) ; Mesotemperado inferior & 174.69 & 18.90 \\ 
  3 & Vales sublitorais ; Agrosistema intensivo (mosaico agroforestal) ; Mesotemperado inferior & 118.87 & 12.90 \\ 
  4 & Litoral Cantabro-Atlantico ; Agrosistema intensivo (mosaico agroforestal) ; Termotemperado & 66.57 & 7.20 \\ 
  5 & Serras ; Agrosistema intensivo (plantacion forestal) ; Mesotemperado superior & 62.89 & 6.80 \\ 
  6 & Litoral Cantabro-Atlantico ; Agrosistema intensivo (mosaico agroforestal) ; Mesotemperado inferior & 33.02 & 3.60 \\ 
  7 & Litoral Cantabro-Atlantico ; Rururbano (diseminado) ; Termotemperado & 24.80 & 2.70 \\ 
  8 & Serras ; Turbeira ; Mesotemperado superior & 24.40 & 2.60 \\ 
  9 & Litoral Cantabro-Atlantico ; Agrosistema intensivo (superficie de cultivo) ; Termotemperado & 19.12 & 2.10 \\ 
  10 & Vales sublitorais ; Agrosistema intensivo (mosaico agroforestal) ; Termotemperado & 19.39 & 2.10 \\ 
  11 & Serras ; Agrosistema intensivo (mosaico agroforestal) ; Mesotemperado superior & 18.10 & 2.00 \\ 
  12 & Serras ; Agrosistema intensivo (plantacion forestal) ; Mesotemperado inferior & 15.38 & 1.70 \\ 
  13 & Serras ; Matogueira e rochedo ; Mesotemperado superior & 15.85 & 1.70 \\ 
  14 & Vales sublitorais ; Bosque ; Mesotemperado superior & 11.24 & 1.20 \\ 
  15 & Litoral Cantabro-Atlantico ; Urbano ; Termotemperado & 10.22 & 1.10 \\ 
  16 & Vales sublitorais ; Agrosistema extensivo ; Mesotemperado inferior & 9.05 & 1.00 \\ 
  17 & Total & 842.54 & 91.30 \\ 
   \hline
\end{tabular}
\end{table}
% latex table generated in R 3.2.5 by xtable 1.8-0 package
% Fri May  6 19:53:39 2016
\begin{table}[p]
\centering
\caption{Principais tipos de paisaxe,  Costa Sur - Baixo Miño ( 3 )} 
\label{Tipos 3}
\begin{tabular}{rlrr}
  \hline
 & Tipo & Área (km²) & Porcentaxe \\ 
  \hline
1 & Vales sublitorais ; Agrosistema intensivo (plantacion forestal) ; Termotemperado & 199.01 & 16.80 \\ 
  2 & Serras ; Matogueira e rochedo ; Mesotemperado superior & 131.12 & 11.10 \\ 
  3 & Vales sublitorais ; Agrosistema intensivo (mosaico agroforestal) ; Termotemperado & 89.27 & 7.50 \\ 
  4 & Chairas e vales interiores ; Agrosistema intensivo (plantacion forestal) ; Termotemperado & 79.36 & 6.70 \\ 
  5 & Litoral Cantabro-Atlantico ; Agrosistema intensivo (plantacion forestal) ; Termotemperado & 79.12 & 6.70 \\ 
  6 & Litoral Cantabro-Atlantico ; Agrosistema intensivo (mosaico agroforestal) ; Termotemperado & 68.94 & 5.80 \\ 
  7 & Litoral Cantabro-Atlantico ; Rururbano (diseminado) ; Termotemperado & 58.86 & 5.00 \\ 
  8 & Serras ; Agrosistema intensivo (plantacion forestal) ; Termotemperado & 44.79 & 3.80 \\ 
  9 & Serras ; Matogueira e rochedo ; Mesotemperado inferior & 45.47 & 3.80 \\ 
  10 & Serras ; Agrosistema intensivo (plantacion forestal) ; Mesotemperado inferior & 39.94 & 3.40 \\ 
  11 & Vales sublitorais ; Rururbano (diseminado) ; Termotemperado & 33.32 & 2.80 \\ 
  12 & Vales sublitorais ; Matogueira e rochedo ; Termotemperado & 32.14 & 2.70 \\ 
  13 & Vales sublitorais ; Agrosistema extensivo ; Termotemperado & 30.68 & 2.60 \\ 
  14 & Chairas e vales interiores ; Matogueira e rochedo ; Mesotemperado inferior & 23.20 & 2.00 \\ 
  15 & Vales sublitorais ; Bosque ; Termotemperado & 17.02 & 1.40 \\ 
  16 & Chairas e vales interiores ; Agrosistema intensivo (mosaico agroforestal) ; Termotemperado & 13.31 & 1.10 \\ 
  17 & Litoral Cantabro-Atlantico ; Urbano ; Termotemperado & 13.02 & 1.10 \\ 
  18 & Chairas e vales interiores ; Vinedo ; Termotemperado & 11.27 & 1.00 \\ 
  19 & Total & 1009.84 & 85.30 \\ 
   \hline
\end{tabular}
\end{table}
% latex table generated in R 3.2.5 by xtable 1.8-0 package
% Fri May  6 19:53:39 2016
\begin{table}[p]
\centering
\caption{Principais tipos de paisaxe,  Ribeiras Encaixadas do Miño e do Sil ( 4 )} 
\label{Tipos 4}
\begin{tabular}{rlrr}
  \hline
 & Tipo & Área (km²) & Porcentaxe \\ 
  \hline
1 & Serras ; Matogueira e rochedo ; Supra e orotemperado & 224.55 & 9.10 \\ 
  2 & Chairas e vales interiores ; Agrosistema intensivo (plantacion forestal) ; Termotemperado & 168.29 & 6.80 \\ 
  3 & Chairas e vales interiores ; Agrosistema extensivo ; Mesotemperado inferior & 162.15 & 6.60 \\ 
  4 & Chairas e vales interiores ; Agrosistema intensivo (mosaico agroforestal) ; Mesotemperado inferior & 132.77 & 5.40 \\ 
  5 & Chairas e vales interiores ; Agrosistema intensivo (plantacion forestal) ; Mesotemperado inferior & 108.11 & 4.40 \\ 
  6 & Chairas e vales interiores ; Bosque ; Mesotemperado inferior & 95.19 & 3.90 \\ 
  7 & Chairas e vales interiores ; Matogueira e rochedo ; Mesotemperado inferior & 89.19 & 3.60 \\ 
  8 & Serras ; Agrosistema intensivo (mosaico agroforestal) ; Mesotemperado superior & 81.98 & 3.30 \\ 
  9 & Serras ; Matogueira e rochedo ; Mesotemperado superior & 76.70 & 3.10 \\ 
  10 & Chairas e vales interiores ; Bosque ; Termotemperado & 74.94 & 3.00 \\ 
  11 & Serras ; Matogueira e rochedo ; Mesotemperado inferior & 68.89 & 2.80 \\ 
  12 & Canons ; Bosque ; Mesotemperado inferior & 65.90 & 2.70 \\ 
  13 & Serras ; Agrosistema intensivo (plantacion forestal) ; Mesotemperado inferior & 65.56 & 2.70 \\ 
  14 & Chairas e vales interiores ; Matogueira e rochedo ; Termotemperado & 59.99 & 2.40 \\ 
  15 & Serras ; Bosque ; Mesotemperado inferior & 57.28 & 2.30 \\ 
  16 & Chairas e vales interiores ; Matogueira e rochedo ;  & 55.54 & 2.20 \\ 
  17 & Chairas e vales interiores ; Vinedo ; Termotemperado & 54.85 & 2.20 \\ 
  18 & Serras ; Agrosistema extensivo ; Mesotemperado inferior & 48.82 & 2.00 \\ 
  19 &  ; Lamina de auga ;  & 46.84 & 1.90 \\ 
  20 & Chairas e vales interiores ; Agrosistema intensivo (plantacion forestal) ;  & 45.24 & 1.80 \\ 
  21 & Chairas e vales interiores ; Vinedo ;  & 41.97 & 1.70 \\ 
  22 & Serras ; Agrosistema extensivo ; Mesotemperado superior & 42.26 & 1.70 \\ 
  23 & Serras ; Agrosistema intensivo (mosaico agroforestal) ; Mesotemperado inferior & 42.18 & 1.70 \\ 
  24 & Chairas e vales interiores ; Agrosistema extensivo ; Termotemperado & 38.48 & 1.60 \\ 
  25 & Canons ; Agrosistema intensivo (plantacion forestal) ; Termotemperado & 34.25 & 1.40 \\ 
  26 & Canons ; Agrosistema intensivo (plantacion forestal) ;  & 30.85 & 1.20 \\ 
  27 & Chairas e vales interiores ; Agrosistema intensivo (mosaico agroforestal) ; Termotemperado & 26.07 & 1.10 \\ 
  28 & Serras ; Agrosistema intensivo (plantacion forestal) ; Mesotemperado superior & 27.58 & 1.10 \\ 
  29 & Canons ; Bosque ; Termotemperado & 25.26 & 1.00 \\ 
  30 & Canons ; Matogueira e rochedo ;  & 23.70 & 1.00 \\ 
  31 & Chairas e vales interiores ; Urbano ; Termotemperado & 24.89 & 1.00 \\ 
  32 & Total & 2140.27 & 86.70 \\ 
   \hline
\end{tabular}
\end{table}
% latex table generated in R 3.2.5 by xtable 1.8-0 package
% Fri May  6 19:53:39 2016
\begin{table}[p]
\centering
\caption{Principais tipos de paisaxe,  Serras Orientais ( 5 )} 
\label{Tipos 5}
\begin{tabular}{rlrr}
  \hline
 & Tipo & Área (km²) & Porcentaxe \\ 
  \hline
1 & Serras ; Matogueira e rochedo ; Supra e orotemperado & 389.28 & 15.60 \\ 
  2 & Serras ; Agrosistema extensivo ; Supra e orotemperado & 222.35 & 8.90 \\ 
  3 & Serras ; Bosque ; Supra e orotemperado & 191.24 & 7.70 \\ 
  4 & Serras ; Agrosistema extensivo ; Mesotemperado superior & 182.18 & 7.30 \\ 
  5 & Serras ; Agrosistema intensivo (plantacion forestal) ; Supra e orotemperado & 171.07 & 6.90 \\ 
  6 & Serras ; Agrosistema intensivo (mosaico agroforestal) ; Supra e orotemperado & 143.16 & 5.70 \\ 
  7 & Serras ; Matogueira e rochedo ; Mesotemperado superior & 134.97 & 5.40 \\ 
  8 & Vales sublitorais ; Agrosistema extensivo ; Mesotemperado superior & 127.37 & 5.10 \\ 
  9 & Vales sublitorais ; Bosque ; Mesotemperado superior & 103.88 & 4.20 \\ 
  10 & Serras ; Agrosistema intensivo (mosaico agroforestal) ; Mesotemperado superior & 98.41 & 3.90 \\ 
  11 & Serras ; Agrosistema intensivo (plantacion forestal) ; Mesotemperado superior & 96.24 & 3.90 \\ 
  12 & Vales sublitorais ; Bosque ; Mesotemperado inferior & 95.60 & 3.80 \\ 
  13 & Serras ; Bosque ; Mesotemperado superior & 91.11 & 3.70 \\ 
  14 & Vales sublitorais ; Matogueira e rochedo ; Mesotemperado superior & 61.05 & 2.40 \\ 
  15 & Vales sublitorais ; Agrosistema intensivo (mosaico agroforestal) ; Mesotemperado inferior & 38.58 & 1.50 \\ 
  16 & Vales sublitorais ; Agrosistema extensivo ; Mesotemperado inferior & 35.12 & 1.40 \\ 
  17 & Vales sublitorais ; Agrosistema intensivo (plantacion forestal) ; Mesotemperado superior & 31.68 & 1.30 \\ 
  18 & Vales sublitorais ; Matogueira e rochedo ; Mesotemperado inferior & 29.90 & 1.20 \\ 
  19 & Chairas e vales interiores ; Agrosistema intensivo (plantacion forestal) ; Mesotemperado inferior & 27.17 & 1.10 \\ 
  20 & Vales sublitorais ; Agrosistema intensivo (plantacion forestal) ; Mesotemperado inferior & 26.29 & 1.10 \\ 
  21 & Total & 2296.65 & 92.10 \\ 
   \hline
\end{tabular}
\end{table}
% latex table generated in R 3.2.5 by xtable 1.8-0 package
% Fri May  6 19:53:39 2016
\begin{table}[p]
\centering
\caption{Principais tipos de paisaxe,  Chairas e Fosas Luguesas ( 6 )} 
\label{Tipos 6}
\begin{tabular}{rlrr}
  \hline
 & Tipo & Área (km²) & Porcentaxe \\ 
  \hline
1 & Chairas e vales interiores ; Agrosistema intensivo (mosaico agroforestal) ; Mesotemperado superior & 916.96 & 20.10 \\ 
  2 & Serras ; Agrosistema intensivo (mosaico agroforestal) ; Mesotemperado superior & 446.39 & 9.80 \\ 
  3 & Chairas e vales interiores ; Agrosistema extensivo ; Mesotemperado superior & 441.38 & 9.70 \\ 
  4 & Chairas e vales interiores ; Agrosistema intensivo (mosaico agroforestal) ; Mesotemperado inferior & 430.63 & 9.40 \\ 
  5 & Chairas e vales interiores ; Agrosistema extensivo ; Mesotemperado inferior & 309.88 & 6.80 \\ 
  6 & Serras ; Agrosistema extensivo ; Mesotemperado superior & 270.81 & 5.90 \\ 
  7 & Chairas e vales interiores ; Agrosistema intensivo (superficie de cultivo) ; Mesotemperado superior & 263.34 & 5.80 \\ 
  8 & Chairas e vales interiores ; Agrosistema intensivo (plantacion forestal) ; Mesotemperado superior & 180.97 & 4.00 \\ 
  9 & Chairas e vales interiores ; Agrosistema intensivo (plantacion forestal) ; Mesotemperado inferior & 151.02 & 3.30 \\ 
  10 & Serras ; Agrosistema intensivo (superficie de cultivo) ; Mesotemperado superior & 138.52 & 3.00 \\ 
  11 & Chairas e vales interiores ; Agrosistema intensivo (superficie de cultivo) ; Mesotemperado inferior & 107.51 & 2.40 \\ 
  12 & Serras ; Agrosistema intensivo (plantacion forestal) ; Mesotemperado superior & 104.52 & 2.30 \\ 
  13 & Chairas e vales interiores ; Bosque ; Mesotemperado superior & 92.18 & 2.00 \\ 
  14 & Chairas e vales interiores ; Bosque ; Mesotemperado inferior & 87.47 & 1.90 \\ 
  15 & Serras ; Matogueira e rochedo ; Mesotemperado superior & 88.36 & 1.90 \\ 
  16 & Chairas e vales interiores ; Matogueira e rochedo ; Mesotemperado inferior & 63.52 & 1.40 \\ 
  17 & Chairas e vales interiores ; Matogueira e rochedo ; Mesotemperado superior & 65.12 & 1.40 \\ 
  18 & Serras ; Turbeira ; Mesotemperado superior & 49.32 & 1.10 \\ 
  19 & Total & 4207.90 & 92.20 \\ 
   \hline
\end{tabular}
\end{table}
% latex table generated in R 3.2.5 by xtable 1.8-0 package
% Fri May  6 19:53:39 2016
\begin{table}[p]
\centering
\caption{Principais tipos de paisaxe,  Galicia Central ( 7 )} 
\label{Tipos 7}
\begin{tabular}{rlrr}
  \hline
 & Tipo & Área (km²) & Porcentaxe \\ 
  \hline
1 & Vales sublitorais ; Agrosistema intensivo (mosaico agroforestal) ; Mesotemperado inferior & 2218.51 & 43.10 \\ 
  2 & Serras ; Matogueira e rochedo ; Mesotemperado superior & 355.35 & 6.90 \\ 
  3 & Vales sublitorais ; Agrosistema intensivo (plantacion forestal) ; Mesotemperado inferior & 300.83 & 5.80 \\ 
  4 & Vales sublitorais ; Agrosistema extensivo ; Mesotemperado inferior & 268.36 & 5.20 \\ 
  5 & Serras ; Agrosistema extensivo ; Mesotemperado superior & 230.47 & 4.50 \\ 
  6 & Serras ; Agrosistema intensivo (mosaico agroforestal) ; Mesotemperado superior & 202.46 & 3.90 \\ 
  7 & Vales sublitorais ; Agrosistema intensivo (superficie de cultivo) ; Mesotemperado inferior & 163.76 & 3.20 \\ 
  8 & Serras ; Agrosistema intensivo (plantacion forestal) ; Mesotemperado superior & 93.52 & 1.80 \\ 
  9 & Vales sublitorais ; Agrosistema intensivo (plantacion forestal) ; Termotemperado & 92.58 & 1.80 \\ 
  10 & Vales sublitorais ; Agrosistema extensivo ; Mesotemperado superior & 82.76 & 1.60 \\ 
  11 & Vales sublitorais ; Matogueira e rochedo ; Mesotemperado inferior & 84.77 & 1.60 \\ 
  12 & Serras ; Agrosistema intensivo (superficie de cultivo) ; Mesotemperado superior & 74.39 & 1.40 \\ 
  13 & Serras ; Agrosistema intensivo (mosaico agroforestal) ; Mesotemperado inferior & 68.40 & 1.30 \\ 
  14 & Chairas e vales interiores ; Agrosistema intensivo (mosaico agroforestal) ; Mesotemperado inferior & 60.27 & 1.20 \\ 
  15 & Serras ; Agrosistema extensivo ; Mesotemperado inferior & 60.48 & 1.20 \\ 
  16 & Chairas e vales interiores ; Agrosistema intensivo (plantacion forestal) ; Mesotemperado inferior & 51.76 & 1.00 \\ 
  17 & Total & 4408.67 & 85.50 \\ 
   \hline
\end{tabular}
\end{table}
% latex table generated in R 3.2.5 by xtable 1.8-0 package
% Fri May  6 19:53:39 2016
\begin{table}[p]
\centering
\caption{Principais tipos de paisaxe,  Chairas, Fosas e Serras Ourensás ( 8 )} 
\label{Tipos 8}
\begin{tabular}{rlrr}
  \hline
 & Tipo & Área (km²) & Porcentaxe \\ 
  \hline
1 & Serras ; Matogueira e rochedo ; Supra e orotemperado & 399.90 & 14.10 \\ 
  2 & Serras ; Matogueira e rochedo ; Mesotemperado superior & 298.99 & 10.50 \\ 
  3 & Chairas e vales interiores ; Agrosistema intensivo (superficie de cultivo) ; Mesotemperado inferior & 214.30 & 7.50 \\ 
  4 & Chairas e vales interiores ; Agrosistema extensivo ; Mesotemperado inferior & 210.07 & 7.40 \\ 
  5 & Chairas e vales interiores ; Bosque ; Mesotemperado inferior & 175.79 & 6.20 \\ 
  6 & Serras ; Agrosistema intensivo (plantacion forestal) ; Mesotemperado inferior & 161.59 & 5.70 \\ 
  7 & Serras ; Agrosistema extensivo ; Mesotemperado inferior & 148.91 & 5.20 \\ 
  8 & Serras ; Matogueira e rochedo ; Mesotemperado inferior & 146.41 & 5.10 \\ 
  9 & Chairas e vales interiores ; Matogueira e rochedo ; Mesotemperado inferior & 137.16 & 4.80 \\ 
  10 & Serras ; Bosque ; Mesotemperado inferior & 132.75 & 4.70 \\ 
  11 & Chairas e vales interiores ; Agrosistema intensivo (plantacion forestal) ; Termotemperado & 107.63 & 3.80 \\ 
  12 & Chairas e vales interiores ; Agrosistema intensivo (plantacion forestal) ; Mesotemperado inferior & 93.92 & 3.30 \\ 
  13 & Serras ; Agrosistema extensivo ; Mesotemperado superior & 69.38 & 2.40 \\ 
  14 & Serras ; Bosque ; Mesotemperado superior & 63.08 & 2.20 \\ 
  15 & Serras ; Agrosistema extensivo ; Supra e orotemperado & 55.68 & 2.00 \\ 
  16 & Chairas e vales interiores ; Matogueira e rochedo ; Termotemperado & 51.12 & 1.80 \\ 
  17 & Serras ; Agrosistema intensivo (plantacion forestal) ; Supra e orotemperado & 43.98 & 1.50 \\ 
  18 & Serras ; Agrosistema intensivo (plantacion forestal) ; Mesotemperado superior & 40.09 & 1.40 \\ 
  19 & Serras ; Bosque ; Supra e orotemperado & 33.57 & 1.20 \\ 
  20 & Chairas e vales interiores ; Agrosistema intensivo (mosaico agroforestal) ; Mesotemperado inferior & 30.68 & 1.10 \\ 
  21 & Total & 2615.00 & 91.90 \\ 
   \hline
\end{tabular}
\end{table}
% latex table generated in R 3.2.5 by xtable 1.8-0 package
% Fri May  6 19:53:39 2016
\begin{table}[p]
\centering
\caption{Principais tipos de paisaxe,  Serras Surorientais ( 9 )} 
\label{Tipos 9}
\begin{tabular}{rlrr}
  \hline
 & Tipo & Área (km²) & Porcentaxe \\ 
  \hline
1 & Serras ; Matogueira e rochedo ; Supra e orotemperado & 1040.66 & 47.30 \\ 
  2 & Serras ; Agrosistema intensivo (plantacion forestal) ; Supra e orotemperado & 225.09 & 10.20 \\ 
  3 & Serras ; Agrosistema extensivo ; Mesotemperado superior & 137.35 & 6.20 \\ 
  4 & Serras ; Bosque ; Mesotemperado superior & 123.40 & 5.60 \\ 
  5 & Serras ; Agrosistema extensivo ; Mesotemperado inferior & 105.52 & 4.80 \\ 
  6 & Serras ; Matogueira e rochedo ; Mesotemperado superior & 73.47 & 3.30 \\ 
  7 & Serras ; Bosque ; Mesotemperado inferior & 67.40 & 3.10 \\ 
  8 & Serras ; Agrosistema extensivo ; Supra e orotemperado & 64.80 & 2.90 \\ 
  9 & Serras ; Matogueira e rochedo ; Mesotemperado inferior & 60.52 & 2.70 \\ 
  10 &  ; Lamina de auga ;  & 30.45 & 1.40 \\ 
  11 & Serras ; Agrosistema intensivo (plantacion forestal) ; Mesotemperado superior & 29.37 & 1.30 \\ 
  12 & Serras ; Bosque ; Supra e orotemperado & 27.68 & 1.30 \\ 
  13 & Canons ; Matogueira e rochedo ; Mesotemperado inferior & 27.27 & 1.20 \\ 
  14 & Canons ; Bosque ; Mesotemperado inferior & 22.10 & 1.00 \\ 
  15 & Total & 2035.08 & 92.30 \\ 
   \hline
\end{tabular}
\end{table}
% latex table generated in R 3.2.5 by xtable 1.8-0 package
% Fri May  6 19:53:39 2016
\begin{table}[p]
\centering
\caption{Principais tipos de paisaxe,  Galicia Setentrional ( 10 )} 
\label{Tipos 10}
\begin{tabular}{rlrr}
  \hline
 & Tipo & Área (km²) & Porcentaxe \\ 
  \hline
1 & Vales sublitorais ; Agrosistema intensivo (plantacion forestal) ; Mesotemperado inferior & 281.73 & 17.30 \\ 
  2 & Vales sublitorais ; Agrosistema intensivo (mosaico agroforestal) ; Mesotemperado inferior & 221.92 & 13.60 \\ 
  3 & Serras ; Turbeira ; Mesotemperado superior & 202.51 & 12.40 \\ 
  4 & Litoral Cantabro-Atlantico ; Agrosistema intensivo (plantacion forestal) ; Termotemperado & 123.91 & 7.60 \\ 
  5 & Serras ; Matogueira e rochedo ; Mesotemperado superior & 110.47 & 6.80 \\ 
  6 & Litoral Cantabro-Atlantico ; Agrosistema intensivo (mosaico agroforestal) ; Termotemperado & 89.99 & 5.50 \\ 
  7 & Serras ; Agrosistema intensivo (plantacion forestal) ; Mesotemperado superior & 76.26 & 4.70 \\ 
  8 & Serras ; Agrosistema intensivo (mosaico agroforestal) ; Mesotemperado superior & 60.24 & 3.70 \\ 
  9 & Litoral Cantabro-Atlantico ; Agrosistema intensivo (plantacion forestal) ; Mesotemperado inferior & 45.48 & 2.80 \\ 
  10 & Serras ; Agrosistema extensivo ; Mesotemperado superior & 44.03 & 2.70 \\ 
  11 & Serras ; Agrosistema intensivo (plantacion forestal) ; Mesotemperado inferior & 40.69 & 2.50 \\ 
  12 & Litoral Cantabro-Atlantico ; Matogueira e rochedo ; Termotemperado & 28.87 & 1.80 \\ 
  13 & Serras ; Bosque ; Mesotemperado superior & 26.50 & 1.60 \\ 
  14 & Vales sublitorais ; Bosque ; Mesotemperado inferior & 24.30 & 1.50 \\ 
  15 & Litoral Cantabro-Atlantico ; Rururbano (diseminado) ; Termotemperado & 18.65 & 1.10 \\ 
  16 & Serras ; Agrosistema intensivo (mosaico agroforestal) ; Mesotemperado inferior & 16.31 & 1.00 \\ 
  17 & Total & 1411.86 & 86.60 \\ 
   \hline
\end{tabular}
\end{table}
% latex table generated in R 3.2.5 by xtable 1.8-0 package
% Fri May  6 19:53:39 2016
\begin{table}[p]
\centering
\caption{Principais tipos de paisaxe,  Chairas e Fosas Occidentais ( 11 )} 
\label{Tipos 11}
\begin{tabular}{rlrr}
  \hline
 & Tipo & Área (km²) & Porcentaxe \\ 
  \hline
1 & Vales sublitorais ; Agrosistema intensivo (mosaico agroforestal) ; Mesotemperado inferior & 748.47 & 36.00 \\ 
  2 & Vales sublitorais ; Agrosistema intensivo (plantacion forestal) ; Mesotemperado inferior & 315.64 & 15.20 \\ 
  3 & Litoral Cantabro-Atlantico ; Agrosistema intensivo (mosaico agroforestal) ; Termotemperado & 205.77 & 9.90 \\ 
  4 & Litoral Cantabro-Atlantico ; Agrosistema intensivo (plantacion forestal) ; Termotemperado & 157.72 & 7.60 \\ 
  5 & Vales sublitorais ; Agrosistema intensivo (superficie de cultivo) ; Mesotemperado inferior & 100.42 & 4.80 \\ 
  6 & Vales sublitorais ; Matogueira e rochedo ; Mesotemperado inferior & 99.02 & 4.80 \\ 
  7 & Litoral Cantabro-Atlantico ; Matogueira e rochedo ; Termotemperado & 95.16 & 4.60 \\ 
  8 & Vales sublitorais ; Agrosistema intensivo (plantacion forestal) ; Termotemperado & 71.46 & 3.40 \\ 
  9 & Vales sublitorais ; Agrosistema intensivo (mosaico agroforestal) ; Termotemperado & 44.69 & 2.20 \\ 
  10 & Vales sublitorais ; Matogueira e rochedo ; Mesotemperado superior & 46.03 & 2.20 \\ 
  11 & Vales sublitorais ; Agrosistema extensivo ; Mesotemperado inferior & 19.75 & 1.00 \\ 
  12 & Vales sublitorais ; Agrosistema intensivo (plantacion forestal) ; Mesotemperado superior & 19.93 & 1.00 \\ 
  13 & Vales sublitorais ; Agrosistema intensivo (superficie de cultivo) ; Termotemperado & 19.93 & 1.00 \\ 
  14 & Total & 1943.99 & 93.70 \\ 
   \hline
\end{tabular}
\end{table}
% latex table generated in R 3.2.5 by xtable 1.8-0 package
% Fri May  6 19:53:39 2016
\begin{table}[p]
\centering
\caption{Principais tipos de paisaxe,  Rías Baixas ( 12 )} 
\label{Tipos 12}
\begin{tabular}{rlrr}
  \hline
 & Tipo & Área (km²) & Porcentaxe \\ 
  \hline
1 & Vales sublitorais ; Agrosistema intensivo (plantacion forestal) ; Termotemperado & 443.17 & 16.40 \\ 
  2 & Litoral Cantabro-Atlantico ; Agrosistema intensivo (mosaico agroforestal) ; Termotemperado & 283.19 & 10.50 \\ 
  3 & Litoral Cantabro-Atlantico ; Agrosistema intensivo (plantacion forestal) ; Termotemperado & 275.56 & 10.20 \\ 
  4 & Vales sublitorais ; Agrosistema intensivo (mosaico agroforestal) ; Termotemperado & 205.93 & 7.60 \\ 
  5 & Litoral Cantabro-Atlantico ; Rururbano (diseminado) ; Termotemperado & 191.86 & 7.10 \\ 
  6 & Serras ; Matogueira e rochedo ; Mesotemperado superior & 180.94 & 6.70 \\ 
  7 & Vales sublitorais ; Matogueira e rochedo ; Mesotemperado inferior & 127.14 & 4.70 \\ 
  8 & Vales sublitorais ; Agrosistema intensivo (mosaico agroforestal) ; Mesotemperado inferior & 109.35 & 4.00 \\ 
  9 & Vales sublitorais ; Agrosistema intensivo (plantacion forestal) ; Mesotemperado inferior & 102.94 & 3.80 \\ 
  10 & Litoral Cantabro-Atlantico ; Urbano ; Termotemperado & 88.71 & 3.30 \\ 
  11 & Serras ; Matogueira e rochedo ; Mesotemperado inferior & 88.96 & 3.30 \\ 
  12 & Vales sublitorais ; Matogueira e rochedo ; Termotemperado & 81.08 & 3.00 \\ 
  13 & Serras ; Agrosistema intensivo (plantacion forestal) ; Mesotemperado inferior & 76.04 & 2.80 \\ 
  14 & Litoral Cantabro-Atlantico ; Matogueira e rochedo ; Termotemperado & 46.80 & 1.70 \\ 
  15 & Litoral Cantabro-Atlantico ; Vinedo ; Termotemperado & 39.27 & 1.50 \\ 
  16 & Vales sublitorais ; Rururbano (diseminado) ; Termotemperado & 35.39 & 1.30 \\ 
  17 & Litoral Cantabro-Atlantico ; Agrosistema intensivo (superficie de cultivo) ; Termotemperado & 27.90 & 1.00 \\ 
  18 & no data ; no data ; no data & 27.28 & 1.00 \\ 
  19 & Total & 2431.51 & 89.90 \\ 
   \hline
\end{tabular}
\end{table}
