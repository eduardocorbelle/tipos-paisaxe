% latex table generated in R 3.2.1 by xtable 1.7-4 package
% Tue Oct  6 00:09:19 2015
\begin{table}[p]
\centering
\caption{Principais tipos de paisaxe,  Golfo Ártabro ( 1 )} 
\label{Tipos 1}
\begin{tabular}{rlr}
  \hline
 & Tipo & Porcentaxe \\ 
  \hline
1 & Litoral Cantabro-Atlantico ; Rururbano (diseminado) ; Termotemperado & 19.30 \\ 
  2 & Vales sublitorais ; Agrosistema intensivo (mosaico agroforestal) ; Mesotemperado inferior & 14.40 \\ 
  3 & Litoral Cantabro-Atlantico ; Agrosistema intensivo (mosaico agroforestal) ; Termotemperado & 10.40 \\ 
  4 & Vales sublitorais ; Agrosistema intensivo (plantacion forestal) ; Mesotemperado inferior & 7.90 \\ 
  5 & Vales sublitorais ; Agrosistema intensivo (mosaico agroforestal) ; Mesotemperado superior & 4.40 \\ 
  6 & Vales sublitorais ; Agrosistema intensivo (mosaico agroforestal) ; Termotemperado & 4.10 \\ 
  7 & Litoral Cantabro-Atlantico ; Urbano ; Termotemperado & 3.10 \\ 
  8 & Serras ; Matogueira e rochedo ; Mesotemperado superior & 3.00 \\ 
  9 & Litoral Cantabro-Atlantico ; Agrosistema intensivo (plantacion forestal) ; Termotemperado & 2.80 \\ 
  10 & Vales sublitorais ; Rururbano (diseminado) ; Termotemperado & 2.10 \\ 
  11 & Serras ; Agrosistema extensivo ; Mesotemperado superior & 1.60 \\ 
  12 & Serras ; Agrosistema intensivo (mosaico agroforestal) ; Mesotemperado superior & 1.60 \\ 
  13 & Vales sublitorais ; Agrosistema intensivo (plantacion forestal) ; Termotemperado & 1.60 \\ 
  14 & Serras ; Turbeira ; Mesotemperado superior & 1.50 \\ 
  15 & Vales sublitorais ; Agrosistema intensivo (plantacion forestal) ; Mesotemperado superior & 1.50 \\ 
  16 & Vales sublitorais ; Rururbano (diseminado) ; Mesotemperado inferior & 1.50 \\ 
  17 & Vales sublitorais ; Agrosistema extensivo ; Mesotemperado superior & 1.40 \\ 
  18 & Vales sublitorais ; Matogueira e rochedo ; Mesotemperado inferior & 1.40 \\ 
  19 & Vales sublitorais ; Matogueira e rochedo ; Mesotemperado superior & 1.20 \\ 
  20 & Canons ; Bosque ; Mesotemperado inferior & 1.00 \\ 
  21 & Total & 85.80 \\ 
   \hline
\end{tabular}
\end{table}
% latex table generated in R 3.2.1 by xtable 1.7-4 package
% Tue Oct  6 00:09:19 2015
\begin{table}[p]
\centering
\caption{Principais tipos de paisaxe,  A Mariña - Baixo Eo ( 2 )} 
\label{Tipos 2}
\begin{tabular}{rlr}
  \hline
 & Tipo & Porcentaxe \\ 
  \hline
1 & Vales sublitorais ; Agrosistema intensivo (plantacion forestal) ; Mesotemperado inferior & 14.00 \\ 
  2 & Vales sublitorais ; Agrosistema intensivo (mosaico agroforestal) ; Mesotemperado inferior & 12.60 \\ 
  3 & Litoral Cantabro-Atlantico ; Agrosistema intensivo (plantacion forestal) ; Mesotemperado inferior & 10.80 \\ 
  4 & Litoral Cantabro-Atlantico ; Agrosistema intensivo (mosaico agroforestal) ; Termotemperado & 8.50 \\ 
  5 & Litoral Cantabro-Atlantico ; Agrosistema intensivo (plantacion forestal) ; Termotemperado & 4.60 \\ 
  6 & Litoral Cantabro-Atlantico ; Agrosistema intensivo (mosaico agroforestal) ; Mesotemperado inferior & 4.30 \\ 
  7 & Serras ; Agrosistema intensivo (plantacion forestal) ; Mesotemperado superior & 4.00 \\ 
  8 & Litoral Cantabro-Atlantico ; Rururbano (diseminado) ; Termotemperado & 3.90 \\ 
  9 & Vales sublitorais ; Agrosistema intensivo (mosaico agroforestal) ; Termotemperado & 3.70 \\ 
  10 & Vales sublitorais ; Agrosistema intensivo (plantacion forestal) ; Mesotemperado superior & 3.20 \\ 
  11 & Serras ; Turbeira ; Mesotemperado superior & 2.30 \\ 
  12 & Litoral Cantabro-Atlantico ; Urbano ; Termotemperado & 2.20 \\ 
  13 & Serras ; Agrosistema intensivo (mosaico agroforestal) ; Mesotemperado superior & 2.20 \\ 
  14 & Serras ; Agrosistema intensivo (plantacion forestal) ; Mesotemperado inferior & 2.20 \\ 
  15 & Serras ; Matogueira e rochedo ; Mesotemperado superior & 2.10 \\ 
  16 & Vales sublitorais ; Agrosistema intensivo (plantacion forestal) ; Termotemperado & 2.10 \\ 
  17 & Vales sublitorais ; Agrosistema intensivo (mosaico agroforestal) ; Mesotemperado superior & 1.60 \\ 
  18 & Litoral Cantabro-Atlantico ; Agrosistema intensivo (superficie de cultivo) ; Termotemperado & 1.50 \\ 
  19 & Vales sublitorais ; Matogueira e rochedo ; Mesotemperado inferior & 1.50 \\ 
  20 & Vales sublitorais ; Agrosistema extensivo ; Mesotemperado inferior & 1.30 \\ 
  21 & Vales sublitorais ; Rururbano (diseminado) ; Mesotemperado inferior & 1.20 \\ 
  22 & Total & 89.80 \\ 
   \hline
\end{tabular}
\end{table}
% latex table generated in R 3.2.1 by xtable 1.7-4 package
% Tue Oct  6 00:09:19 2015
\begin{table}[p]
\centering
\caption{Principais tipos de paisaxe,  Costa Sur - Baixo Miño ( 3 )} 
\label{Tipos 3}
\begin{tabular}{rlr}
  \hline
 & Tipo & Porcentaxe \\ 
  \hline
1 & Vales sublitorais ; Rururbano (diseminado) ; Termotemperado & 12.50 \\ 
  2 & Litoral Cantabro-Atlantico ; Rururbano (diseminado) ; Termotemperado & 11.40 \\ 
  3 & Vales sublitorais ; Agrosistema intensivo (plantacion forestal) ; Termotemperado & 8.90 \\ 
  4 & Serras ; Matogueira e rochedo ; Mesotemperado superior & 5.90 \\ 
  5 & Serras ; Matogueira e rochedo ; Mesotemperado inferior & 5.10 \\ 
  6 & Serras ; Matogueira e rochedo ; Supra e orotemperado & 4.30 \\ 
  7 & Vales sublitorais ; Matogueira e rochedo ; Termotemperado & 3.40 \\ 
  8 & Litoral Cantabro-Atlantico ; Agrosistema intensivo (mosaico agroforestal) ; Termotemperado & 3.30 \\ 
  9 & Vales sublitorais ; Agrosistema intensivo (mosaico agroforestal) ; Termotemperado & 2.90 \\ 
  10 & Serras ; Agrosistema intensivo (plantacion forestal) ; Termotemperado & 2.80 \\ 
  11 & Chairas e vales interiores ; Matogueira e rochedo ; Mesotemperado inferior & 2.50 \\ 
  12 & Litoral Cantabro-Atlantico ; Agrosistema intensivo (plantacion forestal) ; Termotemperado & 2.50 \\ 
  13 & Chairas e vales interiores ; Rururbano (diseminado) ; Termotemperado & 2.30 \\ 
  14 & Serras ; Agrosistema intensivo (plantacion forestal) ; Mesotemperado inferior & 2.00 \\ 
  15 & Chairas e vales interiores ; Agrosistema intensivo (mosaico agroforestal) ; Termotemperado & 1.90 \\ 
  16 & Chairas e vales interiores ; Agrosistema intensivo (plantacion forestal) ; Termotemperado & 1.80 \\ 
  17 & Serras ; Matogueira e rochedo ; Termotemperado & 1.70 \\ 
  18 & Chairas e vales interiores ; Matogueira e rochedo ; Termotemperado & 1.50 \\ 
  19 & Vales sublitorais ; Agrosistema extensivo ; Termotemperado & 1.50 \\ 
  20 & Litoral Cantabro-Atlantico ; Urbano ; Termotemperado & 1.40 \\ 
  21 & no data ; Rururbano (diseminado) ; Termotemperado & 1.10 \\ 
  22 & Chairas e vales interiores ; Agrosistema intensivo (mosaico agroforestal) ; Mesotemperado inferior & 1.00 \\ 
  23 & Vales sublitorais ; Bosque ; Termotemperado & 1.00 \\ 
  24 & Total & 82.70 \\ 
   \hline
\end{tabular}
\end{table}
% latex table generated in R 3.2.1 by xtable 1.7-4 package
% Tue Oct  6 00:09:19 2015
\begin{table}[p]
\centering
\caption{Principais tipos de paisaxe,  Ribeiras Encaixadas do Miño e do Sil ( 4 )} 
\label{Tipos 4}
\begin{tabular}{rlr}
  \hline
 & Tipo & Porcentaxe \\ 
  \hline
1 & Chairas e vales interiores ; Agrosistema extensivo ; Mesotemperado inferior & 9.20 \\ 
  2 & Serras ; Matogueira e rochedo ; Mesotemperado superior & 5.30 \\ 
  3 & Serras ; Matogueira e rochedo ; Mesotemperado inferior & 5.10 \\ 
  4 & Chairas e vales interiores ; Agrosistema intensivo (mosaico agroforestal) ; Mesotemperado inferior & 4.80 \\ 
  5 & Serras ; Agrosistema extensivo ; Mesotemperado inferior & 4.50 \\ 
  6 & Chairas e vales interiores ; Matogueira e rochedo ; Termotemperado & 4.30 \\ 
  7 & Chairas e vales interiores ; Matogueira e rochedo ; Mesotemperado inferior & 4.20 \\ 
  8 & Serras ; Matogueira e rochedo ; Supra e orotemperado & 4.10 \\ 
  9 & Chairas e vales interiores ; Agrosistema extensivo ; Termotemperado & 3.70 \\ 
  10 & Serras ; Agrosistema extensivo ; Mesotemperado superior & 3.20 \\ 
  11 & Chairas e vales interiores ; Matogueira e rochedo ; Mesomediterráneo & 3.00 \\ 
  12 & Chairas e vales interiores ; Rururbano (diseminado) ; Termotemperado & 2.60 \\ 
  13 & Chairas e vales interiores ; Agrosistema intensivo (plantacion forestal) ; Termotemperado & 2.40 \\ 
  14 & Chairas e vales interiores ; Bosque ; Termotemperado & 2.40 \\ 
  15 & Chairas e vales interiores ; Bosque ; Mesotemperado inferior & 2.30 \\ 
  16 & Chairas e vales interiores ; Viñedo ; Termotemperado & 2.20 \\ 
  17 & Canons ; Bosque ; Mesotemperado inferior & 1.90 \\ 
  18 & Canons ; Matogueira e rochedo ; Mesomediterráneo & 1.80 \\ 
  19 & Serras ; Matogueira e rochedo ; Mesomediterráneo & 1.80 \\ 
  20 & Chairas e vales interiores ; Agrosistema intensivo (mosaico agroforestal) ; Termotemperado & 1.60 \\ 
  21 & Chairas e vales interiores ; Agrosistema intensivo (plantacion forestal) ; Mesotemperado inferior & 1.60 \\ 
  22 & Serras ; Agrosistema intensivo (mosaico agroforestal) ; Mesotemperado superior & 1.50 \\ 
  23 & Serras ; Bosque ; Mesotemperado inferior & 1.50 \\ 
  24 & Canons ; Bosque ; Termotemperado & 1.40 \\ 
  25 & Canons ; Matogueira e rochedo ; Mesotemperado inferior & 1.30 \\ 
  26 & Chairas e vales interiores ; Viñedo ; Mesomediterráneo & 1.30 \\ 
  27 & Canons ; Agrosistema extensivo ; Mesotemperado inferior & 1.20 \\ 
  28 & Chairas e vales interiores ; Agrosistema extensivo ; Mesomediterráneo & 1.20 \\ 
  29 & Chairas e vales interiores ; Rururbano (diseminado) ; Mesotemperado inferior & 1.20 \\ 
  30 & Serras ; Agrosistema intensivo (mosaico agroforestal) ; Mesotemperado inferior & 1.20 \\ 
  31 & Chairas e vales interiores ; Agrosistema intensivo (superficie de cultivo) ; Mesotemperado inferior & 1.00 \\ 
  32 & Serras ; Agrosistema intensivo (superficie de cultivo) ; Mesotemperado superior & 1.00 \\ 
  33 & Total & 85.80 \\ 
   \hline
\end{tabular}
\end{table}
% latex table generated in R 3.2.1 by xtable 1.7-4 package
% Tue Oct  6 00:09:19 2015
\begin{table}[p]
\centering
\caption{Principais tipos de paisaxe,  Serras Orientais ( 5 )} 
\label{Tipos 5}
\begin{tabular}{rlr}
  \hline
 & Tipo & Porcentaxe \\ 
  \hline
1 & Serras ; Agrosistema extensivo ; Supra e orotemperado & 15.20 \\ 
  2 & Serras ; Matogueira e rochedo ; Supra e orotemperado & 14.80 \\ 
  3 & Serras ; Agrosistema extensivo ; Mesotemperado superior & 11.80 \\ 
  4 & Serras ; Matogueira e rochedo ; Mesotemperado superior & 6.90 \\ 
  5 & Vales sublitorais ; Agrosistema extensivo ; Mesotemperado superior & 5.70 \\ 
  6 & Serras ; Bosque ; Supra e orotemperado & 5.20 \\ 
  7 & Vales sublitorais ; Agrosistema extensivo ; Mesotemperado inferior & 3.90 \\ 
  8 & Serras ; Bosque ; Mesotemperado superior & 3.80 \\ 
  9 & Vales sublitorais ; Bosque ; Mesotemperado superior & 3.50 \\ 
  10 & Serras ; Agrosistema intensivo (plantacion forestal) ; Supra e orotemperado & 3.40 \\ 
  11 & Vales sublitorais ; Bosque ; Mesotemperado inferior & 2.80 \\ 
  12 & Vales sublitorais ; Matogueira e rochedo ; Mesotemperado superior & 2.50 \\ 
  13 & Serras ; Agrosistema intensivo (mosaico agroforestal) ; Supra e orotemperado & 2.40 \\ 
  14 & Serras ; Agrosistema intensivo (plantacion forestal) ; Mesotemperado superior & 2.30 \\ 
  15 & Serras ; Agrosistema intensivo (mosaico agroforestal) ; Mesotemperado superior & 2.00 \\ 
  16 & Vales sublitorais ; Matogueira e rochedo ; Mesotemperado inferior & 1.60 \\ 
  17 & Serras ; Matogueira e rochedo ; Mesotemperado inferior & 1.10 \\ 
  18 & Total & 88.90 \\ 
   \hline
\end{tabular}
\end{table}
% latex table generated in R 3.2.1 by xtable 1.7-4 package
% Tue Oct  6 00:09:19 2015
\begin{table}[p]
\centering
\caption{Principais tipos de paisaxe,  Chairas e Fosas Luguesas ( 6 )} 
\label{Tipos 6}
\begin{tabular}{rlr}
  \hline
 & Tipo & Porcentaxe \\ 
  \hline
1 & Chairas e vales interiores ; Agrosistema extensivo ; Mesotemperado superior & 17.50 \\ 
  2 & Chairas e vales interiores ; Agrosistema intensivo (mosaico agroforestal) ; Mesotemperado superior & 14.60 \\ 
  3 & Serras ; Agrosistema extensivo ; Mesotemperado superior & 10.10 \\ 
  4 & Chairas e vales interiores ; Agrosistema extensivo ; Mesotemperado inferior & 9.60 \\ 
  5 & Chairas e vales interiores ; Agrosistema intensivo (mosaico agroforestal) ; Mesotemperado inferior & 6.50 \\ 
  6 & Serras ; Agrosistema intensivo (mosaico agroforestal) ; Mesotemperado superior & 5.40 \\ 
  7 & Chairas e vales interiores ; Agrosistema intensivo (superficie de cultivo) ; Mesotemperado superior & 4.60 \\ 
  8 & Serras ; Agrosistema intensivo (superficie de cultivo) ; Mesotemperado superior & 2.90 \\ 
  9 & Chairas e vales interiores ; Agrosistema intensivo (plantacion forestal) ; Mesotemperado superior & 2.60 \\ 
  10 & Chairas e vales interiores ; Agrosistema intensivo (superficie de cultivo) ; Mesotemperado inferior & 2.30 \\ 
  11 & Chairas e vales interiores ; Matogueira e rochedo ; Mesotemperado inferior & 2.00 \\ 
  12 & Chairas e vales interiores ; Matogueira e rochedo ; Mesotemperado superior & 1.90 \\ 
  13 & Chairas e vales interiores ; Rururbano (diseminado) ; Mesotemperado superior & 1.90 \\ 
  14 & Serras ; Matogueira e rochedo ; Mesotemperado superior & 1.80 \\ 
  15 & Chairas e vales interiores ; Bosque ; Mesotemperado superior & 1.50 \\ 
  16 & Chairas e vales interiores ; Rururbano (diseminado) ; Mesotemperado inferior & 1.40 \\ 
  17 & Chairas e vales interiores ; Bosque ; Mesotemperado inferior & 1.30 \\ 
  18 & Serras ; Agrosistema intensivo (plantacion forestal) ; Mesotemperado superior & 1.30 \\ 
  19 & Serras ; Agrosistema extensivo ; Supra e orotemperado & 1.10 \\ 
  20 & Serras ; Turbeira ; Mesotemperado superior & 1.10 \\ 
  21 & Total & 91.40 \\ 
   \hline
\end{tabular}
\end{table}
% latex table generated in R 3.2.1 by xtable 1.7-4 package
% Tue Oct  6 00:09:19 2015
\begin{table}[p]
\centering
\caption{Principais tipos de paisaxe,  Galicia Central ( 7 )} 
\label{Tipos 7}
\begin{tabular}{rlr}
  \hline
 & Tipo & Porcentaxe \\ 
  \hline
1 & Vales sublitorais ; Agrosistema intensivo (mosaico agroforestal) ; Mesotemperado inferior & 23.80 \\ 
  2 & Vales sublitorais ; Agrosistema extensivo ; Mesotemperado inferior & 10.40 \\ 
  3 & Serras ; Agrosistema extensivo ; Mesotemperado superior & 7.10 \\ 
  4 & Vales sublitorais ; Agrosistema intensivo (mosaico agroforestal) ; Termotemperado & 5.80 \\ 
  5 & Serras ; Matogueira e rochedo ; Mesotemperado superior & 5.00 \\ 
  6 & Vales sublitorais ; Agrosistema intensivo (mosaico agroforestal) ; Mesotemperado superior & 4.40 \\ 
  7 & Vales sublitorais ; Rururbano (diseminado) ; Mesotemperado inferior & 3.70 \\ 
  8 & Vales sublitorais ; Matogueira e rochedo ; Mesotemperado inferior & 3.10 \\ 
  9 & Vales sublitorais ; Agrosistema extensivo ; Mesotemperado superior & 2.90 \\ 
  10 & Serras ; Agrosistema extensivo ; Mesotemperado inferior & 2.70 \\ 
  11 & Vales sublitorais ; Agrosistema intensivo (plantacion forestal) ; Mesotemperado inferior & 2.70 \\ 
  12 & Vales sublitorais ; Agrosistema intensivo (superficie de cultivo) ; Mesotemperado inferior & 2.40 \\ 
  13 & Serras ; Matogueira e rochedo ; Supra e orotemperado & 2.20 \\ 
  14 & Vales sublitorais ; Rururbano (diseminado) ; Termotemperado & 1.90 \\ 
  15 & Serras ; Agrosistema intensivo (mosaico agroforestal) ; Mesotemperado superior & 1.60 \\ 
  16 & Serras ; Matogueira e rochedo ; Mesotemperado inferior & 1.60 \\ 
  17 & Vales sublitorais ; Agrosistema intensivo (plantacion forestal) ; Termotemperado & 1.20 \\ 
  18 & Vales sublitorais ; Matogueira e rochedo ; Mesotemperado superior & 1.10 \\ 
  19 & Vales sublitorais ; Agrosistema extensivo ; Termotemperado & 1.00 \\ 
  20 & Total & 84.60 \\ 
   \hline
\end{tabular}
\end{table}
% latex table generated in R 3.2.1 by xtable 1.7-4 package
% Tue Oct  6 00:09:19 2015
\begin{table}[p]
\centering
\caption{Principais tipos de paisaxe,  Chairas, Fosas e Serras Ourensás ( 8 )} 
\label{Tipos 8}
\begin{tabular}{rlr}
  \hline
 & Tipo & Porcentaxe \\ 
  \hline
1 & Chairas e vales interiores ; Agrosistema extensivo ; Mesotemperado inferior & 12.00 \\ 
  2 & Serras ; Matogueira e rochedo ; Mesotemperado superior & 10.00 \\ 
  3 & Serras ; Agrosistema extensivo ; Mesotemperado inferior & 9.00 \\ 
  4 & Serras ; Matogueira e rochedo ; Mesotemperado inferior & 9.00 \\ 
  5 & Serras ; Matogueira e rochedo ; Supra e orotemperado & 8.40 \\ 
  6 & Chairas e vales interiores ; Agrosistema intensivo (superficie de cultivo) ; Mesotemperado inferior & 7.50 \\ 
  7 & Serras ; Agrosistema extensivo ; Mesotemperado superior & 4.90 \\ 
  8 & Chairas e vales interiores ; Matogueira e rochedo ; Mesotemperado inferior & 4.80 \\ 
  9 & Serras ; Agrosistema extensivo ; Supra e orotemperado & 3.10 \\ 
  10 & Chairas e vales interiores ; Bosque ; Mesotemperado inferior & 2.80 \\ 
  11 & Serras ; Bosque ; Mesotemperado inferior & 2.70 \\ 
  12 & Chairas e vales interiores ; Matogueira e rochedo ; Termotemperado & 2.20 \\ 
  13 & Serras ; Agrosistema intensivo (plantacion forestal) ; Mesotemperado inferior & 2.00 \\ 
  14 & Chairas e vales interiores ; Agrosistema extensivo ; Termotemperado & 1.80 \\ 
  15 & Chairas e vales interiores ; Agrosistema intensivo (plantacion forestal) ; Mesotemperado inferior & 1.70 \\ 
  16 & Serras ; Agrosistema intensivo (superficie de cultivo) ; Mesotemperado inferior & 1.60 \\ 
  17 & Serras ; Bosque ; Mesotemperado superior & 1.60 \\ 
  18 & Chairas e vales interiores ; Agrosistema intensivo (mosaico agroforestal) ; Mesotemperado inferior & 1.40 \\ 
  19 & Serras ; Agrosistema intensivo (superficie de cultivo) ; Mesotemperado superior & 1.30 \\ 
  20 & Chairas e vales interiores ; Agrosistema intensivo (mosaico agroforestal) ; Termotemperado & 1.20 \\ 
  21 & Chairas e vales interiores ; Rururbano (diseminado) ; Mesotemperado inferior & 1.20 \\ 
  22 & Serras ; Agrosistema intensivo (plantacion forestal) ; Mesotemperado superior & 1.10 \\ 
  23 & Serras ; Bosque ; Supra e orotemperado & 1.00 \\ 
  24 & Total & 92.30 \\ 
   \hline
\end{tabular}
\end{table}
% latex table generated in R 3.2.1 by xtable 1.7-4 package
% Tue Oct  6 00:09:19 2015
\begin{table}[p]
\centering
\caption{Principais tipos de paisaxe,  Serras Surorientais ( 9 )} 
\label{Tipos 9}
\begin{tabular}{rlr}
  \hline
 & Tipo & Porcentaxe \\ 
  \hline
1 & Serras ; Matogueira e rochedo ; Supra e orotemperado & 37.50 \\ 
  2 & Serras ; Agrosistema extensivo ; Mesotemperado superior & 10.40 \\ 
  3 & Serras ; Matogueira e rochedo ; Mesotemperado superior & 9.90 \\ 
  4 & Serras ; Agrosistema extensivo ; Supra e orotemperado & 7.50 \\ 
  5 & Serras ; Agrosistema extensivo ; Mesotemperado inferior & 7.10 \\ 
  6 & Serras ; Agrosistema intensivo (plantacion forestal) ; Supra e orotemperado & 5.10 \\ 
  7 & Serras ; Matogueira e rochedo ; Mesotemperado inferior & 4.20 \\ 
  8 & Serras ; Bosque ; Mesotemperado superior & 2.60 \\ 
  9 & Serras ; Bosque ; Supra e orotemperado & 2.10 \\ 
  10 & Serras ; Bosque ; Mesotemperado inferior & 2.00 \\ 
  11 & Serras ; Agrosistema intensivo (superficie de cultivo) ; Mesotemperado superior & 1.10 \\ 
  12 & Total & 89.50 \\ 
   \hline
\end{tabular}
\end{table}
% latex table generated in R 3.2.1 by xtable 1.7-4 package
% Tue Oct  6 00:09:19 2015
\begin{table}[p]
\centering
\caption{Principais tipos de paisaxe,  Galicia Setentrional ( 10 )} 
\label{Tipos 10}
\begin{tabular}{rlr}
  \hline
 & Tipo & Porcentaxe \\ 
  \hline
1 & Vales sublitorais ; Agrosistema intensivo (mosaico agroforestal) ; Mesotemperado inferior & 12.50 \\ 
  2 & Vales sublitorais ; Agrosistema intensivo (plantacion forestal) ; Mesotemperado inferior & 11.80 \\ 
  3 & Serras ; Matogueira e rochedo ; Mesotemperado superior & 6.60 \\ 
  4 & Serras ; Turbeira ; Mesotemperado superior & 6.60 \\ 
  5 & Litoral Cantabro-Atlantico ; Agrosistema intensivo (mosaico agroforestal) ; Termotemperado & 6.00 \\ 
  6 & Serras ; Turbeira ; Supra e orotemperado & 5.10 \\ 
  7 & Serras ; Agrosistema extensivo ; Mesotemperado superior & 4.40 \\ 
  8 & Litoral Cantabro-Atlantico ; Agrosistema intensivo (plantacion forestal) ; Mesotemperado inferior & 3.80 \\ 
  9 & Litoral Cantabro-Atlantico ; Agrosistema intensivo (plantacion forestal) ; Termotemperado & 3.30 \\ 
  10 & Serras ; Agrosistema intensivo (mosaico agroforestal) ; Mesotemperado superior & 3.10 \\ 
  11 & Litoral Cantabro-Atlantico ; Rururbano (diseminado) ; Termotemperado & 2.80 \\ 
  12 & Serras ; Agrosistema intensivo (plantacion forestal) ; Mesotemperado superior & 2.80 \\ 
  13 & Vales sublitorais ; Agrosistema intensivo (mosaico agroforestal) ; Mesotemperado superior & 2.20 \\ 
  14 & Serras ; Agrosistema intensivo (plantacion forestal) ; Mesotemperado inferior & 2.00 \\ 
  15 & Vales sublitorais ; Agrosistema intensivo (plantacion forestal) ; Mesotemperado superior & 2.00 \\ 
  16 & Litoral Cantabro-Atlantico ; Agrosistema intensivo (mosaico agroforestal) ; Mesotemperado inferior & 1.90 \\ 
  17 & Litoral Cantabro-Atlantico ; Matogueira e rochedo ; Termotemperado & 1.90 \\ 
  18 & Serras ; Agrosistema intensivo (mosaico agroforestal) ; Mesotemperado inferior & 1.70 \\ 
  19 & Serras ; Matogueira e rochedo ; Mesotemperado inferior & 1.50 \\ 
  20 & Serras ; Bosque ; Mesotemperado superior & 1.40 \\ 
  21 & Vales sublitorais ; Agrosistema intensivo (mosaico agroforestal) ; Termotemperado & 1.30 \\ 
  22 & Serras ; Agrosistema extensivo ; Mesotemperado inferior & 1.10 \\ 
  23 & Vales sublitorais ; Agrosistema extensivo ; Mesotemperado superior & 1.10 \\ 
  24 & Vales sublitorais ; Agrosistema intensivo (plantacion forestal) ; Termotemperado & 1.00 \\ 
  25 & Vales sublitorais ; Matogueira e rochedo ; Mesotemperado inferior & 1.00 \\ 
  26 & Vales sublitorais ; Matogueira e rochedo ; Mesotemperado superior & 1.00 \\ 
  27 & Total & 89.90 \\ 
   \hline
\end{tabular}
\end{table}
% latex table generated in R 3.2.1 by xtable 1.7-4 package
% Tue Oct  6 00:09:19 2015
\begin{table}[p]
\centering
\caption{Principais tipos de paisaxe,  Chairas e Fosas Occidentais ( 11 )} 
\label{Tipos 11}
\begin{tabular}{rlr}
  \hline
 & Tipo & Porcentaxe \\ 
  \hline
1 & Vales sublitorais ; Agrosistema intensivo (mosaico agroforestal) ; Mesotemperado inferior & 18.10 \\ 
  2 & Litoral Cantabro-Atlantico ; Agrosistema intensivo (mosaico agroforestal) ; Termotemperado & 8.90 \\ 
  3 & Vales sublitorais ; Agrosistema intensivo (mosaico agroforestal) ; Termotemperado & 8.80 \\ 
  4 & Vales sublitorais ; Agrosistema intensivo (plantacion forestal) ; Mesotemperado inferior & 8.30 \\ 
  5 & Vales sublitorais ; Matogueira e rochedo ; Mesotemperado inferior & 7.20 \\ 
  6 & Litoral Cantabro-Atlantico ; Matogueira e rochedo ; Termotemperado & 5.60 \\ 
  7 & Vales sublitorais ; Agrosistema intensivo (superficie de cultivo) ; Mesotemperado inferior & 4.90 \\ 
  8 & Vales sublitorais ; Agrosistema extensivo ; Mesotemperado inferior & 4.60 \\ 
  9 & Litoral Cantabro-Atlantico ; Agrosistema intensivo (plantacion forestal) ; Termotemperado & 4.30 \\ 
  10 & Vales sublitorais ; Agrosistema intensivo (mosaico agroforestal) ; Mesotemperado superior & 4.00 \\ 
  11 & Vales sublitorais ; Agrosistema intensivo (plantacion forestal) ; Termotemperado & 3.90 \\ 
  12 & Vales sublitorais ; Matogueira e rochedo ; Mesotemperado superior & 3.30 \\ 
  13 & Litoral Cantabro-Atlantico ; Rururbano (diseminado) ; Termotemperado & 2.80 \\ 
  14 & Vales sublitorais ; Rururbano (diseminado) ; Termotemperado & 2.50 \\ 
  15 & Vales sublitorais ; Rururbano (diseminado) ; Mesotemperado inferior & 1.90 \\ 
  16 & Vales sublitorais ; Matogueira e rochedo ; Termotemperado & 1.70 \\ 
  17 & Vales sublitorais ; Agrosistema intensivo (plantacion forestal) ; Mesotemperado superior & 1.60 \\ 
  18 & Total & 92.40 \\ 
   \hline
\end{tabular}
\end{table}
% latex table generated in R 3.2.1 by xtable 1.7-4 package
% Tue Oct  6 00:09:19 2015
\begin{table}[p]
\centering
\caption{Principais tipos de paisaxe,  Rías Baixas ( 12 )} 
\label{Tipos 12}
\begin{tabular}{rlr}
  \hline
 & Tipo & Porcentaxe \\ 
  \hline
1 & Litoral Cantabro-Atlantico ; Rururbano (diseminado) ; Termotemperado & 16.40 \\ 
  2 & Vales sublitorais ; Agrosistema intensivo (plantacion forestal) ; Termotemperado & 8.30 \\ 
  3 & Litoral Cantabro-Atlantico ; Agrosistema intensivo (mosaico agroforestal) ; Termotemperado & 6.60 \\ 
  4 & Vales sublitorais ; Rururbano (diseminado) ; Termotemperado & 6.10 \\ 
  5 & Vales sublitorais ; Agrosistema intensivo (mosaico agroforestal) ; Termotemperado & 5.60 \\ 
  6 & Vales sublitorais ; Matogueira e rochedo ; Mesotemperado inferior & 5.60 \\ 
  7 & Vales sublitorais ; Matogueira e rochedo ; Termotemperado & 5.30 \\ 
  8 & Litoral Cantabro-Atlantico ; Agrosistema intensivo (plantacion forestal) ; Termotemperado & 4.90 \\ 
  9 & Serras ; Matogueira e rochedo ; Mesotemperado inferior & 4.30 \\ 
  10 & Serras ; Matogueira e rochedo ; Mesotemperado superior & 4.10 \\ 
  11 & Vales sublitorais ; Agrosistema intensivo (plantacion forestal) ; Mesotemperado inferior & 4.00 \\ 
  12 & Vales sublitorais ; Agrosistema intensivo (mosaico agroforestal) ; Mesotemperado inferior & 3.80 \\ 
  13 & Litoral Cantabro-Atlantico ; Matogueira e rochedo ; Termotemperado & 2.60 \\ 
  14 & Serras ; Matogueira e rochedo ; Supra e orotemperado & 2.50 \\ 
  15 & Litoral Cantabro-Atlantico ; Urbano ; Termotemperado & 2.20 \\ 
  16 & Serras ; Agrosistema intensivo (plantacion forestal) ; Mesotemperado inferior & 1.60 \\ 
  17 & Vales sublitorais ; Agrosistema extensivo ; Mesotemperado inferior & 1.30 \\ 
  18 & Litoral Cantabro-Atlantico ; Viñedo ; Termotemperado & 1.20 \\ 
  19 & Serras ; Agrosistema extensivo ; Mesotemperado inferior & 1.10 \\ 
  20 & Vales sublitorais ; Agrosistema extensivo ; Termotemperado & 1.00 \\ 
  21 & Total & 88.50 \\ 
   \hline
\end{tabular}
\end{table}
