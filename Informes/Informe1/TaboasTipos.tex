% latex table generated in R 3.2.1 by xtable 1.7-4 package
% Fri Oct  9 10:33:40 2015
\begin{table}[p]
\centering
\caption{Principais tipos de paisaxe,  Golfo Ártabro ( 1 )} 
\label{Tipos 1}
\begin{tabular}{rlrr}
  \hline
 & Tipo & Área (km²) & Porcentaxe \\ 
  \hline
1 & Litoral Cantabro-Atlantico ; Rururbano (diseminado) ; Termotemperado & 249.26 & 19.30 \\ 
  2 & Vales sublitorais ; Agrosistema intensivo (mosaico agroforestal) ; Mesotemperado inferior & 185.89 & 14.40 \\ 
  3 & Litoral Cantabro-Atlantico ; Agrosistema intensivo (mosaico agroforestal) ; Termotemperado & 134.82 & 10.40 \\ 
  4 & Vales sublitorais ; Agrosistema intensivo (plantacion forestal) ; Mesotemperado inferior & 101.37 & 7.90 \\ 
  5 & Vales sublitorais ; Agrosistema intensivo (mosaico agroforestal) ; Mesotemperado superior & 56.41 & 4.40 \\ 
  6 & Vales sublitorais ; Agrosistema intensivo (mosaico agroforestal) ; Termotemperado & 52.42 & 4.10 \\ 
  7 & Litoral Cantabro-Atlantico ; Urbano ; Termotemperado & 40.04 & 3.10 \\ 
  8 & Serras ; Matogueira e rochedo ; Mesotemperado superior & 38.34 & 3.00 \\ 
  9 & Litoral Cantabro-Atlantico ; Agrosistema intensivo (plantacion forestal) ; Termotemperado & 36.56 & 2.80 \\ 
  10 & Vales sublitorais ; Rururbano (diseminado) ; Termotemperado & 27.28 & 2.10 \\ 
  11 & Serras ; Agrosistema extensivo ; Mesotemperado superior & 20.50 & 1.60 \\ 
  12 & Serras ; Agrosistema intensivo (mosaico agroforestal) ; Mesotemperado superior & 21.20 & 1.60 \\ 
  13 & Vales sublitorais ; Agrosistema intensivo (plantacion forestal) ; Termotemperado & 21.27 & 1.60 \\ 
  14 & Serras ; Turbeira ; Mesotemperado superior & 19.36 & 1.50 \\ 
  15 & Vales sublitorais ; Agrosistema intensivo (plantacion forestal) ; Mesotemperado superior & 19.37 & 1.50 \\ 
  16 & Vales sublitorais ; Rururbano (diseminado) ; Mesotemperado inferior & 19.64 & 1.50 \\ 
  17 & Vales sublitorais ; Agrosistema extensivo ; Mesotemperado superior & 17.46 & 1.40 \\ 
  18 & Vales sublitorais ; Matogueira e rochedo ; Mesotemperado inferior & 18.44 & 1.40 \\ 
  19 & Vales sublitorais ; Matogueira e rochedo ; Mesotemperado superior & 15.00 & 1.20 \\ 
  20 & Canons ; Bosque ; Mesotemperado inferior & 13.47 & 1.00 \\ 
  21 & Total & 1108.10 & 85.80 \\ 
   \hline
\end{tabular}
\end{table}
% latex table generated in R 3.2.1 by xtable 1.7-4 package
% Fri Oct  9 10:33:41 2015
\begin{table}[p]
\centering
\caption{Principais tipos de paisaxe,  A Mariña - Baixo Eo ( 2 )} 
\label{Tipos 2}
\begin{tabular}{rlrr}
  \hline
 & Tipo & Área (km²) & Porcentaxe \\ 
  \hline
1 & Vales sublitorais ; Agrosistema intensivo (plantacion forestal) ; Mesotemperado inferior & 129.64 & 14.00 \\ 
  2 & Vales sublitorais ; Agrosistema intensivo (mosaico agroforestal) ; Mesotemperado inferior & 116.09 & 12.60 \\ 
  3 & Litoral Cantabro-Atlantico ; Agrosistema intensivo (plantacion forestal) ; Mesotemperado inferior & 100.27 & 10.80 \\ 
  4 & Litoral Cantabro-Atlantico ; Agrosistema intensivo (mosaico agroforestal) ; Termotemperado & 78.46 & 8.50 \\ 
  5 & Litoral Cantabro-Atlantico ; Agrosistema intensivo (plantacion forestal) ; Termotemperado & 42.72 & 4.60 \\ 
  6 & Litoral Cantabro-Atlantico ; Agrosistema intensivo (mosaico agroforestal) ; Mesotemperado inferior & 39.40 & 4.30 \\ 
  7 & Serras ; Agrosistema intensivo (plantacion forestal) ; Mesotemperado superior & 37.05 & 4.00 \\ 
  8 & Litoral Cantabro-Atlantico ; Rururbano (diseminado) ; Termotemperado & 36.26 & 3.90 \\ 
  9 & Vales sublitorais ; Agrosistema intensivo (mosaico agroforestal) ; Termotemperado & 34.27 & 3.70 \\ 
  10 & Vales sublitorais ; Agrosistema intensivo (plantacion forestal) ; Mesotemperado superior & 29.60 & 3.20 \\ 
  11 & Serras ; Turbeira ; Mesotemperado superior & 21.70 & 2.30 \\ 
  12 & Litoral Cantabro-Atlantico ; Urbano ; Termotemperado & 20.15 & 2.20 \\ 
  13 & Serras ; Agrosistema intensivo (mosaico agroforestal) ; Mesotemperado superior & 20.46 & 2.20 \\ 
  14 & Serras ; Agrosistema intensivo (plantacion forestal) ; Mesotemperado inferior & 20.70 & 2.20 \\ 
  15 & Serras ; Matogueira e rochedo ; Mesotemperado superior & 19.15 & 2.10 \\ 
  16 & Vales sublitorais ; Agrosistema intensivo (plantacion forestal) ; Termotemperado & 19.70 & 2.10 \\ 
  17 & Vales sublitorais ; Agrosistema intensivo (mosaico agroforestal) ; Mesotemperado superior & 15.16 & 1.60 \\ 
  18 & Litoral Cantabro-Atlantico ; Agrosistema intensivo (superficie de cultivo) ; Termotemperado & 13.50 & 1.50 \\ 
  19 & Vales sublitorais ; Matogueira e rochedo ; Mesotemperado inferior & 13.70 & 1.50 \\ 
  20 & Vales sublitorais ; Agrosistema extensivo ; Mesotemperado inferior & 12.22 & 1.30 \\ 
  21 & Vales sublitorais ; Rururbano (diseminado) ; Mesotemperado inferior & 10.73 & 1.20 \\ 
  22 & Total & 830.93 & 89.80 \\ 
   \hline
\end{tabular}
\end{table}
% latex table generated in R 3.2.1 by xtable 1.7-4 package
% Fri Oct  9 10:33:41 2015
\begin{table}[p]
\centering
\caption{Principais tipos de paisaxe,  Costa Sur - Baixo Miño ( 3 )} 
\label{Tipos 3}
\begin{tabular}{rlrr}
  \hline
 & Tipo & Área (km²) & Porcentaxe \\ 
  \hline
1 & Vales sublitorais ; Rururbano (diseminado) ; Termotemperado & 147.82 & 12.50 \\ 
  2 & Litoral Cantabro-Atlantico ; Rururbano (diseminado) ; Termotemperado & 134.98 & 11.40 \\ 
  3 & Vales sublitorais ; Agrosistema intensivo (plantacion forestal) ; Termotemperado & 105.78 & 8.90 \\ 
  4 & Serras ; Matogueira e rochedo ; Mesotemperado superior & 69.70 & 5.90 \\ 
  5 & Serras ; Matogueira e rochedo ; Mesotemperado inferior & 59.84 & 5.10 \\ 
  6 & Serras ; Matogueira e rochedo ; Supra e orotemperado & 50.35 & 4.30 \\ 
  7 & Vales sublitorais ; Matogueira e rochedo ; Termotemperado & 40.45 & 3.40 \\ 
  8 & Litoral Cantabro-Atlantico ; Agrosistema intensivo (mosaico agroforestal) ; Termotemperado & 38.66 & 3.30 \\ 
  9 & Vales sublitorais ; Agrosistema intensivo (mosaico agroforestal) ; Termotemperado & 34.55 & 2.90 \\ 
  10 & Serras ; Agrosistema intensivo (plantacion forestal) ; Termotemperado & 32.95 & 2.80 \\ 
  11 & Chairas e vales interiores ; Matogueira e rochedo ; Mesotemperado inferior & 29.44 & 2.50 \\ 
  12 & Litoral Cantabro-Atlantico ; Agrosistema intensivo (plantacion forestal) ; Termotemperado & 29.68 & 2.50 \\ 
  13 & Chairas e vales interiores ; Rururbano (diseminado) ; Termotemperado & 27.39 & 2.30 \\ 
  14 & Serras ; Agrosistema intensivo (plantacion forestal) ; Mesotemperado inferior & 24.09 & 2.00 \\ 
  15 & Chairas e vales interiores ; Agrosistema intensivo (mosaico agroforestal) ; Termotemperado & 22.70 & 1.90 \\ 
  16 & Chairas e vales interiores ; Agrosistema intensivo (plantacion forestal) ; Termotemperado & 21.85 & 1.80 \\ 
  17 & Serras ; Matogueira e rochedo ; Termotemperado & 20.68 & 1.70 \\ 
  18 & Chairas e vales interiores ; Matogueira e rochedo ; Termotemperado & 17.21 & 1.50 \\ 
  19 & Vales sublitorais ; Agrosistema extensivo ; Termotemperado & 18.00 & 1.50 \\ 
  20 & Litoral Cantabro-Atlantico ; Urbano ; Termotemperado & 17.06 & 1.40 \\ 
  21 & no data ; Rururbano (diseminado) ; Termotemperado & 13.40 & 1.10 \\ 
  22 & Chairas e vales interiores ; Agrosistema intensivo (mosaico agroforestal) ; Mesotemperado inferior & 11.60 & 1.00 \\ 
  23 & Vales sublitorais ; Bosque ; Termotemperado & 12.18 & 1.00 \\ 
  24 & Total & 980.36 & 82.70 \\ 
   \hline
\end{tabular}
\end{table}
% latex table generated in R 3.2.1 by xtable 1.7-4 package
% Fri Oct  9 10:33:41 2015
\begin{table}[p]
\centering
\caption{Principais tipos de paisaxe,  Ribeiras Encaixadas do Miño e do Sil ( 4 )} 
\label{Tipos 4}
\begin{tabular}{rlrr}
  \hline
 & Tipo & Área (km²) & Porcentaxe \\ 
  \hline
1 & Chairas e vales interiores ; Agrosistema extensivo ; Mesotemperado inferior & 226.57 & 9.20 \\ 
  2 & Serras ; Matogueira e rochedo ; Mesotemperado superior & 130.18 & 5.30 \\ 
  3 & Serras ; Matogueira e rochedo ; Mesotemperado inferior & 126.40 & 5.10 \\ 
  4 & Chairas e vales interiores ; Agrosistema intensivo (mosaico agroforestal) ; Mesotemperado inferior & 118.55 & 4.80 \\ 
  5 & Serras ; Agrosistema extensivo ; Mesotemperado inferior & 110.09 & 4.50 \\ 
  6 & Chairas e vales interiores ; Matogueira e rochedo ; Termotemperado & 105.75 & 4.30 \\ 
  7 & Chairas e vales interiores ; Matogueira e rochedo ; Mesotemperado inferior & 103.19 & 4.20 \\ 
  8 & Serras ; Matogueira e rochedo ; Supra e orotemperado & 100.24 & 4.10 \\ 
  9 & Chairas e vales interiores ; Agrosistema extensivo ; Termotemperado & 91.77 & 3.70 \\ 
  10 & Serras ; Agrosistema extensivo ; Mesotemperado superior & 79.50 & 3.20 \\ 
  11 & Chairas e vales interiores ; Matogueira e rochedo ; Mesomediterráneo & 74.02 & 3.00 \\ 
  12 & Chairas e vales interiores ; Rururbano (diseminado) ; Termotemperado & 64.13 & 2.60 \\ 
  13 & Chairas e vales interiores ; Agrosistema intensivo (plantacion forestal) ; Termotemperado & 60.42 & 2.40 \\ 
  14 & Chairas e vales interiores ; Bosque ; Termotemperado & 58.81 & 2.40 \\ 
  15 & Chairas e vales interiores ; Bosque ; Mesotemperado inferior & 58.08 & 2.30 \\ 
  16 & Chairas e vales interiores ; Viñedo ; Termotemperado & 54.56 & 2.20 \\ 
  17 & Canons ; Bosque ; Mesotemperado inferior & 47.00 & 1.90 \\ 
  18 & Canons ; Matogueira e rochedo ; Mesomediterráneo & 43.68 & 1.80 \\ 
  19 & Serras ; Matogueira e rochedo ; Mesomediterráneo & 45.10 & 1.80 \\ 
  20 & Chairas e vales interiores ; Agrosistema intensivo (mosaico agroforestal) ; Termotemperado & 38.34 & 1.60 \\ 
  21 & Chairas e vales interiores ; Agrosistema intensivo (plantacion forestal) ; Mesotemperado inferior & 38.94 & 1.60 \\ 
  22 & Serras ; Agrosistema intensivo (mosaico agroforestal) ; Mesotemperado superior & 38.27 & 1.50 \\ 
  23 & Serras ; Bosque ; Mesotemperado inferior & 36.08 & 1.50 \\ 
  24 & Canons ; Bosque ; Termotemperado & 34.73 & 1.40 \\ 
  25 & Canons ; Matogueira e rochedo ; Mesotemperado inferior & 32.00 & 1.30 \\ 
  26 & Chairas e vales interiores ; Viñedo ; Mesomediterráneo & 33.13 & 1.30 \\ 
  27 & Canons ; Agrosistema extensivo ; Mesotemperado inferior & 29.29 & 1.20 \\ 
  28 & Chairas e vales interiores ; Agrosistema extensivo ; Mesomediterráneo & 30.61 & 1.20 \\ 
  29 & Chairas e vales interiores ; Rururbano (diseminado) ; Mesotemperado inferior & 30.34 & 1.20 \\ 
  30 & Serras ; Agrosistema intensivo (mosaico agroforestal) ; Mesotemperado inferior & 28.86 & 1.20 \\ 
  31 & Chairas e vales interiores ; Agrosistema intensivo (superficie de cultivo) ; Mesotemperado inferior & 23.93 & 1.00 \\ 
  32 & Serras ; Agrosistema intensivo (superficie de cultivo) ; Mesotemperado superior & 24.71 & 1.00 \\ 
  33 & Total & 2117.27 & 85.80 \\ 
   \hline
\end{tabular}
\end{table}
% latex table generated in R 3.2.1 by xtable 1.7-4 package
% Fri Oct  9 10:33:41 2015
\begin{table}[p]
\centering
\caption{Principais tipos de paisaxe,  Serras Orientais ( 5 )} 
\label{Tipos 5}
\begin{tabular}{rlrr}
  \hline
 & Tipo & Área (km²) & Porcentaxe \\ 
  \hline
1 & Serras ; Agrosistema extensivo ; Supra e orotemperado & 378.25 & 15.20 \\ 
  2 & Serras ; Matogueira e rochedo ; Supra e orotemperado & 369.67 & 14.80 \\ 
  3 & Serras ; Agrosistema extensivo ; Mesotemperado superior & 295.32 & 11.80 \\ 
  4 & Serras ; Matogueira e rochedo ; Mesotemperado superior & 172.18 & 6.90 \\ 
  5 & Vales sublitorais ; Agrosistema extensivo ; Mesotemperado superior & 141.35 & 5.70 \\ 
  6 & Serras ; Bosque ; Supra e orotemperado & 128.80 & 5.20 \\ 
  7 & Vales sublitorais ; Agrosistema extensivo ; Mesotemperado inferior & 97.14 & 3.90 \\ 
  8 & Serras ; Bosque ; Mesotemperado superior & 94.25 & 3.80 \\ 
  9 & Vales sublitorais ; Bosque ; Mesotemperado superior & 86.92 & 3.50 \\ 
  10 & Serras ; Agrosistema intensivo (plantacion forestal) ; Supra e orotemperado & 85.44 & 3.40 \\ 
  11 & Vales sublitorais ; Bosque ; Mesotemperado inferior & 70.38 & 2.80 \\ 
  12 & Vales sublitorais ; Matogueira e rochedo ; Mesotemperado superior & 61.69 & 2.50 \\ 
  13 & Serras ; Agrosistema intensivo (mosaico agroforestal) ; Supra e orotemperado & 60.90 & 2.40 \\ 
  14 & Serras ; Agrosistema intensivo (plantacion forestal) ; Mesotemperado superior & 58.11 & 2.30 \\ 
  15 & Serras ; Agrosistema intensivo (mosaico agroforestal) ; Mesotemperado superior & 48.86 & 2.00 \\ 
  16 & Vales sublitorais ; Matogueira e rochedo ; Mesotemperado inferior & 39.81 & 1.60 \\ 
  17 & Serras ; Matogueira e rochedo ; Mesotemperado inferior & 27.49 & 1.10 \\ 
  18 & Total & 2216.56 & 88.90 \\ 
   \hline
\end{tabular}
\end{table}
% latex table generated in R 3.2.1 by xtable 1.7-4 package
% Fri Oct  9 10:33:41 2015
\begin{table}[p]
\centering
\caption{Principais tipos de paisaxe,  Chairas e Fosas Luguesas ( 6 )} 
\label{Tipos 6}
\begin{tabular}{rlrr}
  \hline
 & Tipo & Área (km²) & Porcentaxe \\ 
  \hline
1 & Chairas e vales interiores ; Agrosistema extensivo ; Mesotemperado superior & 797.87 & 17.50 \\ 
  2 & Chairas e vales interiores ; Agrosistema intensivo (mosaico agroforestal) ; Mesotemperado superior & 664.75 & 14.60 \\ 
  3 & Serras ; Agrosistema extensivo ; Mesotemperado superior & 462.36 & 10.10 \\ 
  4 & Chairas e vales interiores ; Agrosistema extensivo ; Mesotemperado inferior & 436.22 & 9.60 \\ 
  5 & Chairas e vales interiores ; Agrosistema intensivo (mosaico agroforestal) ; Mesotemperado inferior & 295.64 & 6.50 \\ 
  6 & Serras ; Agrosistema intensivo (mosaico agroforestal) ; Mesotemperado superior & 246.83 & 5.40 \\ 
  7 & Chairas e vales interiores ; Agrosistema intensivo (superficie de cultivo) ; Mesotemperado superior & 208.92 & 4.60 \\ 
  8 & Serras ; Agrosistema intensivo (superficie de cultivo) ; Mesotemperado superior & 132.67 & 2.90 \\ 
  9 & Chairas e vales interiores ; Agrosistema intensivo (plantacion forestal) ; Mesotemperado superior & 118.95 & 2.60 \\ 
  10 & Chairas e vales interiores ; Agrosistema intensivo (superficie de cultivo) ; Mesotemperado inferior & 106.27 & 2.30 \\ 
  11 & Chairas e vales interiores ; Matogueira e rochedo ; Mesotemperado inferior & 89.08 & 2.00 \\ 
  12 & Chairas e vales interiores ; Matogueira e rochedo ; Mesotemperado superior & 87.81 & 1.90 \\ 
  13 & Chairas e vales interiores ; Rururbano (diseminado) ; Mesotemperado superior & 88.58 & 1.90 \\ 
  14 & Serras ; Matogueira e rochedo ; Mesotemperado superior & 81.24 & 1.80 \\ 
  15 & Chairas e vales interiores ; Bosque ; Mesotemperado superior & 69.92 & 1.50 \\ 
  16 & Chairas e vales interiores ; Rururbano (diseminado) ; Mesotemperado inferior & 64.35 & 1.40 \\ 
  17 & Chairas e vales interiores ; Bosque ; Mesotemperado inferior & 58.90 & 1.30 \\ 
  18 & Serras ; Agrosistema intensivo (plantacion forestal) ; Mesotemperado superior & 60.75 & 1.30 \\ 
  19 & Serras ; Agrosistema extensivo ; Supra e orotemperado & 49.88 & 1.10 \\ 
  20 & Serras ; Turbeira ; Mesotemperado superior & 51.82 & 1.10 \\ 
  21 & Total & 4172.81 & 91.40 \\ 
   \hline
\end{tabular}
\end{table}
% latex table generated in R 3.2.1 by xtable 1.7-4 package
% Fri Oct  9 10:33:41 2015
\begin{table}[p]
\centering
\caption{Principais tipos de paisaxe,  Galicia Central ( 7 )} 
\label{Tipos 7}
\begin{tabular}{rlrr}
  \hline
 & Tipo & Área (km²) & Porcentaxe \\ 
  \hline
1 & Vales sublitorais ; Agrosistema intensivo (mosaico agroforestal) ; Mesotemperado inferior & 1225.74 & 23.80 \\ 
  2 & Vales sublitorais ; Agrosistema extensivo ; Mesotemperado inferior & 535.39 & 10.40 \\ 
  3 & Serras ; Agrosistema extensivo ; Mesotemperado superior & 363.58 & 7.10 \\ 
  4 & Vales sublitorais ; Agrosistema intensivo (mosaico agroforestal) ; Termotemperado & 298.29 & 5.80 \\ 
  5 & Serras ; Matogueira e rochedo ; Mesotemperado superior & 257.51 & 5.00 \\ 
  6 & Vales sublitorais ; Agrosistema intensivo (mosaico agroforestal) ; Mesotemperado superior & 226.65 & 4.40 \\ 
  7 & Vales sublitorais ; Rururbano (diseminado) ; Mesotemperado inferior & 188.56 & 3.70 \\ 
  8 & Vales sublitorais ; Matogueira e rochedo ; Mesotemperado inferior & 160.57 & 3.10 \\ 
  9 & Vales sublitorais ; Agrosistema extensivo ; Mesotemperado superior & 150.12 & 2.90 \\ 
  10 & Serras ; Agrosistema extensivo ; Mesotemperado inferior & 137.54 & 2.70 \\ 
  11 & Vales sublitorais ; Agrosistema intensivo (plantacion forestal) ; Mesotemperado inferior & 140.19 & 2.70 \\ 
  12 & Vales sublitorais ; Agrosistema intensivo (superficie de cultivo) ; Mesotemperado inferior & 126.21 & 2.40 \\ 
  13 & Serras ; Matogueira e rochedo ; Supra e orotemperado & 113.08 & 2.20 \\ 
  14 & Vales sublitorais ; Rururbano (diseminado) ; Termotemperado & 99.95 & 1.90 \\ 
  15 & Serras ; Agrosistema intensivo (mosaico agroforestal) ; Mesotemperado superior & 82.78 & 1.60 \\ 
  16 & Serras ; Matogueira e rochedo ; Mesotemperado inferior & 83.06 & 1.60 \\ 
  17 & Vales sublitorais ; Agrosistema intensivo (plantacion forestal) ; Termotemperado & 62.40 & 1.20 \\ 
  18 & Vales sublitorais ; Matogueira e rochedo ; Mesotemperado superior & 55.11 & 1.10 \\ 
  19 & Vales sublitorais ; Agrosistema extensivo ; Termotemperado & 49.53 & 1.00 \\ 
  20 & Total & 4356.26 & 84.60 \\ 
   \hline
\end{tabular}
\end{table}
% latex table generated in R 3.2.1 by xtable 1.7-4 package
% Fri Oct  9 10:33:41 2015
\begin{table}[p]
\centering
\caption{Principais tipos de paisaxe,  Chairas, Fosas e Serras Ourensás ( 8 )} 
\label{Tipos 8}
\begin{tabular}{rlrr}
  \hline
 & Tipo & Área (km²) & Porcentaxe \\ 
  \hline
1 & Chairas e vales interiores ; Agrosistema extensivo ; Mesotemperado inferior & 342.76 & 12.00 \\ 
  2 & Serras ; Matogueira e rochedo ; Mesotemperado superior & 284.07 & 10.00 \\ 
  3 & Serras ; Agrosistema extensivo ; Mesotemperado inferior & 256.71 & 9.00 \\ 
  4 & Serras ; Matogueira e rochedo ; Mesotemperado inferior & 255.53 & 9.00 \\ 
  5 & Serras ; Matogueira e rochedo ; Supra e orotemperado & 239.60 & 8.40 \\ 
  6 & Chairas e vales interiores ; Agrosistema intensivo (superficie de cultivo) ; Mesotemperado inferior & 212.46 & 7.50 \\ 
  7 & Serras ; Agrosistema extensivo ; Mesotemperado superior & 140.13 & 4.90 \\ 
  8 & Chairas e vales interiores ; Matogueira e rochedo ; Mesotemperado inferior & 136.15 & 4.80 \\ 
  9 & Serras ; Agrosistema extensivo ; Supra e orotemperado & 87.82 & 3.10 \\ 
  10 & Chairas e vales interiores ; Bosque ; Mesotemperado inferior & 79.78 & 2.80 \\ 
  11 & Serras ; Bosque ; Mesotemperado inferior & 76.28 & 2.70 \\ 
  12 & Chairas e vales interiores ; Matogueira e rochedo ; Termotemperado & 61.48 & 2.20 \\ 
  13 & Serras ; Agrosistema intensivo (plantacion forestal) ; Mesotemperado inferior & 56.85 & 2.00 \\ 
  14 & Chairas e vales interiores ; Agrosistema extensivo ; Termotemperado & 50.49 & 1.80 \\ 
  15 & Chairas e vales interiores ; Agrosistema intensivo (plantacion forestal) ; Mesotemperado inferior & 46.95 & 1.70 \\ 
  16 & Serras ; Agrosistema intensivo (superficie de cultivo) ; Mesotemperado inferior & 46.12 & 1.60 \\ 
  17 & Serras ; Bosque ; Mesotemperado superior & 45.85 & 1.60 \\ 
  18 & Chairas e vales interiores ; Agrosistema intensivo (mosaico agroforestal) ; Mesotemperado inferior & 41.06 & 1.40 \\ 
  19 & Serras ; Agrosistema intensivo (superficie de cultivo) ; Mesotemperado superior & 35.88 & 1.30 \\ 
  20 & Chairas e vales interiores ; Agrosistema intensivo (mosaico agroforestal) ; Termotemperado & 34.03 & 1.20 \\ 
  21 & Chairas e vales interiores ; Rururbano (diseminado) ; Mesotemperado inferior & 34.69 & 1.20 \\ 
  22 & Serras ; Agrosistema intensivo (plantacion forestal) ; Mesotemperado superior & 31.71 & 1.10 \\ 
  23 & Serras ; Bosque ; Supra e orotemperado & 27.73 & 1.00 \\ 
  24 & Total & 2624.13 & 92.30 \\ 
   \hline
\end{tabular}
\end{table}
% latex table generated in R 3.2.1 by xtable 1.7-4 package
% Fri Oct  9 10:33:41 2015
\begin{table}[p]
\centering
\caption{Principais tipos de paisaxe,  Serras Surorientais ( 9 )} 
\label{Tipos 9}
\begin{tabular}{rlrr}
  \hline
 & Tipo & Área (km²) & Porcentaxe \\ 
  \hline
1 & Serras ; Matogueira e rochedo ; Supra e orotemperado & 825.10 & 37.50 \\ 
  2 & Serras ; Agrosistema extensivo ; Mesotemperado superior & 229.18 & 10.40 \\ 
  3 & Serras ; Matogueira e rochedo ; Mesotemperado superior & 218.13 & 9.90 \\ 
  4 & Serras ; Agrosistema extensivo ; Supra e orotemperado & 165.93 & 7.50 \\ 
  5 & Serras ; Agrosistema extensivo ; Mesotemperado inferior & 155.37 & 7.10 \\ 
  6 & Serras ; Agrosistema intensivo (plantacion forestal) ; Supra e orotemperado & 112.84 & 5.10 \\ 
  7 & Serras ; Matogueira e rochedo ; Mesotemperado inferior & 91.96 & 4.20 \\ 
  8 & Serras ; Bosque ; Mesotemperado superior & 56.72 & 2.60 \\ 
  9 & Serras ; Bosque ; Supra e orotemperado & 45.21 & 2.10 \\ 
  10 & Serras ; Bosque ; Mesotemperado inferior & 43.08 & 2.00 \\ 
  11 & Serras ; Agrosistema intensivo (superficie de cultivo) ; Mesotemperado superior & 25.28 & 1.10 \\ 
  12 & Total & 1968.80 & 89.50 \\ 
   \hline
\end{tabular}
\end{table}
% latex table generated in R 3.2.1 by xtable 1.7-4 package
% Fri Oct  9 10:33:41 2015
\begin{table}[p]
\centering
\caption{Principais tipos de paisaxe,  Galicia Setentrional ( 10 )} 
\label{Tipos 10}
\begin{tabular}{rlrr}
  \hline
 & Tipo & Área (km²) & Porcentaxe \\ 
  \hline
1 & Vales sublitorais ; Agrosistema intensivo (mosaico agroforestal) ; Mesotemperado inferior & 203.47 & 12.50 \\ 
  2 & Vales sublitorais ; Agrosistema intensivo (plantacion forestal) ; Mesotemperado inferior & 192.72 & 11.80 \\ 
  3 & Serras ; Matogueira e rochedo ; Mesotemperado superior & 107.67 & 6.60 \\ 
  4 & Serras ; Turbeira ; Mesotemperado superior & 107.35 & 6.60 \\ 
  5 & Litoral Cantabro-Atlantico ; Agrosistema intensivo (mosaico agroforestal) ; Termotemperado & 97.07 & 6.00 \\ 
  6 & Serras ; Turbeira ; Supra e orotemperado & 83.38 & 5.10 \\ 
  7 & Serras ; Agrosistema extensivo ; Mesotemperado superior & 71.03 & 4.40 \\ 
  8 & Litoral Cantabro-Atlantico ; Agrosistema intensivo (plantacion forestal) ; Mesotemperado inferior & 62.49 & 3.80 \\ 
  9 & Litoral Cantabro-Atlantico ; Agrosistema intensivo (plantacion forestal) ; Termotemperado & 54.04 & 3.30 \\ 
  10 & Serras ; Agrosistema intensivo (mosaico agroforestal) ; Mesotemperado superior & 49.80 & 3.10 \\ 
  11 & Litoral Cantabro-Atlantico ; Rururbano (diseminado) ; Termotemperado & 45.22 & 2.80 \\ 
  12 & Serras ; Agrosistema intensivo (plantacion forestal) ; Mesotemperado superior & 45.39 & 2.80 \\ 
  13 & Vales sublitorais ; Agrosistema intensivo (mosaico agroforestal) ; Mesotemperado superior & 35.83 & 2.20 \\ 
  14 & Serras ; Agrosistema intensivo (plantacion forestal) ; Mesotemperado inferior & 31.89 & 2.00 \\ 
  15 & Vales sublitorais ; Agrosistema intensivo (plantacion forestal) ; Mesotemperado superior & 32.69 & 2.00 \\ 
  16 & Litoral Cantabro-Atlantico ; Agrosistema intensivo (mosaico agroforestal) ; Mesotemperado inferior & 30.24 & 1.90 \\ 
  17 & Litoral Cantabro-Atlantico ; Matogueira e rochedo ; Termotemperado & 31.12 & 1.90 \\ 
  18 & Serras ; Agrosistema intensivo (mosaico agroforestal) ; Mesotemperado inferior & 27.05 & 1.70 \\ 
  19 & Serras ; Matogueira e rochedo ; Mesotemperado inferior & 24.89 & 1.50 \\ 
  20 & Serras ; Bosque ; Mesotemperado superior & 21.99 & 1.40 \\ 
  21 & Vales sublitorais ; Agrosistema intensivo (mosaico agroforestal) ; Termotemperado & 20.50 & 1.30 \\ 
  22 & Serras ; Agrosistema extensivo ; Mesotemperado inferior & 17.60 & 1.10 \\ 
  23 & Vales sublitorais ; Agrosistema extensivo ; Mesotemperado superior & 18.07 & 1.10 \\ 
  24 & Vales sublitorais ; Agrosistema intensivo (plantacion forestal) ; Termotemperado & 16.32 & 1.00 \\ 
  25 & Vales sublitorais ; Matogueira e rochedo ; Mesotemperado inferior & 16.13 & 1.00 \\ 
  26 & Vales sublitorais ; Matogueira e rochedo ; Mesotemperado superior & 16.98 & 1.00 \\ 
  27 & Total & 1460.93 & 89.90 \\ 
   \hline
\end{tabular}
\end{table}
% latex table generated in R 3.2.1 by xtable 1.7-4 package
% Fri Oct  9 10:33:41 2015
\begin{table}[p]
\centering
\caption{Principais tipos de paisaxe,  Chairas e Fosas Occidentais ( 11 )} 
\label{Tipos 11}
\begin{tabular}{rlrr}
  \hline
 & Tipo & Área (km²) & Porcentaxe \\ 
  \hline
1 & Vales sublitorais ; Agrosistema intensivo (mosaico agroforestal) ; Mesotemperado inferior & 374.79 & 18.10 \\ 
  2 & Litoral Cantabro-Atlantico ; Agrosistema intensivo (mosaico agroforestal) ; Termotemperado & 183.90 & 8.90 \\ 
  3 & Vales sublitorais ; Agrosistema intensivo (mosaico agroforestal) ; Termotemperado & 183.34 & 8.80 \\ 
  4 & Vales sublitorais ; Agrosistema intensivo (plantacion forestal) ; Mesotemperado inferior & 173.01 & 8.30 \\ 
  5 & Vales sublitorais ; Matogueira e rochedo ; Mesotemperado inferior & 148.54 & 7.20 \\ 
  6 & Litoral Cantabro-Atlantico ; Matogueira e rochedo ; Termotemperado & 116.54 & 5.60 \\ 
  7 & Vales sublitorais ; Agrosistema intensivo (superficie de cultivo) ; Mesotemperado inferior & 102.60 & 4.90 \\ 
  8 & Vales sublitorais ; Agrosistema extensivo ; Mesotemperado inferior & 94.66 & 4.60 \\ 
  9 & Litoral Cantabro-Atlantico ; Agrosistema intensivo (plantacion forestal) ; Termotemperado & 89.54 & 4.30 \\ 
  10 & Vales sublitorais ; Agrosistema intensivo (mosaico agroforestal) ; Mesotemperado superior & 83.26 & 4.00 \\ 
  11 & Vales sublitorais ; Agrosistema intensivo (plantacion forestal) ; Termotemperado & 80.70 & 3.90 \\ 
  12 & Vales sublitorais ; Matogueira e rochedo ; Mesotemperado superior & 68.50 & 3.30 \\ 
  13 & Litoral Cantabro-Atlantico ; Rururbano (diseminado) ; Termotemperado & 57.58 & 2.80 \\ 
  14 & Vales sublitorais ; Rururbano (diseminado) ; Termotemperado & 51.48 & 2.50 \\ 
  15 & Vales sublitorais ; Rururbano (diseminado) ; Mesotemperado inferior & 39.12 & 1.90 \\ 
  16 & Vales sublitorais ; Matogueira e rochedo ; Termotemperado & 34.92 & 1.70 \\ 
  17 & Vales sublitorais ; Agrosistema intensivo (plantacion forestal) ; Mesotemperado superior & 32.91 & 1.60 \\ 
  18 & Total & 1915.39 & 92.40 \\ 
   \hline
\end{tabular}
\end{table}
% latex table generated in R 3.2.1 by xtable 1.7-4 package
% Fri Oct  9 10:33:41 2015
\begin{table}[p]
\centering
\caption{Principais tipos de paisaxe,  Rías Baixas ( 12 )} 
\label{Tipos 12}
\begin{tabular}{rlrr}
  \hline
 & Tipo & Área (km²) & Porcentaxe \\ 
  \hline
1 & Litoral Cantabro-Atlantico ; Rururbano (diseminado) ; Termotemperado & 442.96 & 16.40 \\ 
  2 & Vales sublitorais ; Agrosistema intensivo (plantacion forestal) ; Termotemperado & 222.83 & 8.30 \\ 
  3 & Litoral Cantabro-Atlantico ; Agrosistema intensivo (mosaico agroforestal) ; Termotemperado & 177.25 & 6.60 \\ 
  4 & Vales sublitorais ; Rururbano (diseminado) ; Termotemperado & 163.59 & 6.10 \\ 
  5 & Vales sublitorais ; Agrosistema intensivo (mosaico agroforestal) ; Termotemperado & 151.69 & 5.60 \\ 
  6 & Vales sublitorais ; Matogueira e rochedo ; Mesotemperado inferior & 151.93 & 5.60 \\ 
  7 & Vales sublitorais ; Matogueira e rochedo ; Termotemperado & 142.35 & 5.30 \\ 
  8 & Litoral Cantabro-Atlantico ; Agrosistema intensivo (plantacion forestal) ; Termotemperado & 133.46 & 4.90 \\ 
  9 & Serras ; Matogueira e rochedo ; Mesotemperado inferior & 116.47 & 4.30 \\ 
  10 & Serras ; Matogueira e rochedo ; Mesotemperado superior & 109.91 & 4.10 \\ 
  11 & Vales sublitorais ; Agrosistema intensivo (plantacion forestal) ; Mesotemperado inferior & 107.66 & 4.00 \\ 
  12 & Vales sublitorais ; Agrosistema intensivo (mosaico agroforestal) ; Mesotemperado inferior & 101.56 & 3.80 \\ 
  13 & Litoral Cantabro-Atlantico ; Matogueira e rochedo ; Termotemperado & 70.42 & 2.60 \\ 
  14 & Serras ; Matogueira e rochedo ; Supra e orotemperado & 67.15 & 2.50 \\ 
  15 & Litoral Cantabro-Atlantico ; Urbano ; Termotemperado & 59.70 & 2.20 \\ 
  16 & Serras ; Agrosistema intensivo (plantacion forestal) ; Mesotemperado inferior & 43.18 & 1.60 \\ 
  17 & Vales sublitorais ; Agrosistema extensivo ; Mesotemperado inferior & 34.89 & 1.30 \\ 
  18 & Litoral Cantabro-Atlantico ; Viñedo ; Termotemperado & 33.54 & 1.20 \\ 
  19 & Serras ; Agrosistema extensivo ; Mesotemperado inferior & 28.92 & 1.10 \\ 
  20 & Vales sublitorais ; Agrosistema extensivo ; Termotemperado & 25.94 & 1.00 \\ 
  21 & Total & 2385.40 & 88.50 \\ 
   \hline
\end{tabular}
\end{table}
