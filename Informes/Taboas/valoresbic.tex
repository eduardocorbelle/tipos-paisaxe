% latex table generated in R 3.2.5 by xtable 1.8-0 package
% Mon May  9 13:23:06 2016
\begin{table}[p]
\centering
\caption{Frecuencia de aparición de BIC e frecuencia de tipos asociados, GAP Golfo Ártabro} 
\label{vbic1}
\begin{tabular}{lrrr}
  \hline
Tipo de paisaxe & F.Aparic (\%) & F.Tipo (\%) & Ratio \\ 
  \hline
category 0; Conxunto Historico; category 0 & 27.78 & 0.08 & 336.39 \\ 
  Litoral Cantabro-Atlantico; Agrosistema intensivo (mosaico agroforestal); Termotemperado & 16.67 & 18.64 & 0.89 \\ 
  Litoral Cantabro-Atlantico; Urbano; Termotemperado & 14.81 & 4.93 & 3.00 \\ 
  Litoral Cantabro-Atlantico; Rururbano (diseminado); Termotemperado & 9.26 & 7.02 & 1.32 \\ 
  Litoral Cantabro-Atlantico; Agrosistema intensivo (plantacion forestal); Termotemperado & 5.56 & 5.79 & 0.96 \\ 
  Litoral Cantabro-Atlantico; Matogueira e rochedo; Termotemperado & 5.56 & 1.19 & 4.66 \\ 
  Vales sublitorais; Agrosistema intensivo (mosaico agroforestal); Mesotemperado inferior & 5.56 & 21.75 & 0.26 \\ 
   & 5.56 &  &  \\ 
  Canons; Bosque; Mesotemperado inferior & 1.85 & 1.56 & 1.19 \\ 
  Litoral Cantabro-Atlantico; Agrosistema extensivo; Termotemperado & 1.85 & 1.02 & 1.82 \\ 
  Serras; Agrosistema intensivo (plantacion forestal); Mesotemperado inferior & 1.85 & 0.51 & 3.65 \\ 
  Vales sublitorais; Agrosistema intensivo (plantacion forestal); Mesotemperado inferior & 1.85 & 13.21 & 0.14 \\ 
  Vales sublitorais; Matogueira e rochedo; Termotemperado & 1.85 & 0.24 & 7.64 \\ 
   \hline
\end{tabular}
\end{table}
% latex table generated in R 3.2.5 by xtable 1.8-0 package
% Mon May  9 13:23:06 2016
\begin{table}[p]
\centering
\caption{Frecuencia de aparición de BIC e frecuencia de tipos asociados, GAP A Mariña - Baixo Eo} 
\label{vbic2}
\begin{tabular}{lrrr}
  \hline
Tipo de paisaxe & F.Aparic (\%) & F.Tipo (\%) & Ratio \\ 
  \hline
category 0; Conxunto Historico; category 0 & 43.75 & 0.10 & 455.45 \\ 
  Litoral Cantabro-Atlantico; Agrosistema intensivo (mosaico agroforestal); Mesotemperado inferior & 12.50 & 3.57 & 3.50 \\ 
  Litoral Cantabro-Atlantico; Agrosistema intensivo (mosaico agroforestal); Termotemperado & 12.50 & 7.20 & 1.74 \\ 
  Litoral Cantabro-Atlantico; Matogueira e rochedo; Mesotemperado inferior & 6.25 & 0.05 & 133.22 \\ 
  Vales sublitorais; Agrosistema intensivo (mosaico agroforestal); Mesotemperado inferior & 6.25 & 12.86 & 0.49 \\ 
  Vales sublitorais; Agrosistema intensivo (plantacion forestal); Mesotemperado inferior & 6.25 & 23.68 & 0.26 \\ 
  Vales sublitorais; Agrosistema intensivo (superficie de cultivo); Mesotemperado inferior & 6.25 & 0.59 & 10.67 \\ 
  Vales sublitorais; Urbano; Termotemperado & 6.25 & 0.04 & 146.29 \\ 
   \hline
\end{tabular}
\end{table}
% latex table generated in R 3.2.5 by xtable 1.8-0 package
% Mon May  9 13:23:06 2016
\begin{table}[p]
\centering
\caption{Frecuencia de aparición de BIC e frecuencia de tipos asociados, GAP Costa Sur - Baixo Miño} 
\label{vbic3}
\begin{tabular}{lrrr}
  \hline
Tipo de paisaxe & F.Aparic (\%) & F.Tipo (\%) & Ratio \\ 
  \hline
Vales sublitorais; Matogueira e rochedo; Mesotemperado inferior & 35.38 & 0.91 & 38.88 \\ 
  category 0; Conxunto Historico; category 0 & 9.23 & 0.22 & 41.29 \\ 
  Vales sublitorais; Agrosistema intensivo (plantacion forestal); Termotemperado & 9.23 & 16.82 & 0.55 \\ 
  Litoral Cantabro-Atlantico; Urbano; Termotemperado & 7.69 & 1.10 & 6.99 \\ 
  Vales sublitorais; Agrosistema intensivo (mosaico agroforestal); Termotemperado & 7.69 & 7.55 & 1.02 \\ 
  Litoral Cantabro-Atlantico; Agrosistema intensivo (plantacion forestal); Termotemperado & 6.15 & 6.69 & 0.92 \\ 
  Litoral Cantabro-Atlantico; Rururbano (diseminado); Termotemperado & 6.15 & 4.97 & 1.24 \\ 
  Vales sublitorais; Matogueira e rochedo; Termotemperado & 4.62 & 2.72 & 1.70 \\ 
  Serras; Agrosistema intensivo (plantacion forestal); Mesotemperado inferior & 3.08 & 3.38 & 0.91 \\ 
  Vales sublitorais; Agrosistema intensivo (plantacion forestal); Mesotemperado inferior & 3.08 & 0.48 & 6.39 \\ 
  Litoral Cantabro-Atlantico; Agrosistema intensivo (mosaico agroforestal); Termotemperado & 1.54 & 5.83 & 0.26 \\ 
  Serras; Agrosistema intensivo (plantacion forestal); Termotemperado & 1.54 & 3.79 & 0.41 \\ 
  Serras; Matogueira e rochedo; Mesotemperado inferior & 1.54 & 3.84 & 0.40 \\ 
  Serras; Matogueira e rochedo; Mesotemperado superior & 1.54 & 11.08 & 0.14 \\ 
  Vales sublitorais; Urbano; Termotemperado & 1.54 & 0.38 & 4.06 \\ 
   \hline
\end{tabular}
\end{table}
% latex table generated in R 3.2.5 by xtable 1.8-0 package
% Mon May  9 13:23:06 2016
\begin{table}[p]
\centering
\caption{Frecuencia de aparición de BIC e frecuencia de tipos asociados, GAP Ribeiras Encaixadas do Miño e do Sil} 
\label{vbic4}
\begin{tabular}{lrrr}
  \hline
Tipo de paisaxe & F.Aparic (\%) & F.Tipo (\%) & Ratio \\ 
  \hline
category 0; Conxunto Historico; category 0 & 26.53 & 0.04 & 648.09 \\ 
  Canons; Bosque; Mesotemperado inferior & 12.24 & 2.67 & 4.59 \\ 
  Chairas e vales interiores; Agrosistema extensivo; Mesotemperado inferior & 12.24 & 6.56 & 1.87 \\ 
  Chairas e vales interiores; Agrosistema intensivo (mosaico agroforestal); Mesotemperado inferior & 6.12 & 5.37 & 1.14 \\ 
  Chairas e vales interiores; Urbano; Termotemperado & 6.12 & 1.01 & 6.08 \\ 
  Canons; Agrosistema intensivo (plantacion forestal); Termotemperado & 4.08 & 1.39 & 2.95 \\ 
  Chairas e vales interiores; Agrosistema intensivo (plantacion forestal); Termotemperado & 4.08 & 6.81 & 0.60 \\ 
  Chairas e vales interiores; Urbano; Mesotemperado inferior & 4.08 & 0.08 & 51.25 \\ 
  Canons; Agrosistema intensivo (plantacion forestal); category 0 & 2.04 & 1.25 & 1.64 \\ 
  Canons; Matogueira e rochedo; Mesotemperado superior & 2.04 & 0.06 & 36.29 \\ 
  category 0; Lamina de auga; category 0 & 2.04 & 1.89 & 1.08 \\ 
  Chairas e vales interiores; Agrosistema extensivo; Termotemperado & 2.04 & 1.56 & 1.31 \\ 
  Chairas e vales interiores; Agrosistema intensivo (mosaico agroforestal); Mesotemperado superior & 2.04 & 0.19 & 10.91 \\ 
  Chairas e vales interiores; Agrosistema intensivo (mosaico agroforestal); Termotemperado & 2.04 & 1.05 & 1.93 \\ 
  Chairas e vales interiores; Agrosistema intensivo (plantacion forestal); Mesotemperado inferior & 2.04 & 4.37 & 0.47 \\ 
  Chairas e vales interiores; Agrosistema intensivo (superficie de cultivo); Mesotemperado inferior & 2.04 & 0.66 & 3.10 \\ 
  Chairas e vales interiores; Bosque; Mesotemperado inferior & 2.04 & 3.85 & 0.53 \\ 
  Chairas e vales interiores; Urbano; category 0 & 2.04 & 0.19 & 10.60 \\ 
  Chairas e vales interiores; Vinedo; Termotemperado & 2.04 & 2.22 & 0.92 \\ 
  Serras; Agrosistema intensivo (plantacion forestal); Mesotemperado inferior & 2.04 & 2.65 & 0.77 \\ 
   \hline
\end{tabular}
\end{table}
% latex table generated in R 3.2.5 by xtable 1.8-0 package
% Mon May  9 13:23:06 2016
\begin{table}[p]
\centering
\caption{Frecuencia de aparición de BIC e frecuencia de tipos asociados, GAP Serras Orientais} 
\label{vbic5}
\begin{tabular}{lrrr}
  \hline
Tipo de paisaxe & F.Aparic (\%) & F.Tipo (\%) & Ratio \\ 
  \hline
Serras; Agrosistema extensivo; Supra e orotemperado & 20.00 & 8.91 & 2.24 \\ 
  Serras; Bosque; Mesotemperado superior & 20.00 & 3.65 & 5.48 \\ 
  Chairas e vales interiores; Agrosistema extensivo; Mesotemperado superior & 10.00 & 0.25 & 40.52 \\ 
  Serras; Agrosistema extensivo; Mesotemperado superior & 10.00 & 7.30 & 1.37 \\ 
  Serras; Bosque; Supra e orotemperado & 10.00 & 7.67 & 1.30 \\ 
  Vales sublitorais; Bosque; Mesotemperado inferior & 10.00 & 3.83 & 2.61 \\ 
  Chairas e vales interiores; Agrosistema extensivo; category 0 & 5.00 & 0.03 & 148.91 \\ 
  Serras; Agrosistema intensivo (mosaico agroforestal); Mesotemperado superior & 5.00 & 3.95 & 1.27 \\ 
  Serras; Matogueira e rochedo; Mesotemperado superior & 5.00 & 5.41 & 0.92 \\ 
  Vales sublitorais; Bosque; Mesotemperado superior & 5.00 & 4.16 & 1.20 \\ 
   \hline
\end{tabular}
\end{table}
% latex table generated in R 3.2.5 by xtable 1.8-0 package
% Mon May  9 13:23:06 2016
\begin{table}[p]
\centering
\caption{Frecuencia de aparición de BIC e frecuencia de tipos asociados, GAP Chairas e Fosas Luguesas} 
\label{vbic6}
\begin{tabular}{lrrr}
  \hline
Tipo de paisaxe & F.Aparic (\%) & F.Tipo (\%) & Ratio \\ 
  \hline
category 0; Conxunto Historico; category 0 & 18.18 & 0.02 & 912.21 \\ 
  Chairas e vales interiores; Agrosistema intensivo (mosaico agroforestal); Mesotemperado superior & 16.36 & 20.11 & 0.81 \\ 
  Chairas e vales interiores; Agrosistema extensivo; Mesotemperado inferior & 14.55 & 6.80 & 2.14 \\ 
  Chairas e vales interiores; Agrosistema extensivo; Mesotemperado superior & 9.09 & 9.68 & 0.94 \\ 
  Chairas e vales interiores; Agrosistema intensivo (mosaico agroforestal); Mesotemperado inferior & 7.27 & 9.44 & 0.77 \\ 
  Serras; Agrosistema intensivo (superficie de cultivo); Mesotemperado superior & 7.27 & 3.04 & 2.39 \\ 
  Chairas e vales interiores; Urbano; Mesotemperado superior & 5.45 & 0.45 & 12.10 \\ 
  Chairas e vales interiores; Agrosistema intensivo (plantacion forestal); Mesotemperado superior & 3.64 & 3.97 & 0.92 \\ 
  Chairas e vales interiores; Agrosistema intensivo (superficie de cultivo); Mesotemperado inferior & 3.64 & 2.36 & 1.54 \\ 
  Chairas e vales interiores; Urbano; Mesotemperado inferior & 3.64 & 0.19 & 19.44 \\ 
  Chairas e vales interiores; Agrosistema intensivo (plantacion forestal); Mesotemperado inferior & 1.82 & 3.31 & 0.55 \\ 
  Chairas e vales interiores; Agrosistema intensivo (superficie de cultivo); Mesotemperado superior & 1.82 & 5.78 & 0.31 \\ 
  Chairas e vales interiores; Bosque; Mesotemperado inferior & 1.82 & 1.92 & 0.95 \\ 
  Chairas e vales interiores; Bosque; Mesotemperado superior & 1.82 & 2.02 & 0.90 \\ 
  Chairas e vales interiores; Rururbano (diseminado); Mesotemperado inferior & 1.82 & 0.39 & 4.64 \\ 
  Serras; Agrosistema intensivo (mosaico agroforestal); Mesotemperado superior & 1.82 & 9.79 & 0.19 \\ 
   \hline
\end{tabular}
\end{table}
% latex table generated in R 3.2.5 by xtable 1.8-0 package
% Mon May  9 13:23:06 2016
\begin{table}[p]
\centering
\caption{Frecuencia de aparición de BIC e frecuencia de tipos asociados, GAP Galicia Central} 
\label{vbic7}
\begin{tabular}{lrrr}
  \hline
Tipo de paisaxe & F.Aparic (\%) & F.Tipo (\%) & Ratio \\ 
  \hline
Vales sublitorais; Agrosistema intensivo (mosaico agroforestal); Mesotemperado inferior & 25.42 & 43.06 & 0.59 \\ 
  category 0; Conxunto Historico; category 0 & 16.95 & 0.02 & 964.92 \\ 
  Vales sublitorais; Agrosistema extensivo; Mesotemperado inferior & 8.47 & 5.21 & 1.63 \\ 
  Vales sublitorais; Agrosistema intensivo (superficie de cultivo); Mesotemperado inferior & 8.47 & 3.18 & 2.67 \\ 
  Serras; Agrosistema extensivo; Mesotemperado superior & 6.78 & 4.47 & 1.52 \\ 
  Vales sublitorais; Urbano; Mesotemperado inferior & 6.78 & 0.64 & 10.56 \\ 
  Serras; Matogueira e rochedo; Mesotemperado superior & 5.08 & 6.90 & 0.74 \\ 
  Chairas e vales interiores; Agrosistema intensivo (mosaico agroforestal); Mesotemperado inferior & 3.39 & 1.17 & 2.90 \\ 
  Vales sublitorais; Matogueira e rochedo; Mesotemperado inferior & 3.39 & 1.65 & 2.06 \\ 
  Chairas e vales interiores; Agrosistema intensivo (plantacion forestal); Termotemperado & 1.69 & 0.11 & 15.43 \\ 
  Chairas e vales interiores; Urbano; Termotemperado & 1.69 & 0.03 & 66.69 \\ 
  Serras; Agrosistema extensivo; Mesotemperado inferior & 1.69 & 1.17 & 1.44 \\ 
  Serras; Agrosistema intensivo (mosaico agroforestal); Mesotemperado inferior & 1.69 & 1.33 & 1.28 \\ 
  Serras; Agrosistema intensivo (mosaico agroforestal); Mesotemperado superior & 1.69 & 3.93 & 0.43 \\ 
  Vales sublitorais; Agrosistema extensivo; Termotemperado & 1.69 & 0.29 & 5.89 \\ 
  Vales sublitorais; Agrosistema intensivo (plantacion forestal); Mesotemperado inferior & 1.69 & 5.84 & 0.29 \\ 
  Vales sublitorais; Bosque; Mesotemperado superior & 1.69 & 0.14 & 12.24 \\ 
  Vales sublitorais; Bosque; Termotemperado & 1.69 & 0.32 & 5.23 \\ 
   \hline
\end{tabular}
\end{table}
% latex table generated in R 3.2.5 by xtable 1.8-0 package
% Mon May  9 13:23:06 2016
\begin{table}[p]
\centering
\caption{Frecuencia de aparición de BIC e frecuencia de tipos asociados, GAP Chairas, Fosas e Serras Ourensás} 
\label{vbic8}
\begin{tabular}{lrrr}
  \hline
Tipo de paisaxe & F.Aparic (\%) & F.Tipo (\%) & Ratio \\ 
  \hline
Chairas e vales interiores; Agrosistema extensivo; Mesotemperado inferior & 24.00 & 7.38 & 3.25 \\ 
  category 0; Conxunto Historico; category 0 & 8.00 & 0.01 & 789.91 \\ 
  Chairas e vales interiores; Agrosistema intensivo (plantacion forestal); Mesotemperado inferior & 8.00 & 3.30 & 2.42 \\ 
  Chairas e vales interiores; Agrosistema intensivo (superficie de cultivo); Mesotemperado inferior & 8.00 & 7.53 & 1.06 \\ 
  Chairas e vales interiores; Bosque; Mesotemperado inferior & 8.00 & 6.18 & 1.29 \\ 
  Chairas e vales interiores; Rururbano (diseminado); Mesotemperado inferior & 8.00 & 0.32 & 24.72 \\ 
  Chairas e vales interiores; Urbano; Mesotemperado inferior & 8.00 & 0.23 & 34.45 \\ 
  Serras; Matogueira e rochedo; Supra e orotemperado & 8.00 & 14.06 & 0.57 \\ 
  category 0; Lamina de auga; category 0 & 4.00 & 0.71 & 5.67 \\ 
  Serras; Agrosistema extensivo; Mesotemperado inferior & 4.00 & 5.23 & 0.76 \\ 
  Serras; Agrosistema extensivo; Mesotemperado superior & 4.00 & 2.44 & 1.64 \\ 
  Serras; Agrosistema extensivo; Supra e orotemperado & 4.00 & 1.96 & 2.04 \\ 
  Serras; Matogueira e rochedo; Mesotemperado superior & 4.00 & 10.51 & 0.38 \\ 
   \hline
\end{tabular}
\end{table}
% latex table generated in R 3.2.5 by xtable 1.8-0 package
% Mon May  9 13:23:07 2016
\begin{table}[p]
\centering
\caption{Frecuencia de aparición de BIC e frecuencia de tipos asociados, GAP Serras Surorientais} 
\label{vbic9}
\begin{tabular}{lrrr}
  \hline
Tipo de paisaxe & F.Aparic (\%) & F.Tipo (\%) & Ratio \\ 
  \hline
Canons; Agrosistema intensivo (plantacion forestal); category 0 & 25.00 & 0.26 & 96.06 \\ 
  Canons; Matogueira e rochedo; Mesotemperado inferior & 25.00 & 1.24 & 20.18 \\ 
  Canons; Urbano; Mesotemperado inferior & 25.00 & 0.03 & 801.92 \\ 
  Serras; Agrosistema extensivo; Mesotemperado inferior & 25.00 & 4.79 & 5.22 \\ 
   \hline
\end{tabular}
\end{table}
% latex table generated in R 3.2.5 by xtable 1.8-0 package
% Mon May  9 13:23:07 2016
\begin{table}[p]
\centering
\caption{Frecuencia de aparición de BIC e frecuencia de tipos asociados, GAP Galicia Setentrional} 
\label{vbic10}
\begin{tabular}{lrrr}
  \hline
Tipo de paisaxe & F.Aparic (\%) & F.Tipo (\%) & Ratio \\ 
  \hline
Litoral Cantabro-Atlantico; Urbano; Termotemperado & 41.67 & 0.47 & 87.74 \\ 
  Litoral Cantabro-Atlantico; Agrosistema intensivo (mosaico agroforestal); Termotemperado & 16.67 & 5.53 & 3.02 \\ 
  category 0; Conxunto Historico; category 0 & 8.33 & 0.00 & 9866.66 \\ 
  Litoral Cantabro-Atlantico; Agrosistema extensivo; Termotemperado & 8.33 & 0.29 & 28.80 \\ 
  Serras; Agrosistema intensivo (mosaico agroforestal); Mesotemperado superior & 8.33 & 3.70 & 2.25 \\ 
  Vales sublitorais; Agrosistema intensivo (mosaico agroforestal); Mesotemperado inferior & 8.33 & 13.63 & 0.61 \\ 
  Vales sublitorais; Urbano; Mesotemperado inferior & 8.33 & 0.22 & 37.74 \\ 
   \hline
\end{tabular}
\end{table}
% latex table generated in R 3.2.5 by xtable 1.8-0 package
% Mon May  9 13:23:07 2016
\begin{table}[p]
\centering
\caption{Frecuencia de aparición de BIC e frecuencia de tipos asociados, GAP Chairas e Fosas Occidentais} 
\label{vbic11}
\begin{tabular}{lrrr}
  \hline
Tipo de paisaxe & F.Aparic (\%) & F.Tipo (\%) & Ratio \\ 
  \hline
Vales sublitorais; Agrosistema intensivo (mosaico agroforestal); Mesotemperado inferior & 38.46 & 36.05 & 1.07 \\ 
  Litoral Cantabro-Atlantico; Agrosistema intensivo (mosaico agroforestal); Termotemperado & 15.38 & 9.91 & 1.55 \\ 
  category 0; Conxunto Historico; category 0 & 11.54 & 0.12 & 94.76 \\ 
  Vales sublitorais; Agrosistema intensivo (plantacion forestal); Mesotemperado inferior & 11.54 & 15.20 & 0.76 \\ 
  Litoral Cantabro-Atlantico; Agrosistema intensivo (superficie de cultivo); Termotemperado & 7.69 & 0.90 & 8.57 \\ 
  Litoral Cantabro-Atlantico; Rururbano (diseminado); Termotemperado & 7.69 & 0.53 & 14.62 \\ 
  Vales sublitorais; Agrosistema intensivo (mosaico agroforestal); Mesotemperado superior & 3.85 & 0.08 & 46.23 \\ 
  Vales sublitorais; Urbano; Termotemperado & 3.85 & 0.23 & 16.90 \\ 
   \hline
\end{tabular}
\end{table}
% latex table generated in R 3.2.5 by xtable 1.8-0 package
% Mon May  9 13:23:07 2016
\begin{table}[p]
\centering
\caption{Frecuencia de aparición de BIC e frecuencia de tipos asociados, GAP Rías Baixas} 
\label{vbic12}
\begin{tabular}{lrrr}
  \hline
Tipo de paisaxe & F.Aparic (\%) & F.Tipo (\%) & Ratio \\ 
  \hline
Vales sublitorais; Agrosistema intensivo (plantacion forestal); Termotemperado & 24.14 & 16.41 & 1.47 \\ 
  category 0; Conxunto Historico; category 0 & 12.64 & 0.05 & 274.52 \\ 
  Litoral Cantabro-Atlantico; Rururbano (diseminado); Termotemperado & 10.34 & 7.10 & 1.46 \\ 
  Litoral Cantabro-Atlantico; Urbano; Termotemperado & 7.47 & 3.29 & 2.27 \\ 
  Vales sublitorais; Agrosistema intensivo (mosaico agroforestal); Termotemperado & 7.47 & 7.63 & 0.98 \\ 
  Vales sublitorais; Matogueira e rochedo; Mesotemperado inferior & 5.75 & 4.71 & 1.22 \\ 
  Litoral Cantabro-Atlantico; Agrosistema intensivo (mosaico agroforestal); Termotemperado & 5.17 & 10.49 & 0.49 \\ 
  Litoral Cantabro-Atlantico; Agrosistema intensivo (plantacion forestal); Termotemperado & 4.60 & 10.20 & 0.45 \\ 
  Vales sublitorais; Matogueira e rochedo; Termotemperado & 4.60 & 3.00 & 1.53 \\ 
  Litoral Cantabro-Atlantico; Matogueira e rochedo; Termotemperado & 2.30 & 1.73 & 1.33 \\ 
  Serras; Agrosistema intensivo (plantacion forestal); Mesotemperado inferior & 2.30 & 2.82 & 0.82 \\ 
  Vales sublitorais; Agrosistema intensivo (mosaico agroforestal); Mesotemperado inferior & 2.30 & 4.05 & 0.57 \\ 
  category 0; Praias e cantis; category 0 & 1.72 & 0.56 & 3.06 \\ 
  Litoral Cantabro-Atlantico; Agrosistema intensivo (superficie de cultivo); Termotemperado & 1.72 & 1.03 & 1.67 \\ 
  Serras; Matogueira e rochedo; Mesotemperado superior & 1.72 & 6.70 & 0.26 \\ 
  Serras; Matogueira e rochedo; Mesotemperado inferior & 1.15 & 3.29 & 0.35 \\ 
  Vales sublitorais; Agrosistema intensivo (superficie de cultivo); Termotemperado & 1.15 & 0.09 & 12.95 \\ 
  Vales sublitorais; Urbano; Termotemperado & 1.15 & 0.55 & 2.10 \\ 
   \hline
\end{tabular}
\end{table}
