% latex table generated in R 3.2.5 by xtable 1.8-0 package
% Mon May  9 13:23:06 2016
\begin{table}[p]
\centering
\caption{Frecuencia de aparición de Lugares de Importancia Comunitaria e frecuencia de tipos asociados Golfo Ártabro} 
\label{vnatura1}
\begin{tabular}{lrrr}
  \hline
Tipo de paisaxe & F.Aparic (\%) & F.Tipo (\%) & Ratio \\ 
  \hline
Canons; Bosque; Mesotemperado inferior & 16.05 & 1.56 & 10.28 \\ 
  Serras; Matogueira e rochedo; Mesotemperado superior & 10.97 & 3.06 & 3.59 \\ 
  Serras; Turbeira; Mesotemperado superior & 9.03 & 1.53 & 5.90 \\ 
  Vales sublitorais; Agrosistema intensivo (mosaico agroforestal); Mesotemperado inferior & 6.56 & 21.75 & 0.30 \\ 
  Serras; Bosque; Mesotemperado superior & 5.92 & 0.71 & 8.36 \\ 
  category 0; Lamina de auga; category 0 & 5.59 & 0.55 & 10.19 \\ 
  Vales sublitorais; Agrosistema intensivo (plantacion forestal); Mesotemperado inferior & 4.44 & 13.21 & 0.34 \\ 
  Serras; Agrosistema intensivo (mosaico agroforestal); Mesotemperado superior & 3.65 & 2.44 & 1.50 \\ 
  Vales sublitorais; Agrosistema intensivo (mosaico agroforestal); Mesotemperado superior & 3.09 & 1.47 & 2.10 \\ 
  Vales sublitorais; Bosque; Mesotemperado inferior & 2.66 & 0.78 & 3.40 \\ 
  Litoral Cantabro-Atlantico; Agrosistema intensivo (mosaico agroforestal); Termotemperado & 2.46 & 18.64 & 0.13 \\ 
  Serras; Agrosistema intensivo (plantacion forestal); Mesotemperado superior & 2.45 & 0.88 & 2.78 \\ 
  Canons; Agrosistema intensivo (mosaico agroforestal); Mesotemperado inferior & 2.45 & 0.25 & 9.83 \\ 
  Litoral Cantabro-Atlantico; Agrosistema intensivo (plantacion forestal); Termotemperado & 2.42 & 5.79 & 0.42 \\ 
  Vales sublitorais; Turbeira; Mesotemperado superior & 2.23 & 0.50 & 4.50 \\ 
  Vales sublitorais; Agrosistema extensivo; Mesotemperado inferior & 2.08 & 0.47 & 4.41 \\ 
  Serras; Agrosistema extensivo; Mesotemperado superior & 2.04 & 0.80 & 2.56 \\ 
  Serras; Agrosistema intensivo (mosaico agroforestal); Mesotemperado inferior & 1.39 & 0.16 & 8.52 \\ 
  Vales sublitorais; Matogueira e rochedo; Mesotemperado superior & 1.35 & 1.17 & 1.15 \\ 
  Litoral Cantabro-Atlantico; Bosque; Termotemperado & 1.17 & 0.54 & 2.19 \\ 
  Canons; Agrosistema intensivo (plantacion forestal); Mesotemperado inferior & 1.13 & 0.11 & 10.23 \\ 
  Vales sublitorais; Agrosistema extensivo; Mesotemperado superior & 1.03 & 0.68 & 1.51 \\ 
   \hline
\end{tabular}
\end{table}
% latex table generated in R 3.2.5 by xtable 1.8-0 package
% Mon May  9 13:23:06 2016
\begin{table}[p]
\centering
\caption{Frecuencia de aparición de Lugares de Importancia Comunitaria e frecuencia de tipos asociados A Mariña - Baixo Eo} 
\label{vnatura2}
\begin{tabular}{lrrr}
  \hline
Tipo de paisaxe & F.Aparic (\%) & F.Tipo (\%) & Ratio \\ 
  \hline
Serras; Turbeira; Mesotemperado superior & 32.83 & 2.64 & 12.44 \\ 
  Serras; Agrosistema intensivo (plantacion forestal); Mesotemperado superior & 14.33 & 6.80 & 2.11 \\ 
  Vales sublitorais; Agrosistema intensivo (plantacion forestal); Mesotemperado inferior & 13.58 & 23.68 & 0.57 \\ 
  Litoral Cantabro-Atlantico; Rururbano (diseminado); Termotemperado & 6.78 & 2.68 & 2.53 \\ 
  Serras; Agrosistema intensivo (plantacion forestal); Mesotemperado inferior & 5.98 & 1.66 & 3.60 \\ 
  Litoral Cantabro-Atlantico; Agrosistema intensivo (mosaico agroforestal); Termotemperado & 5.95 & 7.20 & 0.83 \\ 
   & 3.42 &  &  \\ 
  Litoral Cantabro-Atlantico; Agrosistema intensivo (plantacion forestal); Mesotemperado inferior & 2.54 & 18.89 & 0.13 \\ 
  category 0; Praias e cantis; category 0 & 2.11 & 0.25 & 8.60 \\ 
  Litoral Cantabro-Atlantico; Agrosistema intensivo (superficie de cultivo); Termotemperado & 1.96 & 2.07 & 0.95 \\ 
  Vales sublitorais; Agrosistema intensivo (mosaico agroforestal); Mesotemperado inferior & 1.63 & 12.86 & 0.13 \\ 
  category 0; Lamina de auga; category 0 & 1.62 & 0.13 & 12.91 \\ 
  Litoral Cantabro-Atlantico; Agrosistema intensivo (plantacion forestal); Termotemperado & 1.37 & 0.51 & 2.68 \\ 
  Serras; Agrosistema intensivo (mosaico agroforestal); Mesotemperado inferior & 1.32 & 0.26 & 5.02 \\ 
  Serras; Matogueira e rochedo; Mesotemperado superior & 1.10 & 1.71 & 0.64 \\ 
   \hline
\end{tabular}
\end{table}
% latex table generated in R 3.2.5 by xtable 1.8-0 package
% Mon May  9 13:23:06 2016
\begin{table}[p]
\centering
\caption{Frecuencia de aparición de Lugares de Importancia Comunitaria e frecuencia de tipos asociados Costa Sur - Baixo Miño} 
\label{vnatura3}
\begin{tabular}{lrrr}
  \hline
Tipo de paisaxe & F.Aparic (\%) & F.Tipo (\%) & Ratio \\ 
  \hline
Litoral Cantabro-Atlantico; Agrosistema intensivo (plantacion forestal); Termotemperado & 11.79 & 6.69 & 1.76 \\ 
  Serras; Agrosistema intensivo (plantacion forestal); Mesotemperado inferior & 10.79 & 3.38 & 3.20 \\ 
  Litoral Cantabro-Atlantico; Agrosistema intensivo (mosaico agroforestal); Termotemperado & 10.27 & 5.83 & 1.76 \\ 
  Litoral Cantabro-Atlantico; Bosque; Termotemperado & 8.88 & 0.40 & 22.41 \\ 
  Litoral Cantabro-Atlantico; Agrosistema extensivo; Termotemperado & 8.77 & 0.61 & 14.29 \\ 
  category 0; Lamina de auga; category 0 & 8.02 & 0.46 & 17.53 \\ 
  Chairas e vales interiores; Matogueira e rochedo; Mesotemperado inferior & 4.77 & 1.96 & 2.43 \\ 
  Litoral Cantabro-Atlantico; Rururbano (diseminado); Termotemperado & 4.54 & 4.97 & 0.91 \\ 
  Vales sublitorais; Agrosistema intensivo (plantacion forestal); Termotemperado & 4.50 & 16.82 & 0.27 \\ 
  Serras; Matogueira e rochedo; Supra e orotemperado & 4.12 & 0.14 & 29.70 \\ 
  Chairas e vales interiores; Agrosistema intensivo (plantacion forestal); Termotemperado & 2.84 & 6.71 & 0.42 \\ 
  Vales sublitorais; Agrosistema intensivo (mosaico agroforestal); Termotemperado & 2.11 & 7.55 & 0.28 \\ 
  Vales sublitorais; Agrosistema extensivo; Termotemperado & 1.74 & 2.59 & 0.67 \\ 
  Litoral Cantabro-Atlantico; Urbano; Termotemperado & 1.64 & 1.10 & 1.49 \\ 
  Litoral Cantabro-Atlantico; Matogueira e rochedo; Termotemperado & 1.48 & 0.52 & 2.85 \\ 
  Chairas e vales interiores; Agrosistema intensivo (plantacion forestal); Mesotemperado inferior & 1.44 & 0.44 & 3.28 \\ 
  Litoral Cantabro-Atlantico; Vinedo; Termotemperado & 1.35 & 0.70 & 1.91 \\ 
  Vales sublitorais; Bosque; Termotemperado & 1.33 & 1.44 & 0.92 \\ 
  Chairas e vales interiores; Vinedo; Termotemperado & 1.17 & 0.95 & 1.22 \\ 
   & 1.04 &  &  \\ 
  Serras; Matogueira e rochedo; Termotemperado & 1.02 & 0.40 & 2.51 \\ 
   \hline
\end{tabular}
\end{table}
% latex table generated in R 3.2.5 by xtable 1.8-0 package
% Mon May  9 13:23:06 2016
\begin{table}[p]
\centering
\caption{Frecuencia de aparición de Lugares de Importancia Comunitaria e frecuencia de tipos asociados Ribeiras Encaixadas do Miño e do Sil} 
\label{vnatura4}
\begin{tabular}{lrrr}
  \hline
Tipo de paisaxe & F.Aparic (\%) & F.Tipo (\%) & Ratio \\ 
  \hline
Serras; Matogueira e rochedo; Supra e orotemperado & 17.79 & 9.08 & 1.96 \\ 
  Canons; Bosque; Mesotemperado inferior & 11.11 & 2.67 & 4.17 \\ 
  Canons; Agrosistema intensivo (plantacion forestal); Termotemperado & 7.40 & 1.39 & 5.34 \\ 
  Serras; Bosque; Mesotemperado inferior & 7.20 & 2.32 & 3.11 \\ 
  Chairas e vales interiores; Matogueira e rochedo; category 0 & 5.13 & 2.25 & 2.28 \\ 
  Canons; Bosque; Termotemperado & 4.84 & 1.02 & 4.74 \\ 
  Canons; Matogueira e rochedo; Mesotemperado inferior & 3.87 & 0.67 & 5.74 \\ 
  Chairas e vales interiores; Agrosistema intensivo (plantacion forestal); Mesotemperado inferior & 3.71 & 4.37 & 0.85 \\ 
  category 0; Lamina de auga; category 0 & 3.40 & 1.89 & 1.79 \\ 
  Serras; Matogueira e rochedo; Mesotemperado superior & 3.29 & 3.10 & 1.06 \\ 
  Canons; Agrosistema intensivo (plantacion forestal); category 0 & 2.70 & 1.25 & 2.17 \\ 
  Serras; Agrosistema extensivo; Mesotemperado inferior & 2.70 & 1.98 & 1.37 \\ 
  Chairas e vales interiores; Bosque; category 0 & 2.69 & 0.80 & 3.35 \\ 
  Canons; Matogueira e rochedo; category 0 & 2.52 & 0.96 & 2.63 \\ 
  Serras; Matogueira e rochedo; Mesotemperado inferior & 2.04 & 2.79 & 0.73 \\ 
  Chairas e vales interiores; Agrosistema extensivo; Mesotemperado inferior & 2.03 & 6.56 & 0.31 \\ 
  Canons; Matogueira e rochedo; Termotemperado & 2.02 & 0.41 & 4.96 \\ 
  Serras; Matogueira e rochedo; category 0 & 1.67 & 0.48 & 3.48 \\ 
  Chairas e vales interiores; Matogueira e rochedo; Mesotemperado inferior & 1.58 & 3.61 & 0.44 \\ 
  Serras; Bosque; Supra e orotemperado & 1.28 & 0.47 & 2.69 \\ 
  Chairas e vales interiores; Agrosistema intensivo (plantacion forestal); category 0 & 1.26 & 1.83 & 0.69 \\ 
  Serras; Agrosistema intensivo (plantacion forestal); Mesotemperado inferior & 1.19 & 2.65 & 0.45 \\ 
  Chairas e vales interiores; Agrosistema extensivo; category 0 & 1.16 & 0.60 & 1.94 \\ 
   \hline
\end{tabular}
\end{table}
% latex table generated in R 3.2.5 by xtable 1.8-0 package
% Mon May  9 13:23:06 2016
\begin{table}[p]
\centering
\caption{Frecuencia de aparición de Lugares de Importancia Comunitaria e frecuencia de tipos asociados Serras Orientais} 
\label{vnatura5}
\begin{tabular}{lrrr}
  \hline
Tipo de paisaxe & F.Aparic (\%) & F.Tipo (\%) & Ratio \\ 
  \hline
Serras; Matogueira e rochedo; Supra e orotemperado & 25.31 & 15.61 & 1.62 \\ 
  Serras; Bosque; Supra e orotemperado & 13.82 & 7.67 & 1.80 \\ 
  Serras; Agrosistema extensivo; Supra e orotemperado & 12.16 & 8.91 & 1.36 \\ 
  Serras; Agrosistema intensivo (plantacion forestal); Supra e orotemperado & 7.76 & 6.86 & 1.13 \\ 
  Serras; Matogueira e rochedo; Mesotemperado superior & 5.29 & 5.41 & 0.98 \\ 
  Serras; Agrosistema intensivo (mosaico agroforestal); Supra e orotemperado & 5.02 & 5.74 & 0.88 \\ 
  Vales sublitorais; Agrosistema extensivo; Mesotemperado superior & 3.68 & 5.11 & 0.72 \\ 
  Vales sublitorais; Bosque; Mesotemperado inferior & 3.01 & 3.83 & 0.79 \\ 
  Vales sublitorais; Bosque; Mesotemperado superior & 2.95 & 4.16 & 0.71 \\ 
  Serras; Bosque; Mesotemperado superior & 2.51 & 3.65 & 0.69 \\ 
  Serras; Agrosistema intensivo (plantacion forestal); Mesotemperado superior & 2.27 & 3.86 & 0.59 \\ 
  Serras; Agrosistema extensivo; Mesotemperado superior & 2.01 & 7.30 & 0.27 \\ 
  Vales sublitorais; Matogueira e rochedo; Mesotemperado inferior & 1.92 & 1.20 & 1.60 \\ 
  Vales sublitorais; Matogueira e rochedo; Mesotemperado superior & 1.69 & 2.45 & 0.69 \\ 
  Serras; Agrosistema intensivo (mosaico agroforestal); Mesotemperado superior & 1.15 & 3.95 & 0.29 \\ 
  Chairas e vales interiores; Agrosistema intensivo (plantacion forestal); Mesotemperado inferior & 1.14 & 1.09 & 1.05 \\ 
  Vales sublitorais; Agrosistema extensivo; Mesotemperado inferior & 1.12 & 1.41 & 0.79 \\ 
   \hline
\end{tabular}
\end{table}
% latex table generated in R 3.2.5 by xtable 1.8-0 package
% Mon May  9 13:23:06 2016
\begin{table}[p]
\centering
\caption{Frecuencia de aparición de Lugares de Importancia Comunitaria e frecuencia de tipos asociados Chairas e Fosas Luguesas} 
\label{vnatura6}
\begin{tabular}{lrrr}
  \hline
Tipo de paisaxe & F.Aparic (\%) & F.Tipo (\%) & Ratio \\ 
  \hline
Serras; Turbeira; Mesotemperado superior & 16.66 & 1.08 & 15.40 \\ 
  Chairas e vales interiores; Agrosistema extensivo; Mesotemperado superior & 14.67 & 9.68 & 1.51 \\ 
  Chairas e vales interiores; Agrosistema intensivo (mosaico agroforestal); Mesotemperado superior & 7.51 & 20.11 & 0.37 \\ 
  Serras; Turbeira; Supra e orotemperado & 7.04 & 0.60 & 11.79 \\ 
  Serras; Matogueira e rochedo; Supra e orotemperado & 6.72 & 0.48 & 14.02 \\ 
  Chairas e vales interiores; Bosque; Mesotemperado superior & 6.45 & 2.02 & 3.19 \\ 
  Serras; Agrosistema extensivo; Mesotemperado superior & 5.68 & 5.94 & 0.96 \\ 
  Chairas e vales interiores; Agrosistema extensivo; Mesotemperado inferior & 4.62 & 6.80 & 0.68 \\ 
  Serras; Agrosistema intensivo (plantacion forestal); Supra e orotemperado & 3.46 & 0.59 & 5.88 \\ 
  Chairas e vales interiores; Bosque; Mesotemperado inferior & 3.46 & 1.92 & 1.80 \\ 
  Chairas e vales interiores; Agrosistema intensivo (superficie de cultivo); Mesotemperado superior & 3.27 & 5.78 & 0.57 \\ 
  Serras; Bosque; Mesotemperado superior & 2.70 & 0.71 & 3.81 \\ 
  Serras; Agrosistema intensivo (plantacion forestal); Mesotemperado superior & 2.54 & 2.29 & 1.11 \\ 
  Serras; Bosque; Supra e orotemperado & 2.47 & 0.23 & 10.90 \\ 
  Serras; Agrosistema intensivo (mosaico agroforestal); Mesotemperado superior & 2.16 & 9.79 & 0.22 \\ 
  Serras; Agrosistema intensivo (mosaico agroforestal); Supra e orotemperado & 1.67 & 0.34 & 4.94 \\ 
  Serras; Matogueira e rochedo; Mesotemperado superior & 1.60 & 1.94 & 0.83 \\ 
  category 0; Lamina de auga; category 0 & 1.08 & 0.09 & 11.34 \\ 
  Chairas e vales interiores; Agrosistema intensivo (mosaico agroforestal); Mesotemperado inferior & 1.07 & 9.44 & 0.11 \\ 
   \hline
\end{tabular}
\end{table}
% latex table generated in R 3.2.5 by xtable 1.8-0 package
% Mon May  9 13:23:06 2016
\begin{table}[p]
\centering
\caption{Frecuencia de aparición de Lugares de Importancia Comunitaria e frecuencia de tipos asociados Galicia Central} 
\label{vnatura7}
\begin{tabular}{lrrr}
  \hline
Tipo de paisaxe & F.Aparic (\%) & F.Tipo (\%) & Ratio \\ 
  \hline
Serras; Matogueira e rochedo; Mesotemperado superior & 41.48 & 6.90 & 6.01 \\ 
  Vales sublitorais; Agrosistema intensivo (mosaico agroforestal); Mesotemperado inferior & 11.63 & 43.06 & 0.27 \\ 
  Vales sublitorais; Matogueira e rochedo; Mesotemperado inferior & 7.36 & 1.65 & 4.47 \\ 
  Serras; Agrosistema intensivo (plantacion forestal); Supra e orotemperado & 7.05 & 0.54 & 13.16 \\ 
  Serras; Agrosistema extensivo; Mesotemperado superior & 5.52 & 4.47 & 1.23 \\ 
  Vales sublitorais; Agrosistema extensivo; Mesotemperado inferior & 3.69 & 5.21 & 0.71 \\ 
  Serras; Agrosistema intensivo (plantacion forestal); Mesotemperado superior & 2.76 & 1.82 & 1.52 \\ 
  Vales sublitorais; Agrosistema intensivo (plantacion forestal); Mesotemperado inferior & 2.39 & 5.84 & 0.41 \\ 
  Serras; Matogueira e rochedo; Supra e orotemperado & 2.13 & 0.53 & 4.01 \\ 
  Serras; Agrosistema intensivo (mosaico agroforestal); Mesotemperado superior & 1.90 & 3.93 & 0.48 \\ 
  Vales sublitorais; Bosque; Termotemperado & 1.51 & 0.32 & 4.67 \\ 
  Vales sublitorais; Agrosistema intensivo (mosaico agroforestal); Mesotemperado superior & 1.40 & 0.57 & 2.44 \\ 
  Serras; Agrosistema intensivo (superficie de cultivo); Mesotemperado superior & 1.25 & 1.44 & 0.86 \\ 
   \hline
\end{tabular}
\end{table}
% latex table generated in R 3.2.5 by xtable 1.8-0 package
% Mon May  9 13:23:06 2016
\begin{table}[p]
\centering
\caption{Frecuencia de aparición de Lugares de Importancia Comunitaria e frecuencia de tipos asociados Chairas, Fosas e Serras Ourensás} 
\label{vnatura8}
\begin{tabular}{lrrr}
  \hline
Tipo de paisaxe & F.Aparic (\%) & F.Tipo (\%) & Ratio \\ 
  \hline
Serras; Matogueira e rochedo; Supra e orotemperado & 53.15 & 14.06 & 3.78 \\ 
  Chairas e vales interiores; Agrosistema intensivo (superficie de cultivo); Mesotemperado inferior & 12.75 & 7.53 & 1.69 \\ 
  Serras; Matogueira e rochedo; Mesotemperado inferior & 5.24 & 5.15 & 1.02 \\ 
  Serras; Bosque; Supra e orotemperado & 3.55 & 1.18 & 3.01 \\ 
  Serras; Matogueira e rochedo; Mesotemperado superior & 3.20 & 10.51 & 0.30 \\ 
  Chairas e vales interiores; Agrosistema extensivo; Mesotemperado inferior & 2.45 & 7.38 & 0.33 \\ 
  Chairas e vales interiores; Matogueira e rochedo; Termotemperado & 1.90 & 1.80 & 1.06 \\ 
  Chairas e vales interiores; Matogueira e rochedo; Mesotemperado inferior & 1.72 & 4.82 & 0.36 \\ 
  Serras; Agrosistema intensivo (plantacion forestal); Mesotemperado inferior & 1.71 & 5.68 & 0.30 \\ 
  Serras; Agrosistema intensivo (plantacion forestal); Supra e orotemperado & 1.63 & 1.55 & 1.05 \\ 
  category 0; Lamina de auga; category 0 & 1.62 & 0.71 & 2.30 \\ 
  Serras; Bosque; Mesotemperado inferior & 1.37 & 4.67 & 0.29 \\ 
  Serras; Bosque; Mesotemperado superior & 1.05 & 2.22 & 0.47 \\ 
  Chairas e vales interiores; Bosque; Mesotemperado inferior & 1.03 & 6.18 & 0.17 \\ 
   \hline
\end{tabular}
\end{table}
% latex table generated in R 3.2.5 by xtable 1.8-0 package
% Mon May  9 13:23:07 2016
\begin{table}[p]
\centering
\caption{Frecuencia de aparición de Lugares de Importancia Comunitaria e frecuencia de tipos asociados Serras Surorientais} 
\label{vnatura9}
\begin{tabular}{lrrr}
  \hline
Tipo de paisaxe & F.Aparic (\%) & F.Tipo (\%) & Ratio \\ 
  \hline
Serras; Matogueira e rochedo; Supra e orotemperado & 69.77 & 47.28 & 1.48 \\ 
  Serras; Agrosistema intensivo (plantacion forestal); Supra e orotemperado & 6.09 & 10.23 & 0.60 \\ 
  Serras; Bosque; Mesotemperado superior & 5.21 & 5.61 & 0.93 \\ 
  Serras; Agrosistema extensivo; Supra e orotemperado & 3.20 & 2.94 & 1.09 \\ 
  Canons; Matogueira e rochedo; Mesotemperado inferior & 2.76 & 1.24 & 2.23 \\ 
  Serras; Bosque; Supra e orotemperado & 2.53 & 1.26 & 2.02 \\ 
  Serras; Agrosistema extensivo; Mesotemperado superior & 2.49 & 6.24 & 0.40 \\ 
  Serras; Matogueira e rochedo; Mesotemperado superior & 1.80 & 3.34 & 0.54 \\ 
  Serras; Agrosistema extensivo; Mesotemperado inferior & 1.15 & 4.79 & 0.24 \\ 
  Serras; Bosque; Mesotemperado inferior & 1.02 & 3.06 & 0.33 \\ 
   \hline
\end{tabular}
\end{table}
% latex table generated in R 3.2.5 by xtable 1.8-0 package
% Mon May  9 13:23:07 2016
\begin{table}[p]
\centering
\caption{Frecuencia de aparición de Lugares de Importancia Comunitaria e frecuencia de tipos asociados Galicia Setentrional} 
\label{vnatura10}
\begin{tabular}{lrrr}
  \hline
Tipo de paisaxe & F.Aparic (\%) & F.Tipo (\%) & Ratio \\ 
  \hline
Serras; Turbeira; Mesotemperado superior & 45.97 & 12.44 & 3.70 \\ 
  Serras; Agrosistema intensivo (plantacion forestal); Mesotemperado superior & 8.79 & 4.68 & 1.88 \\ 
  Litoral Cantabro-Atlantico; Matogueira e rochedo; Termotemperado & 7.51 & 1.77 & 4.23 \\ 
  category 0; Praias e cantis; category 0 & 4.91 & 0.94 & 5.20 \\ 
  Litoral Cantabro-Atlantico; Agrosistema intensivo (plantacion forestal); Termotemperado & 3.89 & 7.61 & 0.51 \\ 
  Serras; Matogueira e rochedo; Mesotemperado superior & 3.36 & 6.79 & 0.50 \\ 
  Vales sublitorais; Agrosistema intensivo (plantacion forestal); Mesotemperado inferior & 3.02 & 17.31 & 0.17 \\ 
  Litoral Cantabro-Atlantico; Agrosistema intensivo (mosaico agroforestal); Termotemperado & 2.31 & 5.53 & 0.42 \\ 
  Serras; Agrosistema extensivo; Mesotemperado superior & 2.18 & 2.70 & 0.80 \\ 
  Serras; Bosque; Mesotemperado superior & 1.92 & 1.63 & 1.18 \\ 
  Serras; Agrosistema intensivo (mosaico agroforestal); Mesotemperado superior & 1.84 & 3.70 & 0.50 \\ 
  Litoral Cantabro-Atlantico; Agrosistema intensivo (plantacion forestal); Mesotemperado inferior & 1.67 & 2.79 & 0.60 \\ 
   & 1.54 &  &  \\ 
  Litoral Cantabro-Atlantico; Matogueira e rochedo; Mesotemperado inferior & 1.33 & 0.28 & 4.71 \\ 
  Serras; Matogueira e rochedo; Supra e orotemperado & 1.20 & 0.32 & 3.78 \\ 
  Litoral Cantabro-Atlantico; Rururbano (diseminado); Termotemperado & 1.15 & 1.15 & 1.01 \\ 
  category 0; Lamina de auga; category 0 & 1.09 & 0.89 & 1.23 \\ 
   \hline
\end{tabular}
\end{table}
% latex table generated in R 3.2.5 by xtable 1.8-0 package
% Mon May  9 13:23:07 2016
\begin{table}[p]
\centering
\caption{Frecuencia de aparición de Lugares de Importancia Comunitaria e frecuencia de tipos asociados Chairas e Fosas Occidentais} 
\label{vnatura11}
\begin{tabular}{lrrr}
  \hline
Tipo de paisaxe & F.Aparic (\%) & F.Tipo (\%) & Ratio \\ 
  \hline
Litoral Cantabro-Atlantico; Matogueira e rochedo; Termotemperado & 39.24 & 4.58 & 8.56 \\ 
  Vales sublitorais; Matogueira e rochedo; Mesotemperado inferior & 14.31 & 4.77 & 3.00 \\ 
  Litoral Cantabro-Atlantico; Agrosistema intensivo (plantacion forestal); Termotemperado & 9.50 & 7.60 & 1.25 \\ 
  category 0; Praias e cantis; category 0 & 7.92 & 0.54 & 14.73 \\ 
  Litoral Cantabro-Atlantico; Agrosistema intensivo (mosaico agroforestal); Termotemperado & 6.73 & 9.91 & 0.68 \\ 
  Vales sublitorais; Agrosistema intensivo (plantacion forestal); Mesotemperado inferior & 6.50 & 15.20 & 0.43 \\ 
   & 4.40 &  &  \\ 
  Litoral Cantabro-Atlantico; Agrosistema intensivo (superficie de cultivo); Termotemperado & 2.15 & 0.90 & 2.40 \\ 
  Vales sublitorais; Agrosistema intensivo (mosaico agroforestal); Mesotemperado inferior & 1.95 & 36.05 & 0.05 \\ 
  category 0; Lamina de auga; category 0 & 1.58 & 0.46 & 3.42 \\ 
  Vales sublitorais; Agrosistema intensivo (plantacion forestal); Termotemperado & 1.26 & 3.44 & 0.37 \\ 
  Vales sublitorais; Matogueira e rochedo; Termotemperado & 1.05 & 0.80 & 1.31 \\ 
  Litoral Cantabro-Atlantico; Agrosistema extensivo; Termotemperado & 1.04 & 0.13 & 8.36 \\ 
  Litoral Cantabro-Atlantico; Rururbano (diseminado); Termotemperado & 1.01 & 0.53 & 1.92 \\ 
   \hline
\end{tabular}
\end{table}
% latex table generated in R 3.2.5 by xtable 1.8-0 package
% Mon May  9 13:23:07 2016
\begin{table}[p]
\centering
\caption{Frecuencia de aparición de Lugares de Importancia Comunitaria e frecuencia de tipos asociados Rías Baixas} 
\label{vnatura12}
\begin{tabular}{lrrr}
  \hline
Tipo de paisaxe & F.Aparic (\%) & F.Tipo (\%) & Ratio \\ 
  \hline
Serras; Matogueira e rochedo; Mesotemperado superior & 34.05 & 6.70 & 5.08 \\ 
   & 13.92 &  &  \\ 
  Litoral Cantabro-Atlantico; Matogueira e rochedo; Termotemperado & 7.69 & 1.73 & 4.44 \\ 
  Litoral Cantabro-Atlantico; Agrosistema intensivo (plantacion forestal); Termotemperado & 6.85 & 10.20 & 0.67 \\ 
  category 0; Praias e cantis; category 0 & 6.40 & 0.56 & 11.34 \\ 
  Litoral Cantabro-Atlantico; Agrosistema intensivo (mosaico agroforestal); Termotemperado & 3.87 & 10.49 & 0.37 \\ 
  Vales sublitorais; Bosque; Mesotemperado inferior & 3.67 & 0.74 & 4.93 \\ 
  Litoral Cantabro-Atlantico; Rururbano (diseminado); Termotemperado & 2.41 & 7.10 & 0.34 \\ 
  Litoral Cantabro-Atlantico; Agrosistema extensivo; Termotemperado & 2.13 & 0.60 & 3.54 \\ 
  Serras; Agrosistema extensivo; Mesotemperado inferior & 2.07 & 0.57 & 3.61 \\ 
  category 0; Lamina de auga; category 0 & 2.01 & 0.19 & 10.51 \\ 
  Litoral Cantabro-Atlantico; Vinedo; Termotemperado & 1.79 & 1.45 & 1.23 \\ 
  Vales sublitorais; Agrosistema extensivo; Mesotemperado inferior & 1.76 & 0.71 & 2.50 \\ 
  Litoral Cantabro-Atlantico; Agrosistema intensivo (superficie de cultivo); Termotemperado & 1.76 & 1.03 & 1.70 \\ 
  Vales sublitorais; Matogueira e rochedo; Mesotemperado inferior & 1.60 & 4.71 & 0.34 \\ 
  Serras; Bosque; Mesotemperado superior & 1.34 & 0.29 & 4.56 \\ 
  Serras; Bosque; Mesotemperado inferior & 1.17 & 0.47 & 2.47 \\ 
   \hline
\end{tabular}
\end{table}
