% latex table generated in R 3.2.5 by xtable 1.8-0 package
% Mon May  9 13:23:06 2016
\begin{table}[p]
\centering
\caption{Frecuencia de aparición de valores naturais identificados na participación pública e frecuencia de tipos asociados Golfo Ártabro} 
\label{vsixotnat1}
\begin{tabular}{lrrr}
  \hline
Tipo de paisaxe & F.Aparic (\%) & F.Tipo (\%) & Ratio \\ 
  \hline
Canons; Bosque; Mesotemperado inferior & 13.89 & 1.56 & 8.90 \\ 
  Vales sublitorais; Bosque; Mesotemperado inferior & 13.89 & 0.78 & 17.72 \\ 
  Litoral Cantabro-Atlantico; Agrosistema intensivo (mosaico agroforestal); Termotemperado & 8.33 & 18.64 & 0.45 \\ 
  Litoral Cantabro-Atlantico; Rururbano (diseminado); Termotemperado & 8.33 & 7.02 & 1.19 \\ 
  Canons; Matogueira e rochedo; Mesotemperado inferior & 5.56 & 0.08 & 67.80 \\ 
  category 0; Praias e cantis; category 0 & 5.56 & 0.12 & 45.52 \\ 
  Litoral Cantabro-Atlantico; Agrosistema extensivo; Termotemperado & 5.56 & 1.02 & 5.47 \\ 
  Litoral Cantabro-Atlantico; Agrosistema intensivo (plantacion forestal); Termotemperado & 5.56 & 5.79 & 0.96 \\ 
  Litoral Cantabro-Atlantico; Matogueira e rochedo; Termotemperado & 5.56 & 1.19 & 4.66 \\ 
  Vales sublitorais; Agrosistema intensivo (mosaico agroforestal); Mesotemperado inferior & 5.56 & 21.75 & 0.26 \\ 
   & 5.56 &  &  \\ 
  category 0; Conxunto Historico; category 0 & 2.78 & 0.08 & 33.64 \\ 
  category 0; Lamina de auga; category 0 & 2.78 & 0.55 & 5.06 \\ 
  Serras; Agrosistema extensivo; Mesotemperado superior & 2.78 & 0.80 & 3.49 \\ 
  Serras; Turbeira; Mesotemperado superior & 2.78 & 1.53 & 1.82 \\ 
  Vales sublitorais; Agrosistema intensivo (plantacion forestal); Mesotemperado inferior & 2.78 & 13.21 & 0.21 \\ 
  Vales sublitorais; Matogueira e rochedo; Mesotemperado inferior & 2.78 & 0.64 & 4.36 \\ 
   \hline
\end{tabular}
\end{table}
% latex table generated in R 3.2.5 by xtable 1.8-0 package
% Mon May  9 13:23:06 2016
\begin{table}[p]
\centering
\caption{Frecuencia de aparición de valores naturais identificados na participación pública e frecuencia de tipos asociados A Mariña - Baixo Eo} 
\label{vsixotnat2}
\begin{tabular}{lrrr}
  \hline
Tipo de paisaxe & F.Aparic (\%) & F.Tipo (\%) & Ratio \\ 
  \hline
Vales sublitorais; Agrosistema intensivo (plantacion forestal); Mesotemperado inferior & 19.35 & 23.68 & 0.82 \\ 
  Litoral Cantabro-Atlantico; Agrosistema intensivo (plantacion forestal); Mesotemperado inferior & 12.90 & 18.89 & 0.68 \\ 
   & 12.90 &  &  \\ 
  Litoral Cantabro-Atlantico; Agrosistema intensivo (mosaico agroforestal); Termotemperado & 9.68 & 7.20 & 1.34 \\ 
  Litoral Cantabro-Atlantico; Urbano; Termotemperado & 9.68 & 1.10 & 8.76 \\ 
  category 0; Conxunto Historico; category 0 & 6.45 & 0.10 & 67.16 \\ 
  Litoral Cantabro-Atlantico; Matogueira e rochedo; Mesotemperado inferior & 6.45 & 0.05 & 137.52 \\ 
  Serras; Turbeira; Mesotemperado superior & 6.45 & 2.64 & 2.44 \\ 
  Litoral Cantabro-Atlantico; Agrosistema intensivo (mosaico agroforestal); Mesotemperado inferior & 3.23 & 3.57 & 0.90 \\ 
  Litoral Cantabro-Atlantico; Rururbano (diseminado); Termotemperado & 3.23 & 2.68 & 1.20 \\ 
  Serras; Bosque; Mesotemperado superior & 3.23 & 0.70 & 4.58 \\ 
  Serras; Matogueira e rochedo; Mesotemperado superior & 3.23 & 1.71 & 1.88 \\ 
  Vales sublitorais; Bosque; Mesotemperado superior & 3.23 & 1.22 & 2.65 \\ 
   \hline
\end{tabular}
\end{table}
% latex table generated in R 3.2.5 by xtable 1.8-0 package
% Mon May  9 13:23:06 2016
\begin{table}[p]
\centering
\caption{Frecuencia de aparición de valores naturais identificados na participación pública e frecuencia de tipos asociados Costa Sur - Baixo Miño} 
\label{vsixotnat3}
\begin{tabular}{lrrr}
  \hline
Tipo de paisaxe & F.Aparic (\%) & F.Tipo (\%) & Ratio \\ 
  \hline
Serras; Agrosistema intensivo (plantacion forestal); Termotemperado & 19.05 & 3.79 & 5.03 \\ 
  Serras; Matogueira e rochedo; Mesotemperado inferior & 17.46 & 3.84 & 4.54 \\ 
  Vales sublitorais; Agrosistema intensivo (plantacion forestal); Termotemperado & 11.11 & 16.82 & 0.66 \\ 
  Litoral Cantabro-Atlantico; Agrosistema intensivo (plantacion forestal); Termotemperado & 7.94 & 6.69 & 1.19 \\ 
  Serras; Turbeira; Mesotemperado inferior & 7.94 & 0.12 & 65.84 \\ 
  Serras; Agrosistema intensivo (plantacion forestal); Mesotemperado inferior & 6.35 & 3.38 & 1.88 \\ 
  Litoral Cantabro-Atlantico; Agrosistema intensivo (mosaico agroforestal); Termotemperado & 4.76 & 5.83 & 0.82 \\ 
  category 0; Lamina de auga; category 0 & 3.17 & 0.46 & 6.94 \\ 
  Litoral Cantabro-Atlantico; Agrosistema intensivo (superficie de cultivo); Termotemperado & 3.17 & 0.16 & 19.35 \\ 
  Vales sublitorais; Bosque; Termotemperado & 3.17 & 1.44 & 2.21 \\ 
  Chairas e vales interiores; Agrosistema intensivo (mosaico agroforestal); Mesotemperado inferior & 1.59 & 0.84 & 1.88 \\ 
  Chairas e vales interiores; Agrosistema intensivo (plantacion forestal); Termotemperado & 1.59 & 6.71 & 0.24 \\ 
  Litoral Cantabro-Atlantico; Agrosistema extensivo; Termotemperado & 1.59 & 0.61 & 2.59 \\ 
  Litoral Cantabro-Atlantico; Rururbano (diseminado); Termotemperado & 1.59 & 4.97 & 0.32 \\ 
  Litoral Cantabro-Atlantico; Vinedo; Termotemperado & 1.59 & 0.70 & 2.26 \\ 
  Serras; Agrosistema extensivo; Mesotemperado superior & 1.59 & 0.28 & 5.69 \\ 
  Serras; Matogueira e rochedo; Supra e orotemperado & 1.59 & 0.14 & 11.45 \\ 
  Vales sublitorais; Agrosistema intensivo (superficie de cultivo); Termotemperado & 1.59 & 0.10 & 16.08 \\ 
  Vales sublitorais; Matogueira e rochedo; Termotemperado & 1.59 & 2.72 & 0.58 \\ 
  Vales sublitorais; Rururbano (diseminado); Termotemperado & 1.59 & 2.82 & 0.56 \\ 
   \hline
\end{tabular}
\end{table}
% latex table generated in R 3.2.5 by xtable 1.8-0 package
% Mon May  9 13:23:06 2016
\begin{table}[p]
\centering
\caption{Frecuencia de aparición de valores naturais identificados na participación pública e frecuencia de tipos asociados Ribeiras Encaixadas do Miño e do Sil} 
\label{vsixotnat4}
\begin{tabular}{lrrr}
  \hline
Tipo de paisaxe & F.Aparic (\%) & F.Tipo (\%) & Ratio \\ 
  \hline
Canons; Bosque; Mesotemperado inferior & 26.67 & 2.67 & 10.00 \\ 
  Chairas e vales interiores; Bosque; Termotemperado & 11.67 & 3.03 & 3.85 \\ 
  category 0; Lamina de auga; category 0 & 6.67 & 1.89 & 3.52 \\ 
  Serras; Matogueira e rochedo; Supra e orotemperado & 6.67 & 9.08 & 0.73 \\ 
  Chairas e vales interiores; Agrosistema intensivo (plantacion forestal); Mesotemperado inferior & 5.00 & 4.37 & 1.14 \\ 
  Serras; Bosque; Mesotemperado inferior & 5.00 & 2.32 & 2.16 \\ 
  Chairas e vales interiores; Vinedo; Termotemperado & 3.33 & 2.22 & 1.50 \\ 
  Serras; Bosque; Supra e orotemperado & 3.33 & 0.47 & 7.02 \\ 
  Serras; Matogueira e rochedo; Mesotemperado superior & 3.33 & 3.10 & 1.07 \\ 
  Canons; Agrosistema intensivo (plantacion forestal); category 0 & 1.67 & 1.25 & 1.34 \\ 
  Canons; Agrosistema intensivo (plantacion forestal); Termotemperado & 1.67 & 1.39 & 1.20 \\ 
  Canons; Bosque; category 0 & 1.67 & 0.39 & 4.23 \\ 
  Canons; Matogueira e rochedo; category 0 & 1.67 & 0.96 & 1.74 \\ 
  Canons; Matogueira e rochedo; Mesotemperado inferior & 1.67 & 0.67 & 2.47 \\ 
  Chairas e vales interiores; Agrosistema extensivo; Termotemperado & 1.67 & 1.56 & 1.07 \\ 
  Chairas e vales interiores; Agrosistema intensivo (plantacion forestal); category 0 & 1.67 & 1.83 & 0.91 \\ 
  Chairas e vales interiores; Agrosistema intensivo (plantacion forestal); Termotemperado & 1.67 & 6.81 & 0.24 \\ 
  Chairas e vales interiores; Bosque; category 0 & 1.67 & 0.80 & 2.08 \\ 
  Chairas e vales interiores; Matogueira e rochedo; category 0 & 1.67 & 2.25 & 0.74 \\ 
  Chairas e vales interiores; Matogueira e rochedo; Mesotemperado inferior & 1.67 & 3.61 & 0.46 \\ 
  Chairas e vales interiores; Vinedo; category 0 & 1.67 & 1.70 & 0.98 \\ 
  Serras; Agrosistema extensivo; Mesotemperado inferior & 1.67 & 1.98 & 0.84 \\ 
  Serras; Agrosistema extensivo; Mesotemperado superior & 1.67 & 1.71 & 0.97 \\ 
  Serras; Agrosistema intensivo (plantacion forestal); Mesotemperado inferior & 1.67 & 2.65 & 0.63 \\ 
  Serras; Bosque; Termotemperado & 1.67 & 0.00 & 346.92 \\ 
  Serras; Matogueira e rochedo; Mesotemperado inferior & 1.67 & 2.79 & 0.60 \\ 
   \hline
\end{tabular}
\end{table}
% latex table generated in R 3.2.5 by xtable 1.8-0 package
% Mon May  9 13:23:06 2016
\begin{table}[p]
\centering
\caption{Frecuencia de aparición de valores naturais identificados na participación pública e frecuencia de tipos asociados Serras Orientais} 
\label{vsixotnat5}
\begin{tabular}{lrrr}
  \hline
Tipo de paisaxe & F.Aparic (\%) & F.Tipo (\%) & Ratio \\ 
  \hline
Serras; Bosque; Supra e orotemperado & 24.32 & 7.67 & 3.17 \\ 
  Serras; Matogueira e rochedo; Supra e orotemperado & 17.57 & 15.61 & 1.13 \\ 
  Serras; Agrosistema extensivo; Supra e orotemperado & 12.16 & 8.91 & 1.36 \\ 
  Vales sublitorais; Bosque; Mesotemperado inferior & 6.76 & 3.83 & 1.76 \\ 
  Serras; Matogueira e rochedo; Mesotemperado superior & 5.41 & 5.41 & 1.00 \\ 
  Serras; Agrosistema intensivo (mosaico agroforestal); Mesotemperado superior & 4.05 & 3.95 & 1.03 \\ 
  Serras; Agrosistema intensivo (mosaico agroforestal); Supra e orotemperado & 4.05 & 5.74 & 0.71 \\ 
  Vales sublitorais; Agrosistema extensivo; Mesotemperado inferior & 4.05 & 1.41 & 2.88 \\ 
  Vales sublitorais; Bosque; Mesotemperado superior & 4.05 & 4.16 & 0.97 \\ 
  category 0; Lamina de auga; category 0 & 2.70 & 0.16 & 16.79 \\ 
  Serras; Agrosistema extensivo; Mesotemperado superior & 2.70 & 7.30 & 0.37 \\ 
  Serras; Agrosistema intensivo (plantacion forestal); Supra e orotemperado & 2.70 & 6.86 & 0.39 \\ 
  Serras; Bosque; Mesotemperado superior & 2.70 & 3.65 & 0.74 \\ 
  Vales sublitorais; Agrosistema extensivo; Mesotemperado superior & 2.70 & 5.11 & 0.53 \\ 
  Canons; Bosque; Mesotemperado inferior & 1.35 & 0.44 & 3.10 \\ 
  Chairas e vales interiores; Bosque; category 0 & 1.35 & 0.19 & 6.97 \\ 
  Serras; Bosque; Mesotemperado inferior & 1.35 & 0.55 & 2.46 \\ 
   \hline
\end{tabular}
\end{table}
% latex table generated in R 3.2.5 by xtable 1.8-0 package
% Mon May  9 13:23:06 2016
\begin{table}[p]
\centering
\caption{Frecuencia de aparición de valores naturais identificados na participación pública e frecuencia de tipos asociados Chairas e Fosas Luguesas} 
\label{vsixotnat6}
\begin{tabular}{lrrr}
  \hline
Tipo de paisaxe & F.Aparic (\%) & F.Tipo (\%) & Ratio \\ 
  \hline
Chairas e vales interiores; Agrosistema extensivo; Mesotemperado inferior & 18.92 & 6.80 & 2.78 \\ 
  Chairas e vales interiores; Bosque; Mesotemperado inferior & 16.22 & 1.92 & 8.45 \\ 
  Chairas e vales interiores; Agrosistema extensivo; Mesotemperado superior & 10.81 & 9.68 & 1.12 \\ 
  Serras; Agrosistema intensivo (plantacion forestal); Supra e orotemperado & 10.81 & 0.59 & 18.38 \\ 
  Chairas e vales interiores; Agrosistema intensivo (plantacion forestal); Mesotemperado inferior & 8.11 & 3.31 & 2.45 \\ 
  category 0; Lamina de auga; category 0 & 5.41 & 0.09 & 56.91 \\ 
  Chairas e vales interiores; Agrosistema intensivo (mosaico agroforestal); Mesotemperado superior & 5.41 & 20.11 & 0.27 \\ 
  Chairas e vales interiores; Bosque; Mesotemperado superior & 5.41 & 2.02 & 2.67 \\ 
  Chairas e vales interiores; Urbano; Mesotemperado superior & 2.70 & 0.45 & 5.99 \\ 
  Serras; Agrosistema extensivo; Mesotemperado superior & 2.70 & 5.94 & 0.46 \\ 
  Serras; Agrosistema extensivo; Supra e orotemperado & 2.70 & 0.37 & 7.31 \\ 
  Serras; Agrosistema intensivo (mosaico agroforestal); Mesotemperado superior & 2.70 & 9.79 & 0.28 \\ 
  Serras; Agrosistema intensivo (plantacion forestal); Mesotemperado superior & 2.70 & 2.29 & 1.18 \\ 
  Serras; Matogueira e rochedo; Supra e orotemperado & 2.70 & 0.48 & 5.64 \\ 
  Serras; Turbeira; Mesotemperado superior & 2.70 & 1.08 & 2.50 \\ 
   \hline
\end{tabular}
\end{table}
% latex table generated in R 3.2.5 by xtable 1.8-0 package
% Mon May  9 13:23:06 2016
\begin{table}[p]
\centering
\caption{Frecuencia de aparición de valores naturais identificados na participación pública e frecuencia de tipos asociados Galicia Central} 
\label{vsixotnat7}
\begin{tabular}{lrrr}
  \hline
Tipo de paisaxe & F.Aparic (\%) & F.Tipo (\%) & Ratio \\ 
  \hline
Vales sublitorais; Agrosistema intensivo (mosaico agroforestal); Mesotemperado inferior & 29.09 & 43.06 & 0.68 \\ 
  Vales sublitorais; Agrosistema extensivo; Mesotemperado inferior & 10.91 & 5.21 & 2.09 \\ 
  Serras; Agrosistema extensivo; Mesotemperado superior & 7.27 & 4.47 & 1.63 \\ 
  Serras; Matogueira e rochedo; Mesotemperado superior & 7.27 & 6.90 & 1.05 \\ 
  Vales sublitorais; Agrosistema intensivo (plantacion forestal); Mesotemperado inferior & 6.36 & 5.84 & 1.09 \\ 
  Serras; Agrosistema intensivo (mosaico agroforestal); Mesotemperado superior & 2.73 & 3.93 & 0.69 \\ 
  Serras; Agrosistema intensivo (plantacion forestal); Supra e orotemperado & 2.73 & 0.54 & 5.09 \\ 
  Serras; Bosque; Mesotemperado inferior & 2.73 & 0.45 & 6.04 \\ 
  Vales sublitorais; Bosque; Mesotemperado inferior & 2.73 & 0.80 & 3.41 \\ 
  Vales sublitorais; Matogueira e rochedo; Mesotemperado inferior & 2.73 & 1.65 & 1.66 \\ 
  Vales sublitorais; Urbano; Mesotemperado inferior & 2.73 & 0.64 & 4.25 \\ 
  Chairas e vales interiores; Bosque; Mesotemperado inferior & 1.82 & 0.68 & 2.68 \\ 
  Serras; Agrosistema intensivo (superficie de cultivo); Mesotemperado superior & 1.82 & 1.44 & 1.26 \\ 
  Serras; Matogueira e rochedo; Mesotemperado inferior & 1.82 & 0.62 & 2.93 \\ 
  Vales sublitorais; Agrosistema extensivo; Termotemperado & 1.82 & 0.29 & 6.32 \\ 
  Vales sublitorais; Agrosistema intensivo (mosaico agroforestal); Mesotemperado superior & 1.82 & 0.57 & 3.17 \\ 
  Vales sublitorais; Agrosistema intensivo (plantacion forestal); Termotemperado & 1.82 & 1.80 & 1.01 \\ 
  Vales sublitorais; Bosque; Termotemperado & 1.82 & 0.32 & 5.61 \\ 
   \hline
\end{tabular}
\end{table}
% latex table generated in R 3.2.5 by xtable 1.8-0 package
% Mon May  9 13:23:06 2016
\begin{table}[p]
\centering
\caption{Frecuencia de aparición de valores naturais identificados na participación pública e frecuencia de tipos asociados Chairas, Fosas e Serras Ourensás} 
\label{vsixotnat8}
\begin{tabular}{lrrr}
  \hline
Tipo de paisaxe & F.Aparic (\%) & F.Tipo (\%) & Ratio \\ 
  \hline
Serras; Matogueira e rochedo; Supra e orotemperado & 16.92 & 14.06 & 1.20 \\ 
  Serras; Matogueira e rochedo; Mesotemperado superior & 13.85 & 10.51 & 1.32 \\ 
  Chairas e vales interiores; Matogueira e rochedo; Mesotemperado inferior & 9.23 & 4.82 & 1.91 \\ 
  Chairas e vales interiores; Agrosistema extensivo; Mesotemperado inferior & 7.69 & 7.38 & 1.04 \\ 
  Chairas e vales interiores; Bosque; Mesotemperado inferior & 7.69 & 6.18 & 1.24 \\ 
  Chairas e vales interiores; Agrosistema intensivo (plantacion forestal); Termotemperado & 6.15 & 3.78 & 1.63 \\ 
  Chairas e vales interiores; Agrosistema intensivo (superficie de cultivo); Mesotemperado inferior & 6.15 & 7.53 & 0.82 \\ 
  Serras; Agrosistema extensivo; Mesotemperado inferior & 4.62 & 5.23 & 0.88 \\ 
  Serras; Matogueira e rochedo; Mesotemperado inferior & 4.62 & 5.15 & 0.90 \\ 
  Chairas e vales interiores; Vinedo; Termotemperado & 3.08 & 0.90 & 3.42 \\ 
  Serras; Bosque; Supra e orotemperado & 3.08 & 1.18 & 2.61 \\ 
  category 0; Lamina de auga; category 0 & 1.54 & 0.71 & 2.18 \\ 
  Chairas e vales interiores; Agrosistema extensivo; Termotemperado & 1.54 & 0.66 & 2.33 \\ 
  Chairas e vales interiores; Agrosistema intensivo (mosaico agroforestal); Termotemperado & 1.54 & 0.24 & 6.33 \\ 
  Chairas e vales interiores; Agrosistema intensivo (superficie de cultivo); Termotemperado & 1.54 & 0.03 & 52.81 \\ 
  Chairas e vales interiores; Matogueira e rochedo; Termotemperado & 1.54 & 1.80 & 0.86 \\ 
  Chairas e vales interiores; Rururbano (diseminado); Mesotemperado inferior & 1.54 & 0.32 & 4.75 \\ 
  Chairas e vales interiores; Urbano; Mesotemperado inferior & 1.54 & 0.23 & 6.62 \\ 
  Serras; Agrosistema intensivo (plantacion forestal); Mesotemperado inferior & 1.54 & 5.68 & 0.27 \\ 
  Serras; Agrosistema intensivo (plantacion forestal); Mesotemperado superior & 1.54 & 1.41 & 1.09 \\ 
  Serras; Agrosistema intensivo (plantacion forestal); Supra e orotemperado & 1.54 & 1.55 & 1.00 \\ 
  Serras; Bosque; Mesotemperado superior & 1.54 & 2.22 & 0.69 \\ 
   \hline
\end{tabular}
\end{table}
% latex table generated in R 3.2.5 by xtable 1.8-0 package
% Mon May  9 13:23:07 2016
\begin{table}[p]
\centering
\caption{Frecuencia de aparición de valores naturais identificados na participación pública e frecuencia de tipos asociados Serras Surorientais} 
\label{vsixotnat9}
\begin{tabular}{lrrr}
  \hline
Tipo de paisaxe & F.Aparic (\%) & F.Tipo (\%) & Ratio \\ 
  \hline
Serras; Matogueira e rochedo; Supra e orotemperado & 45.21 & 47.28 & 0.96 \\ 
  Canons; Vinedo; category 0 & 9.59 & 0.03 & 302.08 \\ 
  Serras; Agrosistema extensivo; Mesotemperado inferior & 6.85 & 4.79 & 1.43 \\ 
  Serras; Bosque; Mesotemperado superior & 6.85 & 5.61 & 1.22 \\ 
  Serras; Agrosistema intensivo (plantacion forestal); Supra e orotemperado & 5.48 & 10.23 & 0.54 \\ 
  Serras; Matogueira e rochedo; Mesotemperado inferior & 5.48 & 2.75 & 1.99 \\ 
  Canons; Agrosistema intensivo (plantacion forestal); category 0 & 2.74 & 0.26 & 10.53 \\ 
  category 0; Lamina de auga; category 0 & 2.74 & 1.38 & 1.98 \\ 
  Serras; Agrosistema extensivo; Mesotemperado superior & 2.74 & 6.24 & 0.44 \\ 
  Serras; Bosque; Mesotemperado inferior & 2.74 & 3.06 & 0.89 \\ 
  Serras; Bosque; Supra e orotemperado & 2.74 & 1.26 & 2.18 \\ 
  Canons; Vinedo; Mesotemperado inferior & 1.37 & 0.07 & 20.81 \\ 
  Serras; Agrosistema extensivo; Supra e orotemperado & 1.37 & 2.94 & 0.47 \\ 
  Serras; Agrosistema intensivo (mosaico agroforestal); Mesotemperado inferior & 1.37 & 0.76 & 1.81 \\ 
  Serras; Agrosistema intensivo (mosaico agroforestal); Mesotemperado superior & 1.37 & 0.81 & 1.70 \\ 
  Serras; Agrosistema intensivo (superficie de cultivo); Mesotemperado superior & 1.37 & 0.72 & 1.91 \\ 
   \hline
\end{tabular}
\end{table}
% latex table generated in R 3.2.5 by xtable 1.8-0 package
% Mon May  9 13:23:07 2016
\begin{table}[p]
\centering
\caption{Frecuencia de aparición de valores naturais identificados na participación pública e frecuencia de tipos asociados Galicia Setentrional} 
\label{vsixotnat10}
\begin{tabular}{lrrr}
  \hline
Tipo de paisaxe & F.Aparic (\%) & F.Tipo (\%) & Ratio \\ 
  \hline
Serras; Turbeira; Mesotemperado superior & 26.32 & 12.44 & 2.12 \\ 
  category 0; Praias e cantis; category 0 & 14.47 & 0.94 & 15.33 \\ 
   & 10.53 &  &  \\ 
  Litoral Cantabro-Atlantico; Agrosistema intensivo (mosaico agroforestal); Termotemperado & 6.58 & 5.53 & 1.19 \\ 
  Litoral Cantabro-Atlantico; Agrosistema intensivo (plantacion forestal); Termotemperado & 6.58 & 7.61 & 0.86 \\ 
  Vales sublitorais; Bosque; Mesotemperado inferior & 6.58 & 1.49 & 4.41 \\ 
  Litoral Cantabro-Atlantico; Matogueira e rochedo; Termotemperado & 5.26 & 1.77 & 2.97 \\ 
  Serras; Matogueira e rochedo; Mesotemperado superior & 3.95 & 6.79 & 0.58 \\ 
  Vales sublitorais; Agrosistema intensivo (plantacion forestal); Mesotemperado inferior & 3.95 & 17.31 & 0.23 \\ 
  Serras; Agrosistema intensivo (plantacion forestal); Mesotemperado superior & 2.63 & 4.68 & 0.56 \\ 
  Serras; Bosque; Mesotemperado inferior & 2.63 & 0.22 & 12.00 \\ 
  Litoral Cantabro-Atlantico; Agrosistema intensivo (plantacion forestal); Mesotemperado inferior & 1.32 & 2.79 & 0.47 \\ 
  Litoral Cantabro-Atlantico; Matogueira e rochedo; Mesotemperado inferior & 1.32 & 0.28 & 4.66 \\ 
  Litoral Cantabro-Atlantico; Urbano; Termotemperado & 1.32 & 0.47 & 2.77 \\ 
  Serras; Agrosistema extensivo; Mesotemperado superior & 1.32 & 2.70 & 0.49 \\ 
  Serras; Agrosistema intensivo (mosaico agroforestal); Mesotemperado superior & 1.32 & 3.70 & 0.36 \\ 
  Vales sublitorais; Agrosistema intensivo (mosaico agroforestal); Mesotemperado inferior & 1.32 & 13.63 & 0.10 \\ 
  Vales sublitorais; Agrosistema intensivo (mosaico agroforestal); Mesotemperado superior & 1.32 & 0.82 & 1.60 \\ 
  Vales sublitorais; Rururbano (diseminado); Mesotemperado inferior & 1.32 & 0.08 & 16.09 \\ 
   \hline
\end{tabular}
\end{table}
% latex table generated in R 3.2.5 by xtable 1.8-0 package
% Mon May  9 13:23:07 2016
\begin{table}[p]
\centering
\caption{Frecuencia de aparición de valores naturais identificados na participación pública e frecuencia de tipos asociados Chairas e Fosas Occidentais} 
\label{vsixotnat11}
\begin{tabular}{lrrr}
  \hline
Tipo de paisaxe & F.Aparic (\%) & F.Tipo (\%) & Ratio \\ 
  \hline
category 0; Praias e cantis; category 0 & 19.86 & 0.54 & 36.94 \\ 
  Vales sublitorais; Agrosistema intensivo (plantacion forestal); Mesotemperado inferior & 19.18 & 15.20 & 1.26 \\ 
  Litoral Cantabro-Atlantico; Matogueira e rochedo; Termotemperado & 16.44 & 4.58 & 3.59 \\ 
  Vales sublitorais; Agrosistema intensivo (mosaico agroforestal); Mesotemperado inferior & 10.96 & 36.05 & 0.30 \\ 
   & 10.96 &  &  \\ 
  Litoral Cantabro-Atlantico; Agrosistema intensivo (plantacion forestal); Termotemperado & 7.53 & 7.60 & 0.99 \\ 
  category 0; Conxunto Historico; category 0 & 2.74 & 0.12 & 22.50 \\ 
  Vales sublitorais; Matogueira e rochedo; Mesotemperado inferior & 2.74 & 4.77 & 0.57 \\ 
  Litoral Cantabro-Atlantico; Rururbano (diseminado); Termotemperado & 1.37 & 0.53 & 2.60 \\ 
  Litoral Cantabro-Atlantico; Urbano; Termotemperado & 1.37 & 0.66 & 2.08 \\ 
  Vales sublitorais; Agrosistema extensivo; Mesotemperado inferior & 1.37 & 0.95 & 1.44 \\ 
  Vales sublitorais; Agrosistema intensivo (plantacion forestal); Termotemperado & 1.37 & 3.44 & 0.40 \\ 
  Vales sublitorais; Matogueira e rochedo; Mesotemperado superior & 1.37 & 2.22 & 0.62 \\ 
   \hline
\end{tabular}
\end{table}
% latex table generated in R 3.2.5 by xtable 1.8-0 package
% Mon May  9 13:23:07 2016
\begin{table}[p]
\centering
\caption{Frecuencia de aparición de valores naturais identificados na participación pública e frecuencia de tipos asociados Rías Baixas} 
\label{vsixotnat12}
\begin{tabular}{lrrr}
  \hline
Tipo de paisaxe & F.Aparic (\%) & F.Tipo (\%) & Ratio \\ 
  \hline
Serras; Matogueira e rochedo; Mesotemperado superior & 15.57 & 6.70 & 2.32 \\ 
  Serras; Matogueira e rochedo; Mesotemperado inferior & 12.30 & 3.29 & 3.73 \\ 
  Vales sublitorais; Agrosistema intensivo (plantacion forestal); Termotemperado & 11.48 & 16.41 & 0.70 \\ 
   & 11.48 &  &  \\ 
  Vales sublitorais; Matogueira e rochedo; Mesotemperado inferior & 9.84 & 4.71 & 2.09 \\ 
  Litoral Cantabro-Atlantico; Matogueira e rochedo; Termotemperado & 7.38 & 1.73 & 4.26 \\ 
  Litoral Cantabro-Atlantico; Agrosistema intensivo (plantacion forestal); Termotemperado & 6.56 & 10.20 & 0.64 \\ 
  category 0; Praias e cantis; category 0 & 5.74 & 0.56 & 10.17 \\ 
  Serras; Agrosistema intensivo (plantacion forestal); Mesotemperado inferior & 4.10 & 2.82 & 1.46 \\ 
  Litoral Cantabro-Atlantico; Rururbano (diseminado); Termotemperado & 2.46 & 7.10 & 0.35 \\ 
  Vales sublitorais; Agrosistema intensivo (mosaico agroforestal); Termotemperado & 2.46 & 7.63 & 0.32 \\ 
  Litoral Cantabro-Atlantico; Agrosistema intensivo (mosaico agroforestal); Termotemperado & 1.64 & 10.49 & 0.16 \\ 
   \hline
\end{tabular}
\end{table}
