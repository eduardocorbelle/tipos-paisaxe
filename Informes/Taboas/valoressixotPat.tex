% latex table generated in R 3.2.5 by xtable 1.8-0 package
% Mon May  9 13:23:06 2016
\begin{table}[p]
\centering
\caption{Frecuencia de aparición de valores patrimoniais identificados na participación pública e frecuencia de tipos asociados Golfo Ártabro} 
\label{vsixotpat1}
\begin{tabular}{lrrr}
  \hline
Tipo de paisaxe & F.Aparic (\%) & F.Tipo (\%) & Ratio \\ 
  \hline
category 0; Conxunto Historico; category 0 & 16.67 & 0.08 & 201.84 \\ 
  Vales sublitorais; Agrosistema intensivo (mosaico agroforestal); Mesotemperado inferior & 16.67 & 21.75 & 0.77 \\ 
  Canons; Bosque; Mesotemperado inferior & 11.11 & 1.56 & 7.12 \\ 
  Litoral Cantabro-Atlantico; Agrosistema intensivo (plantacion forestal); Termotemperado & 11.11 & 5.79 & 1.92 \\ 
   & 11.11 &  &  \\ 
  Litoral Cantabro-Atlantico; Agrosistema intensivo (mosaico agroforestal); Termotemperado & 8.33 & 18.64 & 0.45 \\ 
  Litoral Cantabro-Atlantico; Urbano; Termotemperado & 8.33 & 4.93 & 1.69 \\ 
  Litoral Cantabro-Atlantico; Agrosistema extensivo; Termotemperado & 5.56 & 1.02 & 5.47 \\ 
  Serras; Agrosistema intensivo (plantacion forestal); Mesotemperado inferior & 2.78 & 0.51 & 5.47 \\ 
  Vales sublitorais; Agrosistema intensivo (plantacion forestal); Mesotemperado inferior & 2.78 & 13.21 & 0.21 \\ 
  Vales sublitorais; Matogueira e rochedo; Mesotemperado superior & 2.78 & 1.17 & 2.37 \\ 
  Vales sublitorais; Urbano; Termotemperado & 2.78 & 0.22 & 12.85 \\ 
   \hline
\end{tabular}
\end{table}
% latex table generated in R 3.2.5 by xtable 1.8-0 package
% Mon May  9 13:23:06 2016
\begin{table}[p]
\centering
\caption{Frecuencia de aparición de valores patrimoniais identificados na participación pública e frecuencia de tipos asociados A Mariña - Baixo Eo} 
\label{vsixotpat2}
\begin{tabular}{lrrr}
  \hline
Tipo de paisaxe & F.Aparic (\%) & F.Tipo (\%) & Ratio \\ 
  \hline
Vales sublitorais; Agrosistema intensivo (plantacion forestal); Mesotemperado inferior & 34.00 & 23.68 & 1.44 \\ 
  Vales sublitorais; Agrosistema intensivo (mosaico agroforestal); Mesotemperado inferior & 18.00 & 12.86 & 1.40 \\ 
  category 0; Conxunto Historico; category 0 & 8.00 & 0.10 & 83.28 \\ 
  Litoral Cantabro-Atlantico; Agrosistema intensivo (plantacion forestal); Mesotemperado inferior & 6.00 & 18.89 & 0.32 \\ 
  Litoral Cantabro-Atlantico; Matogueira e rochedo; Mesotemperado inferior & 6.00 & 0.05 & 127.89 \\ 
  Vales sublitorais; Agrosistema extensivo; Mesotemperado inferior & 6.00 & 0.98 & 6.13 \\ 
  Litoral Cantabro-Atlantico; Agrosistema intensivo (mosaico agroforestal); Mesotemperado inferior & 4.00 & 3.57 & 1.12 \\ 
  Litoral Cantabro-Atlantico; Agrosistema intensivo (mosaico agroforestal); Termotemperado & 4.00 & 7.20 & 0.56 \\ 
  Vales sublitorais; Agrosistema intensivo (superficie de cultivo); Mesotemperado inferior & 4.00 & 0.59 & 6.83 \\ 
  category 0; Praias e cantis; category 0 & 2.00 & 0.25 & 8.15 \\ 
  Serras; Agrosistema intensivo (plantacion forestal); Mesotemperado superior & 2.00 & 6.80 & 0.29 \\ 
  Serras; Turbeira; Mesotemperado superior & 2.00 & 2.64 & 0.76 \\ 
  Vales sublitorais; Bosque; Mesotemperado superior & 2.00 & 1.22 & 1.65 \\ 
   & 2.00 &  &  \\ 
   \hline
\end{tabular}
\end{table}
% latex table generated in R 3.2.5 by xtable 1.8-0 package
% Mon May  9 13:23:06 2016
\begin{table}[p]
\centering
\caption{Frecuencia de aparición de valores patrimoniais identificados na participación pública e frecuencia de tipos asociados Costa Sur - Baixo Miño} 
\label{vsixotpat3}
\begin{tabular}{lrrr}
  \hline
Tipo de paisaxe & F.Aparic (\%) & F.Tipo (\%) & Ratio \\ 
  \hline
Litoral Cantabro-Atlantico; Agrosistema intensivo (mosaico agroforestal); Termotemperado & 28.12 & 5.83 & 4.83 \\ 
  Litoral Cantabro-Atlantico; Agrosistema intensivo (plantacion forestal); Termotemperado & 15.62 & 6.69 & 2.34 \\ 
  Vales sublitorais; Agrosistema intensivo (mosaico agroforestal); Termotemperado & 15.62 & 7.55 & 2.07 \\ 
  Vales sublitorais; Agrosistema intensivo (plantacion forestal); Termotemperado & 15.62 & 16.82 & 0.93 \\ 
  category 0; Conxunto Historico; category 0 & 12.50 & 0.22 & 55.91 \\ 
  Chairas e vales interiores; Agrosistema intensivo (mosaico agroforestal); Mesotemperado inferior & 3.12 & 0.84 & 3.71 \\ 
  Chairas e vales interiores; Agrosistema intensivo (plantacion forestal); Termotemperado & 3.12 & 6.71 & 0.47 \\ 
  Serras; Agrosistema intensivo (plantacion forestal); Termotemperado & 3.12 & 3.79 & 0.83 \\ 
  Serras; Matogueira e rochedo; Mesotemperado inferior & 3.12 & 3.84 & 0.81 \\ 
   \hline
\end{tabular}
\end{table}
% latex table generated in R 3.2.5 by xtable 1.8-0 package
% Mon May  9 13:23:06 2016
\begin{table}[p]
\centering
\caption{Frecuencia de aparición de valores patrimoniais identificados na participación pública e frecuencia de tipos asociados Ribeiras Encaixadas do Miño e do Sil} 
\label{vsixotpat4}
\begin{tabular}{lrrr}
  \hline
Tipo de paisaxe & F.Aparic (\%) & F.Tipo (\%) & Ratio \\ 
  \hline
Canons; Bosque; Mesotemperado inferior & 11.11 & 2.67 & 4.17 \\ 
  category 0; Conxunto Historico; category 0 & 11.11 & 0.04 & 271.42 \\ 
  Chairas e vales interiores; Agrosistema intensivo (plantacion forestal); Termotemperado & 9.52 & 6.81 & 1.40 \\ 
  Canons; Agrosistema intensivo (plantacion forestal); category 0 & 6.35 & 1.25 & 5.09 \\ 
  Canons; Bosque; category 0 & 4.76 & 0.39 & 12.08 \\ 
  Canons; Vinedo; Mesotemperado inferior & 4.76 & 0.25 & 19.08 \\ 
  category 0; Lamina de auga; category 0 & 4.76 & 1.89 & 2.51 \\ 
  Chairas e vales interiores; Agrosistema intensivo (mosaico agroforestal); Mesotemperado inferior & 4.76 & 5.37 & 0.89 \\ 
  Chairas e vales interiores; Bosque; Termotemperado & 4.76 & 3.03 & 1.57 \\ 
  Canons; Matogueira e rochedo; category 0 & 3.17 & 0.96 & 3.31 \\ 
  Chairas e vales interiores; Agrosistema intensivo (plantacion forestal); category 0 & 3.17 & 1.83 & 1.73 \\ 
  Chairas e vales interiores; Agrosistema intensivo (plantacion forestal); Mesotemperado inferior & 3.17 & 4.37 & 0.73 \\ 
  Chairas e vales interiores; Bosque; category 0 & 3.17 & 0.80 & 3.96 \\ 
  Chairas e vales interiores; Vinedo; category 0 & 3.17 & 1.70 & 1.87 \\ 
  Canons; Matogueira e rochedo; Termotemperado & 1.59 & 0.41 & 3.90 \\ 
  Canons; Vinedo; Termotemperado & 1.59 & 0.51 & 3.11 \\ 
  Chairas e vales interiores; Agrosistema intensivo (mosaico agroforestal); Termotemperado & 1.59 & 1.05 & 1.50 \\ 
  Chairas e vales interiores; Rururbano (diseminado); Termotemperado & 1.59 & 0.91 & 1.75 \\ 
  Chairas e vales interiores; Urbano; Mesotemperado inferior & 1.59 & 0.08 & 19.93 \\ 
  Chairas e vales interiores; Urbano; Termotemperado & 1.59 & 1.01 & 1.58 \\ 
  Chairas e vales interiores; Vinedo; Termotemperado & 1.59 & 2.22 & 0.72 \\ 
  Serras; Agrosistema extensivo; Mesotemperado superior & 1.59 & 1.71 & 0.93 \\ 
  Serras; Agrosistema intensivo (plantacion forestal); Mesotemperado inferior & 1.59 & 2.65 & 0.60 \\ 
  Serras; Agrosistema intensivo (superficie de cultivo); Mesotemperado inferior & 1.59 & 0.12 & 13.76 \\ 
  Serras; Bosque; Mesotemperado inferior & 1.59 & 2.32 & 0.69 \\ 
  Serras; Matogueira e rochedo; Mesotemperado superior & 1.59 & 3.10 & 0.51 \\ 
  Serras; Matogueira e rochedo; Supra e orotemperado & 1.59 & 9.08 & 0.17 \\ 
  Serras; Vinedo; Mesotemperado inferior & 1.59 & 0.12 & 12.99 \\ 
   \hline
\end{tabular}
\end{table}
% latex table generated in R 3.2.5 by xtable 1.8-0 package
% Mon May  9 13:23:06 2016
\begin{table}[p]
\centering
\caption{Frecuencia de aparición de valores patrimoniais identificados na participación pública e frecuencia de tipos asociados Serras Orientais} 
\label{vsixotpat5}
\begin{tabular}{lrrr}
  \hline
Tipo de paisaxe & F.Aparic (\%) & F.Tipo (\%) & Ratio \\ 
  \hline
Serras; Agrosistema extensivo; Supra e orotemperado & 20.00 & 8.91 & 2.24 \\ 
  Serras; Matogueira e rochedo; Supra e orotemperado & 20.00 & 15.61 & 1.28 \\ 
  Serras; Bosque; Supra e orotemperado & 10.00 & 7.67 & 1.30 \\ 
  Vales sublitorais; Bosque; Mesotemperado superior & 10.00 & 4.16 & 2.40 \\ 
  Serras; Agrosistema extensivo; Mesotemperado superior & 6.67 & 7.30 & 0.91 \\ 
  Serras; Agrosistema intensivo (mosaico agroforestal); Mesotemperado superior & 6.67 & 3.95 & 1.69 \\ 
  Vales sublitorais; Bosque; Mesotemperado inferior & 6.67 & 3.83 & 1.74 \\ 
  Canons; Bosque; Mesotemperado inferior & 3.33 & 0.44 & 7.64 \\ 
  Chairas e vales interiores; Agrosistema extensivo; Mesotemperado superior & 3.33 & 0.25 & 13.51 \\ 
  Chairas e vales interiores; Bosque; category 0 & 3.33 & 0.19 & 17.19 \\ 
  Serras; Agrosistema intensivo (mosaico agroforestal); Supra e orotemperado & 3.33 & 5.74 & 0.58 \\ 
  Serras; Agrosistema intensivo (plantacion forestal); Supra e orotemperado & 3.33 & 6.86 & 0.49 \\ 
  Serras; Bosque; Mesotemperado superior & 3.33 & 3.65 & 0.91 \\ 
   \hline
\end{tabular}
\end{table}
% latex table generated in R 3.2.5 by xtable 1.8-0 package
% Mon May  9 13:23:06 2016
\begin{table}[p]
\centering
\caption{Frecuencia de aparición de valores patrimoniais identificados na participación pública e frecuencia de tipos asociados Chairas e Fosas Luguesas} 
\label{vsixotpat6}
\begin{tabular}{lrrr}
  \hline
Tipo de paisaxe & F.Aparic (\%) & F.Tipo (\%) & Ratio \\ 
  \hline
Chairas e vales interiores; Agrosistema intensivo (mosaico agroforestal); Mesotemperado superior & 18.42 & 20.11 & 0.92 \\ 
  Chairas e vales interiores; Bosque; Mesotemperado inferior & 15.79 & 1.92 & 8.23 \\ 
  category 0; Conxunto Historico; category 0 & 10.53 & 0.02 & 528.12 \\ 
  Chairas e vales interiores; Agrosistema extensivo; Mesotemperado inferior & 10.53 & 6.80 & 1.55 \\ 
  Chairas e vales interiores; Agrosistema extensivo; Mesotemperado superior & 10.53 & 9.68 & 1.09 \\ 
  Chairas e vales interiores; Agrosistema intensivo (superficie de cultivo); Mesotemperado superior & 10.53 & 5.78 & 1.82 \\ 
  Chairas e vales interiores; Urbano; Mesotemperado superior & 5.26 & 0.45 & 11.67 \\ 
  category 0; Lamina de auga; category 0 & 2.63 & 0.09 & 27.71 \\ 
  Chairas e vales interiores; Agrosistema intensivo (mosaico agroforestal); Mesotemperado inferior & 2.63 & 9.44 & 0.28 \\ 
  Chairas e vales interiores; Agrosistema intensivo (plantacion forestal); Mesotemperado superior & 2.63 & 3.97 & 0.66 \\ 
  Chairas e vales interiores; Urbano; Mesotemperado inferior & 2.63 & 0.19 & 14.07 \\ 
  Serras; Agrosistema extensivo; Supra e orotemperado & 2.63 & 0.37 & 7.12 \\ 
  Serras; Agrosistema intensivo (mosaico agroforestal); Mesotemperado superior & 2.63 & 9.79 & 0.27 \\ 
  Serras; Agrosistema intensivo (superficie de cultivo); Mesotemperado superior & 2.63 & 3.04 & 0.87 \\ 
   \hline
\end{tabular}
\end{table}
% latex table generated in R 3.2.5 by xtable 1.8-0 package
% Mon May  9 13:23:06 2016
\begin{table}[p]
\centering
\caption{Frecuencia de aparición de valores patrimoniais identificados na participación pública e frecuencia de tipos asociados Galicia Central} 
\label{vsixotpat7}
\begin{tabular}{lrrr}
  \hline
Tipo de paisaxe & F.Aparic (\%) & F.Tipo (\%) & Ratio \\ 
  \hline
Vales sublitorais; Agrosistema intensivo (mosaico agroforestal); Mesotemperado inferior & 35.38 & 43.06 & 0.82 \\ 
  Vales sublitorais; Agrosistema extensivo; Mesotemperado inferior & 8.46 & 5.21 & 1.62 \\ 
  Vales sublitorais; Agrosistema intensivo (superficie de cultivo); Mesotemperado inferior & 7.69 & 3.18 & 2.42 \\ 
  Vales sublitorais; Urbano; Mesotemperado inferior & 6.92 & 0.64 & 10.78 \\ 
  Serras; Agrosistema extensivo; Mesotemperado superior & 5.38 & 4.47 & 1.20 \\ 
  Serras; Matogueira e rochedo; Mesotemperado superior & 5.38 & 6.90 & 0.78 \\ 
  category 0; Conxunto Historico; category 0 & 3.85 & 0.02 & 218.96 \\ 
  Vales sublitorais; Bosque; Mesotemperado inferior & 3.08 & 0.80 & 3.84 \\ 
  Serras; Agrosistema extensivo; Mesotemperado inferior & 2.31 & 1.17 & 1.97 \\ 
  Vales sublitorais; Rururbano (diseminado); Mesotemperado inferior & 2.31 & 0.23 & 10.12 \\ 
  Chairas e vales interiores; Urbano; Termotemperado & 1.54 & 0.03 & 60.54 \\ 
  Vales sublitorais; Agrosistema intensivo (plantacion forestal); Mesotemperado inferior & 1.54 & 5.84 & 0.26 \\ 
  Vales sublitorais; Agrosistema intensivo (plantacion forestal); Termotemperado & 1.54 & 1.80 & 0.86 \\ 
  Vales sublitorais; Rururbano (diseminado); Termotemperado & 1.54 & 0.34 & 4.57 \\ 
   \hline
\end{tabular}
\end{table}
% latex table generated in R 3.2.5 by xtable 1.8-0 package
% Mon May  9 13:23:06 2016
\begin{table}[p]
\centering
\caption{Frecuencia de aparición de valores patrimoniais identificados na participación pública e frecuencia de tipos asociados Chairas, Fosas e Serras Ourensás} 
\label{vsixotpat8}
\begin{tabular}{lrrr}
  \hline
Tipo de paisaxe & F.Aparic (\%) & F.Tipo (\%) & Ratio \\ 
  \hline
Serras; Agrosistema extensivo; Mesotemperado inferior & 11.43 & 5.23 & 2.18 \\ 
  Serras; Matogueira e rochedo; Mesotemperado superior & 10.00 & 10.51 & 0.95 \\ 
  Chairas e vales interiores; Matogueira e rochedo; Mesotemperado inferior & 7.14 & 4.82 & 1.48 \\ 
  Chairas e vales interiores; Agrosistema extensivo; Mesotemperado inferior & 5.71 & 7.38 & 0.77 \\ 
  Chairas e vales interiores; Bosque; Mesotemperado inferior & 5.71 & 6.18 & 0.92 \\ 
  Serras; Agrosistema extensivo; Supra e orotemperado & 5.71 & 1.96 & 2.92 \\ 
  Serras; Matogueira e rochedo; Supra e orotemperado & 5.71 & 14.06 & 0.41 \\ 
  Chairas e vales interiores; Matogueira e rochedo; Termotemperado & 4.29 & 1.80 & 2.39 \\ 
  Serras; Agrosistema intensivo (mosaico agroforestal); Supra e orotemperado & 4.29 & 0.26 & 16.24 \\ 
  Serras; Agrosistema intensivo (superficie de cultivo); Supra e orotemperado & 4.29 & 0.46 & 9.24 \\ 
  Serras; Bosque; Mesotemperado superior & 4.29 & 2.22 & 1.93 \\ 
  Serras; Matogueira e rochedo; Mesotemperado inferior & 4.29 & 5.15 & 0.83 \\ 
  category 0; Conxunto Historico; category 0 & 2.86 & 0.01 & 282.11 \\ 
  Chairas e vales interiores; Agrosistema intensivo (plantacion forestal); Mesotemperado inferior & 2.86 & 3.30 & 0.87 \\ 
  Chairas e vales interiores; Agrosistema intensivo (plantacion forestal); Termotemperado & 2.86 & 3.78 & 0.76 \\ 
  Chairas e vales interiores; Agrosistema intensivo (superficie de cultivo); Mesotemperado inferior & 2.86 & 7.53 & 0.38 \\ 
  Chairas e vales interiores; Rururbano (diseminado); Mesotemperado inferior & 2.86 & 0.32 & 8.83 \\ 
  Chairas e vales interiores; Vinedo; Termotemperado & 2.86 & 0.90 & 3.17 \\ 
  Serras; Bosque; Mesotemperado inferior & 2.86 & 4.67 & 0.61 \\ 
  Chairas e vales interiores; Agrosistema extensivo; Termotemperado & 1.43 & 0.66 & 2.17 \\ 
  Chairas e vales interiores; Agrosistema intensivo (superficie de cultivo); Termotemperado & 1.43 & 0.03 & 49.04 \\ 
  Chairas e vales interiores; NoData; Mesotemperado inferior & 1.43 & 0.00 & 6502.69 \\ 
  Chairas e vales interiores; Urbano; Mesotemperado inferior & 1.43 & 0.23 & 6.15 \\ 
  Serras; Agrosistema intensivo (plantacion forestal); Supra e orotemperado & 1.43 & 1.55 & 0.92 \\ 
   \hline
\end{tabular}
\end{table}
% latex table generated in R 3.2.5 by xtable 1.8-0 package
% Mon May  9 13:23:07 2016
\begin{table}[p]
\centering
\caption{Frecuencia de aparición de valores patrimoniais identificados na participación pública e frecuencia de tipos asociados Serras Surorientais} 
\label{vsixotpat9}
\begin{tabular}{lrrr}
  \hline
Tipo de paisaxe & F.Aparic (\%) & F.Tipo (\%) & Ratio \\ 
  \hline
Canons; Vinedo; category 0 & 20.69 & 0.03 & 651.79 \\ 
  Serras; Agrosistema extensivo; Mesotemperado inferior & 20.69 & 4.79 & 4.32 \\ 
  Serras; Agrosistema extensivo; Supra e orotemperado & 10.34 & 2.94 & 3.51 \\ 
  Serras; Matogueira e rochedo; Mesotemperado inferior & 10.34 & 2.75 & 3.76 \\ 
  Serras; Agrosistema extensivo; Mesotemperado superior & 6.90 & 6.24 & 1.11 \\ 
  Serras; Matogueira e rochedo; Supra e orotemperado & 6.90 & 47.28 & 0.15 \\ 
  Serras; Urbano; category 0 & 6.90 & 0.01 & 1124.54 \\ 
  Canons; Agrosistema intensivo (plantacion forestal); category 0 & 3.45 & 0.26 & 13.25 \\ 
  Canons; Bosque; Mesotemperado inferior & 3.45 & 1.00 & 3.43 \\ 
  Canons; Vinedo; Mesotemperado inferior & 3.45 & 0.07 & 52.39 \\ 
  Serras; Agrosistema intensivo (mosaico agroforestal); Mesotemperado superior & 3.45 & 0.81 & 4.27 \\ 
  Serras; Bosque; Mesotemperado inferior & 3.45 & 3.06 & 1.13 \\ 
   \hline
\end{tabular}
\end{table}
% latex table generated in R 3.2.5 by xtable 1.8-0 package
% Mon May  9 13:23:07 2016
\begin{table}[p]
\centering
\caption{Frecuencia de aparición de valores patrimoniais identificados na participación pública e frecuencia de tipos asociados Galicia Setentrional} 
\label{vsixotpat10}
\begin{tabular}{lrrr}
  \hline
Tipo de paisaxe & F.Aparic (\%) & F.Tipo (\%) & Ratio \\ 
  \hline
Vales sublitorais; Agrosistema intensivo (mosaico agroforestal); Mesotemperado inferior & 27.27 & 13.63 & 2.00 \\ 
  Litoral Cantabro-Atlantico; Agrosistema intensivo (mosaico agroforestal); Termotemperado & 18.18 & 5.53 & 3.29 \\ 
  Vales sublitorais; Agrosistema intensivo (plantacion forestal); Mesotemperado inferior & 18.18 & 17.31 & 1.05 \\ 
  Serras; Turbeira; Mesotemperado superior & 13.64 & 12.44 & 1.10 \\ 
  Litoral Cantabro-Atlantico; Agrosistema intensivo (plantacion forestal); Termotemperado & 4.55 & 7.61 & 0.60 \\ 
  Serras; Agrosistema intensivo (mosaico agroforestal); Mesotemperado superior & 4.55 & 3.70 & 1.23 \\ 
  Serras; Agrosistema intensivo (plantacion forestal); Mesotemperado superior & 4.55 & 4.68 & 0.97 \\ 
  Serras; Bosque; Mesotemperado inferior & 4.55 & 0.22 & 20.73 \\ 
  Vales sublitorais; Urbano; Mesotemperado inferior & 4.55 & 0.22 & 20.59 \\ 
   \hline
\end{tabular}
\end{table}
% latex table generated in R 3.2.5 by xtable 1.8-0 package
% Mon May  9 13:23:07 2016
\begin{table}[p]
\centering
\caption{Frecuencia de aparición de valores patrimoniais identificados na participación pública e frecuencia de tipos asociados Chairas e Fosas Occidentais} 
\label{vsixotpat11}
\begin{tabular}{lrrr}
  \hline
Tipo de paisaxe & F.Aparic (\%) & F.Tipo (\%) & Ratio \\ 
  \hline
Vales sublitorais; Agrosistema intensivo (mosaico agroforestal); Mesotemperado inferior & 30.77 & 36.05 & 0.85 \\ 
  Vales sublitorais; Agrosistema intensivo (plantacion forestal); Mesotemperado inferior & 23.08 & 15.20 & 1.52 \\ 
  Litoral Cantabro-Atlantico; Agrosistema intensivo (mosaico agroforestal); Termotemperado & 15.38 & 9.91 & 1.55 \\ 
  category 0; Conxunto Historico; category 0 & 7.69 & 0.12 & 63.17 \\ 
  Litoral Cantabro-Atlantico; Rururbano (diseminado); Termotemperado & 5.13 & 0.53 & 9.75 \\ 
  Vales sublitorais; Agrosistema intensivo (superficie de cultivo); Termotemperado & 5.13 & 0.96 & 5.34 \\ 
  Litoral Cantabro-Atlantico; Agrosistema intensivo (superficie de cultivo); Termotemperado & 2.56 & 0.90 & 2.86 \\ 
  Vales sublitorais; Agrosistema extensivo; Mesotemperado inferior & 2.56 & 0.95 & 2.70 \\ 
  Vales sublitorais; Agrosistema intensivo (superficie de cultivo); Mesotemperado inferior & 2.56 & 4.84 & 0.53 \\ 
  Vales sublitorais; Rururbano (diseminado); Termotemperado & 2.56 & 0.19 & 13.28 \\ 
  Vales sublitorais; Urbano; Termotemperado & 2.56 & 0.23 & 11.26 \\ 
   \hline
\end{tabular}
\end{table}
% latex table generated in R 3.2.5 by xtable 1.8-0 package
% Mon May  9 13:23:07 2016
\begin{table}[p]
\centering
\caption{Frecuencia de aparición de valores patrimoniais identificados na participación pública e frecuencia de tipos asociados Rías Baixas} 
\label{vsixotpat12}
\begin{tabular}{lrrr}
  \hline
Tipo de paisaxe & F.Aparic (\%) & F.Tipo (\%) & Ratio \\ 
  \hline
Serras; Matogueira e rochedo; Mesotemperado superior & 19.35 & 6.70 & 2.89 \\ 
  Vales sublitorais; Agrosistema intensivo (plantacion forestal); Termotemperado & 12.90 & 16.41 & 0.79 \\ 
  Litoral Cantabro-Atlantico; Agrosistema intensivo (mosaico agroforestal); Termotemperado & 9.68 & 10.49 & 0.92 \\ 
  Litoral Cantabro-Atlantico; Agrosistema intensivo (plantacion forestal); Termotemperado & 9.68 & 10.20 & 0.95 \\ 
  Vales sublitorais; Matogueira e rochedo; Mesotemperado inferior & 9.68 & 4.71 & 2.06 \\ 
  Litoral Cantabro-Atlantico; Matogueira e rochedo; Termotemperado & 6.45 & 1.73 & 3.72 \\ 
  Litoral Cantabro-Atlantico; Rururbano (diseminado); Termotemperado & 6.45 & 7.10 & 0.91 \\ 
  category 0; Praias e cantis; category 0 & 3.23 & 0.56 & 5.72 \\ 
  Litoral Cantabro-Atlantico; Bosque; Termotemperado & 3.23 & 0.07 & 45.12 \\ 
  Litoral Cantabro-Atlantico; Vinedo; Termotemperado & 3.23 & 1.45 & 2.22 \\ 
  Serras; Agrosistema intensivo (plantacion forestal); Mesotemperado inferior & 3.23 & 2.82 & 1.15 \\ 
  Serras; Matogueira e rochedo; Mesotemperado inferior & 3.23 & 3.29 & 0.98 \\ 
  Vales sublitorais; Agrosistema intensivo (mosaico agroforestal); Mesotemperado inferior & 3.23 & 4.05 & 0.80 \\ 
  Vales sublitorais; Matogueira e rochedo; Termotemperado & 3.23 & 3.00 & 1.07 \\ 
  Vales sublitorais; Urbano; Termotemperado & 3.23 & 0.55 & 5.89 \\ 
   \hline
\end{tabular}
\end{table}
