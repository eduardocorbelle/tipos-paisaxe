% latex table generated in R 3.2.5 by xtable 1.8-0 package
% Mon May  9 13:23:06 2016
\begin{table}[p]
\centering
\caption{Frecuencia de aparición de valores estéticos identificados na participación pública e frecuencia de tipos asociados Golfo Ártabro} 
\label{vsixotest1}
\begin{tabular}{lrrr}
  \hline
Tipo de paisaxe & F.Aparic (\%) & F.Tipo (\%) & Ratio \\ 
  \hline
category 0; Conxunto Historico; category 0 & 13.56 & 0.08 & 164.21 \\ 
  Litoral Cantabro-Atlantico; Agrosistema intensivo (mosaico agroforestal); Termotemperado & 11.86 & 18.64 & 0.64 \\ 
  Vales sublitorais; Agrosistema intensivo (mosaico agroforestal); Mesotemperado inferior & 11.86 & 21.75 & 0.55 \\ 
  Canons; Bosque; Mesotemperado inferior & 8.47 & 1.56 & 5.43 \\ 
  Vales sublitorais; Bosque; Mesotemperado inferior & 8.47 & 0.78 & 10.81 \\ 
  Litoral Cantabro-Atlantico; Agrosistema extensivo; Termotemperado & 5.08 & 1.02 & 5.00 \\ 
  Litoral Cantabro-Atlantico; Agrosistema intensivo (plantacion forestal); Termotemperado & 5.08 & 5.79 & 0.88 \\ 
  Litoral Cantabro-Atlantico; Matogueira e rochedo; Termotemperado & 5.08 & 1.19 & 4.26 \\ 
  Litoral Cantabro-Atlantico; Urbano; Termotemperado & 5.08 & 4.93 & 1.03 \\ 
  Vales sublitorais; Agrosistema intensivo (plantacion forestal); Mesotemperado inferior & 5.08 & 13.21 & 0.38 \\ 
  Canons; Matogueira e rochedo; Mesotemperado inferior & 3.39 & 0.08 & 41.37 \\ 
  Litoral Cantabro-Atlantico; Rururbano (diseminado); Termotemperado & 3.39 & 7.02 & 0.48 \\ 
  Vales sublitorais; Matogueira e rochedo; Mesotemperado inferior & 3.39 & 0.64 & 5.32 \\ 
  category 0; Lamina de auga; category 0 & 1.69 & 0.55 & 3.09 \\ 
  Serras; Agrosistema intensivo (plantacion forestal); Mesotemperado inferior & 1.69 & 0.51 & 3.34 \\ 
  Serras; Turbeira; Mesotemperado superior & 1.69 & 1.53 & 1.11 \\ 
  Vales sublitorais; Matogueira e rochedo; Mesotemperado superior & 1.69 & 1.17 & 1.45 \\ 
  Vales sublitorais; Urbano; Termotemperado & 1.69 & 0.22 & 7.84 \\ 
   & 1.69 &  &  \\ 
   \hline
\end{tabular}
\end{table}
% latex table generated in R 3.2.5 by xtable 1.8-0 package
% Mon May  9 13:23:06 2016
\begin{table}[p]
\centering
\caption{Frecuencia de aparición de valores estéticos identificados na participación pública e frecuencia de tipos asociados A Mariña - Baixo Eo} 
\label{vsixotest2}
\begin{tabular}{lrrr}
  \hline
Tipo de paisaxe & F.Aparic (\%) & F.Tipo (\%) & Ratio \\ 
  \hline
Vales sublitorais; Agrosistema intensivo (plantacion forestal); Mesotemperado inferior & 25.58 & 23.68 & 1.08 \\ 
  Litoral Cantabro-Atlantico; Agrosistema intensivo (plantacion forestal); Mesotemperado inferior & 11.63 & 18.89 & 0.62 \\ 
  Litoral Cantabro-Atlantico; Agrosistema intensivo (mosaico agroforestal); Mesotemperado inferior & 6.98 & 3.57 & 1.95 \\ 
  Litoral Cantabro-Atlantico; Agrosistema intensivo (mosaico agroforestal); Termotemperado & 6.98 & 7.20 & 0.97 \\ 
  Litoral Cantabro-Atlantico; Matogueira e rochedo; Mesotemperado inferior & 6.98 & 0.05 & 148.71 \\ 
  category 0; Praias e cantis; category 0 & 4.65 & 0.25 & 18.96 \\ 
  Litoral Cantabro-Atlantico; Rururbano (diseminado); Termotemperado & 4.65 & 2.68 & 1.73 \\ 
  Litoral Cantabro-Atlantico; Urbano; Termotemperado & 4.65 & 1.10 & 4.21 \\ 
  Vales sublitorais; Agrosistema intensivo (mosaico agroforestal); Mesotemperado inferior & 4.65 & 12.86 & 0.36 \\ 
   & 4.65 &  &  \\ 
  category 0; Conxunto Historico; category 0 & 2.33 & 0.10 & 24.21 \\ 
  Litoral Cantabro-Atlantico; Matogueira e rochedo; Termotemperado & 2.33 & 0.43 & 5.42 \\ 
  Serras; Agrosistema intensivo (plantacion forestal); Mesotemperado superior & 2.33 & 6.80 & 0.34 \\ 
  Serras; Bosque; Mesotemperado superior & 2.33 & 0.70 & 3.30 \\ 
  Serras; Matogueira e rochedo; Mesotemperado superior & 2.33 & 1.71 & 1.36 \\ 
  Serras; Turbeira; Mesotemperado superior & 2.33 & 2.64 & 0.88 \\ 
  Vales sublitorais; Agrosistema extensivo; Mesotemperado inferior & 2.33 & 0.98 & 2.37 \\ 
  Vales sublitorais; Bosque; Mesotemperado superior & 2.33 & 1.22 & 1.91 \\ 
   \hline
\end{tabular}
\end{table}
% latex table generated in R 3.2.5 by xtable 1.8-0 package
% Mon May  9 13:23:06 2016
\begin{table}[p]
\centering
\caption{Frecuencia de aparición de valores estéticos identificados na participación pública e frecuencia de tipos asociados Costa Sur - Baixo Miño} 
\label{vsixotest3}
\begin{tabular}{lrrr}
  \hline
Tipo de paisaxe & F.Aparic (\%) & F.Tipo (\%) & Ratio \\ 
  \hline
Serras; Agrosistema intensivo (plantacion forestal); Termotemperado & 14.13 & 3.79 & 3.73 \\ 
  Serras; Matogueira e rochedo; Mesotemperado inferior & 14.13 & 3.84 & 3.68 \\ 
  Vales sublitorais; Agrosistema intensivo (plantacion forestal); Termotemperado & 13.04 & 16.82 & 0.78 \\ 
  category 0; Conxunto Historico; category 0 & 8.70 & 0.22 & 38.90 \\ 
  Litoral Cantabro-Atlantico; Agrosistema intensivo (mosaico agroforestal); Termotemperado & 5.43 & 5.83 & 0.93 \\ 
  Litoral Cantabro-Atlantico; Agrosistema intensivo (plantacion forestal); Termotemperado & 5.43 & 6.69 & 0.81 \\ 
  Serras; Agrosistema intensivo (plantacion forestal); Mesotemperado inferior & 5.43 & 3.38 & 1.61 \\ 
  Serras; Turbeira; Mesotemperado inferior & 5.43 & 0.12 & 45.08 \\ 
  Vales sublitorais; Agrosistema intensivo (mosaico agroforestal); Termotemperado & 5.43 & 7.55 & 0.72 \\ 
  Vales sublitorais; Matogueira e rochedo; Termotemperado & 3.26 & 2.72 & 1.20 \\ 
  Litoral Cantabro-Atlantico; Agrosistema intensivo (superficie de cultivo); Termotemperado & 2.17 & 0.16 & 13.25 \\ 
  Vales sublitorais; Bosque; Termotemperado & 2.17 & 1.44 & 1.51 \\ 
  category 0; Lamina de auga; category 0 & 1.09 & 0.46 & 2.38 \\ 
  category 0; Praias e cantis; category 0 & 1.09 & 0.07 & 15.79 \\ 
  Chairas e vales interiores; Agrosistema intensivo (mosaico agroforestal); Mesotemperado inferior & 1.09 & 0.84 & 1.29 \\ 
  Chairas e vales interiores; Agrosistema intensivo (plantacion forestal); Termotemperado & 1.09 & 6.71 & 0.16 \\ 
  Chairas e vales interiores; Matogueira e rochedo; Mesotemperado inferior & 1.09 & 1.96 & 0.55 \\ 
  Litoral Cantabro-Atlantico; Agrosistema extensivo; Termotemperado & 1.09 & 0.61 & 1.77 \\ 
  Litoral Cantabro-Atlantico; Matogueira e rochedo; Termotemperado & 1.09 & 0.52 & 2.09 \\ 
  Litoral Cantabro-Atlantico; Rururbano (diseminado); Termotemperado & 1.09 & 4.97 & 0.22 \\ 
  Litoral Cantabro-Atlantico; Vinedo; Termotemperado & 1.09 & 0.70 & 1.55 \\ 
  Serras; Agrosistema extensivo; Mesotemperado superior & 1.09 & 0.28 & 3.89 \\ 
  Serras; Matogueira e rochedo; Supra e orotemperado & 1.09 & 0.14 & 7.84 \\ 
  Vales sublitorais; Agrosistema intensivo (superficie de cultivo); Termotemperado & 1.09 & 0.10 & 11.01 \\ 
  Vales sublitorais; Extractivo; Termotemperado & 1.09 & 0.17 & 6.52 \\ 
  Vales sublitorais; Rururbano (diseminado); Termotemperado & 1.09 & 2.82 & 0.39 \\ 
   \hline
\end{tabular}
\end{table}
% latex table generated in R 3.2.5 by xtable 1.8-0 package
% Mon May  9 13:23:06 2016
\begin{table}[p]
\centering
\caption{Frecuencia de aparición de valores estéticos identificados na participación pública e frecuencia de tipos asociados Ribeiras Encaixadas do Miño e do Sil} 
\label{vsixotest4}
\begin{tabular}{lrrr}
  \hline
Tipo de paisaxe & F.Aparic (\%) & F.Tipo (\%) & Ratio \\ 
  \hline
Canons; Bosque; Mesotemperado inferior & 17.59 & 2.67 & 6.60 \\ 
  category 0; Lamina de auga; category 0 & 8.33 & 1.89 & 4.40 \\ 
  Chairas e vales interiores; Bosque; Termotemperado & 8.33 & 3.03 & 2.75 \\ 
  Chairas e vales interiores; Agrosistema intensivo (plantacion forestal); Termotemperado & 4.63 & 6.81 & 0.68 \\ 
  Canons; Agrosistema intensivo (plantacion forestal); category 0 & 3.70 & 1.25 & 2.97 \\ 
  category 0; Conxunto Historico; category 0 & 3.70 & 0.04 & 90.47 \\ 
  Chairas e vales interiores; Agrosistema intensivo (plantacion forestal); category 0 & 3.70 & 1.83 & 2.02 \\ 
  Serras; Matogueira e rochedo; Supra e orotemperado & 3.70 & 9.08 & 0.41 \\ 
  Canons; Vinedo; Mesotemperado inferior & 2.78 & 0.25 & 11.13 \\ 
  Chairas e vales interiores; Agrosistema intensivo (mosaico agroforestal); Mesotemperado inferior & 2.78 & 5.37 & 0.52 \\ 
  Chairas e vales interiores; Agrosistema intensivo (plantacion forestal); Mesotemperado inferior & 2.78 & 4.37 & 0.64 \\ 
  Serras; Bosque; Mesotemperado inferior & 2.78 & 2.32 & 1.20 \\ 
  Serras; Bosque; Supra e orotemperado & 2.78 & 0.47 & 5.85 \\ 
  Serras; Matogueira e rochedo; Mesotemperado superior & 2.78 & 3.10 & 0.90 \\ 
  Canons; Agrosistema intensivo (plantacion forestal); Termotemperado & 1.85 & 1.39 & 1.34 \\ 
  Canons; Bosque; category 0 & 1.85 & 0.39 & 4.70 \\ 
  Canons; Matogueira e rochedo; category 0 & 1.85 & 0.96 & 1.93 \\ 
  Chairas e vales interiores; Matogueira e rochedo; Mesotemperado inferior & 1.85 & 3.61 & 0.51 \\ 
  Chairas e vales interiores; Rururbano (diseminado); Termotemperado & 1.85 & 0.91 & 2.04 \\ 
  Chairas e vales interiores; Urbano; Termotemperado & 1.85 & 1.01 & 1.84 \\ 
  Serras; Agrosistema extensivo; Mesotemperado inferior & 1.85 & 1.98 & 0.94 \\ 
   \hline
\end{tabular}
\end{table}
% latex table generated in R 3.2.5 by xtable 1.8-0 package
% Mon May  9 13:23:06 2016
\begin{table}[p]
\centering
\caption{Frecuencia de aparición de valores estéticos identificados na participación pública e frecuencia de tipos asociados Serras Orientais} 
\label{vsixotest5}
\begin{tabular}{lrrr}
  \hline
Tipo de paisaxe & F.Aparic (\%) & F.Tipo (\%) & Ratio \\ 
  \hline
Serras; Bosque; Supra e orotemperado & 20.51 & 7.67 & 2.68 \\ 
  Serras; Matogueira e rochedo; Supra e orotemperado & 14.10 & 15.61 & 0.90 \\ 
  Serras; Agrosistema extensivo; Supra e orotemperado & 11.54 & 8.91 & 1.29 \\ 
  Vales sublitorais; Bosque; Mesotemperado inferior & 8.97 & 3.83 & 2.34 \\ 
  Serras; Agrosistema intensivo (mosaico agroforestal); Mesotemperado superior & 6.41 & 3.95 & 1.62 \\ 
  Serras; Agrosistema extensivo; Mesotemperado superior & 5.13 & 7.30 & 0.70 \\ 
  Serras; Matogueira e rochedo; Mesotemperado superior & 5.13 & 5.41 & 0.95 \\ 
  Serras; Agrosistema intensivo (mosaico agroforestal); Supra e orotemperado & 3.85 & 5.74 & 0.67 \\ 
  Vales sublitorais; Agrosistema extensivo; Mesotemperado inferior & 3.85 & 1.41 & 2.73 \\ 
  Vales sublitorais; Bosque; Mesotemperado superior & 3.85 & 4.16 & 0.92 \\ 
  Canons; Bosque; Mesotemperado inferior & 2.56 & 0.44 & 5.88 \\ 
  category 0; Lamina de auga; category 0 & 2.56 & 0.16 & 15.93 \\ 
  Serras; Agrosistema intensivo (plantacion forestal); Supra e orotemperado & 2.56 & 6.86 & 0.37 \\ 
  Serras; Bosque; Mesotemperado superior & 2.56 & 3.65 & 0.70 \\ 
  Vales sublitorais; Agrosistema extensivo; Mesotemperado superior & 2.56 & 5.11 & 0.50 \\ 
  Chairas e vales interiores; Agrosistema extensivo; Mesotemperado superior & 1.28 & 0.25 & 5.19 \\ 
  Chairas e vales interiores; Bosque; category 0 & 1.28 & 0.19 & 6.61 \\ 
  Serras; Bosque; Mesotemperado inferior & 1.28 & 0.55 & 2.34 \\ 
   \hline
\end{tabular}
\end{table}
% latex table generated in R 3.2.5 by xtable 1.8-0 package
% Mon May  9 13:23:06 2016
\begin{table}[p]
\centering
\caption{Frecuencia de aparición de valores estéticos identificados na participación pública e frecuencia de tipos asociados Chairas e Fosas Luguesas} 
\label{vsixotest6}
\begin{tabular}{lrrr}
  \hline
Tipo de paisaxe & F.Aparic (\%) & F.Tipo (\%) & Ratio \\ 
  \hline
Chairas e vales interiores; Agrosistema extensivo; Mesotemperado inferior & 11.11 & 6.80 & 1.63 \\ 
  Chairas e vales interiores; Agrosistema extensivo; Mesotemperado superior & 11.11 & 9.68 & 1.15 \\ 
  Chairas e vales interiores; Agrosistema intensivo (mosaico agroforestal); Mesotemperado superior & 11.11 & 20.11 & 0.55 \\ 
  Chairas e vales interiores; Bosque; Mesotemperado inferior & 11.11 & 1.92 & 5.79 \\ 
  category 0; Conxunto Historico; category 0 & 7.94 & 0.02 & 398.19 \\ 
  Chairas e vales interiores; Urbano; Mesotemperado superior & 7.94 & 0.45 & 17.60 \\ 
  Serras; Agrosistema intensivo (plantacion forestal); Supra e orotemperado & 6.35 & 0.59 & 10.79 \\ 
  Chairas e vales interiores; Agrosistema intensivo (superficie de cultivo); Mesotemperado superior & 4.76 & 5.78 & 0.82 \\ 
  Chairas e vales interiores; Bosque; Mesotemperado superior & 4.76 & 2.02 & 2.36 \\ 
  Chairas e vales interiores; Agrosistema intensivo (plantacion forestal); Mesotemperado inferior & 3.17 & 3.31 & 0.96 \\ 
  Serras; Agrosistema intensivo (mosaico agroforestal); Mesotemperado superior & 3.17 & 9.79 & 0.32 \\ 
  Serras; Turbeira; Mesotemperado superior & 3.17 & 1.08 & 2.93 \\ 
  category 0; Lamina de auga; category 0 & 1.59 & 0.09 & 16.71 \\ 
  Chairas e vales interiores; Agrosistema intensivo (mosaico agroforestal); Mesotemperado inferior & 1.59 & 9.44 & 0.17 \\ 
  Chairas e vales interiores; Agrosistema intensivo (plantacion forestal); Mesotemperado superior & 1.59 & 3.97 & 0.40 \\ 
  Chairas e vales interiores; Urbano; Mesotemperado inferior & 1.59 & 0.19 & 8.49 \\ 
  Serras; Agrosistema extensivo; Mesotemperado superior & 1.59 & 5.94 & 0.27 \\ 
  Serras; Agrosistema extensivo; Supra e orotemperado & 1.59 & 0.37 & 4.29 \\ 
  Serras; Agrosistema intensivo (plantacion forestal); Mesotemperado superior & 1.59 & 2.29 & 0.69 \\ 
  Serras; Agrosistema intensivo (superficie de cultivo); Mesotemperado superior & 1.59 & 3.04 & 0.52 \\ 
  Serras; Matogueira e rochedo; Supra e orotemperado & 1.59 & 0.48 & 3.31 \\ 
   \hline
\end{tabular}
\end{table}
% latex table generated in R 3.2.5 by xtable 1.8-0 package
% Mon May  9 13:23:06 2016
\begin{table}[p]
\centering
\caption{Frecuencia de aparición de valores estéticos identificados na participación pública e frecuencia de tipos asociados Galicia Central} 
\label{vsixotest7}
\begin{tabular}{lrrr}
  \hline
Tipo de paisaxe & F.Aparic (\%) & F.Tipo (\%) & Ratio \\ 
  \hline
Vales sublitorais; Agrosistema intensivo (mosaico agroforestal); Mesotemperado inferior & 26.94 & 43.06 & 0.63 \\ 
  Vales sublitorais; Agrosistema extensivo; Mesotemperado inferior & 9.84 & 5.21 & 1.89 \\ 
  Vales sublitorais; Urbano; Mesotemperado inferior & 6.74 & 0.64 & 10.49 \\ 
  Vales sublitorais; Agrosistema intensivo (superficie de cultivo); Mesotemperado inferior & 6.22 & 3.18 & 1.96 \\ 
  Serras; Agrosistema extensivo; Mesotemperado superior & 5.70 & 4.47 & 1.27 \\ 
  category 0; Lamina de auga; category 0 & 4.15 & 0.30 & 13.90 \\ 
  Serras; Matogueira e rochedo; Mesotemperado superior & 3.63 & 6.90 & 0.53 \\ 
  Vales sublitorais; Agrosistema intensivo (plantacion forestal); Mesotemperado inferior & 3.63 & 5.84 & 0.62 \\ 
  category 0; Conxunto Historico; category 0 & 3.11 & 0.02 & 176.99 \\ 
  Vales sublitorais; Bosque; Mesotemperado inferior & 2.59 & 0.80 & 3.24 \\ 
  Serras; Agrosistema intensivo (mosaico agroforestal); Mesotemperado superior & 2.07 & 3.93 & 0.53 \\ 
  Vales sublitorais; Matogueira e rochedo; Mesotemperado inferior & 2.07 & 1.65 & 1.26 \\ 
  Chairas e vales interiores; Urbano; Termotemperado & 1.55 & 0.03 & 61.16 \\ 
  Serras; Agrosistema extensivo; Mesotemperado inferior & 1.55 & 1.17 & 1.32 \\ 
  Serras; Agrosistema intensivo (plantacion forestal); Supra e orotemperado & 1.55 & 0.54 & 2.90 \\ 
  Serras; Agrosistema intensivo (superficie de cultivo); Mesotemperado superior & 1.55 & 1.44 & 1.08 \\ 
  Serras; Bosque; Mesotemperado inferior & 1.55 & 0.45 & 3.44 \\ 
  Vales sublitorais; Rururbano (diseminado); Mesotemperado inferior & 1.55 & 0.23 & 6.81 \\ 
  Chairas e vales interiores; Bosque; Mesotemperado inferior & 1.04 & 0.68 & 1.53 \\ 
  Serras; Matogueira e rochedo; Mesotemperado inferior & 1.04 & 0.62 & 1.67 \\ 
  Vales sublitorais; Agrosistema extensivo; Termotemperado & 1.04 & 0.29 & 3.60 \\ 
  Vales sublitorais; Agrosistema intensivo (mosaico agroforestal); Mesotemperado superior & 1.04 & 0.57 & 1.80 \\ 
  Vales sublitorais; Agrosistema intensivo (plantacion forestal); Termotemperado & 1.04 & 1.80 & 0.58 \\ 
  Vales sublitorais; Bosque; Termotemperado & 1.04 & 0.32 & 3.20 \\ 
  Vales sublitorais; Matogueira e rochedo; Termotemperado & 1.04 & 0.43 & 2.42 \\ 
  Vales sublitorais; Rururbano (diseminado); Termotemperado & 1.04 & 0.34 & 3.08 \\ 
   \hline
\end{tabular}
\end{table}
% latex table generated in R 3.2.5 by xtable 1.8-0 package
% Mon May  9 13:23:07 2016
\begin{table}[p]
\centering
\caption{Frecuencia de aparición de valores estéticos identificados na participación pública e frecuencia de tipos asociados Chairas, Fosas e Serras Ourensás} 
\label{vsixotest8}
\begin{tabular}{lrrr}
  \hline
Tipo de paisaxe & F.Aparic (\%) & F.Tipo (\%) & Ratio \\ 
  \hline
Serras; Matogueira e rochedo; Supra e orotemperado & 13.73 & 14.06 & 0.98 \\ 
  Serras; Matogueira e rochedo; Mesotemperado superior & 10.78 & 10.51 & 1.03 \\ 
  Serras; Agrosistema extensivo; Mesotemperado inferior & 6.86 & 5.23 & 1.31 \\ 
  Chairas e vales interiores; Agrosistema extensivo; Mesotemperado inferior & 5.88 & 7.38 & 0.80 \\ 
  Chairas e vales interiores; Matogueira e rochedo; Mesotemperado inferior & 5.88 & 4.82 & 1.22 \\ 
  Serras; Matogueira e rochedo; Mesotemperado inferior & 5.88 & 5.15 & 1.14 \\ 
  Chairas e vales interiores; Bosque; Mesotemperado inferior & 4.90 & 6.18 & 0.79 \\ 
  Serras; Bosque; Mesotemperado superior & 4.90 & 2.22 & 2.21 \\ 
  Chairas e vales interiores; Agrosistema intensivo (plantacion forestal); Termotemperado & 3.92 & 3.78 & 1.04 \\ 
  Chairas e vales interiores; Agrosistema intensivo (superficie de cultivo); Mesotemperado inferior & 3.92 & 7.53 & 0.52 \\ 
  Chairas e vales interiores; Matogueira e rochedo; Termotemperado & 2.94 & 1.80 & 1.64 \\ 
  Serras; Agrosistema intensivo (mosaico agroforestal); Supra e orotemperado & 2.94 & 0.26 & 11.15 \\ 
  category 0; Conxunto Historico; category 0 & 1.96 & 0.01 & 193.61 \\ 
  Chairas e vales interiores; Agrosistema extensivo; Termotemperado & 1.96 & 0.66 & 2.98 \\ 
  Chairas e vales interiores; Agrosistema intensivo (plantacion forestal); Mesotemperado inferior & 1.96 & 3.30 & 0.59 \\ 
  Chairas e vales interiores; Rururbano (diseminado); Mesotemperado inferior & 1.96 & 0.32 & 6.06 \\ 
  Chairas e vales interiores; Urbano; Mesotemperado inferior & 1.96 & 0.23 & 8.44 \\ 
  Serras; Agrosistema extensivo; Mesotemperado superior & 1.96 & 2.44 & 0.80 \\ 
  Serras; Agrosistema extensivo; Supra e orotemperado & 1.96 & 1.96 & 1.00 \\ 
  Serras; Agrosistema intensivo (plantacion forestal); Mesotemperado superior & 1.96 & 1.41 & 1.39 \\ 
  Serras; Agrosistema intensivo (superficie de cultivo); Supra e orotemperado & 1.96 & 0.46 & 4.23 \\ 
  Serras; Bosque; Supra e orotemperado & 1.96 & 1.18 & 1.66 \\ 
   \hline
\end{tabular}
\end{table}
% latex table generated in R 3.2.5 by xtable 1.8-0 package
% Mon May  9 13:23:07 2016
\begin{table}[p]
\centering
\caption{Frecuencia de aparición de valores estéticos identificados na participación pública e frecuencia de tipos asociados Serras Surorientais} 
\label{vsixotest9}
\begin{tabular}{lrrr}
  \hline
Tipo de paisaxe & F.Aparic (\%) & F.Tipo (\%) & Ratio \\ 
  \hline
Serras; Matogueira e rochedo; Supra e orotemperado & 32.32 & 47.28 & 0.68 \\ 
  Serras; Matogueira e rochedo; Mesotemperado inferior & 12.12 & 2.75 & 4.41 \\ 
  Serras; Agrosistema extensivo; Mesotemperado inferior & 10.10 & 4.79 & 2.11 \\ 
  Serras; Bosque; Mesotemperado superior & 8.08 & 5.61 & 1.44 \\ 
  Canons; Vinedo; category 0 & 7.07 & 0.03 & 222.75 \\ 
  Serras; Agrosistema intensivo (plantacion forestal); Supra e orotemperado & 6.06 & 10.23 & 0.59 \\ 
  category 0; Lamina de auga; category 0 & 4.04 & 1.38 & 2.92 \\ 
  Canons; Agrosistema intensivo (plantacion forestal); category 0 & 3.03 & 0.26 & 11.64 \\ 
  Serras; Agrosistema extensivo; Mesotemperado superior & 3.03 & 6.24 & 0.49 \\ 
  Serras; Bosque; Mesotemperado inferior & 3.03 & 3.06 & 0.99 \\ 
  Serras; Bosque; Supra e orotemperado & 2.02 & 1.26 & 1.61 \\ 
  Serras; Urbano; category 0 & 2.02 & 0.01 & 329.41 \\ 
  Canons; Bosque; Mesotemperado inferior & 1.01 & 1.00 & 1.01 \\ 
  Canons; Vinedo; Mesotemperado inferior & 1.01 & 0.07 & 15.35 \\ 
  Serras; Agrosistema extensivo; Supra e orotemperado & 1.01 & 2.94 & 0.34 \\ 
  Serras; Agrosistema intensivo (mosaico agroforestal); Mesotemperado inferior & 1.01 & 0.76 & 1.33 \\ 
  Serras; Agrosistema intensivo (mosaico agroforestal); Mesotemperado superior & 1.01 & 0.81 & 1.25 \\ 
  Serras; Agrosistema intensivo (superficie de cultivo); Mesotemperado superior & 1.01 & 0.72 & 1.41 \\ 
  Serras; Urbano; Mesotemperado inferior & 1.01 & 0.02 & 40.71 \\ 
   \hline
\end{tabular}
\end{table}
% latex table generated in R 3.2.5 by xtable 1.8-0 package
% Mon May  9 13:23:07 2016
\begin{table}[p]
\centering
\caption{Frecuencia de aparición de valores estéticos identificados na participación pública e frecuencia de tipos asociados Galicia Setentrional} 
\label{vsixotest10}
\begin{tabular}{lrrr}
  \hline
Tipo de paisaxe & F.Aparic (\%) & F.Tipo (\%) & Ratio \\ 
  \hline
Serras; Turbeira; Mesotemperado superior & 22.22 & 12.44 & 1.79 \\ 
  Vales sublitorais; Agrosistema intensivo (plantacion forestal); Mesotemperado inferior & 13.89 & 17.31 & 0.80 \\ 
  Vales sublitorais; Agrosistema intensivo (mosaico agroforestal); Mesotemperado inferior & 8.33 & 13.63 & 0.61 \\ 
  Litoral Cantabro-Atlantico; Agrosistema intensivo (mosaico agroforestal); Termotemperado & 6.94 & 5.53 & 1.26 \\ 
  Serras; Matogueira e rochedo; Mesotemperado superior & 5.56 & 6.79 & 0.82 \\ 
  Vales sublitorais; Bosque; Mesotemperado inferior & 5.56 & 1.49 & 3.72 \\ 
  Litoral Cantabro-Atlantico; Agrosistema intensivo (plantacion forestal); Mesotemperado inferior & 4.17 & 2.79 & 1.49 \\ 
  Litoral Cantabro-Atlantico; Agrosistema intensivo (plantacion forestal); Termotemperado & 4.17 & 7.61 & 0.55 \\ 
  Serras; Agrosistema intensivo (plantacion forestal); Mesotemperado superior & 4.17 & 4.68 & 0.89 \\ 
  category 0; Lamina de auga; category 0 & 2.78 & 0.89 & 3.12 \\ 
  category 0; Praias e cantis; category 0 & 2.78 & 0.94 & 2.94 \\ 
  Serras; Agrosistema extensivo; Mesotemperado superior & 2.78 & 2.70 & 1.03 \\ 
  Serras; Agrosistema intensivo (mosaico agroforestal); Mesotemperado superior & 2.78 & 3.70 & 0.75 \\ 
  Serras; Bosque; Mesotemperado inferior & 2.78 & 0.22 & 12.67 \\ 
  Vales sublitorais; Matogueira e rochedo; Mesotemperado inferior & 2.78 & 0.62 & 4.45 \\ 
  Litoral Cantabro-Atlantico; Agrosistema intensivo (superficie de cultivo); Termotemperado & 1.39 & 0.47 & 2.93 \\ 
  Litoral Cantabro-Atlantico; Matogueira e rochedo; Termotemperado & 1.39 & 1.77 & 0.78 \\ 
  Litoral Cantabro-Atlantico; Urbano; Mesotemperado inferior & 1.39 & 0.03 & 43.75 \\ 
  Vales sublitorais; Agrosistema intensivo (mosaico agroforestal); Mesotemperado superior & 1.39 & 0.82 & 1.69 \\ 
  Vales sublitorais; Rururbano (diseminado); Mesotemperado inferior & 1.39 & 0.08 & 16.98 \\ 
   & 1.39 &  &  \\ 
   \hline
\end{tabular}
\end{table}
% latex table generated in R 3.2.5 by xtable 1.8-0 package
% Mon May  9 13:23:07 2016
\begin{table}[p]
\centering
\caption{Frecuencia de aparición de valores estéticos identificados na participación pública e frecuencia de tipos asociados Chairas e Fosas Occidentais} 
\label{vsixotest11}
\begin{tabular}{lrrr}
  \hline
Tipo de paisaxe & F.Aparic (\%) & F.Tipo (\%) & Ratio \\ 
  \hline
Vales sublitorais; Agrosistema intensivo (plantacion forestal); Mesotemperado inferior & 27.50 & 15.20 & 1.81 \\ 
  Vales sublitorais; Agrosistema intensivo (mosaico agroforestal); Mesotemperado inferior & 20.00 & 36.05 & 0.55 \\ 
  Litoral Cantabro-Atlantico; Agrosistema intensivo (plantacion forestal); Termotemperado & 9.17 & 7.60 & 1.21 \\ 
  category 0; Praias e cantis; category 0 & 8.33 & 0.54 & 15.50 \\ 
  Litoral Cantabro-Atlantico; Matogueira e rochedo; Termotemperado & 7.50 & 4.58 & 1.64 \\ 
  category 0; Conxunto Historico; category 0 & 5.83 & 0.12 & 47.91 \\ 
  Litoral Cantabro-Atlantico; Urbano; Termotemperado & 4.17 & 0.66 & 6.33 \\ 
  Litoral Cantabro-Atlantico; Agrosistema intensivo (mosaico agroforestal); Termotemperado & 2.50 & 9.91 & 0.25 \\ 
  Vales sublitorais; Agrosistema extensivo; Mesotemperado inferior & 2.50 & 0.95 & 2.63 \\ 
  Vales sublitorais; Matogueira e rochedo; Mesotemperado inferior & 2.50 & 4.77 & 0.52 \\ 
  Vales sublitorais; Agrosistema intensivo (plantacion forestal); Termotemperado & 1.67 & 3.44 & 0.48 \\ 
  Vales sublitorais; Agrosistema intensivo (superficie de cultivo); Mesotemperado inferior & 1.67 & 4.84 & 0.34 \\ 
  Vales sublitorais; Matogueira e rochedo; Mesotemperado superior & 1.67 & 2.22 & 0.75 \\ 
   \hline
\end{tabular}
\end{table}
% latex table generated in R 3.2.5 by xtable 1.8-0 package
% Mon May  9 13:23:07 2016
\begin{table}[p]
\centering
\caption{Frecuencia de aparición de valores estéticos identificados na participación pública e frecuencia de tipos asociados Rías Baixas} 
\label{vsixotest12}
\begin{tabular}{lrrr}
  \hline
Tipo de paisaxe & F.Aparic (\%) & F.Tipo (\%) & Ratio \\ 
  \hline
Serras; Matogueira e rochedo; Mesotemperado superior & 18.26 & 6.70 & 2.73 \\ 
  Vales sublitorais; Agrosistema intensivo (plantacion forestal); Termotemperado & 18.26 & 16.41 & 1.11 \\ 
  Serras; Matogueira e rochedo; Mesotemperado inferior & 17.39 & 3.29 & 5.28 \\ 
  Vales sublitorais; Matogueira e rochedo; Mesotemperado inferior & 10.43 & 4.71 & 2.22 \\ 
  Serras; Agrosistema intensivo (plantacion forestal); Mesotemperado inferior & 5.22 & 2.82 & 1.85 \\ 
  Litoral Cantabro-Atlantico; Agrosistema intensivo (plantacion forestal); Termotemperado & 4.35 & 10.20 & 0.43 \\ 
  Litoral Cantabro-Atlantico; Matogueira e rochedo; Termotemperado & 3.48 & 1.73 & 2.01 \\ 
  Litoral Cantabro-Atlantico; Agrosistema intensivo (mosaico agroforestal); Termotemperado & 2.61 & 10.49 & 0.25 \\ 
  Litoral Cantabro-Atlantico; Rururbano (diseminado); Termotemperado & 2.61 & 7.10 & 0.37 \\ 
  Litoral Cantabro-Atlantico; Urbano; Termotemperado & 2.61 & 3.29 & 0.79 \\ 
  Vales sublitorais; Agrosistema intensivo (mosaico agroforestal); Termotemperado & 2.61 & 7.63 & 0.34 \\ 
  category 0; Praias e cantis; category 0 & 1.74 & 0.56 & 3.08 \\ 
  Vales sublitorais; Agrosistema intensivo (mosaico agroforestal); Mesotemperado inferior & 1.74 & 4.05 & 0.43 \\ 
   \hline
\end{tabular}
\end{table}
