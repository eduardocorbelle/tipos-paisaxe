% latex table generated in R 3.2.5 by xtable 1.8-0 package
% Mon May  9 13:23:06 2016
\begin{table}[p]
\centering
\caption{Frecuencia de aparición dos Camiños de Santiago (área de influencia de 500 m a ambos lados) e frecuencia de tipos asociados Golfo Ártabro} 
\label{vcamino1}
\begin{tabular}{lrrr}
  \hline
Tipo de paisaxe & F.Aparic (\%) & F.Tipo (\%) & Ratio \\ 
  \hline
Litoral Cantabro-Atlantico; Agrosistema intensivo (mosaico agroforestal); Termotemperado & 25.40 & 18.64 & 1.36 \\ 
  Litoral Cantabro-Atlantico; Urbano; Termotemperado & 21.42 & 4.93 & 4.34 \\ 
  Vales sublitorais; Agrosistema intensivo (mosaico agroforestal); Mesotemperado inferior & 14.09 & 21.75 & 0.65 \\ 
  Litoral Cantabro-Atlantico; Rururbano (diseminado); Termotemperado & 13.37 & 7.02 & 1.90 \\ 
  Litoral Cantabro-Atlantico; Agrosistema intensivo (plantacion forestal); Termotemperado & 6.45 & 5.79 & 1.11 \\ 
  Vales sublitorais; Agrosistema intensivo (plantacion forestal); Mesotemperado inferior & 4.94 & 13.21 & 0.37 \\ 
  Vales sublitorais; Agrosistema intensivo (mosaico agroforestal); Termotemperado & 3.99 & 1.58 & 2.54 \\ 
  Vales sublitorais; Agrosistema intensivo (plantacion forestal); Termotemperado & 2.20 & 0.86 & 2.57 \\ 
  Litoral Cantabro-Atlantico; Agrosistema intensivo (superficie de cultivo); Termotemperado & 2.08 & 1.02 & 2.03 \\ 
  Vales sublitorais; Rururbano (diseminado); Termotemperado & 1.08 & 0.26 & 4.14 \\ 
   \hline
\end{tabular}
\end{table}
% latex table generated in R 3.2.5 by xtable 1.8-0 package
% Mon May  9 13:23:06 2016
\begin{table}[p]
\centering
\caption{Frecuencia de aparición dos Camiños de Santiago (área de influencia de 500 m a ambos lados) e frecuencia de tipos asociados A Mariña - Baixo Eo} 
\label{vcamino2}
\begin{tabular}{lrrr}
  \hline
Tipo de paisaxe & F.Aparic (\%) & F.Tipo (\%) & Ratio \\ 
  \hline
Vales sublitorais; Agrosistema intensivo (mosaico agroforestal); Mesotemperado inferior & 40.14 & 12.86 & 3.12 \\ 
  Vales sublitorais; Agrosistema intensivo (plantacion forestal); Mesotemperado inferior & 27.38 & 23.68 & 1.16 \\ 
  Vales sublitorais; Agrosistema intensivo (mosaico agroforestal); Termotemperado & 7.19 & 2.10 & 3.43 \\ 
  Litoral Cantabro-Atlantico; Agrosistema intensivo (mosaico agroforestal); Termotemperado & 6.76 & 7.20 & 0.94 \\ 
  Vales sublitorais; Agrosistema intensivo (superficie de cultivo); Mesotemperado inferior & 3.97 & 0.59 & 6.78 \\ 
  Vales sublitorais; Rururbano (diseminado); Mesotemperado inferior & 3.90 & 0.41 & 9.44 \\ 
  Serras; Agrosistema intensivo (plantacion forestal); Mesotemperado superior & 2.05 & 6.80 & 0.30 \\ 
  Litoral Cantabro-Atlantico; Rururbano (diseminado); Termotemperado & 1.55 & 2.68 & 0.58 \\ 
  Litoral Cantabro-Atlantico; Urbano; Termotemperado & 1.39 & 1.10 & 1.26 \\ 
  Vales sublitorais; Rururbano (diseminado); Termotemperado & 1.32 & 0.17 & 7.91 \\ 
  Serras; Matogueira e rochedo; Mesotemperado superior & 1.08 & 1.71 & 0.63 \\ 
   \hline
\end{tabular}
\end{table}
% latex table generated in R 3.2.5 by xtable 1.8-0 package
% Mon May  9 13:23:06 2016
\begin{table}[p]
\centering
\caption{Frecuencia de aparición dos Camiños de Santiago (área de influencia de 500 m a ambos lados) e frecuencia de tipos asociados Costa Sur - Baixo Miño} 
\label{vcamino3}
\begin{tabular}{lrrr}
  \hline
Tipo de paisaxe & F.Aparic (\%) & F.Tipo (\%) & Ratio \\ 
  \hline
Litoral Cantabro-Atlantico; Rururbano (diseminado); Termotemperado & 23.30 & 4.97 & 4.68 \\ 
  Litoral Cantabro-Atlantico; Urbano; Termotemperado & 18.59 & 1.10 & 16.89 \\ 
  Vales sublitorais; Rururbano (diseminado); Termotemperado & 13.29 & 2.82 & 4.72 \\ 
  Litoral Cantabro-Atlantico; Bosque; Termotemperado & 10.86 & 0.40 & 27.41 \\ 
  Vales sublitorais; Agrosistema intensivo (plantacion forestal); Termotemperado & 10.04 & 16.82 & 0.60 \\ 
  Litoral Cantabro-Atlantico; Agrosistema intensivo (mosaico agroforestal); Termotemperado & 9.92 & 5.83 & 1.70 \\ 
  Litoral Cantabro-Atlantico; Agrosistema extensivo; Termotemperado & 5.34 & 0.61 & 8.70 \\ 
  Litoral Cantabro-Atlantico; Agrosistema intensivo (plantacion forestal); Termotemperado & 4.93 & 6.69 & 0.74 \\ 
  Vales sublitorais; Agrosistema intensivo (mosaico agroforestal); Termotemperado & 2.23 & 7.55 & 0.30 \\ 
   \hline
\end{tabular}
\end{table}
% latex table generated in R 3.2.5 by xtable 1.8-0 package
% Mon May  9 13:23:06 2016
\begin{table}[p]
\centering
\caption{Frecuencia de aparición dos Camiños de Santiago (área de influencia de 500 m a ambos lados) e frecuencia de tipos asociados Ribeiras Encaixadas do Miño e do Sil} 
\label{vcamino4}
\begin{tabular}{lrrr}
  \hline
Tipo de paisaxe & F.Aparic (\%) & F.Tipo (\%) & Ratio \\ 
  \hline
Chairas e vales interiores; Agrosistema extensivo; Mesotemperado inferior & 21.40 & 6.56 & 3.26 \\ 
  Chairas e vales interiores; Agrosistema intensivo (mosaico agroforestal); Mesotemperado inferior & 19.32 & 5.37 & 3.60 \\ 
  Chairas e vales interiores; Urbano; Termotemperado & 15.59 & 1.01 & 15.48 \\ 
  Chairas e vales interiores; Agrosistema intensivo (plantacion forestal); Mesotemperado inferior & 12.22 & 4.37 & 2.79 \\ 
  Chairas e vales interiores; Rururbano (diseminado); Termotemperado & 5.39 & 0.91 & 5.93 \\ 
  Chairas e vales interiores; Bosque; Mesotemperado inferior & 5.30 & 3.85 & 1.38 \\ 
  Chairas e vales interiores; Matogueira e rochedo; Termotemperado & 4.83 & 2.43 & 1.99 \\ 
  Chairas e vales interiores; Agrosistema intensivo (plantacion forestal); Termotemperado & 3.21 & 6.81 & 0.47 \\ 
  Chairas e vales interiores; Agrosistema intensivo (superficie de cultivo); Mesotemperado inferior & 2.20 & 0.66 & 3.34 \\ 
  Chairas e vales interiores; Bosque; Termotemperado & 2.05 & 3.03 & 0.68 \\ 
  category 0; Lamina de auga; category 0 & 1.27 & 1.89 & 0.67 \\ 
  Chairas e vales interiores; Vinedo; Termotemperado & 1.13 & 2.22 & 0.51 \\ 
  Serras; Agrosistema intensivo (mosaico agroforestal); Mesotemperado superior & 1.10 & 3.32 & 0.33 \\ 
   \hline
\end{tabular}
\end{table}
% latex table generated in R 3.2.5 by xtable 1.8-0 package
% Mon May  9 13:23:06 2016
\begin{table}[p]
\centering
\caption{Frecuencia de aparición dos Camiños de Santiago (área de influencia de 500 m a ambos lados) e frecuencia de tipos asociados Serras Orientais} 
\label{vcamino5}
\begin{tabular}{lrrr}
  \hline
Tipo de paisaxe & F.Aparic (\%) & F.Tipo (\%) & Ratio \\ 
  \hline
Serras; Agrosistema extensivo; Supra e orotemperado & 22.81 & 8.91 & 2.56 \\ 
  Serras; Agrosistema intensivo (mosaico agroforestal); Supra e orotemperado & 17.23 & 5.74 & 3.00 \\ 
  Serras; Agrosistema intensivo (plantacion forestal); Supra e orotemperado & 13.83 & 6.86 & 2.02 \\ 
  Serras; Agrosistema extensivo; Mesotemperado superior & 10.48 & 7.30 & 1.44 \\ 
  Serras; Matogueira e rochedo; Supra e orotemperado & 9.96 & 15.61 & 0.64 \\ 
  Serras; Bosque; Supra e orotemperado & 6.29 & 7.67 & 0.82 \\ 
  Serras; Bosque; Mesotemperado superior & 3.72 & 3.65 & 1.02 \\ 
  Serras; Matogueira e rochedo; Mesotemperado superior & 2.86 & 5.41 & 0.53 \\ 
  Serras; Agrosistema intensivo (mosaico agroforestal); Mesotemperado superior & 2.43 & 3.95 & 0.61 \\ 
  Chairas e vales interiores; Agrosistema extensivo; Mesotemperado superior & 2.13 & 0.25 & 8.64 \\ 
  Chairas e vales interiores; Bosque; Mesotemperado superior & 1.63 & 0.19 & 8.37 \\ 
  Vales sublitorais; Bosque; Mesotemperado superior & 1.54 & 4.16 & 0.37 \\ 
  Serras; Agrosistema intensivo (plantacion forestal); Mesotemperado superior & 1.53 & 3.86 & 0.40 \\ 
  Serras; Agrosistema intensivo (superficie de cultivo); Supra e orotemperado & 1.22 & 0.69 & 1.75 \\ 
   \hline
\end{tabular}
\end{table}
% latex table generated in R 3.2.5 by xtable 1.8-0 package
% Mon May  9 13:23:06 2016
\begin{table}[p]
\centering
\caption{Frecuencia de aparición dos Camiños de Santiago (área de influencia de 500 m a ambos lados) e frecuencia de tipos asociados Chairas e Fosas Luguesas} 
\label{vcamino6}
\begin{tabular}{lrrr}
  \hline
Tipo de paisaxe & F.Aparic (\%) & F.Tipo (\%) & Ratio \\ 
  \hline
Chairas e vales interiores; Agrosistema intensivo (mosaico agroforestal); Mesotemperado superior & 16.71 & 20.11 & 0.83 \\ 
  Serras; Agrosistema intensivo (mosaico agroforestal); Mesotemperado superior & 15.27 & 9.79 & 1.56 \\ 
  Chairas e vales interiores; Agrosistema extensivo; Mesotemperado superior & 11.34 & 9.68 & 1.17 \\ 
  Serras; Agrosistema extensivo; Mesotemperado superior & 7.49 & 5.94 & 1.26 \\ 
  Chairas e vales interiores; Agrosistema intensivo (mosaico agroforestal); Mesotemperado inferior & 7.42 & 9.44 & 0.79 \\ 
  Serras; Agrosistema intensivo (superficie de cultivo); Mesotemperado superior & 7.14 & 3.04 & 2.35 \\ 
  Chairas e vales interiores; Agrosistema extensivo; Mesotemperado inferior & 5.78 & 6.80 & 0.85 \\ 
  Chairas e vales interiores; Agrosistema intensivo (superficie de cultivo); Mesotemperado superior & 5.00 & 5.78 & 0.87 \\ 
  Chairas e vales interiores; Agrosistema intensivo (plantacion forestal); Mesotemperado superior & 2.61 & 3.97 & 0.66 \\ 
  Chairas e vales interiores; Agrosistema intensivo (superficie de cultivo); Mesotemperado inferior & 2.55 & 2.36 & 1.08 \\ 
  Serras; Agrosistema intensivo (plantacion forestal); Mesotemperado superior & 2.01 & 2.29 & 0.88 \\ 
  Chairas e vales interiores; Urbano; Mesotemperado superior & 1.79 & 0.45 & 3.96 \\ 
  Chairas e vales interiores; Matogueira e rochedo; Mesotemperado superior & 1.67 & 1.43 & 1.17 \\ 
  Chairas e vales interiores; Rururbano (diseminado); Mesotemperado inferior & 1.47 & 0.39 & 3.74 \\ 
  Chairas e vales interiores; Bosque; Mesotemperado superior & 1.31 & 2.02 & 0.65 \\ 
  Serras; Agrosistema intensivo (plantacion forestal); Supra e orotemperado & 1.19 & 0.59 & 2.02 \\ 
   \hline
\end{tabular}
\end{table}
% latex table generated in R 3.2.5 by xtable 1.8-0 package
% Mon May  9 13:23:06 2016
\begin{table}[p]
\centering
\caption{Frecuencia de aparición dos Camiños de Santiago (área de influencia de 500 m a ambos lados) e frecuencia de tipos asociados Galicia Central} 
\label{vcamino7}
\begin{tabular}{lrrr}
  \hline
Tipo de paisaxe & F.Aparic (\%) & F.Tipo (\%) & Ratio \\ 
  \hline
Vales sublitorais; Agrosistema intensivo (mosaico agroforestal); Mesotemperado inferior & 50.56 & 43.06 & 1.17 \\ 
  Vales sublitorais; Agrosistema extensivo; Mesotemperado inferior & 6.19 & 5.21 & 1.19 \\ 
  Vales sublitorais; Agrosistema intensivo (superficie de cultivo); Mesotemperado inferior & 5.60 & 3.18 & 1.76 \\ 
  Vales sublitorais; Urbano; Mesotemperado inferior & 4.63 & 0.64 & 7.21 \\ 
  Vales sublitorais; Agrosistema intensivo (plantacion forestal); Mesotemperado inferior & 4.58 & 5.84 & 0.78 \\ 
  Serras; Agrosistema extensivo; Mesotemperado superior & 3.35 & 4.47 & 0.75 \\ 
  Serras; Matogueira e rochedo; Mesotemperado superior & 3.00 & 6.90 & 0.43 \\ 
  Serras; Agrosistema intensivo (mosaico agroforestal); Mesotemperado superior & 2.33 & 3.93 & 0.59 \\ 
  Vales sublitorais; Agrosistema extensivo; Mesotemperado superior & 2.17 & 1.61 & 1.35 \\ 
  Vales sublitorais; Rururbano (diseminado); Termotemperado & 2.08 & 0.34 & 6.17 \\ 
  Vales sublitorais; Matogueira e rochedo; Mesotemperado inferior & 1.56 & 1.65 & 0.95 \\ 
  Vales sublitorais; Rururbano (diseminado); Mesotemperado inferior & 1.24 & 0.23 & 5.41 \\ 
   \hline
\end{tabular}
\end{table}
% latex table generated in R 3.2.5 by xtable 1.8-0 package
% Mon May  9 13:23:06 2016
\begin{table}[p]
\centering
\caption{Frecuencia de aparición dos Camiños de Santiago (área de influencia de 500 m a ambos lados) e frecuencia de tipos asociados Chairas, Fosas e Serras Ourensás} 
\label{vcamino8}
\begin{tabular}{lrrr}
  \hline
Tipo de paisaxe & F.Aparic (\%) & F.Tipo (\%) & Ratio \\ 
  \hline
Chairas e vales interiores; Agrosistema intensivo (superficie de cultivo); Mesotemperado inferior & 16.65 & 7.53 & 2.21 \\ 
  Chairas e vales interiores; Agrosistema extensivo; Mesotemperado inferior & 13.85 & 7.38 & 1.88 \\ 
  Chairas e vales interiores; Bosque; Mesotemperado inferior & 7.29 & 6.18 & 1.18 \\ 
  Serras; Agrosistema extensivo; Mesotemperado inferior & 7.16 & 5.23 & 1.37 \\ 
  Chairas e vales interiores; Matogueira e rochedo; Mesotemperado inferior & 7.11 & 4.82 & 1.47 \\ 
  Serras; Matogueira e rochedo; Mesotemperado superior & 6.28 & 10.51 & 0.60 \\ 
  Serras; Bosque; Mesotemperado inferior & 6.06 & 4.67 & 1.30 \\ 
  Serras; Matogueira e rochedo; Mesotemperado inferior & 4.70 & 5.15 & 0.91 \\ 
  Serras; Agrosistema intensivo (plantacion forestal); Mesotemperado inferior & 4.54 & 5.68 & 0.80 \\ 
  Serras; Matogueira e rochedo; Supra e orotemperado & 4.36 & 14.06 & 0.31 \\ 
  Chairas e vales interiores; Vinedo; Termotemperado & 3.56 & 0.90 & 3.96 \\ 
  Chairas e vales interiores; Agrosistema intensivo (mosaico agroforestal); Mesotemperado inferior & 3.15 & 1.08 & 2.92 \\ 
  Chairas e vales interiores; Urbano; Mesotemperado inferior & 2.49 & 0.23 & 10.73 \\ 
  Chairas e vales interiores; Rururbano (diseminado); Termotemperado & 2.15 & 0.40 & 5.40 \\ 
  Chairas e vales interiores; Agrosistema intensivo (plantacion forestal); Mesotemperado inferior & 1.96 & 3.30 & 0.59 \\ 
  Chairas e vales interiores; Rururbano (diseminado); Mesotemperado inferior & 1.58 & 0.32 & 4.89 \\ 
  Chairas e vales interiores; Urbano; Termotemperado & 1.54 & 0.11 & 14.58 \\ 
  Serras; Agrosistema intensivo (mosaico agroforestal); Mesotemperado inferior & 1.29 & 0.42 & 3.09 \\ 
  Serras; Agrosistema extensivo; Supra e orotemperado & 1.05 & 1.96 & 0.53 \\ 
   \hline
\end{tabular}
\end{table}
% latex table generated in R 3.2.5 by xtable 1.8-0 package
% Mon May  9 13:23:07 2016
\begin{table}[p]
\centering
\caption{Frecuencia de aparición dos Camiños de Santiago (área de influencia de 500 m a ambos lados) e frecuencia de tipos asociados Serras Surorientais} 
\label{vcamino9}
\begin{tabular}{lrrr}
  \hline
Tipo de paisaxe & F.Aparic (\%) & F.Tipo (\%) & Ratio \\ 
  \hline
Serras; Matogueira e rochedo; Supra e orotemperado & 37.18 & 47.28 & 0.79 \\ 
  Serras; Agrosistema intensivo (plantacion forestal); Supra e orotemperado & 17.78 & 10.23 & 1.74 \\ 
  Serras; Agrosistema extensivo; Mesotemperado inferior & 13.83 & 4.79 & 2.88 \\ 
  Serras; Agrosistema extensivo; Supra e orotemperado & 9.65 & 2.94 & 3.28 \\ 
  Serras; Matogueira e rochedo; Mesotemperado superior & 6.08 & 3.34 & 1.82 \\ 
  Serras; Agrosistema intensivo (mosaico agroforestal); Mesotemperado superior & 1.69 & 0.81 & 2.10 \\ 
  Serras; Agrosistema intensivo (mosaico agroforestal); Supra e orotemperado & 1.56 & 0.55 & 2.83 \\ 
  Serras; Agrosistema intensivo (plantacion forestal); Mesotemperado inferior & 1.52 & 0.76 & 2.00 \\ 
  Serras; Matogueira e rochedo; Mesotemperado inferior & 1.46 & 2.75 & 0.53 \\ 
  Serras; Rururbano (diseminado); Supra e orotemperado & 1.30 & 0.09 & 14.23 \\ 
  Serras; Bosque; Mesotemperado inferior & 1.27 & 3.06 & 0.41 \\ 
  Serras; Bosque; Mesotemperado superior & 1.25 & 5.61 & 0.22 \\ 
  Serras; Agrosistema extensivo; Mesotemperado superior & 1.08 & 6.24 & 0.17 \\ 
   \hline
\end{tabular}
\end{table}
% latex table generated in R 3.2.5 by xtable 1.8-0 package
% Mon May  9 13:23:07 2016
\begin{table}[p]
\centering
\caption{Frecuencia de aparición dos Camiños de Santiago (área de influencia de 500 m a ambos lados) e frecuencia de tipos asociados Galicia Setentrional} 
\label{vcamino10}
\begin{tabular}{lrrr}
  \hline
Tipo de paisaxe & F.Aparic (\%) & F.Tipo (\%) & Ratio \\ 
  \hline
\hline
\end{tabular}
\end{table}
% latex table generated in R 3.2.5 by xtable 1.8-0 package
% Mon May  9 13:23:07 2016
\begin{table}[p]
\centering
\caption{Frecuencia de aparición dos Camiños de Santiago (área de influencia de 500 m a ambos lados) e frecuencia de tipos asociados Chairas e Fosas Occidentais} 
\label{vcamino11}
\begin{tabular}{lrrr}
  \hline
Tipo de paisaxe & F.Aparic (\%) & F.Tipo (\%) & Ratio \\ 
  \hline
Litoral Cantabro-Atlantico; Agrosistema intensivo (mosaico agroforestal); Termotemperado & 21.33 & 9.91 & 2.15 \\ 
  Vales sublitorais; Agrosistema intensivo (mosaico agroforestal); Mesotemperado inferior & 19.47 & 36.05 & 0.54 \\ 
  Litoral Cantabro-Atlantico; Agrosistema intensivo (plantacion forestal); Termotemperado & 13.53 & 7.60 & 1.78 \\ 
  Vales sublitorais; Agrosistema intensivo (superficie de cultivo); Mesotemperado inferior & 8.67 & 4.84 & 1.79 \\ 
  Vales sublitorais; Agrosistema intensivo (plantacion forestal); Mesotemperado inferior & 8.26 & 15.20 & 0.54 \\ 
  Vales sublitorais; Agrosistema intensivo (mosaico agroforestal); Termotemperado & 8.12 & 2.15 & 3.77 \\ 
  Vales sublitorais; Agrosistema intensivo (plantacion forestal); Termotemperado & 6.56 & 3.44 & 1.91 \\ 
  Vales sublitorais; Matogueira e rochedo; Mesotemperado inferior & 6.07 & 4.77 & 1.27 \\ 
  Litoral Cantabro-Atlantico; Matogueira e rochedo; Termotemperado & 2.98 & 4.58 & 0.65 \\ 
  Litoral Cantabro-Atlantico; Urbano; Termotemperado & 1.71 & 0.66 & 2.59 \\ 
  category 0; Praias e cantis; category 0 & 1.06 & 0.54 & 1.97 \\ 
   \hline
\end{tabular}
\end{table}
% latex table generated in R 3.2.5 by xtable 1.8-0 package
% Mon May  9 13:23:07 2016
\begin{table}[p]
\centering
\caption{Frecuencia de aparición dos Camiños de Santiago (área de influencia de 500 m a ambos lados) e frecuencia de tipos asociados Rías Baixas} 
\label{vcamino12}
\begin{tabular}{lrrr}
  \hline
Tipo de paisaxe & F.Aparic (\%) & F.Tipo (\%) & Ratio \\ 
  \hline
Litoral Cantabro-Atlantico; Agrosistema intensivo (mosaico agroforestal); Termotemperado & 30.28 & 10.49 & 2.89 \\ 
  Litoral Cantabro-Atlantico; Rururbano (diseminado); Termotemperado & 22.48 & 7.10 & 3.16 \\ 
  Vales sublitorais; Agrosistema intensivo (mosaico agroforestal); Termotemperado & 16.63 & 7.63 & 2.18 \\ 
  Litoral Cantabro-Atlantico; Urbano; Termotemperado & 7.39 & 3.29 & 2.25 \\ 
  Litoral Cantabro-Atlantico; Agrosistema intensivo (plantacion forestal); Termotemperado & 5.09 & 10.20 & 0.50 \\ 
  Vales sublitorais; Agrosistema intensivo (plantacion forestal); Termotemperado & 4.76 & 16.41 & 0.29 \\ 
  Vales sublitorais; Agrosistema intensivo (mosaico agroforestal); Mesotemperado inferior & 2.38 & 4.05 & 0.59 \\ 
  Vales sublitorais; Agrosistema intensivo (superficie de cultivo); Termotemperado & 1.81 & 0.09 & 20.39 \\ 
  Litoral Cantabro-Atlantico; Vinedo; Termotemperado & 1.80 & 1.45 & 1.24 \\ 
  Vales sublitorais; Matogueira e rochedo; Termotemperado & 1.39 & 3.00 & 0.46 \\ 
  Litoral Cantabro-Atlantico; Agrosistema extensivo; Termotemperado & 1.36 & 0.60 & 2.26 \\ 
  Litoral Cantabro-Atlantico; Agrosistema intensivo (superficie de cultivo); Termotemperado & 1.33 & 1.03 & 1.28 \\ 
  Vales sublitorais; Rururbano (diseminado); Termotemperado & 1.17 & 1.31 & 0.90 \\ 
   \hline
\end{tabular}
\end{table}
