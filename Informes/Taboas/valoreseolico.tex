% latex table generated in R 3.2.5 by xtable 1.8-0 package
% Mon May  9 13:23:06 2016
\begin{table}[p]
\centering
\caption{Frecuencia de aparición de xeneradores eólicos e frecuencia de tipos asociados Golfo Ártabro} 
\label{veolico1}
\begin{tabular}{lrrr}
  \hline
Tipo de paisaxe & F.Aparic (\%) & F.Tipo (\%) & Ratio \\ 
  \hline
Serras; Turbeira; Mesotemperado superior & 40.26 & 1.53 & 26.32 \\ 
  Serras; Matogueira e rochedo; Mesotemperado superior & 25.97 & 3.06 & 8.49 \\ 
  Vales sublitorais; Agrosistema intensivo (mosaico agroforestal); Mesotemperado inferior & 9.09 & 21.75 & 0.42 \\ 
  Serras; Agrosistema intensivo (mosaico agroforestal); Mesotemperado superior & 6.49 & 2.44 & 2.67 \\ 
  Serras; Turbeira; Supra e orotemperado & 6.49 & 0.03 & 221.98 \\ 
  Serras; Agrosistema extensivo; Mesotemperado superior & 5.19 & 0.80 & 6.52 \\ 
  Vales sublitorais; Turbeira; Mesotemperado superior & 3.90 & 0.50 & 7.86 \\ 
  Serras; Agrosistema intensivo (plantacion forestal); Mesotemperado superior & 2.60 & 0.88 & 2.94 \\ 
   \hline
\end{tabular}
\end{table}
% latex table generated in R 3.2.5 by xtable 1.8-0 package
% Mon May  9 13:23:06 2016
\begin{table}[p]
\centering
\caption{Frecuencia de aparición de xeneradores eólicos e frecuencia de tipos asociados A Mariña - Baixo Eo} 
\label{veolico2}
\begin{tabular}{lrrr}
  \hline
Tipo de paisaxe & F.Aparic (\%) & F.Tipo (\%) & Ratio \\ 
  \hline
Serras; Turbeira; Mesotemperado superior & 63.37 & 2.64 & 24.01 \\ 
  Serras; Agrosistema intensivo (plantacion forestal); Mesotemperado superior & 20.79 & 6.80 & 3.06 \\ 
  Serras; Agrosistema intensivo (plantacion forestal); Mesotemperado inferior & 9.90 & 1.66 & 5.95 \\ 
  Serras; Matogueira e rochedo; Mesotemperado superior & 4.95 & 1.71 & 2.89 \\ 
   \hline
\end{tabular}
\end{table}
% latex table generated in R 3.2.5 by xtable 1.8-0 package
% Mon May  9 13:23:06 2016
\begin{table}[p]
\centering
\caption{Frecuencia de aparición de xeneradores eólicos e frecuencia de tipos asociados Costa Sur - Baixo Miño} 
\label{veolico3}
\begin{tabular}{lrrr}
  \hline
Tipo de paisaxe & F.Aparic (\%) & F.Tipo (\%) & Ratio \\ 
  \hline
Serras; Matogueira e rochedo; Mesotemperado superior & 90.12 & 11.08 & 8.13 \\ 
  Serras; Agrosistema intensivo (plantacion forestal); Supra e orotemperado & 9.26 & 0.66 & 13.98 \\ 
   \hline
\end{tabular}
\end{table}
% latex table generated in R 3.2.5 by xtable 1.8-0 package
% Mon May  9 13:23:06 2016
\begin{table}[p]
\centering
\caption{Frecuencia de aparición de xeneradores eólicos e frecuencia de tipos asociados Ribeiras Encaixadas do Miño e do Sil} 
\label{veolico4}
\begin{tabular}{lrrr}
  \hline
Tipo de paisaxe & F.Aparic (\%) & F.Tipo (\%) & Ratio \\ 
  \hline
Serras; Matogueira e rochedo; Mesotemperado inferior & 50.00 & 2.79 & 17.94 \\ 
  Serras; Agrosistema intensivo (plantacion forestal); Mesotemperado superior & 23.44 & 1.12 & 21.01 \\ 
  Serras; Bosque; Supra e orotemperado & 12.50 & 0.47 & 26.33 \\ 
  Serras; Agrosistema intensivo (plantacion forestal); Supra e orotemperado & 4.69 & 0.76 & 6.13 \\ 
  Serras; Agrosistema intensivo (plantacion forestal); Mesotemperado inferior & 3.12 & 2.65 & 1.18 \\ 
  Serras; Matogueira e rochedo; Mesotemperado superior & 3.12 & 3.10 & 1.01 \\ 
  Serras; Bosque; Mesotemperado inferior & 1.56 & 2.32 & 0.67 \\ 
  Serras; Matogueira e rochedo; Supra e orotemperado & 1.56 & 9.08 & 0.17 \\ 
   \hline
\end{tabular}
\end{table}
% latex table generated in R 3.2.5 by xtable 1.8-0 package
% Mon May  9 13:23:06 2016
\begin{table}[p]
\centering
\caption{Frecuencia de aparición de xeneradores eólicos e frecuencia de tipos asociados Serras Orientais} 
\label{veolico5}
\begin{tabular}{lrrr}
  \hline
Tipo de paisaxe & F.Aparic (\%) & F.Tipo (\%) & Ratio \\ 
  \hline
Serras; Matogueira e rochedo; Mesotemperado superior & 35.62 & 5.41 & 6.58 \\ 
  Serras; Matogueira e rochedo; Supra e orotemperado & 16.44 & 15.61 & 1.05 \\ 
  Serras; Agrosistema extensivo; Mesotemperado superior & 13.70 & 7.30 & 1.88 \\ 
  Serras; Agrosistema intensivo (mosaico agroforestal); Mesotemperado superior & 11.64 & 3.95 & 2.95 \\ 
  Serras; Agrosistema intensivo (plantacion forestal); Supra e orotemperado & 10.27 & 6.86 & 1.50 \\ 
  Serras; Agrosistema intensivo (plantacion forestal); Mesotemperado superior & 4.79 & 3.86 & 1.24 \\ 
  Serras; Agrosistema intensivo (mosaico agroforestal); Supra e orotemperado & 3.42 & 5.74 & 0.60 \\ 
  Serras; Agrosistema extensivo; Supra e orotemperado & 2.74 & 8.91 & 0.31 \\ 
   & 1.37 &  &  \\ 
   \hline
\end{tabular}
\end{table}
% latex table generated in R 3.2.5 by xtable 1.8-0 package
% Mon May  9 13:23:06 2016
\begin{table}[p]
\centering
\caption{Frecuencia de aparición de xeneradores eólicos e frecuencia de tipos asociados Chairas e Fosas Luguesas} 
\label{veolico6}
\begin{tabular}{lrrr}
  \hline
Tipo de paisaxe & F.Aparic (\%) & F.Tipo (\%) & Ratio \\ 
  \hline
Serras; Turbeira; Mesotemperado superior & 23.55 & 1.08 & 21.77 \\ 
  Serras; Agrosistema intensivo (plantacion forestal); Supra e orotemperado & 17.13 & 0.59 & 29.11 \\ 
  Serras; Turbeira; Supra e orotemperado & 17.13 & 0.60 & 28.69 \\ 
  Serras; Matogueira e rochedo; Mesotemperado superior & 13.15 & 1.94 & 6.79 \\ 
  Serras; Agrosistema intensivo (plantacion forestal); Mesotemperado superior & 7.95 & 2.29 & 3.47 \\ 
  Serras; Matogueira e rochedo; Supra e orotemperado & 7.95 & 0.48 & 16.60 \\ 
  Chairas e vales interiores; Agrosistema intensivo (plantacion forestal); Mesotemperado superior & 3.98 & 3.97 & 1.00 \\ 
  Serras; Bosque; Supra e orotemperado & 3.36 & 0.23 & 14.85 \\ 
  Serras; Agrosistema intensivo (mosaico agroforestal); Mesotemperado superior & 3.06 & 9.79 & 0.31 \\ 
  category 0; Lamina de auga; category 0 & 1.22 & 0.09 & 12.88 \\ 
  Serras; Agrosistema intensivo (superficie de cultivo); Mesotemperado superior & 1.22 & 3.04 & 0.40 \\ 
   \hline
\end{tabular}
\end{table}
% latex table generated in R 3.2.5 by xtable 1.8-0 package
% Mon May  9 13:23:06 2016
\begin{table}[p]
\centering
\caption{Frecuencia de aparición de xeneradores eólicos e frecuencia de tipos asociados Galicia Central} 
\label{veolico7}
\begin{tabular}{lrrr}
  \hline
Tipo de paisaxe & F.Aparic (\%) & F.Tipo (\%) & Ratio \\ 
  \hline
Serras; Matogueira e rochedo; Mesotemperado superior & 62.66 & 6.90 & 9.08 \\ 
  Serras; Agrosistema intensivo (plantacion forestal); Supra e orotemperado & 6.91 & 0.54 & 12.89 \\ 
  Serras; Matogueira e rochedo; Mesotemperado inferior & 6.39 & 0.62 & 10.31 \\ 
  Serras; Agrosistema intensivo (plantacion forestal); Mesotemperado superior & 4.35 & 1.82 & 2.40 \\ 
  Serras; Agrosistema intensivo (plantacion forestal); Mesotemperado inferior & 3.07 & 0.64 & 4.79 \\ 
  Serras; Matogueira e rochedo; Supra e orotemperado & 3.07 & 0.53 & 5.77 \\ 
  Serras; Agrosistema extensivo; Mesotemperado superior & 2.56 & 4.47 & 0.57 \\ 
  Vales sublitorais; Matogueira e rochedo; Mesotemperado superior & 2.05 & 0.85 & 2.42 \\ 
  Serras; Agrosistema intensivo (mosaico agroforestal); Mesotemperado superior & 1.79 & 3.93 & 0.46 \\ 
  Serras; Agrosistema intensivo (mosaico agroforestal); Supra e orotemperado & 1.53 & 0.22 & 7.05 \\ 
  Vales sublitorais; Matogueira e rochedo; Mesotemperado inferior & 1.28 & 1.65 & 0.78 \\ 
  Serras; Agrosistema extensivo; Supra e orotemperado & 1.02 & 0.08 & 13.63 \\ 
  Serras; Agrosistema intensivo (superficie de cultivo); Mesotemperado superior & 1.02 & 1.44 & 0.71 \\ 
   \hline
\end{tabular}
\end{table}
% latex table generated in R 3.2.5 by xtable 1.8-0 package
% Mon May  9 13:23:06 2016
\begin{table}[p]
\centering
\caption{Frecuencia de aparición de xeneradores eólicos e frecuencia de tipos asociados Chairas, Fosas e Serras Ourensás} 
\label{veolico8}
\begin{tabular}{lrrr}
  \hline
Tipo de paisaxe & F.Aparic (\%) & F.Tipo (\%) & Ratio \\ 
  \hline
Serras; Matogueira e rochedo; Mesotemperado superior & 53.33 & 10.51 & 5.07 \\ 
  Serras; Matogueira e rochedo; Supra e orotemperado & 27.62 & 14.06 & 1.96 \\ 
  Serras; Agrosistema intensivo (plantacion forestal); Supra e orotemperado & 9.52 & 1.55 & 6.16 \\ 
  Serras; Matogueira e rochedo; Mesotemperado inferior & 6.67 & 5.15 & 1.30 \\ 
  Serras; Bosque; Supra e orotemperado & 1.90 & 1.18 & 1.61 \\ 
   \hline
\end{tabular}
\end{table}
% latex table generated in R 3.2.5 by xtable 1.8-0 package
% Mon May  9 13:23:07 2016
\begin{table}[p]
\centering
\caption{Frecuencia de aparición de xeneradores eólicos e frecuencia de tipos asociados Serras Surorientais} 
\label{veolico9}
\begin{tabular}{lrrr}
  \hline
Tipo de paisaxe & F.Aparic (\%) & F.Tipo (\%) & Ratio \\ 
  \hline
Serras; Matogueira e rochedo; Supra e orotemperado & 87.76 & 47.28 & 1.86 \\ 
  Serras; Agrosistema extensivo; Supra e orotemperado & 8.16 & 2.94 & 2.77 \\ 
  Serras; Agrosistema intensivo (plantacion forestal); Supra e orotemperado & 4.08 & 10.23 & 0.40 \\ 
   \hline
\end{tabular}
\end{table}
% latex table generated in R 3.2.5 by xtable 1.8-0 package
% Mon May  9 13:23:07 2016
\begin{table}[p]
\centering
\caption{Frecuencia de aparición de xeneradores eólicos e frecuencia de tipos asociados Galicia Setentrional} 
\label{veolico10}
\begin{tabular}{lrrr}
  \hline
Tipo de paisaxe & F.Aparic (\%) & F.Tipo (\%) & Ratio \\ 
  \hline
Serras; Turbeira; Mesotemperado superior & 46.17 & 12.44 & 3.71 \\ 
  Serras; Matogueira e rochedo; Mesotemperado superior & 28.30 & 6.79 & 4.17 \\ 
  Serras; Agrosistema intensivo (plantacion forestal); Mesotemperado superior & 6.40 & 4.68 & 1.37 \\ 
  Serras; Agrosistema intensivo (mosaico agroforestal); Mesotemperado superior & 2.84 & 3.70 & 0.77 \\ 
  Serras; Agrosistema intensivo (plantacion forestal); Mesotemperado inferior & 2.77 & 2.50 & 1.11 \\ 
  Vales sublitorais; Agrosistema intensivo (plantacion forestal); Mesotemperado inferior & 2.77 & 17.31 & 0.16 \\ 
  Vales sublitorais; Matogueira e rochedo; Mesotemperado inferior & 2.31 & 0.62 & 3.70 \\ 
  Vales sublitorais; Agrosistema intensivo (mosaico agroforestal); Mesotemperado inferior & 1.65 & 13.63 & 0.12 \\ 
  Serras; Turbeira; Mesotemperado inferior & 1.58 & 0.25 & 6.33 \\ 
  Serras; Matogueira e rochedo; Mesotemperado inferior & 1.45 & 0.56 & 2.61 \\ 
  Vales sublitorais; Turbeira; Mesotemperado inferior & 1.06 & 0.11 & 9.20 \\ 
   \hline
\end{tabular}
\end{table}
% latex table generated in R 3.2.5 by xtable 1.8-0 package
% Mon May  9 13:23:07 2016
\begin{table}[p]
\centering
\caption{Frecuencia de aparición de xeneradores eólicos e frecuencia de tipos asociados Chairas e Fosas Occidentais} 
\label{veolico11}
\begin{tabular}{lrrr}
  \hline
Tipo de paisaxe & F.Aparic (\%) & F.Tipo (\%) & Ratio \\ 
  \hline
Vales sublitorais; Agrosistema intensivo (plantacion forestal); Mesotemperado inferior & 25.86 & 15.20 & 1.70 \\ 
  Litoral Cantabro-Atlantico; Matogueira e rochedo; Termotemperado & 22.27 & 4.58 & 4.86 \\ 
  Vales sublitorais; Matogueira e rochedo; Mesotemperado inferior & 17.64 & 4.77 & 3.70 \\ 
  Vales sublitorais; Agrosistema intensivo (plantacion forestal); Mesotemperado superior & 10.46 & 0.96 & 10.90 \\ 
  Vales sublitorais; Matogueira e rochedo; Mesotemperado superior & 8.82 & 2.22 & 3.98 \\ 
  Vales sublitorais; Turbeira; Mesotemperado superior & 4.63 & 0.39 & 12.00 \\ 
  Vales sublitorais; Matogueira e rochedo; Termotemperado & 4.19 & 0.80 & 5.25 \\ 
  Vales sublitorais; Agrosistema intensivo (mosaico agroforestal); Mesotemperado inferior & 2.99 & 36.05 & 0.08 \\ 
  Litoral Cantabro-Atlantico; Agrosistema intensivo (plantacion forestal); Termotemperado & 2.69 & 7.60 & 0.35 \\ 
   \hline
\end{tabular}
\end{table}
% latex table generated in R 3.2.5 by xtable 1.8-0 package
% Mon May  9 13:23:07 2016
\begin{table}[p]
\centering
\caption{Frecuencia de aparición de xeneradores eólicos e frecuencia de tipos asociados Rías Baixas} 
\label{veolico12}
\begin{tabular}{lrrr}
  \hline
Tipo de paisaxe & F.Aparic (\%) & F.Tipo (\%) & Ratio \\ 
  \hline
Serras; Matogueira e rochedo; Mesotemperado superior & 31.35 & 6.70 & 4.68 \\ 
  Serras; Matogueira e rochedo; Mesotemperado inferior & 24.35 & 3.29 & 7.39 \\ 
  Vales sublitorais; Matogueira e rochedo; Mesotemperado superior & 15.80 & 0.39 & 40.36 \\ 
  Vales sublitorais; Matogueira e rochedo; Mesotemperado inferior & 7.77 & 4.71 & 1.65 \\ 
  Vales sublitorais; Agrosistema intensivo (plantacion forestal); Termotemperado & 5.18 & 16.41 & 0.32 \\ 
  Vales sublitorais; Agrosistema intensivo (plantacion forestal); Mesotemperado superior & 4.15 & 0.17 & 24.02 \\ 
  Vales sublitorais; Turbeira; Mesotemperado superior & 3.11 & 0.04 & 73.04 \\ 
  Vales sublitorais; Agrosistema intensivo (mosaico agroforestal); Mesotemperado inferior & 2.59 & 4.05 & 0.64 \\ 
  Vales sublitorais; Agrosistema intensivo (plantacion forestal); Mesotemperado inferior & 2.33 & 3.81 & 0.61 \\ 
  Serras; Agrosistema intensivo (plantacion forestal); Mesotemperado inferior & 2.07 & 2.82 & 0.74 \\ 
  Serras; Turbeira; Supra e orotemperado & 1.04 & 0.03 & 37.22 \\ 
   \hline
\end{tabular}
\end{table}
