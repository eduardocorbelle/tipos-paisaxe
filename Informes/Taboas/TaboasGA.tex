% latex table generated in R 3.2.5 by xtable 1.8-0 package
% Mon May  9 10:56:57 2016
\begin{table}[p]
\centering
\caption{Grandes Áreas paisaxísticas e código asignado} 
\label{xtaboa0}
\begin{tabular}{rr}
  \hline
 & Código \\ 
  \hline
Golfo Ártabro &   1 \\ 
  A Mariña - Baixo Eo &   2 \\ 
  Costa Sur - Baixo Miño &   3 \\ 
  Ribeiras Encaixadas do Miño e do Sil &   4 \\ 
  Serras Orientais &   5 \\ 
  Chairas e Fosas Luguesas &   6 \\ 
  Galicia Central &   7 \\ 
  Chairas, Fosas e Serras Ourensás &   8 \\ 
  Serras Surorientais &   9 \\ 
  Galicia Setentrional &  10 \\ 
  Chairas e Fosas Occidentais &  11 \\ 
  Rías Baixas &  12 \\ 
   \hline
\end{tabular}
\end{table}
% latex table generated in R 3.2.5 by xtable 1.8-0 package
% Mon May  9 10:56:57 2016
\begin{table}[p]
\centering
\caption{Grandes unidades morfolóxicas por Grandes Áreas paisaxísticas (datos en km²)} 
\label{xtaboa1}
\begin{tabular}{rrrrrrrrrrrrr}
  \hline
 & 1 & 2 & 3 & 4 & 5 & 6 & 7 & 8 & 9 & 10 & 11 & 12 \\ 
  \hline
 & 10 & 4 & 9 & 48 & 4 & 5 & 16 & 20 & 30 & 30 & 23 & 22 \\ 
  Canons & 31 & 0 & 0 & 294 & 37 & 6 & 0 & 0 & 73 & 0 & 0 & 0 \\ 
  Chairas e vales interiores & 0 & 1 & 168 & 1292 & 74 & 3212 & 229 & 1122 & 3 & 0 & 0 & 0 \\ 
  no data & 4 & 2 & 3 & 0 & 2 & 0 & 0 & 4 & 0 & 5 & 9 & 27 \\ 
  Serras & 139 & 152 & 307 & 837 & 1794 & 1309 & 1308 & 1696 & 2095 & 631 & 0 & 460 \\ 
  Vales sublitorais & 580 & 424 & 441 & 1 & 583 & 27 & 3598 & 2 & 0 & 618 & 1536 & 1217 \\ 
   \hline
\end{tabular}
\end{table}
% latex table generated in R 3.2.5 by xtable 1.8-0 package
% Mon May  9 10:56:57 2016
\begin{table}[p]
\centering
\caption{Grandes unidades morfolóxicas por Grandes Áreas paisaxísticas (datos en porcentaxe)} 
\label{xtaboa1p}
\begin{tabular}{rrrrrrrrrrrrr}
  \hline
 & 1 & 2 & 3 & 4 & 5 & 6 & 7 & 8 & 9 & 10 & 11 & 12 \\ 
  \hline
 & 0.8 & 0.5 & 0.7 & 1.9 & 0.2 & 0.1 & 0.3 & 0.7 & 1.4 & 1.8 & 1.1 & 0.8 \\ 
  Canons & 2.4 & 0.0 & 0.0 & 11.9 & 1.5 & 0.1 & 0.0 & 0.0 & 3.3 & 0.0 & 0.0 & 0.0 \\ 
  Chairas e vales interiores & 0.0 & 0.1 & 14.2 & 52.3 & 3.0 & 70.4 & 4.5 & 39.4 & 0.1 & 0.0 & 0.0 & 0.0 \\ 
  no data & 0.3 & 0.2 & 0.3 & 0.0 & 0.1 & 0.0 & 0.0 & 0.1 & 0.0 & 0.3 & 0.4 & 1.0 \\ 
  Serras & 10.8 & 16.5 & 26.0 & 33.9 & 71.9 & 28.7 & 25.4 & 59.6 & 95.2 & 38.8 & 0.0 & 17.0 \\ 
  Vales sublitorais & 45.0 & 45.8 & 37.3 & 0.0 & 23.4 & 0.6 & 69.8 & 0.1 & 0.0 & 38.0 & 74.0 & 45.1 \\ 
   \hline
\end{tabular}
\end{table}
% latex table generated in R 3.2.5 by xtable 1.8-0 package
% Mon May  9 10:56:58 2016
\begin{table}[p]
\centering
\caption{Clases de cuberta por Grandes Áreas paisaxísticas (datos en km²)} 
\label{xtaboa2}
\begin{tabular}{rrrrrrrrrrrrr}
  \hline
 & 1 & 2 & 3 & 4 & 5 & 6 & 7 & 8 & 9 & 10 & 11 & 12 \\ 
  \hline
 & 0 & 0 & 0 & 0 & 0 & 0 & 0 & 0 & 0 & 0 & 0 & 0 \\ 
  Agrosistema extensivo & 42 & 15 & 62 & 328 & 589 & 1058 & 694 & 503 & 317 & 80 & 25 & 91 \\ 
  Agrosistema intensivo (mosaico agroforestal) & 600 & 263 & 189 & 310 & 311 & 1842 & 2614 & 70 & 47 & 422 & 1001 & 615 \\ 
  Agrosistema intensivo (plantacion forestal) & 286 & 488 & 465 & 524 & 377 & 484 & 643 & 449 & 281 & 580 & 566 & 939 \\ 
  Agrosistema intensivo (superficie de cultivo) & 21 & 27 & 3 & 43 & 37 & 523 & 320 & 251 & 32 & 18 & 141 & 34 \\ 
  Bosque & 49 & 23 & 40 & 381 & 533 & 230 & 171 & 416 & 242 & 60 & 0 & 54 \\ 
  Conxunto Historico & 1 & 1 & 3 & 1 & 0 & 1 & 1 & 0 & 0 & 0 & 3 & 1 \\ 
  Extractivo & 2 & 1 & 8 & 5 & 5 & 4 & 11 & 4 & 23 & 9 & 3 & 2 \\ 
  Matogueira e rochedo & 84 & 32 & 265 & 641 & 634 & 256 & 592 & 1036 & 1216 & 179 & 257 & 537 \\ 
  no data & 4 & 2 & 3 & 0 & 2 & 0 & 0 & 4 & 0 & 5 & 9 & 27 \\ 
  NoData & 1 & 0 & 0 & 0 & 0 & 0 & 1 & 0 & 0 & 0 & 1 & 2 \\ 
  Praias e cantis & 2 & 2 & 1 & 0 & 0 & 0 & 0 & 0 & 0 & 15 & 11 & 15 \\ 
  Rururbano (diseminado) & 96 & 32 & 95 & 32 & 0 & 35 & 31 & 25 & 6 & 21 & 16 & 228 \\ 
  Turbeira & 27 & 25 & 1 & 0 & 2 & 89 & 7 & 23 & 1 & 211 & 10 & 4 \\ 
  Urbano & 71 & 11 & 19 & 32 & 1 & 34 & 51 & 12 & 2 & 13 & 24 & 106 \\ 
  Vinedo & 0 & 0 & 21 & 128 & 0 & 0 & 1 & 31 & 4 & 0 & 0 & 39 \\ 
   \hline
\end{tabular}
\end{table}
% latex table generated in R 3.2.5 by xtable 1.8-0 package
% Mon May  9 10:56:58 2016
\begin{table}[p]
\centering
\caption{Clases de cuberta por Grandes Áreas paisaxísticas (datos en porcentaxe)} 
\label{xtaboa2p}
\begin{tabular}{rrrrrrrrrrrrr}
  \hline
 & 1 & 2 & 3 & 4 & 5 & 6 & 7 & 8 & 9 & 10 & 11 & 12 \\ 
  \hline
 & 0.0 & 0.0 & 0.0 & 0.0 & 0.0 & 0.0 & 0.0 & 0.0 & 0.0 & 0.0 & 0.0 & 0.0 \\ 
  Agrosistema extensivo & 3.2 & 1.6 & 5.3 & 13.3 & 23.6 & 23.2 & 13.5 & 17.7 & 14.4 & 4.9 & 1.2 & 3.4 \\ 
  Agrosistema intensivo (mosaico agroforestal) & 46.5 & 28.5 & 16.0 & 12.5 & 12.5 & 40.4 & 50.7 & 2.4 & 2.1 & 25.9 & 48.2 & 22.8 \\ 
  Agrosistema intensivo (plantacion forestal) & 22.1 & 52.8 & 39.3 & 21.2 & 15.1 & 10.6 & 12.5 & 15.8 & 12.8 & 35.6 & 27.2 & 34.8 \\ 
  Agrosistema intensivo (superficie de cultivo) & 1.6 & 2.9 & 0.3 & 1.7 & 1.5 & 11.5 & 6.2 & 8.8 & 1.4 & 1.1 & 6.8 & 1.3 \\ 
  Bosque & 3.8 & 2.5 & 3.4 & 15.4 & 21.4 & 5.1 & 3.3 & 14.6 & 11.0 & 3.7 & 0.0 & 2.0 \\ 
  Conxunto Historico & 0.1 & 0.1 & 0.2 & 0.0 & 0.0 & 0.0 & 0.0 & 0.0 & 0.0 & 0.0 & 0.1 & 0.0 \\ 
  Extractivo & 0.1 & 0.2 & 0.7 & 0.2 & 0.2 & 0.1 & 0.2 & 0.1 & 1.0 & 0.5 & 0.1 & 0.1 \\ 
  Matogueira e rochedo & 6.5 & 3.4 & 22.4 & 25.9 & 25.4 & 5.6 & 11.5 & 36.4 & 55.2 & 11.0 & 12.4 & 19.9 \\ 
  no data & 0.3 & 0.2 & 0.3 & 0.0 & 0.1 & 0.0 & 0.0 & 0.1 & 0.0 & 0.3 & 0.4 & 1.0 \\ 
  NoData & 0.1 & 0.0 & 0.0 & 0.0 & 0.0 & 0.0 & 0.0 & 0.0 & 0.0 & 0.0 & 0.0 & 0.1 \\ 
  Praias e cantis & 0.1 & 0.2 & 0.1 & 0.0 & 0.0 & 0.0 & 0.0 & 0.0 & 0.0 & 0.9 & 0.5 & 0.6 \\ 
  Rururbano (diseminado) & 7.5 & 3.5 & 8.1 & 1.3 & 0.0 & 0.8 & 0.6 & 0.9 & 0.3 & 1.3 & 0.8 & 8.5 \\ 
  Turbeira & 2.1 & 2.7 & 0.1 & 0.0 & 0.1 & 1.9 & 0.1 & 0.8 & 0.0 & 13.0 & 0.5 & 0.1 \\ 
  Urbano & 5.5 & 1.2 & 1.6 & 1.3 & 0.0 & 0.8 & 1.0 & 0.4 & 0.1 & 0.8 & 1.2 & 3.9 \\ 
  Vinedo & 0.0 & 0.0 & 1.8 & 5.2 & 0.0 & 0.0 & 0.0 & 1.1 & 0.2 & 0.0 & 0.0 & 1.5 \\ 
   \hline
\end{tabular}
\end{table}
% latex table generated in R 3.2.5 by xtable 1.8-0 package
% Mon May  9 10:56:58 2016
\begin{table}[p]
\centering
\caption{Termotipos por Grandes Áreas paisaxísticas (datos en km²)} 
\label{xtaboa3}
\begin{tabular}{rrrrrrrrrrrrr}
  \hline
 & 1 & 2 & 3 & 4 & 5 & 6 & 7 & 8 & 9 & 10 & 11 & 12 \\ 
  \hline
 & 10 & 4 & 9 & 336 & 12 & 5 & 16 & 20 & 56 & 30 & 23 & 22 \\ 
  Mesotemperado inferior & 527 & 597 & 172 & 1031 & 323 & 1246 & 3561 & 1508 & 333 & 707 & 1291 & 593 \\ 
  Mesotemperado superior & 184 & 160 & 144 & 290 & 988 & 3154 & 1255 & 507 & 407 & 577 & 83 & 253 \\ 
  Supra e orotemperado & 3 & 0 & 10 & 258 & 1143 & 133 & 79 & 567 & 1405 & 20 & 0 & 2 \\ 
  Termotemperado & 564 & 161 & 846 & 556 & 26 & 21 & 241 & 239 & 0 & 288 & 670 & 1804 \\ 
   \hline
\end{tabular}
\end{table}
% latex table generated in R 3.2.5 by xtable 1.8-0 package
% Mon May  9 10:56:58 2016
\begin{table}[p]
\centering
\caption{Termotipos por Grandes Áreas paisaxísticas (datos en porcentaxe)} 
\label{xtaboa3p}
\begin{tabular}{rrrrrrrrrrrrr}
  \hline
 & 1 & 2 & 3 & 4 & 5 & 6 & 7 & 8 & 9 & 10 & 11 & 12 \\ 
  \hline
 & 0.8 & 0.5 & 0.7 & 13.6 & 0.5 & 0.1 & 0.3 & 0.7 & 2.5 & 1.8 & 1.1 & 0.8 \\ 
  Mesotemperado inferior & 40.8 & 64.6 & 14.5 & 41.7 & 13.0 & 27.3 & 69.1 & 53.0 & 15.1 & 43.5 & 62.2 & 22.0 \\ 
  Mesotemperado superior & 14.2 & 17.3 & 12.2 & 11.7 & 39.6 & 69.2 & 24.4 & 17.8 & 18.5 & 35.4 & 4.0 & 9.4 \\ 
  Supra e orotemperado & 0.2 & 0.0 & 0.8 & 10.4 & 45.8 & 2.9 & 1.5 & 19.9 & 63.8 & 1.3 & 0.0 & 0.1 \\ 
  Termotemperado & 43.7 & 17.4 & 71.5 & 22.5 & 1.1 & 0.5 & 4.7 & 8.4 & 0.0 & 17.7 & 32.3 & 66.8 \\ 
   \hline
\end{tabular}
\end{table}
