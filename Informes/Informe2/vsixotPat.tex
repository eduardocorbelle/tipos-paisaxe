% latex table generated in R 3.2.2 by xtable 1.8-0 package
% Tue Nov 24 18:51:08 2015
\begin{table}[p]
\centering
\caption{Frecuencia de aparición de valores patrimoniais identificados na participación pública e frecuencia de tipos asociados} 
\label{vsixotpat}
\begin{tabular}{lrr}
  \hline
Tipo de paisaxe & Frec. aparición (\%) & Frecuencia do tipo (\%) \\ 
  \hline
Vales sublitorais; Agrosistema intensivo (mosaico agroforestal); Mesotemperado inferior & 6.83 & 7.56 \\ 
  Litoral Cantabro-Atlantico; Rururbano (diseminado); Termotemperado & 5.08 & 3.30 \\ 
  Vales sublitorais; Rururbano (diseminado); Termotemperado & 4.38 & 1.68 \\ 
  Serras; Agrosistema extensivo; Mesotemperado superior & 4.03 & 5.77 \\ 
  Serras; Agrosistema extensivo; Mesotemperado inferior & 3.85 & 2.57 \\ 
  Vales sublitorais; Agrosistema intensivo (plantacion forestal); Mesotemperado inferior & 3.85 & 2.90 \\ 
  Vales sublitorais; Rururbano (diseminado); Mesotemperado inferior & 3.85 & 0.98 \\ 
  Serras; Agrosistema extensivo; Supra e orotemperado & 3.50 & 2.50 \\ 
  Vales sublitorais; Agrosistema extensivo; Mesotemperado inferior & 3.50 & 2.74 \\ 
  Vales sublitorais; Agrosistema intensivo (mosaico agroforestal); Termotemperado & 3.15 & 2.62 \\ 
  Serras; Matogueira e rochedo; Mesotemperado superior & 2.98 & 5.04 \\ 
  Chairas e vales interiores; Agrosistema extensivo; Mesotemperado inferior & 2.28 & 3.59 \\ 
  Serras; Matogueira e rochedo; Mesotemperado inferior & 1.93 & 2.73 \\ 
  Litoral Cantabro-Atlantico; Agrosistema intensivo (mosaico agroforestal); Termotemperado & 1.75 & 2.43 \\ 
  Canons; Bosque; Mesotemperado inferior & 1.58 & 0.30 \\ 
  Vales sublitorais; Agrosistema intensivo (plantacion forestal); Termotemperado & 1.58 & 1.79 \\ 
  Vales sublitorais; Agrosistema intensivo (superficie de cultivo); Mesotemperado inferior & 1.58 & 0.81 \\ 
  Chairas e vales interiores; Bosque; Mesotemperado inferior & 1.40 & 0.77 \\ 
  Chairas e vales interiores; Conxunto Historico; Termotemperado & 1.40 & 0.00 \\ 
  Chairas e vales interiores; Rururbano (diseminado); Mesotemperado inferior & 1.40 & 0.51 \\ 
  Litoral Cantabro-Atlantico; Conxunto Historico; Termotemperado & 1.40 & 0.02 \\ 
  Canons; Matogueira e rochedo; Mesomediterráneo & 1.23 & 0.19 \\ 
  Chairas e vales interiores; Agrosistema extensivo; Mesotemperado superior & 1.23 & 2.78 \\ 
  Chairas e vales interiores; Matogueira e rochedo; Mesotemperado inferior & 1.23 & 1.38 \\ 
  Serras; Matogueira e rochedo; Supra e orotemperado & 1.23 & 6.13 \\ 
  Vales sublitorais; Urbano; Mesotemperado inferior & 1.23 & 0.11 \\ 
  Canons; Viñedo; Mesotemperado inferior & 1.05 & 0.02 \\ 
  Chairas e vales interiores; Agrosistema intensivo (mosaico agroforestal); Mesotemperado inferior & 0.88 & 1.71 \\ 
  Chairas e vales interiores; Rururbano (diseminado); Mesotemperado superior & 0.88 & 0.30 \\ 
  Litoral Cantabro-Atlantico; Agrosistema intensivo (mosaico agroforestal); Mesotemperado inferior & 0.88 & 0.29 \\ 
  Litoral Cantabro-Atlantico; Agrosistema intensivo (plantacion forestal); Termotemperado & 0.88 & 1.32 \\ 
  Litoral Cantabro-Atlantico; Conxunto Historico; no data & 0.88 & 0.00 \\ 
  Serras; Bosque; Mesotemperado superior & 0.88 & 1.00 \\ 
  Serras; Turbeira; Mesotemperado superior & 0.88 & 0.71 \\ 
  Vales sublitorais; Bosque; Mesotemperado inferior & 0.88 & 0.50 \\ 
  Vales sublitorais; Conxunto Historico; Mesotemperado inferior & 0.88 & 0.00 \\ 
  Canons; Bosque; Termotemperado & 0.70 & 0.15 \\ 
  Canons; Viñedo; Mesomediterráneo & 0.70 & 0.01 \\ 
  Chairas e vales interiores; Agrosistema intensivo (mosaico agroforestal); Mesotemperado superior & 0.70 & 2.27 \\ 
  Chairas e vales interiores; Agrosistema intensivo (mosaico agroforestal); Termotemperado & 0.70 & 0.34 \\ 
  Chairas e vales interiores; Matogueira e rochedo; Termotemperado & 0.70 & 0.66 \\ 
  Serras; Agrosistema intensivo (mosaico agroforestal); Mesotemperado superior & 0.70 & 1.78 \\ 
  Serras; Conxunto Historico; Termotemperado & 0.70 & 0.01 \\ 
  Vales sublitorais; Agrosistema intensivo (plantacion forestal); Mesotemperado superior & 0.70 & 0.60 \\ 
  Vales sublitorais; Bosque; Mesotemperado superior & 0.70 & 0.39 \\ 
   & 0.70 &  \\ 
  Chairas e vales interiores; Agrosistema extensivo; Termotemperado & 0.53 & 0.61 \\ 
  Chairas e vales interiores; Bosque; Termotemperado & 0.53 & 0.29 \\ 
  Chairas e vales interiores; Rururbano (diseminado); Termotemperado & 0.53 & 0.42 \\ 
  Chairas e vales interiores; Urbano; Termotemperado & 0.53 & 0.08 \\ 
  Litoral Cantabro-Atlantico; Matogueira e rochedo; Mesotemperado inferior & 0.53 & 0.10 \\ 
  Litoral Cantabro-Atlantico; Rururbano (diseminado); Mesotemperado inferior & 0.53 & 0.04 \\ 
  Serras; Agrosistema intensivo (superficie de cultivo); Mesotemperado inferior & 0.53 & 0.28 \\ 
  Vales sublitorais; Matogueira e rochedo; Termotemperado & 0.53 & 0.93 \\ 
   \hline
\end{tabular}
\end{table}
