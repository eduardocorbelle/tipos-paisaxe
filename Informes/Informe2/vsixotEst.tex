% latex table generated in R 3.2.2 by xtable 1.8-0 package
% Tue Nov 24 18:51:08 2015
\begin{table}[p]
\centering
\caption{Frecuencia de aparición de valores estéticos identificados na participación pública e frecuencia de tipos asociados} 
\label{vsixotest}
\begin{tabular}{lrr}
  \hline
Tipo de paisaxe & Frec. aparición (\%) & Frecuencia do tipo (\%) \\ 
  \hline
Serras; Matogueira e rochedo; Mesotemperado superior & 5.70 & 5.04 \\ 
  Serras; Matogueira e rochedo; Supra e orotemperado & 4.49 & 6.13 \\ 
  Serras; Matogueira e rochedo; Mesotemperado inferior & 4.24 & 2.73 \\ 
  Vales sublitorais; Agrosistema extensivo; Mesotemperado inferior & 4.06 & 2.74 \\ 
  Serras; Agrosistema extensivo; Mesotemperado superior & 3.89 & 5.77 \\ 
  Vales sublitorais; Agrosistema intensivo (plantacion forestal); Mesotemperado inferior & 3.80 & 2.90 \\ 
  Vales sublitorais; Agrosistema intensivo (mosaico agroforestal); Mesotemperado inferior & 3.20 & 7.56 \\ 
  Vales sublitorais; Agrosistema intensivo (mosaico agroforestal); Termotemperado & 3.03 & 2.62 \\ 
  Serras; Agrosistema extensivo; Mesotemperado inferior & 2.94 & 2.57 \\ 
  Litoral Cantabro-Atlantico; Rururbano (diseminado); Termotemperado & 2.77 & 3.30 \\ 
  Vales sublitorais; Rururbano (diseminado); Termotemperado & 2.77 & 1.68 \\ 
  Vales sublitorais; Agrosistema intensivo (plantacion forestal); Termotemperado & 2.33 & 1.79 \\ 
  Serras; Agrosistema extensivo; Supra e orotemperado & 2.25 & 2.50 \\ 
  Vales sublitorais; Matogueira e rochedo; Mesotemperado inferior & 1.99 & 1.90 \\ 
  Chairas e vales interiores; Agrosistema extensivo; Mesotemperado inferior & 1.82 & 3.59 \\ 
  Litoral Cantabro-Atlantico; Agrosistema intensivo (plantacion forestal); Termotemperado & 1.82 & 1.32 \\ 
  Litoral Cantabro-Atlantico; Matogueira e rochedo; Termotemperado & 1.82 & 0.83 \\ 
  Vales sublitorais; Rururbano (diseminado); Mesotemperado inferior & 1.73 & 0.98 \\ 
  Vales sublitorais; Bosque; Mesotemperado inferior & 1.38 & 0.50 \\ 
  Canons; Bosque; Mesotemperado inferior & 1.21 & 0.30 \\ 
  Serras; Bosque; Mesotemperado superior & 1.21 & 1.00 \\ 
  Vales sublitorais; Urbano; Mesotemperado inferior & 1.12 & 0.11 \\ 
  Chairas e vales interiores; Matogueira e rochedo; Mesotemperado inferior & 1.04 & 1.38 \\ 
  Litoral Cantabro-Atlantico; Urbano; Termotemperado & 1.04 & 0.53 \\ 
  Serras; Agrosistema intensivo (plantacion forestal); Supra e orotemperado & 1.04 & 0.93 \\ 
  Serras; Bosque; Supra e orotemperado & 1.04 & 0.74 \\ 
   & 1.04 &  \\ 
  Chairas e vales interiores; Bosque; Mesotemperado inferior & 0.95 & 0.77 \\ 
  Chairas e vales interiores; Bosque; Termotemperado & 0.95 & 0.29 \\ 
  Chairas e vales interiores; Rururbano (diseminado); Mesotemperado inferior & 0.95 & 0.51 \\ 
  Litoral Cantabro-Atlantico; Agrosistema intensivo (mosaico agroforestal); Termotemperado & 0.95 & 2.43 \\ 
  Litoral Cantabro-Atlantico; Conxunto Historico; Termotemperado & 0.95 & 0.02 \\ 
  Serras; Bosque; Mesotemperado inferior & 0.95 & 0.69 \\ 
  Vales sublitorais; Matogueira e rochedo; Termotemperado & 0.95 & 0.93 \\ 
  Chairas e vales interiores; Agrosistema extensivo; Mesotemperado superior & 0.86 & 2.78 \\ 
  Litoral Cantabro-Atlantico; Matogueira e rochedo; no data & 0.86 & 0.09 \\ 
  Serras; Agrosistema intensivo (plantacion forestal); Mesotemperado inferior & 0.86 & 0.83 \\ 
  Serras; Turbeira; Mesotemperado superior & 0.86 & 0.71 \\ 
  Serras; Turbeira; Supra e orotemperado & 0.86 & 0.40 \\ 
  Vales sublitorais; Agrosistema intensivo (superficie de cultivo); Mesotemperado inferior & 0.86 & 0.81 \\ 
  Canons; Bosque; Termotemperado & 0.78 & 0.15 \\ 
  Canons; Matogueira e rochedo; Mesomediterráneo & 0.69 & 0.19 \\ 
  Serras; Conxunto Historico; Termotemperado & 0.69 & 0.01 \\ 
  Canons; Matogueira e rochedo; Mesotemperado inferior & 0.61 & 0.20 \\ 
  Canons; Viñedo; Mesotemperado inferior & 0.61 & 0.02 \\ 
  Chairas e vales interiores; Conxunto Historico; Termotemperado & 0.61 & 0.00 \\ 
  Litoral Cantabro-Atlantico; Conxunto Historico; no data & 0.61 & 0.00 \\ 
  Serras; Turbeira; Mesotemperado inferior & 0.61 & 0.05 \\ 
  Vales sublitorais; Matogueira e rochedo; Mesotemperado superior & 0.61 & 0.85 \\ 
  Chairas e vales interiores; Agrosistema intensivo (mosaico agroforestal); Mesotemperado inferior & 0.52 & 1.71 \\ 
  Chairas e vales interiores; Matogueira e rochedo; Termotemperado & 0.52 & 0.66 \\ 
  Chairas e vales interiores; Rururbano (diseminado); Termotemperado & 0.52 & 0.42 \\ 
  Litoral Cantabro-Atlantico; Agrosistema intensivo (plantacion forestal); Mesotemperado inferior & 0.52 & 0.62 \\ 
  Serras; Agrosistema intensivo (plantacion forestal); Termotemperado & 0.52 & 0.15 \\ 
  Vales sublitorais; Agrosistema extensivo; Mesotemperado superior & 0.52 & 1.21 \\ 
  Vales sublitorais; Bosque; Termotemperado & 0.52 & 0.14 \\ 
  Vales sublitorais; Conxunto Historico; Mesotemperado inferior & 0.52 & 0.00 \\ 
   \hline
\end{tabular}
\end{table}
