\documentclass[11pt,a4paper]{article}
\usepackage[]{graphicx}
\usepackage[]{color}
\usepackage[english,spanish,galician,activeacute]{babel}
\usepackage[T1]{fontenc}
\usepackage[utf8x]{inputenc}
\usepackage{lmodern}
%\usepackage[adobe-utopia]{mathdesign}
\usepackage{url}
\usepackage[unicode=true, pdfusetitle, bookmarks=true, bookmarksnumbered=true, bookmarksopen=FALSE, 
            breaklinks=false, pdfborder={0 0 1}, backref=false, colorlinks=TRUE, linktocpage=TRUE, 
            allcolors=blue]{hyperref}
\usepackage{natbib}
\usepackage[textwidth=35mm, textsize=footnotesize, disable]{todonotes}
\usepackage{microtype}
\usepackage{booktabs}
\usepackage{longtable}
\usepackage{pdflscape}
\usepackage{morefloats}

\title{Tipos de paisaxe e valores da paisaxe}
\author{Eduardo Corbelle Rico\thanks{Laboratorio do territorio (LaboraTe), Departamento de Enxe\~nería Agroforestal, Universidade de Santiago de Compostela. Correo-e: \href{mailto:eduardo.corbelle@usc.es}{eduardo.corbelle@usc.es}.}}
\date{\today}
\graphicspath{{"./Figuras/"}}

\begin{document}

\maketitle




\section{Presentación}



\begin{landscape}
 \begin{footnotesize}

% latex table generated in R 3.2.2 by xtable 1.8-0 package
% Tue Nov 24 18:51:08 2015
\begin{table}[p]
\centering
\caption{Frecuencia de aparición de BIC e frecuencia de tipos asociados} 
\label{vbic}
\begin{tabular}{lrr}
  \hline
Tipo de paisaxe & Frec. aparición (\%) & Frecuencia do tipo (\%) \\ 
  \hline
Litoral Cantabro-Atlantico; Rururbano (diseminado); Termotemperado & 10.20 & 3.30 \\ 
  Litoral Cantabro-Atlantico; Conxunto Historico; Termotemperado & 7.33 & 0.02 \\ 
  Vales sublitorais; Matogueira e rochedo; Mesotemperado inferior & 6.08 & 1.90 \\ 
  Vales sublitorais; Agrosistema intensivo (mosaico agroforestal); Mesotemperado inferior & 5.19 & 7.56 \\ 
  Chairas e vales interiores; Agrosistema extensivo; Mesotemperado inferior & 5.01 & 3.59 \\ 
  Vales sublitorais; Rururbano (diseminado); Termotemperado & 5.01 & 1.68 \\ 
  Vales sublitorais; Matogueira e rochedo; Termotemperado & 4.11 & 0.93 \\ 
  Litoral Cantabro-Atlantico; Urbano; Termotemperado & 2.86 & 0.53 \\ 
  Serras; Agrosistema extensivo; Mesotemperado superior & 2.68 & 5.77 \\ 
  Litoral Cantabro-Atlantico; Agrosistema intensivo (mosaico agroforestal); Termotemperado & 2.50 & 2.43 \\ 
  Vales sublitorais; Agrosistema intensivo (plantacion forestal); Termotemperado & 2.50 & 1.79 \\ 
  Chairas e vales interiores; Conxunto Historico; Termotemperado & 2.15 & 0.00 \\ 
  Vales sublitorais; Conxunto Historico; Mesotemperado inferior & 2.15 & 0.00 \\ 
  Chairas e vales interiores; Rururbano (diseminado); Mesotemperado inferior & 1.97 & 0.51 \\ 
  Vales sublitorais; Agrosistema extensivo; Mesotemperado inferior & 1.97 & 2.74 \\ 
  Vales sublitorais; Agrosistema intensivo (mosaico agroforestal); Termotemperado & 1.97 & 2.62 \\ 
  Litoral Cantabro-Atlantico; Conxunto Historico; no data & 1.61 & 0.00 \\ 
  Litoral Cantabro-Atlantico; Urbano; no data & 1.61 & 0.04 \\ 
  Vales sublitorais; Rururbano (diseminado); Mesotemperado inferior & 1.61 & 0.98 \\ 
  Chairas e vales interiores; Agrosistema extensivo; Mesotemperado superior & 1.43 & 2.78 \\ 
  Serras; Agrosistema extensivo; Mesotemperado inferior & 1.43 & 2.57 \\ 
  Vales sublitorais; Agrosistema intensivo (plantacion forestal); Mesotemperado inferior & 1.43 & 2.90 \\ 
  Serras; Matogueira e rochedo; Mesotemperado inferior & 1.25 & 2.73 \\ 
  Canons; Bosque; Mesotemperado inferior & 1.07 & 0.30 \\ 
  Chairas e vales interiores; Agrosistema intensivo (mosaico agroforestal); Mesotemperado superior & 1.07 & 2.27 \\ 
  Chairas e vales interiores; Conxunto Historico; Mesotemperado superior & 1.07 & 0.00 \\ 
  Chairas e vales interiores; Agrosistema extensivo; Termotemperado & 0.89 & 0.61 \\ 
  Chairas e vales interiores; Agrosistema intensivo (mosaico agroforestal); Mesotemperado inferior & 0.89 & 1.71 \\ 
  Chairas e vales interiores; Conxunto Historico; Mesotemperado inferior & 0.89 & 0.00 \\ 
  Litoral Cantabro-Atlantico; Agrosistema intensivo (plantacion forestal); Termotemperado & 0.89 & 1.32 \\ 
  Serras; Agrosistema intensivo (plantacion forestal); Mesotemperado inferior & 0.89 & 0.83 \\ 
  Serras; Matogueira e rochedo; Mesotemperado superior & 0.89 & 5.04 \\ 
  Chairas e vales interiores; Rururbano (diseminado); Mesotemperado superior & 0.72 & 0.30 \\ 
  Chairas e vales interiores; Urbano; Termotemperado & 0.72 & 0.08 \\ 
  Litoral Cantabro-Atlantico; Matogueira e rochedo; Termotemperado & 0.72 & 0.83 \\ 
  Litoral Cantabro-Atlantico; Rururbano (diseminado); no data & 0.72 & 0.10 \\ 
  Chairas e vales interiores; Agrosistema intensivo (mosaico agroforestal); Termotemperado & 0.54 & 0.34 \\ 
  Chairas e vales interiores; Urbano; Mesotemperado inferior & 0.54 & 0.06 \\ 
  Litoral Cantabro-Atlantico; Agrosistema intensivo (mosaico agroforestal); Mesotemperado inferior & 0.54 & 0.29 \\ 
  Litoral Cantabro-Atlantico; Agrosistema intensivo (mosaico agroforestal); no data & 0.54 & 0.04 \\ 
  Litoral Cantabro-Atlantico; Matogueira e rochedo; no data & 0.54 & 0.09 \\ 
  Serras; Agrosistema extensivo; Supra e orotemperado & 0.54 & 2.50 \\ 
  Serras; Agrosistema intensivo (mosaico agroforestal); Mesotemperado superior & 0.54 & 1.78 \\ 
  Serras; Conxunto Historico; Termotemperado & 0.54 & 0.01 \\ 
  Serras; Matogueira e rochedo; Supra e orotemperado & 0.54 & 6.13 \\ 
  Vales sublitorais; Agrosistema intensivo (superficie de cultivo); Termotemperado & 0.54 & 0.10 \\ 
  Vales sublitorais; Urbano; Termotemperado & 0.54 & 0.07 \\ 
   & 0.54 &  \\ 
   \hline
\end{tabular}
\end{table}


% latex table generated in R 3.2.2 by xtable 1.8-0 package
% Tue Nov 24 18:51:08 2015
\begin{table}[p]
\centering
\caption{Frecuencia de aparición de valores naturais identificados na participación pública e frecuencia de tipos asociados} 
\label{vsixotnat}
\begin{tabular}{lrr}
  \hline
Tipo de paisaxe & Frec. aparición (\%) & Frecuencia do tipo (\%) \\ 
  \hline
 & 10.06 &  \\ 
  Litoral Cantabro-Atlantico; Matogueira e rochedo; Termotemperado & 6.98 & 0.83 \\ 
  Serras; Matogueira e rochedo; Mesotemperado superior & 5.85 & 5.04 \\ 
  Serras; Matogueira e rochedo; Supra e orotemperado & 5.44 & 6.13 \\ 
  Serras; Matogueira e rochedo; Mesotemperado inferior & 3.90 & 2.73 \\ 
  Litoral Cantabro-Atlantico; Rururbano (diseminado); Termotemperado & 3.59 & 3.30 \\ 
  Vales sublitorais; Agrosistema extensivo; Mesotemperado inferior & 3.29 & 2.74 \\ 
  Serras; Agrosistema extensivo; Mesotemperado superior & 2.77 & 5.77 \\ 
  Vales sublitorais; Agrosistema intensivo (plantacion forestal); Mesotemperado inferior & 2.77 & 2.90 \\ 
  Litoral Cantabro-Atlantico; Agrosistema intensivo (plantacion forestal); Termotemperado & 2.57 & 1.32 \\ 
  Vales sublitorais; Agrosistema intensivo (mosaico agroforestal); Termotemperado & 2.36 & 2.62 \\ 
  Serras; Agrosistema extensivo; Mesotemperado inferior & 2.16 & 2.57 \\ 
  Serras; Agrosistema extensivo; Supra e orotemperado & 2.16 & 2.50 \\ 
  Vales sublitorais; Rururbano (diseminado); Termotemperado & 2.16 & 1.68 \\ 
  Vales sublitorais; Matogueira e rochedo; Mesotemperado inferior & 1.85 & 1.90 \\ 
  Chairas e vales interiores; Agrosistema extensivo; Mesotemperado inferior & 1.75 & 3.59 \\ 
  Vales sublitorais; Agrosistema intensivo (plantacion forestal); Termotemperado & 1.75 & 1.79 \\ 
  Vales sublitorais; Agrosistema intensivo (mosaico agroforestal); Mesotemperado inferior & 1.64 & 7.56 \\ 
  Litoral Cantabro-Atlantico; Matogueira e rochedo; no data & 1.54 & 0.09 \\ 
  Serras; Bosque; Supra e orotemperado & 1.33 & 0.74 \\ 
  Vales sublitorais; Bosque; Mesotemperado inferior & 1.33 & 0.50 \\ 
  Serras; Bosque; Mesotemperado superior & 1.23 & 1.00 \\ 
  Serras; Turbeira; Mesotemperado superior & 1.23 & 0.71 \\ 
  Canons; Bosque; Mesotemperado inferior & 1.13 & 0.30 \\ 
  Chairas e vales interiores; Matogueira e rochedo; Mesotemperado inferior & 1.13 & 1.38 \\ 
  Litoral Cantabro-Atlantico; Agrosistema intensivo (mosaico agroforestal); Termotemperado & 1.13 & 2.43 \\ 
  Serras; Turbeira; Supra e orotemperado & 1.13 & 0.40 \\ 
  Serras; Agrosistema intensivo (plantacion forestal); Supra e orotemperado & 1.03 & 0.93 \\ 
  Chairas e vales interiores; Bosque; Mesotemperado inferior & 0.92 & 0.77 \\ 
  Chairas e vales interiores; Bosque; Termotemperado & 0.92 & 0.29 \\ 
  Serras; Agrosistema intensivo (plantacion forestal); Mesotemperado inferior & 0.92 & 0.83 \\ 
  Chairas e vales interiores; Agrosistema intensivo (mosaico agroforestal); Mesotemperado inferior & 0.82 & 1.71 \\ 
  Serras; Bosque; Mesotemperado inferior & 0.82 & 0.69 \\ 
  Canons; Bosque; Termotemperado & 0.72 & 0.15 \\ 
  Canons; Matogueira e rochedo; Mesotemperado inferior & 0.72 & 0.20 \\ 
  Litoral Cantabro-Atlantico; Agrosistema intensivo (mosaico agroforestal); no data & 0.62 & 0.04 \\ 
  Litoral Cantabro-Atlantico; Agrosistema intensivo (plantacion forestal); Mesotemperado inferior & 0.62 & 0.62 \\ 
  Litoral Cantabro-Atlantico; Conxunto Historico; Termotemperado & 0.62 & 0.02 \\ 
  Serras; Agrosistema intensivo (plantacion forestal); Termotemperado & 0.62 & 0.15 \\ 
  Serras; Turbeira; Mesotemperado inferior & 0.62 & 0.05 \\ 
  Vales sublitorais; Agrosistema extensivo; Mesotemperado superior & 0.62 & 1.21 \\ 
  Vales sublitorais; Bosque; Termotemperado & 0.62 & 0.14 \\ 
  Vales sublitorais; Matogueira e rochedo; Termotemperado & 0.62 & 0.93 \\ 
  Vales sublitorais; Urbano; Mesotemperado inferior & 0.62 & 0.11 \\ 
  Chairas e vales interiores; Viñedo; Termotemperado & 0.51 & 0.28 \\ 
  Vales sublitorais; Matogueira e rochedo; Mesotemperado superior & 0.51 & 0.85 \\ 
  Vales sublitorais; Rururbano (diseminado); Mesotemperado inferior & 0.51 & 0.98 \\ 
   \hline
\end{tabular}
\end{table}


% latex table generated in R 3.2.2 by xtable 1.8-0 package
% Tue Nov 24 18:51:08 2015
\begin{table}[p]
\centering
\caption{Frecuencia de aparición de valores patrimoniais identificados na participación pública e frecuencia de tipos asociados} 
\label{vsixotpat}
\begin{tabular}{lrr}
  \hline
Tipo de paisaxe & Frec. aparición (\%) & Frecuencia do tipo (\%) \\ 
  \hline
Vales sublitorais; Agrosistema intensivo (mosaico agroforestal); Mesotemperado inferior & 6.83 & 7.56 \\ 
  Litoral Cantabro-Atlantico; Rururbano (diseminado); Termotemperado & 5.08 & 3.30 \\ 
  Vales sublitorais; Rururbano (diseminado); Termotemperado & 4.38 & 1.68 \\ 
  Serras; Agrosistema extensivo; Mesotemperado superior & 4.03 & 5.77 \\ 
  Serras; Agrosistema extensivo; Mesotemperado inferior & 3.85 & 2.57 \\ 
  Vales sublitorais; Agrosistema intensivo (plantacion forestal); Mesotemperado inferior & 3.85 & 2.90 \\ 
  Vales sublitorais; Rururbano (diseminado); Mesotemperado inferior & 3.85 & 0.98 \\ 
  Serras; Agrosistema extensivo; Supra e orotemperado & 3.50 & 2.50 \\ 
  Vales sublitorais; Agrosistema extensivo; Mesotemperado inferior & 3.50 & 2.74 \\ 
  Vales sublitorais; Agrosistema intensivo (mosaico agroforestal); Termotemperado & 3.15 & 2.62 \\ 
  Serras; Matogueira e rochedo; Mesotemperado superior & 2.98 & 5.04 \\ 
  Chairas e vales interiores; Agrosistema extensivo; Mesotemperado inferior & 2.28 & 3.59 \\ 
  Serras; Matogueira e rochedo; Mesotemperado inferior & 1.93 & 2.73 \\ 
  Litoral Cantabro-Atlantico; Agrosistema intensivo (mosaico agroforestal); Termotemperado & 1.75 & 2.43 \\ 
  Canons; Bosque; Mesotemperado inferior & 1.58 & 0.30 \\ 
  Vales sublitorais; Agrosistema intensivo (plantacion forestal); Termotemperado & 1.58 & 1.79 \\ 
  Vales sublitorais; Agrosistema intensivo (superficie de cultivo); Mesotemperado inferior & 1.58 & 0.81 \\ 
  Chairas e vales interiores; Bosque; Mesotemperado inferior & 1.40 & 0.77 \\ 
  Chairas e vales interiores; Conxunto Historico; Termotemperado & 1.40 & 0.00 \\ 
  Chairas e vales interiores; Rururbano (diseminado); Mesotemperado inferior & 1.40 & 0.51 \\ 
  Litoral Cantabro-Atlantico; Conxunto Historico; Termotemperado & 1.40 & 0.02 \\ 
  Canons; Matogueira e rochedo; Mesomediterráneo & 1.23 & 0.19 \\ 
  Chairas e vales interiores; Agrosistema extensivo; Mesotemperado superior & 1.23 & 2.78 \\ 
  Chairas e vales interiores; Matogueira e rochedo; Mesotemperado inferior & 1.23 & 1.38 \\ 
  Serras; Matogueira e rochedo; Supra e orotemperado & 1.23 & 6.13 \\ 
  Vales sublitorais; Urbano; Mesotemperado inferior & 1.23 & 0.11 \\ 
  Canons; Viñedo; Mesotemperado inferior & 1.05 & 0.02 \\ 
  Chairas e vales interiores; Agrosistema intensivo (mosaico agroforestal); Mesotemperado inferior & 0.88 & 1.71 \\ 
  Chairas e vales interiores; Rururbano (diseminado); Mesotemperado superior & 0.88 & 0.30 \\ 
  Litoral Cantabro-Atlantico; Agrosistema intensivo (mosaico agroforestal); Mesotemperado inferior & 0.88 & 0.29 \\ 
  Litoral Cantabro-Atlantico; Agrosistema intensivo (plantacion forestal); Termotemperado & 0.88 & 1.32 \\ 
  Litoral Cantabro-Atlantico; Conxunto Historico; no data & 0.88 & 0.00 \\ 
  Serras; Bosque; Mesotemperado superior & 0.88 & 1.00 \\ 
  Serras; Turbeira; Mesotemperado superior & 0.88 & 0.71 \\ 
  Vales sublitorais; Bosque; Mesotemperado inferior & 0.88 & 0.50 \\ 
  Vales sublitorais; Conxunto Historico; Mesotemperado inferior & 0.88 & 0.00 \\ 
  Canons; Bosque; Termotemperado & 0.70 & 0.15 \\ 
  Canons; Viñedo; Mesomediterráneo & 0.70 & 0.01 \\ 
  Chairas e vales interiores; Agrosistema intensivo (mosaico agroforestal); Mesotemperado superior & 0.70 & 2.27 \\ 
  Chairas e vales interiores; Agrosistema intensivo (mosaico agroforestal); Termotemperado & 0.70 & 0.34 \\ 
  Chairas e vales interiores; Matogueira e rochedo; Termotemperado & 0.70 & 0.66 \\ 
  Serras; Agrosistema intensivo (mosaico agroforestal); Mesotemperado superior & 0.70 & 1.78 \\ 
  Serras; Conxunto Historico; Termotemperado & 0.70 & 0.01 \\ 
  Vales sublitorais; Agrosistema intensivo (plantacion forestal); Mesotemperado superior & 0.70 & 0.60 \\ 
  Vales sublitorais; Bosque; Mesotemperado superior & 0.70 & 0.39 \\ 
   & 0.70 &  \\ 
  Chairas e vales interiores; Agrosistema extensivo; Termotemperado & 0.53 & 0.61 \\ 
  Chairas e vales interiores; Bosque; Termotemperado & 0.53 & 0.29 \\ 
  Chairas e vales interiores; Rururbano (diseminado); Termotemperado & 0.53 & 0.42 \\ 
  Chairas e vales interiores; Urbano; Termotemperado & 0.53 & 0.08 \\ 
  Litoral Cantabro-Atlantico; Matogueira e rochedo; Mesotemperado inferior & 0.53 & 0.10 \\ 
  Litoral Cantabro-Atlantico; Rururbano (diseminado); Mesotemperado inferior & 0.53 & 0.04 \\ 
  Serras; Agrosistema intensivo (superficie de cultivo); Mesotemperado inferior & 0.53 & 0.28 \\ 
  Vales sublitorais; Matogueira e rochedo; Termotemperado & 0.53 & 0.93 \\ 
   \hline
\end{tabular}
\end{table}


% latex table generated in R 3.2.2 by xtable 1.8-0 package
% Tue Nov 24 18:51:08 2015
\begin{table}[p]
\centering
\caption{Frecuencia de aparición de valores estéticos identificados na participación pública e frecuencia de tipos asociados} 
\label{vsixotest}
\begin{tabular}{lrr}
  \hline
Tipo de paisaxe & Frec. aparición (\%) & Frecuencia do tipo (\%) \\ 
  \hline
Serras; Matogueira e rochedo; Mesotemperado superior & 5.70 & 5.04 \\ 
  Serras; Matogueira e rochedo; Supra e orotemperado & 4.49 & 6.13 \\ 
  Serras; Matogueira e rochedo; Mesotemperado inferior & 4.24 & 2.73 \\ 
  Vales sublitorais; Agrosistema extensivo; Mesotemperado inferior & 4.06 & 2.74 \\ 
  Serras; Agrosistema extensivo; Mesotemperado superior & 3.89 & 5.77 \\ 
  Vales sublitorais; Agrosistema intensivo (plantacion forestal); Mesotemperado inferior & 3.80 & 2.90 \\ 
  Vales sublitorais; Agrosistema intensivo (mosaico agroforestal); Mesotemperado inferior & 3.20 & 7.56 \\ 
  Vales sublitorais; Agrosistema intensivo (mosaico agroforestal); Termotemperado & 3.03 & 2.62 \\ 
  Serras; Agrosistema extensivo; Mesotemperado inferior & 2.94 & 2.57 \\ 
  Litoral Cantabro-Atlantico; Rururbano (diseminado); Termotemperado & 2.77 & 3.30 \\ 
  Vales sublitorais; Rururbano (diseminado); Termotemperado & 2.77 & 1.68 \\ 
  Vales sublitorais; Agrosistema intensivo (plantacion forestal); Termotemperado & 2.33 & 1.79 \\ 
  Serras; Agrosistema extensivo; Supra e orotemperado & 2.25 & 2.50 \\ 
  Vales sublitorais; Matogueira e rochedo; Mesotemperado inferior & 1.99 & 1.90 \\ 
  Chairas e vales interiores; Agrosistema extensivo; Mesotemperado inferior & 1.82 & 3.59 \\ 
  Litoral Cantabro-Atlantico; Agrosistema intensivo (plantacion forestal); Termotemperado & 1.82 & 1.32 \\ 
  Litoral Cantabro-Atlantico; Matogueira e rochedo; Termotemperado & 1.82 & 0.83 \\ 
  Vales sublitorais; Rururbano (diseminado); Mesotemperado inferior & 1.73 & 0.98 \\ 
  Vales sublitorais; Bosque; Mesotemperado inferior & 1.38 & 0.50 \\ 
  Canons; Bosque; Mesotemperado inferior & 1.21 & 0.30 \\ 
  Serras; Bosque; Mesotemperado superior & 1.21 & 1.00 \\ 
  Vales sublitorais; Urbano; Mesotemperado inferior & 1.12 & 0.11 \\ 
  Chairas e vales interiores; Matogueira e rochedo; Mesotemperado inferior & 1.04 & 1.38 \\ 
  Litoral Cantabro-Atlantico; Urbano; Termotemperado & 1.04 & 0.53 \\ 
  Serras; Agrosistema intensivo (plantacion forestal); Supra e orotemperado & 1.04 & 0.93 \\ 
  Serras; Bosque; Supra e orotemperado & 1.04 & 0.74 \\ 
   & 1.04 &  \\ 
  Chairas e vales interiores; Bosque; Mesotemperado inferior & 0.95 & 0.77 \\ 
  Chairas e vales interiores; Bosque; Termotemperado & 0.95 & 0.29 \\ 
  Chairas e vales interiores; Rururbano (diseminado); Mesotemperado inferior & 0.95 & 0.51 \\ 
  Litoral Cantabro-Atlantico; Agrosistema intensivo (mosaico agroforestal); Termotemperado & 0.95 & 2.43 \\ 
  Litoral Cantabro-Atlantico; Conxunto Historico; Termotemperado & 0.95 & 0.02 \\ 
  Serras; Bosque; Mesotemperado inferior & 0.95 & 0.69 \\ 
  Vales sublitorais; Matogueira e rochedo; Termotemperado & 0.95 & 0.93 \\ 
  Chairas e vales interiores; Agrosistema extensivo; Mesotemperado superior & 0.86 & 2.78 \\ 
  Litoral Cantabro-Atlantico; Matogueira e rochedo; no data & 0.86 & 0.09 \\ 
  Serras; Agrosistema intensivo (plantacion forestal); Mesotemperado inferior & 0.86 & 0.83 \\ 
  Serras; Turbeira; Mesotemperado superior & 0.86 & 0.71 \\ 
  Serras; Turbeira; Supra e orotemperado & 0.86 & 0.40 \\ 
  Vales sublitorais; Agrosistema intensivo (superficie de cultivo); Mesotemperado inferior & 0.86 & 0.81 \\ 
  Canons; Bosque; Termotemperado & 0.78 & 0.15 \\ 
  Canons; Matogueira e rochedo; Mesomediterráneo & 0.69 & 0.19 \\ 
  Serras; Conxunto Historico; Termotemperado & 0.69 & 0.01 \\ 
  Canons; Matogueira e rochedo; Mesotemperado inferior & 0.61 & 0.20 \\ 
  Canons; Viñedo; Mesotemperado inferior & 0.61 & 0.02 \\ 
  Chairas e vales interiores; Conxunto Historico; Termotemperado & 0.61 & 0.00 \\ 
  Litoral Cantabro-Atlantico; Conxunto Historico; no data & 0.61 & 0.00 \\ 
  Serras; Turbeira; Mesotemperado inferior & 0.61 & 0.05 \\ 
  Vales sublitorais; Matogueira e rochedo; Mesotemperado superior & 0.61 & 0.85 \\ 
  Chairas e vales interiores; Agrosistema intensivo (mosaico agroforestal); Mesotemperado inferior & 0.52 & 1.71 \\ 
  Chairas e vales interiores; Matogueira e rochedo; Termotemperado & 0.52 & 0.66 \\ 
  Chairas e vales interiores; Rururbano (diseminado); Termotemperado & 0.52 & 0.42 \\ 
  Litoral Cantabro-Atlantico; Agrosistema intensivo (plantacion forestal); Mesotemperado inferior & 0.52 & 0.62 \\ 
  Serras; Agrosistema intensivo (plantacion forestal); Termotemperado & 0.52 & 0.15 \\ 
  Vales sublitorais; Agrosistema extensivo; Mesotemperado superior & 0.52 & 1.21 \\ 
  Vales sublitorais; Bosque; Termotemperado & 0.52 & 0.14 \\ 
  Vales sublitorais; Conxunto Historico; Mesotemperado inferior & 0.52 & 0.00 \\ 
   \hline
\end{tabular}
\end{table}


% latex table generated in R 3.2.2 by xtable 1.8-0 package
% Tue Nov 24 18:51:08 2015
\begin{table}[p]
\centering
\caption{Frecuencia de aparición dos Camiños de Santiago (área de influencia de 500 m a ambos lados) e frecuencia de tipos asociados} 
\label{vcamino}
\begin{tabular}{lrr}
  \hline
Tipo de paisaxe & Frec. aparición (\%) & Frecuencia do tipo (\%) \\ 
  \hline
Vales sublitorais; Agrosistema intensivo (mosaico agroforestal); Mesotemperado inferior & 11.29 & 7.55 \\ 
  Litoral Cantabro-Atlantico; Rururbano (diseminado); Termotemperado & 6.22 & 3.29 \\ 
  Serras; Agrosistema extensivo; Mesotemperado superior & 5.12 & 5.77 \\ 
  Serras; Agrosistema extensivo; Supra e orotemperado & 4.00 & 2.50 \\ 
  Chairas e vales interiores; Agrosistema extensivo; Mesotemperado inferior & 3.82 & 3.59 \\ 
  Chairas e vales interiores; Agrosistema extensivo; Mesotemperado superior & 3.80 & 2.78 \\ 
  Vales sublitorais; Agrosistema extensivo; Mesotemperado inferior & 3.79 & 2.74 \\ 
  Vales sublitorais; Rururbano (diseminado); Mesotemperado inferior & 3.68 & 0.98 \\ 
  Vales sublitorais; Agrosistema intensivo (mosaico agroforestal); Termotemperado & 3.47 & 2.62 \\ 
  Vales sublitorais; Rururbano (diseminado); Termotemperado & 3.26 & 1.68 \\ 
  Serras; Matogueira e rochedo; Supra e orotemperado & 3.22 & 6.13 \\ 
  Litoral Cantabro-Atlantico; Agrosistema intensivo (mosaico agroforestal); Termotemperado & 3.21 & 2.43 \\ 
  Serras; Agrosistema intensivo (mosaico agroforestal); Mesotemperado superior & 2.23 & 1.78 \\ 
  Chairas e vales interiores; Agrosistema intensivo (mosaico agroforestal); Mesotemperado superior & 2.11 & 2.27 \\ 
  Vales sublitorais; Agrosistema intensivo (plantacion forestal); Mesotemperado inferior & 2.11 & 2.90 \\ 
  Litoral Cantabro-Atlantico; Urbano; Termotemperado & 2.03 & 0.53 \\ 
  Vales sublitorais; Agrosistema intensivo (superficie de cultivo); Mesotemperado inferior & 1.90 & 0.81 \\ 
  Serras; Agrosistema extensivo; Mesotemperado inferior & 1.87 & 2.57 \\ 
  Chairas e vales interiores; Agrosistema intensivo (superficie de cultivo); Mesotemperado inferior & 1.69 & 1.16 \\ 
  Serras; Agrosistema intensivo (superficie de cultivo); Mesotemperado superior & 1.62 & 0.95 \\ 
  Vales sublitorais; Agrosistema intensivo (mosaico agroforestal); Mesotemperado superior & 1.59 & 1.52 \\ 
  Chairas e vales interiores; Agrosistema intensivo (mosaico agroforestal); Mesotemperado inferior & 1.55 & 1.71 \\ 
  Chairas e vales interiores; Rururbano (diseminado); Mesotemperado inferior & 1.51 & 0.51 \\ 
  Serras; Matogueira e rochedo; Mesotemperado superior & 1.44 & 5.04 \\ 
  Vales sublitorais; Agrosistema extensivo; Mesotemperado superior & 1.34 & 1.21 \\ 
  Serras; Matogueira e rochedo; Mesotemperado inferior & 1.20 & 2.73 \\ 
   & 1.13 &  \\ 
  Vales sublitorais; Matogueira e rochedo; Mesotemperado inferior & 0.99 & 1.90 \\ 
  Litoral Cantabro-Atlantico; Agrosistema intensivo (plantacion forestal); Termotemperado & 0.95 & 1.32 \\ 
  Vales sublitorais; Urbano; Mesotemperado inferior & 0.88 & 0.11 \\ 
  Vales sublitorais; Agrosistema intensivo (plantacion forestal); Termotemperado & 0.87 & 1.79 \\ 
  Serras; Agrosistema intensivo (mosaico agroforestal); Supra e orotemperado & 0.82 & 0.37 \\ 
  Chairas e vales interiores; Urbano; Termotemperado & 0.70 & 0.08 \\ 
  Chairas e vales interiores; Rururbano (diseminado); Mesotemperado superior & 0.69 & 0.30 \\ 
  Serras; Agrosistema intensivo (plantacion forestal); Supra e orotemperado & 0.69 & 0.93 \\ 
  Chairas e vales interiores; Rururbano (diseminado); Termotemperado & 0.68 & 0.42 \\ 
  Chairas e vales interiores; Matogueira e rochedo; Mesotemperado inferior & 0.60 & 1.38 \\ 
  Litoral Cantabro-Atlantico; Matogueira e rochedo; Termotemperado & 0.59 & 0.83 \\ 
  Serras; Agrosistema intensivo (mosaico agroforestal); Mesotemperado inferior & 0.55 & 0.50 \\ 
  Vales sublitorais; Matogueira e rochedo; Termotemperado & 0.54 & 0.93 \\ 
   \hline
\end{tabular}
\end{table}


% latex table generated in R 3.2.2 by xtable 1.8-0 package
% Tue Nov 24 18:51:08 2015
\begin{table}[p]
\centering
\caption{Frecuencia de aparición de Lugares de Importancia Comunitaria e frecuencia de tipos asociados} 
\label{vnatura}
\begin{tabular}{lrr}
  \hline
Tipo de paisaxe & Frec. aparición (\%) & Frecuencia do tipo (\%) \\ 
  \hline
Serras; Matogueira e rochedo; Supra e orotemperado & 24.32 & 6.05 \\ 
   & 9.80 &  \\ 
  Serras; Agrosistema extensivo; Supra e orotemperado & 7.88 & 2.47 \\ 
  Serras; Matogueira e rochedo; Mesotemperado superior & 7.72 & 4.97 \\ 
  Serras; Agrosistema extensivo; Mesotemperado superior & 4.10 & 5.69 \\ 
  Serras; Bosque; Supra e orotemperado & 3.83 & 0.73 \\ 
  Serras; Turbeira; Mesotemperado superior & 2.46 & 0.70 \\ 
  Serras; Matogueira e rochedo; Mesotemperado inferior & 2.44 & 2.70 \\ 
  Litoral Cantabro-Atlantico; Matogueira e rochedo; Termotemperado & 2.37 & 0.82 \\ 
  Serras; Turbeira; Supra e orotemperado & 2.23 & 0.39 \\ 
  Serras; Agrosistema intensivo (plantacion forestal); Supra e orotemperado & 2.03 & 0.92 \\ 
  Serras; Bosque; Mesotemperado superior & 1.76 & 0.99 \\ 
  Vales sublitorais; Agrosistema extensivo; Mesotemperado inferior & 1.60 & 2.70 \\ 
  Chairas e vales interiores; Agrosistema intensivo (superficie de cultivo); Mesotemperado inferior & 1.39 & 1.15 \\ 
  Vales sublitorais; Matogueira e rochedo; Mesotemperado inferior & 1.28 & 1.87 \\ 
  Serras; Agrosistema intensivo (plantacion forestal); Mesotemperado superior & 1.19 & 1.06 \\ 
  Vales sublitorais; Agrosistema extensivo; Mesotemperado superior & 1.18 & 1.19 \\ 
  Serras; Agrosistema extensivo; Mesotemperado inferior & 0.94 & 2.54 \\ 
  Vales sublitorais; Bosque; Mesotemperado inferior & 0.87 & 0.49 \\ 
  Litoral Cantabro-Atlantico; Agrosistema intensivo (mosaico agroforestal); Termotemperado & 0.85 & 2.40 \\ 
  Canons; Bosque; Mesotemperado inferior & 0.81 & 0.29 \\ 
  Litoral Cantabro-Atlantico; Rururbano (diseminado); Termotemperado & 0.79 & 3.25 \\ 
  Vales sublitorais; Bosque; Mesotemperado superior & 0.77 & 0.38 \\ 
  Chairas e vales interiores; Agrosistema extensivo; Mesotemperado inferior & 0.77 & 3.54 \\ 
  Chairas e vales interiores; Agrosistema extensivo; Mesotemperado superior & 0.72 & 2.74 \\ 
  Serras; Bosque; Mesotemperado inferior & 0.69 & 0.68 \\ 
  Vales sublitorais; Agrosistema intensivo (mosaico agroforestal); Mesotemperado inferior & 0.65 & 7.46 \\ 
  Litoral Cantabro-Atlantico; Agrosistema intensivo (plantacion forestal); Termotemperado & 0.56 & 1.30 \\ 
  Serras; Agrosistema intensivo (mosaico agroforestal); Supra e orotemperado & 0.53 & 0.37 \\ 
  Vales sublitorais; Matogueira e rochedo; Mesotemperado superior & 0.51 & 0.84 \\ 
   \hline
\end{tabular}
\end{table}


% latex table generated in R 3.2.2 by xtable 1.8-0 package
% Tue Nov 24 18:51:08 2015
\begin{table}[p]
\centering
\caption{Frecuencia de aparición de xeneradores eólicos e frecuencia de tipos asociados} 
\label{veolico}
\begin{tabular}{lrr}
  \hline
Tipo de paisaxe & Frec. aparición (\%) & Frecuencia do tipo (\%) \\ 
  \hline
Serras; Matogueira e rochedo; Mesotemperado superior & 17.23 & 5.04 \\ 
  Serras; Matogueira e rochedo; Supra e orotemperado & 16.64 & 6.13 \\ 
  Serras; Turbeira; Mesotemperado superior & 12.21 & 0.71 \\ 
  Serras; Turbeira; Supra e orotemperado & 10.37 & 0.40 \\ 
  Vales sublitorais; Matogueira e rochedo; Mesotemperado superior & 5.39 & 0.85 \\ 
  Vales sublitorais; Matogueira e rochedo; Mesotemperado inferior & 4.56 & 1.90 \\ 
  Litoral Cantabro-Atlantico; Matogueira e rochedo; Termotemperado & 3.97 & 0.83 \\ 
  Serras; Matogueira e rochedo; Mesotemperado inferior & 3.43 & 2.73 \\ 
  Vales sublitorais; Agrosistema intensivo (plantacion forestal); Mesotemperado superior & 3.14 & 0.60 \\ 
  Serras; Agrosistema intensivo (plantacion forestal); Mesotemperado superior & 2.92 & 1.08 \\ 
  Vales sublitorais; Agrosistema intensivo (plantacion forestal); Mesotemperado inferior & 2.75 & 2.90 \\ 
  Serras; Agrosistema intensivo (plantacion forestal); Supra e orotemperado & 2.38 & 0.93 \\ 
   & 2.06 &  \\ 
  Serras; Agrosistema extensivo; Supra e orotemperado & 1.99 & 2.50 \\ 
  Serras; Agrosistema intensivo (mosaico agroforestal); Mesotemperado superior & 1.37 & 1.78 \\ 
  Vales sublitorais; Turbeira; Mesotemperado superior & 1.25 & 0.06 \\ 
  Serras; Agrosistema extensivo; Mesotemperado superior & 1.20 & 5.77 \\ 
  Serras; Agrosistema intensivo (plantacion forestal); Mesotemperado inferior & 1.03 & 0.83 \\ 
  Serras; Turbeira; Mesotemperado inferior & 0.88 & 0.05 \\ 
  Vales sublitorais; Matogueira e rochedo; Termotemperado & 0.69 & 0.93 \\ 
  Serras; Agrosistema intensivo (mosaico agroforestal); Supra e orotemperado & 0.66 & 0.37 \\ 
  Vales sublitorais; Agrosistema intensivo (mosaico agroforestal); Mesotemperado superior & 0.61 & 1.52 \\ 
   \hline
\end{tabular}
\end{table}

 
 \end{footnotesize}
\end{landscape}
\end{document}
