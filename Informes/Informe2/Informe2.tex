\documentclass[11pt,a4paper]{article}
\usepackage[]{graphicx}
\usepackage[]{color}
\usepackage[english,spanish,galician]{babel}
\usepackage[T1]{fontenc}
\usepackage[utf8x]{inputenc}
\usepackage{fourier}
%\usepackage[adobe-utopia]{mathdesign}
\usepackage{url}
\usepackage[unicode=true, pdfusetitle, bookmarks=true, bookmarksnumbered=true, bookmarksopen=FALSE, 
            breaklinks=false, pdfborder={0 0 1}, backref=false, colorlinks=TRUE, linktocpage=TRUE, 
            allcolors=blue]{hyperref}
\usepackage{natbib}
\usepackage[textwidth=35mm, textsize=footnotesize, disable]{todonotes}
\usepackage{microtype}
\usepackage{booktabs}
\usepackage{longtable}
\usepackage{pdflscape}
\usepackage{morefloats}

\title{Tipos de paisaxe e valores da paisaxe}
\author{Eduardo Corbelle Rico\thanks{Laboratorio do territorio (LaboraTe), Departamento de Enxener\'ia Agroforestal, Universidade de Santiago de Compostela. Correo-e: \href{mailto:eduardo.corbelle@usc.es}{eduardo.corbelle@usc.es}.}}
\date{\today}
\graphicspath{{"./Figuras/"}}

\begin{document}

\maketitle




\section{Presentación}

Nos cadros das seguintes páxinas presentamos o resultado de cruzar varias capas de valores paisaxísticos co mapa de unidades de paisaxe. A maioría dos valores utilizados corresponden a entidades de puntos, coa única excepción da Rede Natura 2000, da que dispoñiamos en formato polígono, ou as diferentes trazas do Camiño de Santiago, para as que se considerou unha área de influencia de 500 metros a cada lado. Os valores utilizados ata o momento inclúen:

\begin{enumerate}
 \item Bens de Interese Cultural (BIC, cadros \ref{vbic1} a \ref{vbic12}).
 \item Valores naturais resultantes da participación pública (cadros \ref{vsixotnat1} a \ref{vsixotnat12}).
 \item Valores patrimoniais resultantes da participación pública (cadros \ref{vsixotpat1} a \ref{vsixotpat12}).
 \item Valores estéticos resultantes da participación pública (cadros \ref{vsixotest1} a \ref{vsixotest12}).
 \item Camiños de Santiago (área de influencia de 500 m a cada lado, cadros \ref{vcamino1} a \ref{vcamino12}).
 \item Rede Natura 2000 (cadros \ref{vnatura1} a \ref{vnatura12}).
 \item Xeneradores eólicos (cadros \ref{veolico1} a \ref{veolico12}).
\end{enumerate}

Os cálculos preséntanse desagregados por Grande Área Paisaxística (GAP). Para cada un dos casos citados, os cadros presentan cal é a distribución dos valores identificados por cada tipo de paisaxe (F. Aparic), a porcentaxe de aparición de cada tipo de paisaxe na GAP (F. Tipo), e a relación entre ambos (Ratio). O primeiro valor permite coñecer a que tipo de paisaxe está asociado con maior frecuencia cada un dos valores concretos. A segunda columna presenta a importancia de cada un dos tipos no conxunto da GAP. E a terceira columna permite identificar os tipos de maior concentración relativa de valores. Os cadros están ordenados pola columna F. Aparic., e presentan só os tipos de paisaxe que concentran máis dun 1\% dos puntos (ou área) de cada valor identificado.


\begin{landscape}
 \begin{footnotesize}

% latex table generated in R 3.2.2 by xtable 1.8-0 package
% Wed Dec  2 19:45:31 2015
\begin{table}[p]
\centering
\caption{Frecuencia de aparición de BIC e frecuencia de tipos asociados, GAP Golfo Ártabro} 
\label{vbic1}
\begin{tabular}{lrrr}
  \hline
Tipo de paisaxe & F.Aparic (\%) & F.Tipo (\%) & Ratio \\ 
  \hline
Litoral Cantabro-Atlantico; Rururbano (diseminado); Termotemperado & 29.63 & 3.29 & 9.00 \\ 
  Litoral Cantabro-Atlantico; Conxunto Historico; Termotemperado & 18.52 & 0.02 & 979.20 \\ 
  Litoral Cantabro-Atlantico; Conxunto Historico; no data & 11.11 & 0.00 & 4212.35 \\ 
  Litoral Cantabro-Atlantico; Agrosistema intensivo (mosaico agroforestal); Termotemperado & 5.56 & 2.43 & 2.29 \\ 
  Litoral Cantabro-Atlantico; Urbano; Termotemperado & 5.56 & 0.53 & 10.47 \\ 
  Litoral Cantabro-Atlantico; Agrosistema intensivo (mosaico agroforestal); no data & 3.70 & 0.04 & 87.10 \\ 
  Litoral Cantabro-Atlantico; Agrosistema intensivo (plantacion forestal); Termotemperado & 3.70 & 1.32 & 2.81 \\ 
   & 3.70 &  &  \\ 
  Canons; Bosque; Mesotemperado inferior & 1.85 & 0.30 & 6.27 \\ 
  Litoral Cantabro-Atlantico; Agrosistema intensivo (mosaico agroforestal); Mesotemperado inferior & 1.85 & 0.29 & 6.32 \\ 
  Litoral Cantabro-Atlantico; Matogueira e rochedo; no data & 1.85 & 0.09 & 21.76 \\ 
  Litoral Cantabro-Atlantico; Urbano; no data & 1.85 & 0.04 & 41.36 \\ 
  Serras; Agrosistema intensivo (mosaico agroforestal); Mesotemperado superior & 1.85 & 1.77 & 1.04 \\ 
  Vales sublitorais; Agrosistema intensivo (mosaico agroforestal); Mesotemperado inferior & 1.85 & 7.54 & 0.25 \\ 
  Vales sublitorais; Agrosistema intensivo (plantacion forestal); Termotemperado & 1.85 & 1.79 & 1.03 \\ 
  Vales sublitorais; Matogueira e rochedo; Termotemperado & 1.85 & 0.93 & 1.99 \\ 
  Vales sublitorais; Rururbano (diseminado); Mesotemperado inferior & 1.85 & 0.98 & 1.89 \\ 
  Vales sublitorais; Rururbano (diseminado); Termotemperado & 1.85 & 1.68 & 1.10 \\ 
   \hline
\end{tabular}
\end{table}
% latex table generated in R 3.2.2 by xtable 1.8-0 package
% Wed Dec  2 19:45:31 2015
\begin{table}[p]
\centering
\caption{Frecuencia de aparición de BIC e frecuencia de tipos asociados, GAP A Mariña - Baixo Eo} 
\label{vbic2}
\begin{tabular}{lrrr}
  \hline
Tipo de paisaxe & F.Aparic (\%) & F.Tipo (\%) & Ratio \\ 
  \hline
Litoral Cantabro-Atlantico; Conxunto Historico; Termotemperado & 31.25 & 0.02 & 1652.40 \\ 
  Litoral Cantabro-Atlantico; Agrosistema intensivo (mosaico agroforestal); Termotemperado & 12.50 & 2.43 & 5.15 \\ 
  Vales sublitorais; Agrosistema intensivo (superficie de cultivo); Termotemperado & 12.50 & 0.10 & 127.54 \\ 
  Vales sublitorais; Conxunto Historico; Mesotemperado inferior & 12.50 & 0.00 & 3642.39 \\ 
  Litoral Cantabro-Atlantico; Agrosistema intensivo (mosaico agroforestal); Mesotemperado inferior & 6.25 & 0.29 & 21.33 \\ 
  Litoral Cantabro-Atlantico; Matogueira e rochedo; Mesotemperado inferior & 6.25 & 0.10 & 62.79 \\ 
  Litoral Cantabro-Atlantico; Rururbano (diseminado); Termotemperado & 6.25 & 3.29 & 1.90 \\ 
  Vales sublitorais; Agrosistema intensivo (plantacion forestal); Mesotemperado inferior & 6.25 & 2.90 & 2.16 \\ 
  Vales sublitorais; Rururbano (diseminado); Termotemperado & 6.25 & 1.68 & 3.72 \\ 
   \hline
\end{tabular}
\end{table}
% latex table generated in R 3.2.2 by xtable 1.8-0 package
% Wed Dec  2 19:45:31 2015
\begin{table}[p]
\centering
\caption{Frecuencia de aparición de BIC e frecuencia de tipos asociados, GAP Costa Sur - Baixo Miño} 
\label{vbic3}
\begin{tabular}{lrrr}
  \hline
Tipo de paisaxe & F.Aparic (\%) & F.Tipo (\%) & Ratio \\ 
  \hline
Vales sublitorais; Matogueira e rochedo; Mesotemperado inferior & 38.46 & 1.89 & 20.31 \\ 
  Litoral Cantabro-Atlantico; Rururbano (diseminado); Termotemperado & 9.23 & 3.29 & 2.81 \\ 
  Vales sublitorais; Rururbano (diseminado); Termotemperado & 9.23 & 1.68 & 5.49 \\ 
  Vales sublitorais; Matogueira e rochedo; Termotemperado & 6.15 & 0.93 & 6.60 \\ 
  Litoral Cantabro-Atlantico; Conxunto Historico; Termotemperado & 4.62 & 0.02 & 244.05 \\ 
  Litoral Cantabro-Atlantico; Urbano; no data & 4.62 & 0.04 & 103.07 \\ 
  Serras; Conxunto Historico; Termotemperado & 4.62 & 0.01 & 576.94 \\ 
  Litoral Cantabro-Atlantico; Agrosistema intensivo (plantacion forestal); Termotemperado & 3.08 & 1.32 & 2.34 \\ 
  Litoral Cantabro-Atlantico; Urbano; Termotemperado & 3.08 & 0.53 & 5.80 \\ 
  Serras; Agrosistema intensivo (plantacion forestal); Mesotemperado inferior & 3.08 & 0.82 & 3.74 \\ 
  Serras; Matogueira e rochedo; Mesotemperado inferior & 3.08 & 2.73 & 1.13 \\ 
  Litoral Cantabro-Atlantico; Agrosistema intensivo (mosaico agroforestal); Termotemperado & 1.54 & 2.43 & 0.63 \\ 
  no data; Agrosistema intensivo (mosaico agroforestal); Termotemperado & 1.54 & 0.02 & 86.84 \\ 
  no data; Rururbano (diseminado); Termotemperado & 1.54 & 0.05 & 33.98 \\ 
  Serras; Agrosistema intensivo (plantacion forestal); Termotemperado & 1.54 & 0.15 & 10.09 \\ 
  Vales sublitorais; Agrosistema intensivo (mosaico agroforestal); Termotemperado & 1.54 & 2.62 & 0.59 \\ 
  Vales sublitorais; Agrosistema intensivo (plantacion forestal); Termotemperado & 1.54 & 1.79 & 0.86 \\ 
  Vales sublitorais; Urbano; Termotemperado & 1.54 & 0.07 & 21.25 \\ 
   \hline
\end{tabular}
\end{table}
% latex table generated in R 3.2.2 by xtable 1.8-0 package
% Wed Dec  2 19:45:31 2015
\begin{table}[p]
\centering
\caption{Frecuencia de aparición de BIC e frecuencia de tipos asociados, GAP Ribeiras Encaixadas do Miño e do Sil} 
\label{vbic4}
\begin{tabular}{lrrr}
  \hline
Tipo de paisaxe & F.Aparic (\%) & F.Tipo (\%) & Ratio \\ 
  \hline
Chairas e vales interiores; Conxunto Historico; Termotemperado & 20.41 & 0.00 & 7027.99 \\ 
  Chairas e vales interiores; Agrosistema extensivo; Mesotemperado inferior & 14.29 & 3.58 & 3.99 \\ 
  Canons; Bosque; Mesotemperado inferior & 10.20 & 0.30 & 34.54 \\ 
  Chairas e vales interiores; Rururbano (diseminado); Mesotemperado inferior & 8.16 & 0.51 & 15.95 \\ 
  Chairas e vales interiores; Agrosistema intensivo (mosaico agroforestal); Termotemperado & 6.12 & 0.34 & 18.04 \\ 
  Chairas e vales interiores; Urbano; Termotemperado & 6.12 & 0.08 & 74.59 \\ 
  Chairas e vales interiores; Agrosistema extensivo; Termotemperado & 4.08 & 0.61 & 6.66 \\ 
  Chairas e vales interiores; Agrosistema intensivo (mosaico agroforestal); Mesotemperado inferior & 4.08 & 1.71 & 2.39 \\ 
  Serras; Conxunto Historico; Mesotemperado inferior & 4.08 & 0.00 & 1966.12 \\ 
  Canons; Agrosistema intensivo (mosaico agroforestal); Mesotemperado inferior & 2.04 & 0.04 & 48.50 \\ 
  Canons; Agrosistema intensivo (plantacion forestal); Termotemperado & 2.04 & 0.07 & 28.45 \\ 
  Canons; Bosque; Termotemperado & 2.04 & 0.15 & 13.48 \\ 
  Canons; Matogueira e rochedo; Mesomediterráneo & 2.04 & 0.18 & 11.05 \\ 
  Canons; Matogueira e rochedo; Mesotemperado inferior & 2.04 & 0.20 & 10.19 \\ 
  Canons; Rururbano (diseminado); Mesotemperado inferior & 2.04 & 0.00 & 437.46 \\ 
  Chairas e vales interiores; Conxunto Historico; Mesotemperado inferior & 2.04 & 0.00 & 1479.86 \\ 
  Chairas e vales interiores; Rururbano (diseminado); Mesomediterráneo & 2.04 & 0.05 & 43.84 \\ 
  Chairas e vales interiores; Rururbano (diseminado); Termotemperado & 2.04 & 0.42 & 4.82 \\ 
  Chairas e vales interiores; Viñedo; Termotemperado & 2.04 & 0.28 & 7.35 \\ 
  Serras; Matogueira e rochedo; Mesotemperado inferior & 2.04 & 2.73 & 0.75 \\ 
   \hline
\end{tabular}
\end{table}
% latex table generated in R 3.2.2 by xtable 1.8-0 package
% Wed Dec  2 19:45:31 2015
\begin{table}[p]
\centering
\caption{Frecuencia de aparición de BIC e frecuencia de tipos asociados, GAP Serras Orientais} 
\label{vbic5}
\begin{tabular}{lrrr}
  \hline
Tipo de paisaxe & F.Aparic (\%) & F.Tipo (\%) & Ratio \\ 
  \hline
Serras; Agrosistema extensivo; Mesotemperado superior & 40.00 & 5.76 & 6.95 \\ 
  Chairas e vales interiores; Agrosistema extensivo; Mesotemperado superior & 10.00 & 2.77 & 3.60 \\ 
  Serras; Agrosistema extensivo; Supra e orotemperado & 10.00 & 2.50 & 4.00 \\ 
  Serras; Bosque; Supra e orotemperado & 10.00 & 0.74 & 13.58 \\ 
  Vales sublitorais; Bosque; Mesotemperado superior & 10.00 & 0.39 & 25.91 \\ 
  Chairas e vales interiores; Agrosistema extensivo; Mesomediterráneo & 5.00 & 0.11 & 45.59 \\ 
  Serras; Agrosistema intensivo (mosaico agroforestal); Mesotemperado superior & 5.00 & 1.77 & 2.82 \\ 
  Serras; Matogueira e rochedo; Mesotemperado superior & 5.00 & 5.03 & 0.99 \\ 
  Vales sublitorais; Bosque; Mesotemperado inferior & 5.00 & 0.49 & 10.10 \\ 
   \hline
\end{tabular}
\end{table}
% latex table generated in R 3.2.2 by xtable 1.8-0 package
% Wed Dec  2 19:45:31 2015
\begin{table}[p]
\centering
\caption{Frecuencia de aparición de BIC e frecuencia de tipos asociados, GAP Chairas e Fosas Luguesas} 
\label{vbic6}
\begin{tabular}{lrrr}
  \hline
Tipo de paisaxe & F.Aparic (\%) & F.Tipo (\%) & Ratio \\ 
  \hline
Chairas e vales interiores; Agrosistema extensivo; Mesotemperado inferior & 21.82 & 3.58 & 6.09 \\ 
  Chairas e vales interiores; Agrosistema extensivo; Mesotemperado superior & 10.91 & 2.77 & 3.93 \\ 
  Chairas e vales interiores; Agrosistema intensivo (mosaico agroforestal); Mesotemperado superior & 10.91 & 2.27 & 4.81 \\ 
  Chairas e vales interiores; Conxunto Historico; Mesotemperado superior & 10.91 & 0.00 & 9876.81 \\ 
  Chairas e vales interiores; Rururbano (diseminado); Mesotemperado superior & 7.27 & 0.30 & 24.26 \\ 
  Chairas e vales interiores; Rururbano (diseminado); Mesotemperado inferior & 5.45 & 0.51 & 10.65 \\ 
  Chairas e vales interiores; Agrosistema intensivo (mosaico agroforestal); Mesotemperado inferior & 3.64 & 1.71 & 2.13 \\ 
  Chairas e vales interiores; Conxunto Historico; Mesotemperado inferior & 3.64 & 0.00 & 2636.84 \\ 
  Chairas e vales interiores; Conxunto Historico; Termotemperado & 3.64 & 0.00 & 1252.26 \\ 
  Chairas e vales interiores; Urbano; Mesotemperado inferior & 3.64 & 0.06 & 65.24 \\ 
  Serras; Agrosistema extensivo; Mesotemperado superior & 3.64 & 5.76 & 0.63 \\ 
  Chairas e vales interiores; Agrosistema extensivo; Termotemperado & 1.82 & 0.61 & 2.96 \\ 
  Chairas e vales interiores; Agrosistema intensivo (plantacion forestal); Mesotemperado inferior & 1.82 & 0.55 & 3.30 \\ 
  Chairas e vales interiores; Agrosistema intensivo (plantacion forestal); Mesotemperado superior & 1.82 & 0.42 & 4.31 \\ 
  Chairas e vales interiores; Agrosistema intensivo (superficie de cultivo); Mesotemperado superior & 1.82 & 0.72 & 2.54 \\ 
  Chairas e vales interiores; Bosque; Mesotemperado inferior & 1.82 & 0.77 & 2.38 \\ 
  Serras; Agrosistema extensivo; Mesotemperado inferior & 1.82 & 2.57 & 0.71 \\ 
  Serras; Agrosistema intensivo (superficie de cultivo); Mesotemperado superior & 1.82 & 0.95 & 1.91 \\ 
  Serras; Rururbano (diseminado); Mesotemperado superior & 1.82 & 0.11 & 16.37 \\ 
   \hline
\end{tabular}
\end{table}
% latex table generated in R 3.2.2 by xtable 1.8-0 package
% Wed Dec  2 19:45:31 2015
\begin{table}[p]
\centering
\caption{Frecuencia de aparición de BIC e frecuencia de tipos asociados, GAP Galicia Central} 
\label{vbic7}
\begin{tabular}{lrrr}
  \hline
Tipo de paisaxe & F.Aparic (\%) & F.Tipo (\%) & Ratio \\ 
  \hline
Vales sublitorais; Conxunto Historico; Mesotemperado inferior & 16.95 & 0.00 & 4938.83 \\ 
  Vales sublitorais; Agrosistema extensivo; Mesotemperado inferior & 15.25 & 2.73 & 5.58 \\ 
  Vales sublitorais; Agrosistema intensivo (mosaico agroforestal); Mesotemperado inferior & 15.25 & 7.54 & 2.02 \\ 
  Vales sublitorais; Rururbano (diseminado); Mesotemperado inferior & 8.47 & 0.98 & 8.63 \\ 
  Serras; Agrosistema extensivo; Mesotemperado superior & 6.78 & 5.76 & 1.18 \\ 
  Vales sublitorais; Matogueira e rochedo; Mesotemperado inferior & 5.08 & 1.89 & 2.69 \\ 
  Serras; Agrosistema extensivo; Mesotemperado inferior & 3.39 & 2.57 & 1.32 \\ 
  Serras; Matogueira e rochedo; Supra e orotemperado & 3.39 & 6.12 & 0.55 \\ 
  Vales sublitorais; Agrosistema intensivo (mosaico agroforestal); Termotemperado & 3.39 & 2.62 & 1.29 \\ 
  Vales sublitorais; Agrosistema intensivo (superficie de cultivo); Mesotemperado inferior & 3.39 & 0.81 & 4.18 \\ 
  Chairas e vales interiores; Agrosistema extensivo; Termotemperado & 1.69 & 0.61 & 2.76 \\ 
  Chairas e vales interiores; Agrosistema intensivo (mosaico agroforestal); Mesotemperado inferior & 1.69 & 1.71 & 0.99 \\ 
  Chairas e vales interiores; Bosque; Termotemperado & 1.69 & 0.29 & 5.81 \\ 
  Chairas e vales interiores; Urbano; Termotemperado & 1.69 & 0.08 & 20.65 \\ 
  Serras; Agrosistema intensivo (mosaico agroforestal); Mesotemperado inferior & 1.69 & 0.50 & 3.41 \\ 
  Serras; Matogueira e rochedo; Mesotemperado superior & 1.69 & 5.03 & 0.34 \\ 
  Vales sublitorais; Agrosistema extensivo; Mesotemperado superior & 1.69 & 1.21 & 1.40 \\ 
  Vales sublitorais; Agrosistema extensivo; Termotemperado & 1.69 & 0.36 & 4.73 \\ 
  Vales sublitorais; Bosque; Mesotemperado inferior & 1.69 & 0.49 & 3.43 \\ 
  Vales sublitorais; Rururbano (diseminado); Termotemperado & 1.69 & 1.68 & 1.01 \\ 
  Vales sublitorais; Urbano; Mesotemperado inferior & 1.69 & 0.11 & 15.61 \\ 
   \hline
\end{tabular}
\end{table}
% latex table generated in R 3.2.2 by xtable 1.8-0 package
% Wed Dec  2 19:45:32 2015
\begin{table}[p]
\centering
\caption{Frecuencia de aparición de BIC e frecuencia de tipos asociados, GAP Chairas, Fosas e Serras Ourensás} 
\label{vbic8}
\begin{tabular}{lrrr}
  \hline
Tipo de paisaxe & F.Aparic (\%) & F.Tipo (\%) & Ratio \\ 
  \hline
Chairas e vales interiores; Agrosistema extensivo; Mesotemperado inferior & 36.00 & 3.58 & 10.04 \\ 
  Chairas e vales interiores; Rururbano (diseminado); Mesotemperado inferior & 16.00 & 0.51 & 31.25 \\ 
  Chairas e vales interiores; Agrosistema intensivo (superficie de cultivo); Mesotemperado inferior & 8.00 & 1.16 & 6.91 \\ 
  Chairas e vales interiores; Conxunto Historico; Mesotemperado inferior & 8.00 & 0.00 & 5801.05 \\ 
  Serras; Agrosistema extensivo; Mesotemperado inferior & 8.00 & 2.57 & 3.11 \\ 
  Chairas e vales interiores; Agrosistema extensivo; Termotemperado & 4.00 & 0.61 & 6.52 \\ 
  Chairas e vales interiores; Urbano; Mesotemperado inferior & 4.00 & 0.06 & 71.76 \\ 
  Serras; Agrosistema extensivo; Mesotemperado superior & 4.00 & 5.76 & 0.69 \\ 
  Serras; Agrosistema extensivo; Supra e orotemperado & 4.00 & 2.50 & 1.60 \\ 
  Serras; Matogueira e rochedo; no data & 4.00 & 0.07 & 54.26 \\ 
  Serras; Matogueira e rochedo; Supra e orotemperado & 4.00 & 6.12 & 0.65 \\ 
   \hline
\end{tabular}
\end{table}
% latex table generated in R 3.2.2 by xtable 1.8-0 package
% Wed Dec  2 19:45:32 2015
\begin{table}[p]
\centering
\caption{Frecuencia de aparición de BIC e frecuencia de tipos asociados, GAP Serras Surorientais} 
\label{vbic9}
\begin{tabular}{lrrr}
  \hline
Tipo de paisaxe & F.Aparic (\%) & F.Tipo (\%) & Ratio \\ 
  \hline
Canons; Agrosistema extensivo; Mesomediterráneo & 25.00 & 0.03 & 724.60 \\ 
  Canons; Matogueira e rochedo; Mesomediterráneo & 25.00 & 0.18 & 135.32 \\ 
  Canons; Rururbano (diseminado); Mesotemperado inferior & 25.00 & 0.00 & 5358.88 \\ 
  Serras; Agrosistema extensivo; Mesotemperado inferior & 25.00 & 2.57 & 9.73 \\ 
   \hline
\end{tabular}
\end{table}
% latex table generated in R 3.2.2 by xtable 1.8-0 package
% Wed Dec  2 19:45:32 2015
\begin{table}[p]
\centering
\caption{Frecuencia de aparición de BIC e frecuencia de tipos asociados, GAP Galicia Setentrional} 
\label{vbic10}
\begin{tabular}{lrrr}
  \hline
Tipo de paisaxe & F.Aparic (\%) & F.Tipo (\%) & Ratio \\ 
  \hline
Litoral Cantabro-Atlantico; Rururbano (diseminado); Termotemperado & 41.67 & 3.29 & 12.66 \\ 
  Litoral Cantabro-Atlantico; Agrosistema intensivo (mosaico agroforestal); Termotemperado & 16.67 & 2.43 & 6.86 \\ 
  Litoral Cantabro-Atlantico; Conxunto Historico; Termotemperado & 8.33 & 0.02 & 440.64 \\ 
  Litoral Cantabro-Atlantico; Rururbano (diseminado); no data & 8.33 & 0.10 & 83.53 \\ 
  Serras; Agrosistema intensivo (mosaico agroforestal); Mesotemperado superior & 8.33 & 1.77 & 4.70 \\ 
  Vales sublitorais; Agrosistema intensivo (mosaico agroforestal); Mesotemperado inferior & 8.33 & 7.54 & 1.10 \\ 
  Vales sublitorais; Rururbano (diseminado); Mesotemperado inferior & 8.33 & 0.98 & 8.49 \\ 
   \hline
\end{tabular}
\end{table}
% latex table generated in R 3.2.2 by xtable 1.8-0 package
% Wed Dec  2 19:45:32 2015
\begin{table}[p]
\centering
\caption{Frecuencia de aparición de BIC e frecuencia de tipos asociados, GAP Chairas e Fosas Occidentais} 
\label{vbic11}
\begin{tabular}{lrrr}
  \hline
Tipo de paisaxe & F.Aparic (\%) & F.Tipo (\%) & Ratio \\ 
  \hline
Vales sublitorais; Agrosistema intensivo (mosaico agroforestal); Mesotemperado inferior & 34.62 & 7.54 & 4.59 \\ 
  Litoral Cantabro-Atlantico; Agrosistema intensivo (mosaico agroforestal); Termotemperado & 11.54 & 2.43 & 4.75 \\ 
  Litoral Cantabro-Atlantico; Conxunto Historico; Termotemperado & 11.54 & 0.02 & 610.12 \\ 
  Vales sublitorais; Agrosistema intensivo (plantacion forestal); Mesotemperado inferior & 7.69 & 2.90 & 2.65 \\ 
  Litoral Cantabro-Atlantico; Agrosistema extensivo; no data & 3.85 & 0.01 & 353.97 \\ 
  Litoral Cantabro-Atlantico; Agrosistema intensivo (mosaico agroforestal); Mesotemperado inferior & 3.85 & 0.29 & 13.13 \\ 
  Litoral Cantabro-Atlantico; Agrosistema intensivo (superficie de cultivo); Termotemperado & 3.85 & 0.22 & 17.53 \\ 
  Litoral Cantabro-Atlantico; Rururbano (diseminado); Termotemperado & 3.85 & 3.29 & 1.17 \\ 
  Litoral Cantabro-Atlantico; Urbano; Termotemperado & 3.85 & 0.53 & 7.25 \\ 
  Vales sublitorais; Agrosistema extensivo; Mesotemperado inferior & 3.85 & 2.73 & 1.41 \\ 
  Vales sublitorais; Agrosistema intensivo (superficie de cultivo); Termotemperado & 3.85 & 0.10 & 39.24 \\ 
  Vales sublitorais; Rururbano (diseminado); Mesotemperado inferior & 3.85 & 0.98 & 3.92 \\ 
  Vales sublitorais; Rururbano (diseminado); Termotemperado & 3.85 & 1.68 & 2.29 \\ 
   \hline
\end{tabular}
\end{table}
% latex table generated in R 3.2.2 by xtable 1.8-0 package
% Wed Dec  2 19:45:32 2015
\begin{table}[p]
\centering
\caption{Frecuencia de aparición de BIC e frecuencia de tipos asociados, GAP Rías Baixas} 
\label{vbic12}
\begin{tabular}{lrrr}
  \hline
Tipo de paisaxe & F.Aparic (\%) & F.Tipo (\%) & Ratio \\ 
  \hline
Litoral Cantabro-Atlantico; Rururbano (diseminado); Termotemperado & 16.09 & 3.29 & 4.89 \\ 
  Litoral Cantabro-Atlantico; Conxunto Historico; Termotemperado & 10.92 & 0.02 & 577.39 \\ 
  Vales sublitorais; Matogueira e rochedo; Termotemperado & 10.34 & 0.93 & 11.09 \\ 
  Vales sublitorais; Rururbano (diseminado); Termotemperado & 10.34 & 1.68 & 6.15 \\ 
  Vales sublitorais; Agrosistema intensivo (plantacion forestal); Termotemperado & 6.90 & 1.79 & 3.85 \\ 
  Litoral Cantabro-Atlantico; Urbano; Termotemperado & 5.75 & 0.53 & 10.83 \\ 
  Vales sublitorais; Agrosistema intensivo (mosaico agroforestal); Mesotemperado inferior & 5.17 & 7.54 & 0.69 \\ 
  Vales sublitorais; Agrosistema intensivo (mosaico agroforestal); Termotemperado & 4.60 & 2.62 & 1.76 \\ 
  Vales sublitorais; Matogueira e rochedo; Mesotemperado inferior & 3.45 & 1.89 & 1.82 \\ 
  Litoral Cantabro-Atlantico; Urbano; no data & 2.87 & 0.04 & 64.17 \\ 
  Vales sublitorais; Agrosistema intensivo (plantacion forestal); Mesotemperado inferior & 2.87 & 2.90 & 0.99 \\ 
  Litoral Cantabro-Atlantico; Matogueira e rochedo; Termotemperado & 2.30 & 0.83 & 2.77 \\ 
  Serras; Matogueira e rochedo; Mesotemperado inferior & 2.30 & 2.73 & 0.84 \\ 
  Litoral Cantabro-Atlantico; Agrosistema intensivo (mosaico agroforestal); Termotemperado & 1.72 & 2.43 & 0.71 \\ 
  Litoral Cantabro-Atlantico; Conxunto Historico; no data & 1.72 & 0.00 & 653.64 \\ 
  Litoral Cantabro-Atlantico; Rururbano (diseminado); no data & 1.72 & 0.10 & 17.28 \\ 
  Serras; Agrosistema intensivo (plantacion forestal); Mesotemperado inferior & 1.72 & 0.82 & 2.09 \\ 
  Serras; Matogueira e rochedo; Mesotemperado superior & 1.72 & 5.03 & 0.34 \\ 
  Litoral Cantabro-Atlantico; Matogueira e rochedo; no data & 1.15 & 0.09 & 13.51 \\ 
  Serras; Agrosistema extensivo; Mesotemperado inferior & 1.15 & 2.57 & 0.45 \\ 
  Vales sublitorais; Urbano; Termotemperado & 1.15 & 0.07 & 15.88 \\ 
   \hline
\end{tabular}
\end{table}


\clearpage
% latex table generated in R 3.2.5 by xtable 1.8-0 package
% Mon May  9 13:23:06 2016
\begin{table}[p]
\centering
\caption{Frecuencia de aparición de valores naturais identificados na participación pública e frecuencia de tipos asociados Golfo Ártabro} 
\label{vsixotnat1}
\begin{tabular}{lrrr}
  \hline
Tipo de paisaxe & F.Aparic (\%) & F.Tipo (\%) & Ratio \\ 
  \hline
Canons; Bosque; Mesotemperado inferior & 13.89 & 1.56 & 8.90 \\ 
  Vales sublitorais; Bosque; Mesotemperado inferior & 13.89 & 0.78 & 17.72 \\ 
  Litoral Cantabro-Atlantico; Agrosistema intensivo (mosaico agroforestal); Termotemperado & 8.33 & 18.64 & 0.45 \\ 
  Litoral Cantabro-Atlantico; Rururbano (diseminado); Termotemperado & 8.33 & 7.02 & 1.19 \\ 
  Canons; Matogueira e rochedo; Mesotemperado inferior & 5.56 & 0.08 & 67.80 \\ 
  category 0; Praias e cantis; category 0 & 5.56 & 0.12 & 45.52 \\ 
  Litoral Cantabro-Atlantico; Agrosistema extensivo; Termotemperado & 5.56 & 1.02 & 5.47 \\ 
  Litoral Cantabro-Atlantico; Agrosistema intensivo (plantacion forestal); Termotemperado & 5.56 & 5.79 & 0.96 \\ 
  Litoral Cantabro-Atlantico; Matogueira e rochedo; Termotemperado & 5.56 & 1.19 & 4.66 \\ 
  Vales sublitorais; Agrosistema intensivo (mosaico agroforestal); Mesotemperado inferior & 5.56 & 21.75 & 0.26 \\ 
   & 5.56 &  &  \\ 
  category 0; Conxunto Historico; category 0 & 2.78 & 0.08 & 33.64 \\ 
  category 0; Lamina de auga; category 0 & 2.78 & 0.55 & 5.06 \\ 
  Serras; Agrosistema extensivo; Mesotemperado superior & 2.78 & 0.80 & 3.49 \\ 
  Serras; Turbeira; Mesotemperado superior & 2.78 & 1.53 & 1.82 \\ 
  Vales sublitorais; Agrosistema intensivo (plantacion forestal); Mesotemperado inferior & 2.78 & 13.21 & 0.21 \\ 
  Vales sublitorais; Matogueira e rochedo; Mesotemperado inferior & 2.78 & 0.64 & 4.36 \\ 
   \hline
\end{tabular}
\end{table}
% latex table generated in R 3.2.5 by xtable 1.8-0 package
% Mon May  9 13:23:06 2016
\begin{table}[p]
\centering
\caption{Frecuencia de aparición de valores naturais identificados na participación pública e frecuencia de tipos asociados A Mariña - Baixo Eo} 
\label{vsixotnat2}
\begin{tabular}{lrrr}
  \hline
Tipo de paisaxe & F.Aparic (\%) & F.Tipo (\%) & Ratio \\ 
  \hline
Vales sublitorais; Agrosistema intensivo (plantacion forestal); Mesotemperado inferior & 19.35 & 23.68 & 0.82 \\ 
  Litoral Cantabro-Atlantico; Agrosistema intensivo (plantacion forestal); Mesotemperado inferior & 12.90 & 18.89 & 0.68 \\ 
   & 12.90 &  &  \\ 
  Litoral Cantabro-Atlantico; Agrosistema intensivo (mosaico agroforestal); Termotemperado & 9.68 & 7.20 & 1.34 \\ 
  Litoral Cantabro-Atlantico; Urbano; Termotemperado & 9.68 & 1.10 & 8.76 \\ 
  category 0; Conxunto Historico; category 0 & 6.45 & 0.10 & 67.16 \\ 
  Litoral Cantabro-Atlantico; Matogueira e rochedo; Mesotemperado inferior & 6.45 & 0.05 & 137.52 \\ 
  Serras; Turbeira; Mesotemperado superior & 6.45 & 2.64 & 2.44 \\ 
  Litoral Cantabro-Atlantico; Agrosistema intensivo (mosaico agroforestal); Mesotemperado inferior & 3.23 & 3.57 & 0.90 \\ 
  Litoral Cantabro-Atlantico; Rururbano (diseminado); Termotemperado & 3.23 & 2.68 & 1.20 \\ 
  Serras; Bosque; Mesotemperado superior & 3.23 & 0.70 & 4.58 \\ 
  Serras; Matogueira e rochedo; Mesotemperado superior & 3.23 & 1.71 & 1.88 \\ 
  Vales sublitorais; Bosque; Mesotemperado superior & 3.23 & 1.22 & 2.65 \\ 
   \hline
\end{tabular}
\end{table}
% latex table generated in R 3.2.5 by xtable 1.8-0 package
% Mon May  9 13:23:06 2016
\begin{table}[p]
\centering
\caption{Frecuencia de aparición de valores naturais identificados na participación pública e frecuencia de tipos asociados Costa Sur - Baixo Miño} 
\label{vsixotnat3}
\begin{tabular}{lrrr}
  \hline
Tipo de paisaxe & F.Aparic (\%) & F.Tipo (\%) & Ratio \\ 
  \hline
Serras; Agrosistema intensivo (plantacion forestal); Termotemperado & 19.05 & 3.79 & 5.03 \\ 
  Serras; Matogueira e rochedo; Mesotemperado inferior & 17.46 & 3.84 & 4.54 \\ 
  Vales sublitorais; Agrosistema intensivo (plantacion forestal); Termotemperado & 11.11 & 16.82 & 0.66 \\ 
  Litoral Cantabro-Atlantico; Agrosistema intensivo (plantacion forestal); Termotemperado & 7.94 & 6.69 & 1.19 \\ 
  Serras; Turbeira; Mesotemperado inferior & 7.94 & 0.12 & 65.84 \\ 
  Serras; Agrosistema intensivo (plantacion forestal); Mesotemperado inferior & 6.35 & 3.38 & 1.88 \\ 
  Litoral Cantabro-Atlantico; Agrosistema intensivo (mosaico agroforestal); Termotemperado & 4.76 & 5.83 & 0.82 \\ 
  category 0; Lamina de auga; category 0 & 3.17 & 0.46 & 6.94 \\ 
  Litoral Cantabro-Atlantico; Agrosistema intensivo (superficie de cultivo); Termotemperado & 3.17 & 0.16 & 19.35 \\ 
  Vales sublitorais; Bosque; Termotemperado & 3.17 & 1.44 & 2.21 \\ 
  Chairas e vales interiores; Agrosistema intensivo (mosaico agroforestal); Mesotemperado inferior & 1.59 & 0.84 & 1.88 \\ 
  Chairas e vales interiores; Agrosistema intensivo (plantacion forestal); Termotemperado & 1.59 & 6.71 & 0.24 \\ 
  Litoral Cantabro-Atlantico; Agrosistema extensivo; Termotemperado & 1.59 & 0.61 & 2.59 \\ 
  Litoral Cantabro-Atlantico; Rururbano (diseminado); Termotemperado & 1.59 & 4.97 & 0.32 \\ 
  Litoral Cantabro-Atlantico; Vinedo; Termotemperado & 1.59 & 0.70 & 2.26 \\ 
  Serras; Agrosistema extensivo; Mesotemperado superior & 1.59 & 0.28 & 5.69 \\ 
  Serras; Matogueira e rochedo; Supra e orotemperado & 1.59 & 0.14 & 11.45 \\ 
  Vales sublitorais; Agrosistema intensivo (superficie de cultivo); Termotemperado & 1.59 & 0.10 & 16.08 \\ 
  Vales sublitorais; Matogueira e rochedo; Termotemperado & 1.59 & 2.72 & 0.58 \\ 
  Vales sublitorais; Rururbano (diseminado); Termotemperado & 1.59 & 2.82 & 0.56 \\ 
   \hline
\end{tabular}
\end{table}
% latex table generated in R 3.2.5 by xtable 1.8-0 package
% Mon May  9 13:23:06 2016
\begin{table}[p]
\centering
\caption{Frecuencia de aparición de valores naturais identificados na participación pública e frecuencia de tipos asociados Ribeiras Encaixadas do Miño e do Sil} 
\label{vsixotnat4}
\begin{tabular}{lrrr}
  \hline
Tipo de paisaxe & F.Aparic (\%) & F.Tipo (\%) & Ratio \\ 
  \hline
Canons; Bosque; Mesotemperado inferior & 26.67 & 2.67 & 10.00 \\ 
  Chairas e vales interiores; Bosque; Termotemperado & 11.67 & 3.03 & 3.85 \\ 
  category 0; Lamina de auga; category 0 & 6.67 & 1.89 & 3.52 \\ 
  Serras; Matogueira e rochedo; Supra e orotemperado & 6.67 & 9.08 & 0.73 \\ 
  Chairas e vales interiores; Agrosistema intensivo (plantacion forestal); Mesotemperado inferior & 5.00 & 4.37 & 1.14 \\ 
  Serras; Bosque; Mesotemperado inferior & 5.00 & 2.32 & 2.16 \\ 
  Chairas e vales interiores; Vinedo; Termotemperado & 3.33 & 2.22 & 1.50 \\ 
  Serras; Bosque; Supra e orotemperado & 3.33 & 0.47 & 7.02 \\ 
  Serras; Matogueira e rochedo; Mesotemperado superior & 3.33 & 3.10 & 1.07 \\ 
  Canons; Agrosistema intensivo (plantacion forestal); category 0 & 1.67 & 1.25 & 1.34 \\ 
  Canons; Agrosistema intensivo (plantacion forestal); Termotemperado & 1.67 & 1.39 & 1.20 \\ 
  Canons; Bosque; category 0 & 1.67 & 0.39 & 4.23 \\ 
  Canons; Matogueira e rochedo; category 0 & 1.67 & 0.96 & 1.74 \\ 
  Canons; Matogueira e rochedo; Mesotemperado inferior & 1.67 & 0.67 & 2.47 \\ 
  Chairas e vales interiores; Agrosistema extensivo; Termotemperado & 1.67 & 1.56 & 1.07 \\ 
  Chairas e vales interiores; Agrosistema intensivo (plantacion forestal); category 0 & 1.67 & 1.83 & 0.91 \\ 
  Chairas e vales interiores; Agrosistema intensivo (plantacion forestal); Termotemperado & 1.67 & 6.81 & 0.24 \\ 
  Chairas e vales interiores; Bosque; category 0 & 1.67 & 0.80 & 2.08 \\ 
  Chairas e vales interiores; Matogueira e rochedo; category 0 & 1.67 & 2.25 & 0.74 \\ 
  Chairas e vales interiores; Matogueira e rochedo; Mesotemperado inferior & 1.67 & 3.61 & 0.46 \\ 
  Chairas e vales interiores; Vinedo; category 0 & 1.67 & 1.70 & 0.98 \\ 
  Serras; Agrosistema extensivo; Mesotemperado inferior & 1.67 & 1.98 & 0.84 \\ 
  Serras; Agrosistema extensivo; Mesotemperado superior & 1.67 & 1.71 & 0.97 \\ 
  Serras; Agrosistema intensivo (plantacion forestal); Mesotemperado inferior & 1.67 & 2.65 & 0.63 \\ 
  Serras; Bosque; Termotemperado & 1.67 & 0.00 & 346.92 \\ 
  Serras; Matogueira e rochedo; Mesotemperado inferior & 1.67 & 2.79 & 0.60 \\ 
   \hline
\end{tabular}
\end{table}
% latex table generated in R 3.2.5 by xtable 1.8-0 package
% Mon May  9 13:23:06 2016
\begin{table}[p]
\centering
\caption{Frecuencia de aparición de valores naturais identificados na participación pública e frecuencia de tipos asociados Serras Orientais} 
\label{vsixotnat5}
\begin{tabular}{lrrr}
  \hline
Tipo de paisaxe & F.Aparic (\%) & F.Tipo (\%) & Ratio \\ 
  \hline
Serras; Bosque; Supra e orotemperado & 24.32 & 7.67 & 3.17 \\ 
  Serras; Matogueira e rochedo; Supra e orotemperado & 17.57 & 15.61 & 1.13 \\ 
  Serras; Agrosistema extensivo; Supra e orotemperado & 12.16 & 8.91 & 1.36 \\ 
  Vales sublitorais; Bosque; Mesotemperado inferior & 6.76 & 3.83 & 1.76 \\ 
  Serras; Matogueira e rochedo; Mesotemperado superior & 5.41 & 5.41 & 1.00 \\ 
  Serras; Agrosistema intensivo (mosaico agroforestal); Mesotemperado superior & 4.05 & 3.95 & 1.03 \\ 
  Serras; Agrosistema intensivo (mosaico agroforestal); Supra e orotemperado & 4.05 & 5.74 & 0.71 \\ 
  Vales sublitorais; Agrosistema extensivo; Mesotemperado inferior & 4.05 & 1.41 & 2.88 \\ 
  Vales sublitorais; Bosque; Mesotemperado superior & 4.05 & 4.16 & 0.97 \\ 
  category 0; Lamina de auga; category 0 & 2.70 & 0.16 & 16.79 \\ 
  Serras; Agrosistema extensivo; Mesotemperado superior & 2.70 & 7.30 & 0.37 \\ 
  Serras; Agrosistema intensivo (plantacion forestal); Supra e orotemperado & 2.70 & 6.86 & 0.39 \\ 
  Serras; Bosque; Mesotemperado superior & 2.70 & 3.65 & 0.74 \\ 
  Vales sublitorais; Agrosistema extensivo; Mesotemperado superior & 2.70 & 5.11 & 0.53 \\ 
  Canons; Bosque; Mesotemperado inferior & 1.35 & 0.44 & 3.10 \\ 
  Chairas e vales interiores; Bosque; category 0 & 1.35 & 0.19 & 6.97 \\ 
  Serras; Bosque; Mesotemperado inferior & 1.35 & 0.55 & 2.46 \\ 
   \hline
\end{tabular}
\end{table}
% latex table generated in R 3.2.5 by xtable 1.8-0 package
% Mon May  9 13:23:06 2016
\begin{table}[p]
\centering
\caption{Frecuencia de aparición de valores naturais identificados na participación pública e frecuencia de tipos asociados Chairas e Fosas Luguesas} 
\label{vsixotnat6}
\begin{tabular}{lrrr}
  \hline
Tipo de paisaxe & F.Aparic (\%) & F.Tipo (\%) & Ratio \\ 
  \hline
Chairas e vales interiores; Agrosistema extensivo; Mesotemperado inferior & 18.92 & 6.80 & 2.78 \\ 
  Chairas e vales interiores; Bosque; Mesotemperado inferior & 16.22 & 1.92 & 8.45 \\ 
  Chairas e vales interiores; Agrosistema extensivo; Mesotemperado superior & 10.81 & 9.68 & 1.12 \\ 
  Serras; Agrosistema intensivo (plantacion forestal); Supra e orotemperado & 10.81 & 0.59 & 18.38 \\ 
  Chairas e vales interiores; Agrosistema intensivo (plantacion forestal); Mesotemperado inferior & 8.11 & 3.31 & 2.45 \\ 
  category 0; Lamina de auga; category 0 & 5.41 & 0.09 & 56.91 \\ 
  Chairas e vales interiores; Agrosistema intensivo (mosaico agroforestal); Mesotemperado superior & 5.41 & 20.11 & 0.27 \\ 
  Chairas e vales interiores; Bosque; Mesotemperado superior & 5.41 & 2.02 & 2.67 \\ 
  Chairas e vales interiores; Urbano; Mesotemperado superior & 2.70 & 0.45 & 5.99 \\ 
  Serras; Agrosistema extensivo; Mesotemperado superior & 2.70 & 5.94 & 0.46 \\ 
  Serras; Agrosistema extensivo; Supra e orotemperado & 2.70 & 0.37 & 7.31 \\ 
  Serras; Agrosistema intensivo (mosaico agroforestal); Mesotemperado superior & 2.70 & 9.79 & 0.28 \\ 
  Serras; Agrosistema intensivo (plantacion forestal); Mesotemperado superior & 2.70 & 2.29 & 1.18 \\ 
  Serras; Matogueira e rochedo; Supra e orotemperado & 2.70 & 0.48 & 5.64 \\ 
  Serras; Turbeira; Mesotemperado superior & 2.70 & 1.08 & 2.50 \\ 
   \hline
\end{tabular}
\end{table}
% latex table generated in R 3.2.5 by xtable 1.8-0 package
% Mon May  9 13:23:06 2016
\begin{table}[p]
\centering
\caption{Frecuencia de aparición de valores naturais identificados na participación pública e frecuencia de tipos asociados Galicia Central} 
\label{vsixotnat7}
\begin{tabular}{lrrr}
  \hline
Tipo de paisaxe & F.Aparic (\%) & F.Tipo (\%) & Ratio \\ 
  \hline
Vales sublitorais; Agrosistema intensivo (mosaico agroforestal); Mesotemperado inferior & 29.09 & 43.06 & 0.68 \\ 
  Vales sublitorais; Agrosistema extensivo; Mesotemperado inferior & 10.91 & 5.21 & 2.09 \\ 
  Serras; Agrosistema extensivo; Mesotemperado superior & 7.27 & 4.47 & 1.63 \\ 
  Serras; Matogueira e rochedo; Mesotemperado superior & 7.27 & 6.90 & 1.05 \\ 
  Vales sublitorais; Agrosistema intensivo (plantacion forestal); Mesotemperado inferior & 6.36 & 5.84 & 1.09 \\ 
  Serras; Agrosistema intensivo (mosaico agroforestal); Mesotemperado superior & 2.73 & 3.93 & 0.69 \\ 
  Serras; Agrosistema intensivo (plantacion forestal); Supra e orotemperado & 2.73 & 0.54 & 5.09 \\ 
  Serras; Bosque; Mesotemperado inferior & 2.73 & 0.45 & 6.04 \\ 
  Vales sublitorais; Bosque; Mesotemperado inferior & 2.73 & 0.80 & 3.41 \\ 
  Vales sublitorais; Matogueira e rochedo; Mesotemperado inferior & 2.73 & 1.65 & 1.66 \\ 
  Vales sublitorais; Urbano; Mesotemperado inferior & 2.73 & 0.64 & 4.25 \\ 
  Chairas e vales interiores; Bosque; Mesotemperado inferior & 1.82 & 0.68 & 2.68 \\ 
  Serras; Agrosistema intensivo (superficie de cultivo); Mesotemperado superior & 1.82 & 1.44 & 1.26 \\ 
  Serras; Matogueira e rochedo; Mesotemperado inferior & 1.82 & 0.62 & 2.93 \\ 
  Vales sublitorais; Agrosistema extensivo; Termotemperado & 1.82 & 0.29 & 6.32 \\ 
  Vales sublitorais; Agrosistema intensivo (mosaico agroforestal); Mesotemperado superior & 1.82 & 0.57 & 3.17 \\ 
  Vales sublitorais; Agrosistema intensivo (plantacion forestal); Termotemperado & 1.82 & 1.80 & 1.01 \\ 
  Vales sublitorais; Bosque; Termotemperado & 1.82 & 0.32 & 5.61 \\ 
   \hline
\end{tabular}
\end{table}
% latex table generated in R 3.2.5 by xtable 1.8-0 package
% Mon May  9 13:23:06 2016
\begin{table}[p]
\centering
\caption{Frecuencia de aparición de valores naturais identificados na participación pública e frecuencia de tipos asociados Chairas, Fosas e Serras Ourensás} 
\label{vsixotnat8}
\begin{tabular}{lrrr}
  \hline
Tipo de paisaxe & F.Aparic (\%) & F.Tipo (\%) & Ratio \\ 
  \hline
Serras; Matogueira e rochedo; Supra e orotemperado & 16.92 & 14.06 & 1.20 \\ 
  Serras; Matogueira e rochedo; Mesotemperado superior & 13.85 & 10.51 & 1.32 \\ 
  Chairas e vales interiores; Matogueira e rochedo; Mesotemperado inferior & 9.23 & 4.82 & 1.91 \\ 
  Chairas e vales interiores; Agrosistema extensivo; Mesotemperado inferior & 7.69 & 7.38 & 1.04 \\ 
  Chairas e vales interiores; Bosque; Mesotemperado inferior & 7.69 & 6.18 & 1.24 \\ 
  Chairas e vales interiores; Agrosistema intensivo (plantacion forestal); Termotemperado & 6.15 & 3.78 & 1.63 \\ 
  Chairas e vales interiores; Agrosistema intensivo (superficie de cultivo); Mesotemperado inferior & 6.15 & 7.53 & 0.82 \\ 
  Serras; Agrosistema extensivo; Mesotemperado inferior & 4.62 & 5.23 & 0.88 \\ 
  Serras; Matogueira e rochedo; Mesotemperado inferior & 4.62 & 5.15 & 0.90 \\ 
  Chairas e vales interiores; Vinedo; Termotemperado & 3.08 & 0.90 & 3.42 \\ 
  Serras; Bosque; Supra e orotemperado & 3.08 & 1.18 & 2.61 \\ 
  category 0; Lamina de auga; category 0 & 1.54 & 0.71 & 2.18 \\ 
  Chairas e vales interiores; Agrosistema extensivo; Termotemperado & 1.54 & 0.66 & 2.33 \\ 
  Chairas e vales interiores; Agrosistema intensivo (mosaico agroforestal); Termotemperado & 1.54 & 0.24 & 6.33 \\ 
  Chairas e vales interiores; Agrosistema intensivo (superficie de cultivo); Termotemperado & 1.54 & 0.03 & 52.81 \\ 
  Chairas e vales interiores; Matogueira e rochedo; Termotemperado & 1.54 & 1.80 & 0.86 \\ 
  Chairas e vales interiores; Rururbano (diseminado); Mesotemperado inferior & 1.54 & 0.32 & 4.75 \\ 
  Chairas e vales interiores; Urbano; Mesotemperado inferior & 1.54 & 0.23 & 6.62 \\ 
  Serras; Agrosistema intensivo (plantacion forestal); Mesotemperado inferior & 1.54 & 5.68 & 0.27 \\ 
  Serras; Agrosistema intensivo (plantacion forestal); Mesotemperado superior & 1.54 & 1.41 & 1.09 \\ 
  Serras; Agrosistema intensivo (plantacion forestal); Supra e orotemperado & 1.54 & 1.55 & 1.00 \\ 
  Serras; Bosque; Mesotemperado superior & 1.54 & 2.22 & 0.69 \\ 
   \hline
\end{tabular}
\end{table}
% latex table generated in R 3.2.5 by xtable 1.8-0 package
% Mon May  9 13:23:07 2016
\begin{table}[p]
\centering
\caption{Frecuencia de aparición de valores naturais identificados na participación pública e frecuencia de tipos asociados Serras Surorientais} 
\label{vsixotnat9}
\begin{tabular}{lrrr}
  \hline
Tipo de paisaxe & F.Aparic (\%) & F.Tipo (\%) & Ratio \\ 
  \hline
Serras; Matogueira e rochedo; Supra e orotemperado & 45.21 & 47.28 & 0.96 \\ 
  Canons; Vinedo; category 0 & 9.59 & 0.03 & 302.08 \\ 
  Serras; Agrosistema extensivo; Mesotemperado inferior & 6.85 & 4.79 & 1.43 \\ 
  Serras; Bosque; Mesotemperado superior & 6.85 & 5.61 & 1.22 \\ 
  Serras; Agrosistema intensivo (plantacion forestal); Supra e orotemperado & 5.48 & 10.23 & 0.54 \\ 
  Serras; Matogueira e rochedo; Mesotemperado inferior & 5.48 & 2.75 & 1.99 \\ 
  Canons; Agrosistema intensivo (plantacion forestal); category 0 & 2.74 & 0.26 & 10.53 \\ 
  category 0; Lamina de auga; category 0 & 2.74 & 1.38 & 1.98 \\ 
  Serras; Agrosistema extensivo; Mesotemperado superior & 2.74 & 6.24 & 0.44 \\ 
  Serras; Bosque; Mesotemperado inferior & 2.74 & 3.06 & 0.89 \\ 
  Serras; Bosque; Supra e orotemperado & 2.74 & 1.26 & 2.18 \\ 
  Canons; Vinedo; Mesotemperado inferior & 1.37 & 0.07 & 20.81 \\ 
  Serras; Agrosistema extensivo; Supra e orotemperado & 1.37 & 2.94 & 0.47 \\ 
  Serras; Agrosistema intensivo (mosaico agroforestal); Mesotemperado inferior & 1.37 & 0.76 & 1.81 \\ 
  Serras; Agrosistema intensivo (mosaico agroforestal); Mesotemperado superior & 1.37 & 0.81 & 1.70 \\ 
  Serras; Agrosistema intensivo (superficie de cultivo); Mesotemperado superior & 1.37 & 0.72 & 1.91 \\ 
   \hline
\end{tabular}
\end{table}
% latex table generated in R 3.2.5 by xtable 1.8-0 package
% Mon May  9 13:23:07 2016
\begin{table}[p]
\centering
\caption{Frecuencia de aparición de valores naturais identificados na participación pública e frecuencia de tipos asociados Galicia Setentrional} 
\label{vsixotnat10}
\begin{tabular}{lrrr}
  \hline
Tipo de paisaxe & F.Aparic (\%) & F.Tipo (\%) & Ratio \\ 
  \hline
Serras; Turbeira; Mesotemperado superior & 26.32 & 12.44 & 2.12 \\ 
  category 0; Praias e cantis; category 0 & 14.47 & 0.94 & 15.33 \\ 
   & 10.53 &  &  \\ 
  Litoral Cantabro-Atlantico; Agrosistema intensivo (mosaico agroforestal); Termotemperado & 6.58 & 5.53 & 1.19 \\ 
  Litoral Cantabro-Atlantico; Agrosistema intensivo (plantacion forestal); Termotemperado & 6.58 & 7.61 & 0.86 \\ 
  Vales sublitorais; Bosque; Mesotemperado inferior & 6.58 & 1.49 & 4.41 \\ 
  Litoral Cantabro-Atlantico; Matogueira e rochedo; Termotemperado & 5.26 & 1.77 & 2.97 \\ 
  Serras; Matogueira e rochedo; Mesotemperado superior & 3.95 & 6.79 & 0.58 \\ 
  Vales sublitorais; Agrosistema intensivo (plantacion forestal); Mesotemperado inferior & 3.95 & 17.31 & 0.23 \\ 
  Serras; Agrosistema intensivo (plantacion forestal); Mesotemperado superior & 2.63 & 4.68 & 0.56 \\ 
  Serras; Bosque; Mesotemperado inferior & 2.63 & 0.22 & 12.00 \\ 
  Litoral Cantabro-Atlantico; Agrosistema intensivo (plantacion forestal); Mesotemperado inferior & 1.32 & 2.79 & 0.47 \\ 
  Litoral Cantabro-Atlantico; Matogueira e rochedo; Mesotemperado inferior & 1.32 & 0.28 & 4.66 \\ 
  Litoral Cantabro-Atlantico; Urbano; Termotemperado & 1.32 & 0.47 & 2.77 \\ 
  Serras; Agrosistema extensivo; Mesotemperado superior & 1.32 & 2.70 & 0.49 \\ 
  Serras; Agrosistema intensivo (mosaico agroforestal); Mesotemperado superior & 1.32 & 3.70 & 0.36 \\ 
  Vales sublitorais; Agrosistema intensivo (mosaico agroforestal); Mesotemperado inferior & 1.32 & 13.63 & 0.10 \\ 
  Vales sublitorais; Agrosistema intensivo (mosaico agroforestal); Mesotemperado superior & 1.32 & 0.82 & 1.60 \\ 
  Vales sublitorais; Rururbano (diseminado); Mesotemperado inferior & 1.32 & 0.08 & 16.09 \\ 
   \hline
\end{tabular}
\end{table}
% latex table generated in R 3.2.5 by xtable 1.8-0 package
% Mon May  9 13:23:07 2016
\begin{table}[p]
\centering
\caption{Frecuencia de aparición de valores naturais identificados na participación pública e frecuencia de tipos asociados Chairas e Fosas Occidentais} 
\label{vsixotnat11}
\begin{tabular}{lrrr}
  \hline
Tipo de paisaxe & F.Aparic (\%) & F.Tipo (\%) & Ratio \\ 
  \hline
category 0; Praias e cantis; category 0 & 19.86 & 0.54 & 36.94 \\ 
  Vales sublitorais; Agrosistema intensivo (plantacion forestal); Mesotemperado inferior & 19.18 & 15.20 & 1.26 \\ 
  Litoral Cantabro-Atlantico; Matogueira e rochedo; Termotemperado & 16.44 & 4.58 & 3.59 \\ 
  Vales sublitorais; Agrosistema intensivo (mosaico agroforestal); Mesotemperado inferior & 10.96 & 36.05 & 0.30 \\ 
   & 10.96 &  &  \\ 
  Litoral Cantabro-Atlantico; Agrosistema intensivo (plantacion forestal); Termotemperado & 7.53 & 7.60 & 0.99 \\ 
  category 0; Conxunto Historico; category 0 & 2.74 & 0.12 & 22.50 \\ 
  Vales sublitorais; Matogueira e rochedo; Mesotemperado inferior & 2.74 & 4.77 & 0.57 \\ 
  Litoral Cantabro-Atlantico; Rururbano (diseminado); Termotemperado & 1.37 & 0.53 & 2.60 \\ 
  Litoral Cantabro-Atlantico; Urbano; Termotemperado & 1.37 & 0.66 & 2.08 \\ 
  Vales sublitorais; Agrosistema extensivo; Mesotemperado inferior & 1.37 & 0.95 & 1.44 \\ 
  Vales sublitorais; Agrosistema intensivo (plantacion forestal); Termotemperado & 1.37 & 3.44 & 0.40 \\ 
  Vales sublitorais; Matogueira e rochedo; Mesotemperado superior & 1.37 & 2.22 & 0.62 \\ 
   \hline
\end{tabular}
\end{table}
% latex table generated in R 3.2.5 by xtable 1.8-0 package
% Mon May  9 13:23:07 2016
\begin{table}[p]
\centering
\caption{Frecuencia de aparición de valores naturais identificados na participación pública e frecuencia de tipos asociados Rías Baixas} 
\label{vsixotnat12}
\begin{tabular}{lrrr}
  \hline
Tipo de paisaxe & F.Aparic (\%) & F.Tipo (\%) & Ratio \\ 
  \hline
Serras; Matogueira e rochedo; Mesotemperado superior & 15.57 & 6.70 & 2.32 \\ 
  Serras; Matogueira e rochedo; Mesotemperado inferior & 12.30 & 3.29 & 3.73 \\ 
  Vales sublitorais; Agrosistema intensivo (plantacion forestal); Termotemperado & 11.48 & 16.41 & 0.70 \\ 
   & 11.48 &  &  \\ 
  Vales sublitorais; Matogueira e rochedo; Mesotemperado inferior & 9.84 & 4.71 & 2.09 \\ 
  Litoral Cantabro-Atlantico; Matogueira e rochedo; Termotemperado & 7.38 & 1.73 & 4.26 \\ 
  Litoral Cantabro-Atlantico; Agrosistema intensivo (plantacion forestal); Termotemperado & 6.56 & 10.20 & 0.64 \\ 
  category 0; Praias e cantis; category 0 & 5.74 & 0.56 & 10.17 \\ 
  Serras; Agrosistema intensivo (plantacion forestal); Mesotemperado inferior & 4.10 & 2.82 & 1.46 \\ 
  Litoral Cantabro-Atlantico; Rururbano (diseminado); Termotemperado & 2.46 & 7.10 & 0.35 \\ 
  Vales sublitorais; Agrosistema intensivo (mosaico agroforestal); Termotemperado & 2.46 & 7.63 & 0.32 \\ 
  Litoral Cantabro-Atlantico; Agrosistema intensivo (mosaico agroforestal); Termotemperado & 1.64 & 10.49 & 0.16 \\ 
   \hline
\end{tabular}
\end{table}


\clearpage
% latex table generated in R 3.2.2 by xtable 1.8-0 package
% Wed Dec  2 19:45:31 2015
\begin{table}[p]
\centering
\caption{Frecuencia de aparición de valores patrimoniais identificados na participación pública e frecuencia de tipos asociados Golfo Ártabro} 
\label{vsixotpat1}
\begin{tabular}{lrrr}
  \hline
Tipo de paisaxe & F.Aparic (\%) & F.Tipo (\%) & Ratio \\ 
  \hline
Litoral Cantabro-Atlantico; Rururbano (diseminado); Termotemperado & 16.67 & 3.29 & 5.06 \\ 
  Litoral Cantabro-Atlantico; Conxunto Historico; no data & 13.89 & 0.00 & 5265.44 \\ 
  Canons; Bosque; Mesotemperado inferior & 8.33 & 0.30 & 28.20 \\ 
  Litoral Cantabro-Atlantico; Agrosistema intensivo (plantacion forestal); Termotemperado & 8.33 & 1.32 & 6.33 \\ 
  Vales sublitorais; Agrosistema intensivo (mosaico agroforestal); Mesotemperado inferior & 8.33 & 7.54 & 1.10 \\ 
  Vales sublitorais; Agrosistema intensivo (mosaico agroforestal); Termotemperado & 8.33 & 2.62 & 3.18 \\ 
   & 8.33 &  &  \\ 
  Litoral Cantabro-Atlantico; Conxunto Historico; Termotemperado & 5.56 & 0.02 & 293.76 \\ 
  Litoral Cantabro-Atlantico; Rururbano (diseminado); Mesotemperado inferior & 5.56 & 0.04 & 147.54 \\ 
  Canons; Bosque; Termotemperado & 2.78 & 0.15 & 18.35 \\ 
  Litoral Cantabro-Atlantico; Agrosistema intensivo (mosaico agroforestal); no data & 2.78 & 0.04 & 65.33 \\ 
  Serras; Agrosistema intensivo (mosaico agroforestal); Mesotemperado superior & 2.78 & 1.77 & 1.57 \\ 
  Vales sublitorais; Agrosistema intensivo (plantacion forestal); Termotemperado & 2.78 & 1.79 & 1.55 \\ 
  Vales sublitorais; Matogueira e rochedo; Mesotemperado superior & 2.78 & 0.85 & 3.26 \\ 
  Vales sublitorais; Rururbano (diseminado); Termotemperado & 2.78 & 1.68 & 1.65 \\ 
   \hline
\end{tabular}
\end{table}
% latex table generated in R 3.2.2 by xtable 1.8-0 package
% Wed Dec  2 19:45:31 2015
\begin{table}[p]
\centering
\caption{Frecuencia de aparición de valores patrimoniais identificados na participación pública e frecuencia de tipos asociados A Mariña - Baixo Eo} 
\label{vsixotpat2}
\begin{tabular}{lrrr}
  \hline
Tipo de paisaxe & F.Aparic (\%) & F.Tipo (\%) & Ratio \\ 
  \hline
Vales sublitorais; Agrosistema intensivo (plantacion forestal); Mesotemperado inferior & 26.00 & 2.90 & 8.97 \\ 
  Vales sublitorais; Agrosistema intensivo (mosaico agroforestal); Mesotemperado inferior & 22.00 & 7.54 & 2.92 \\ 
  Vales sublitorais; Rururbano (diseminado); Mesotemperado inferior & 12.00 & 0.98 & 12.22 \\ 
  Litoral Cantabro-Atlantico; Rururbano (diseminado); Termotemperado & 8.00 & 3.29 & 2.43 \\ 
  Litoral Cantabro-Atlantico; Conxunto Historico; Termotemperado & 6.00 & 0.02 & 317.26 \\ 
  Litoral Cantabro-Atlantico; Matogueira e rochedo; Mesotemperado inferior & 6.00 & 0.10 & 60.27 \\ 
  Litoral Cantabro-Atlantico; Agrosistema intensivo (mosaico agroforestal); Mesotemperado inferior & 4.00 & 0.29 & 13.65 \\ 
  Serras; Turbeira; Mesotemperado superior & 4.00 & 0.71 & 5.65 \\ 
  Litoral Cantabro-Atlantico; Agrosistema intensivo (mosaico agroforestal); Termotemperado & 2.00 & 2.43 & 0.82 \\ 
  Litoral Cantabro-Atlantico; Conxunto Historico; Mesotemperado inferior & 2.00 & 0.00 & 3431.24 \\ 
  Litoral Cantabro-Atlantico; Rururbano (diseminado); Mesotemperado inferior & 2.00 & 0.04 & 53.11 \\ 
  Vales sublitorais; Agrosistema extensivo; Mesotemperado inferior & 2.00 & 2.73 & 0.73 \\ 
  Vales sublitorais; Bosque; Mesotemperado superior & 2.00 & 0.39 & 5.18 \\ 
   & 2.00 &  &  \\ 
   \hline
\end{tabular}
\end{table}
% latex table generated in R 3.2.2 by xtable 1.8-0 package
% Wed Dec  2 19:45:31 2015
\begin{table}[p]
\centering
\caption{Frecuencia de aparición de valores patrimoniais identificados na participación pública e frecuencia de tipos asociados Costa Sur - Baixo Miño} 
\label{vsixotpat3}
\begin{tabular}{lrrr}
  \hline
Tipo de paisaxe & F.Aparic (\%) & F.Tipo (\%) & Ratio \\ 
  \hline
Litoral Cantabro-Atlantico; Rururbano (diseminado); Termotemperado & 43.75 & 3.29 & 13.29 \\ 
  Serras; Conxunto Historico; Termotemperado & 12.50 & 0.01 & 1562.54 \\ 
  Vales sublitorais; Agrosistema intensivo (plantacion forestal); Termotemperado & 9.38 & 1.79 & 5.24 \\ 
  Vales sublitorais; Rururbano (diseminado); Termotemperado & 9.38 & 1.68 & 5.58 \\ 
  Chairas e vales interiores; Agrosistema intensivo (mosaico agroforestal); Mesotemperado inferior & 3.12 & 1.71 & 1.83 \\ 
  Chairas e vales interiores; Matogueira e rochedo; Mesotemperado inferior & 3.12 & 1.38 & 2.26 \\ 
  Litoral Cantabro-Atlantico; Agrosistema intensivo (plantacion forestal); Termotemperado & 3.12 & 1.32 & 2.37 \\ 
  no data; Agrosistema intensivo (mosaico agroforestal); Termotemperado & 3.12 & 0.02 & 176.39 \\ 
  Serras; Agrosistema intensivo (plantacion forestal); Mesotemperado inferior & 3.12 & 0.82 & 3.79 \\ 
  Serras; Matogueira e rochedo; Mesotemperado inferior & 3.12 & 2.73 & 1.15 \\ 
  Vales sublitorais; Agrosistema intensivo (mosaico agroforestal); Termotemperado & 3.12 & 2.62 & 1.19 \\ 
  Vales sublitorais; Agrosistema intensivo (plantacion forestal); Mesotemperado inferior & 3.12 & 2.90 & 1.08 \\ 
   \hline
\end{tabular}
\end{table}
% latex table generated in R 3.2.2 by xtable 1.8-0 package
% Wed Dec  2 19:45:31 2015
\begin{table}[p]
\centering
\caption{Frecuencia de aparición de valores patrimoniais identificados na participación pública e frecuencia de tipos asociados Ribeiras Encaixadas do Miño e do Sil} 
\label{vsixotpat4}
\begin{tabular}{lrrr}
  \hline
Tipo de paisaxe & F.Aparic (\%) & F.Tipo (\%) & Ratio \\ 
  \hline
Canons; Matogueira e rochedo; Mesomediterráneo & 9.52 & 0.18 & 51.55 \\ 
  Chairas e vales interiores; Conxunto Historico; Termotemperado & 9.52 & 0.00 & 3279.73 \\ 
  Canons; Viñedo; Mesotemperado inferior & 7.94 & 0.02 & 343.20 \\ 
  Canons; Bosque; Mesotemperado inferior & 6.35 & 0.30 & 21.49 \\ 
  Chairas e vales interiores; Rururbano (diseminado); Mesotemperado inferior & 6.35 & 0.51 & 12.40 \\ 
  Canons; Bosque; Termotemperado & 4.76 & 0.15 & 31.46 \\ 
  Chairas e vales interiores; Bosque; Termotemperado & 4.76 & 0.29 & 16.33 \\ 
  Chairas e vales interiores; Rururbano (diseminado); Termotemperado & 4.76 & 0.42 & 11.25 \\ 
  Canons; Agrosistema extensivo; Mesomediterráneo & 3.17 & 0.03 & 92.01 \\ 
  Chairas e vales interiores; Agrosistema intensivo (mosaico agroforestal); Termotemperado & 3.17 & 0.34 & 9.36 \\ 
  Chairas e vales interiores; Matogueira e rochedo; Mesomediterráneo & 3.17 & 0.27 & 11.84 \\ 
  Chairas e vales interiores; Viñedo; Mesomediterráneo & 3.17 & 0.11 & 28.35 \\ 
  Serras; Agrosistema extensivo; Mesotemperado superior & 3.17 & 5.76 & 0.55 \\ 
  Serras; Matogueira e rochedo; Mesotemperado inferior & 3.17 & 2.73 & 1.16 \\ 
  Canons; Bosque; Mesomediterráneo & 1.59 & 0.03 & 59.11 \\ 
  Canons; Matogueira e rochedo; Termotemperado & 1.59 & 0.09 & 18.41 \\ 
  Canons; Viñedo; Termotemperado & 1.59 & 0.04 & 42.70 \\ 
  Chairas e vales interiores; Agrosistema extensivo; Mesomediterráneo & 1.59 & 0.11 & 14.47 \\ 
  Chairas e vales interiores; Agrosistema intensivo (mosaico agroforestal); Mesomediterráneo & 1.59 & 0.01 & 118.98 \\ 
  Chairas e vales interiores; Agrosistema intensivo (mosaico agroforestal); Mesotemperado inferior & 1.59 & 1.71 & 0.93 \\ 
  Chairas e vales interiores; Agrosistema intensivo (plantacion forestal); Mesotemperado inferior & 1.59 & 0.55 & 2.88 \\ 
  Chairas e vales interiores; Agrosistema intensivo (plantacion forestal); Termotemperado & 1.59 & 0.37 & 4.31 \\ 
  Chairas e vales interiores; Bosque; Mesomediterráneo & 1.59 & 0.06 & 28.14 \\ 
  Chairas e vales interiores; Matogueira e rochedo; Mesotemperado inferior & 1.59 & 1.38 & 1.15 \\ 
  Chairas e vales interiores; Matogueira e rochedo; Termotemperado & 1.59 & 0.66 & 2.42 \\ 
  Chairas e vales interiores; Urbano; Termotemperado & 1.59 & 0.08 & 19.34 \\ 
  Chairas e vales interiores; Viñedo; Termotemperado & 1.59 & 0.28 & 5.72 \\ 
  Serras; Agrosistema extensivo; Mesotemperado inferior & 1.59 & 2.57 & 0.62 \\ 
  Serras; Agrosistema intensivo (superficie de cultivo); Mesotemperado inferior & 1.59 & 0.28 & 5.60 \\ 
  Serras; Conxunto Historico; Mesotemperado inferior & 1.59 & 0.00 & 764.60 \\ 
  Serras; Viñedo; Mesotemperado inferior & 1.59 & 0.02 & 82.44 \\ 
   \hline
\end{tabular}
\end{table}
% latex table generated in R 3.2.2 by xtable 1.8-0 package
% Wed Dec  2 19:45:31 2015
\begin{table}[p]
\centering
\caption{Frecuencia de aparición de valores patrimoniais identificados na participación pública e frecuencia de tipos asociados Serras Orientais} 
\label{vsixotpat5}
\begin{tabular}{lrrr}
  \hline
Tipo de paisaxe & F.Aparic (\%) & F.Tipo (\%) & Ratio \\ 
  \hline
Serras; Agrosistema extensivo; Supra e orotemperado & 30.00 & 2.50 & 12.01 \\ 
  Serras; Agrosistema extensivo; Mesotemperado superior & 16.67 & 5.76 & 2.89 \\ 
  Serras; Matogueira e rochedo; Supra e orotemperado & 10.00 & 6.12 & 1.63 \\ 
  Vales sublitorais; Bosque; Mesotemperado superior & 10.00 & 0.39 & 25.91 \\ 
  Serras; Bosque; Supra e orotemperado & 6.67 & 0.74 & 9.05 \\ 
  Vales sublitorais; Bosque; Mesotemperado inferior & 6.67 & 0.49 & 13.47 \\ 
  Canons; Bosque; Mesotemperado inferior & 3.33 & 0.30 & 11.28 \\ 
  Chairas e vales interiores; Agrosistema extensivo; Mesotemperado inferior & 3.33 & 3.58 & 0.93 \\ 
  Chairas e vales interiores; Agrosistema extensivo; Mesotemperado superior & 3.33 & 2.77 & 1.20 \\ 
  Serras; Agrosistema intensivo (mosaico agroforestal); Mesotemperado superior & 3.33 & 1.77 & 1.88 \\ 
  Serras; Agrosistema intensivo (plantacion forestal); Supra e orotemperado & 3.33 & 0.93 & 3.58 \\ 
  Serras; Bosque; Mesotemperado superior & 3.33 & 1.00 & 3.33 \\ 
   \hline
\end{tabular}
\end{table}
% latex table generated in R 3.2.2 by xtable 1.8-0 package
% Wed Dec  2 19:45:31 2015
\begin{table}[p]
\centering
\caption{Frecuencia de aparición de valores patrimoniais identificados na participación pública e frecuencia de tipos asociados Chairas e Fosas Luguesas} 
\label{vsixotpat6}
\begin{tabular}{lrrr}
  \hline
Tipo de paisaxe & F.Aparic (\%) & F.Tipo (\%) & Ratio \\ 
  \hline
Chairas e vales interiores; Agrosistema extensivo; Mesotemperado inferior & 15.79 & 3.58 & 4.41 \\ 
  Chairas e vales interiores; Agrosistema extensivo; Mesotemperado superior & 15.79 & 2.77 & 5.69 \\ 
  Chairas e vales interiores; Bosque; Mesotemperado inferior & 13.16 & 0.77 & 17.20 \\ 
  Chairas e vales interiores; Rururbano (diseminado); Mesotemperado superior & 13.16 & 0.30 & 43.90 \\ 
  Chairas e vales interiores; Agrosistema intensivo (mosaico agroforestal); Mesotemperado superior & 10.53 & 2.27 & 4.64 \\ 
  Chairas e vales interiores; Agrosistema intensivo (superficie de cultivo); Mesotemperado superior & 5.26 & 0.72 & 7.35 \\ 
  Chairas e vales interiores; Conxunto Historico; Mesotemperado superior & 5.26 & 0.00 & 4765.13 \\ 
  Chairas e vales interiores; Conxunto Historico; Termotemperado & 5.26 & 0.00 & 1812.48 \\ 
  Serras; Agrosistema extensivo; Mesotemperado superior & 5.26 & 5.76 & 0.91 \\ 
  Chairas e vales interiores; Agrosistema intensivo (mosaico agroforestal); Mesotemperado inferior & 2.63 & 1.71 & 1.54 \\ 
  Chairas e vales interiores; Agrosistema intensivo (plantacion forestal); Mesotemperado superior & 2.63 & 0.42 & 6.24 \\ 
  Chairas e vales interiores; Rururbano (diseminado); Mesotemperado inferior & 2.63 & 0.51 & 5.14 \\ 
  Serras; Agrosistema extensivo; Supra e orotemperado & 2.63 & 2.50 & 1.05 \\ 
   \hline
\end{tabular}
\end{table}
% latex table generated in R 3.2.2 by xtable 1.8-0 package
% Wed Dec  2 19:45:32 2015
\begin{table}[p]
\centering
\caption{Frecuencia de aparición de valores patrimoniais identificados na participación pública e frecuencia de tipos asociados Galicia Central} 
\label{vsixotpat7}
\begin{tabular}{lrrr}
  \hline
Tipo de paisaxe & F.Aparic (\%) & F.Tipo (\%) & Ratio \\ 
  \hline
Vales sublitorais; Rururbano (diseminado); Termotemperado & 12.31 & 1.68 & 7.32 \\ 
  Vales sublitorais; Agrosistema intensivo (mosaico agroforestal); Mesotemperado inferior & 11.54 & 7.54 & 1.53 \\ 
  Vales sublitorais; Agrosistema extensivo; Mesotemperado inferior & 10.77 & 2.73 & 3.94 \\ 
  Vales sublitorais; Rururbano (diseminado); Mesotemperado inferior & 9.23 & 0.98 & 9.40 \\ 
  Serras; Agrosistema extensivo; Mesotemperado superior & 6.92 & 5.76 & 1.20 \\ 
  Vales sublitorais; Agrosistema intensivo (mosaico agroforestal); Termotemperado & 6.92 & 2.62 & 2.64 \\ 
  Vales sublitorais; Agrosistema intensivo (superficie de cultivo); Mesotemperado inferior & 6.92 & 0.81 & 8.54 \\ 
  Vales sublitorais; Urbano; Mesotemperado inferior & 5.38 & 0.11 & 49.58 \\ 
  Serras; Matogueira e rochedo; Mesotemperado superior & 4.62 & 5.03 & 0.92 \\ 
  Vales sublitorais; Conxunto Historico; Mesotemperado inferior & 3.85 & 0.00 & 1120.74 \\ 
  Serras; Agrosistema extensivo; Mesotemperado inferior & 3.08 & 2.57 & 1.20 \\ 
  Vales sublitorais; Bosque; Mesotemperado inferior & 2.31 & 0.49 & 4.66 \\ 
  Chairas e vales interiores; Urbano; Termotemperado & 1.54 & 0.08 & 18.74 \\ 
  Vales sublitorais; Agrosistema extensivo; Termotemperado & 1.54 & 0.36 & 4.29 \\ 
  Vales sublitorais; Agrosistema intensivo (plantacion forestal); Termotemperado & 1.54 & 1.79 & 0.86 \\ 
  Vales sublitorais; Matogueira e rochedo; Termotemperado & 1.54 & 0.93 & 1.65 \\ 
   \hline
\end{tabular}
\end{table}
% latex table generated in R 3.2.2 by xtable 1.8-0 package
% Wed Dec  2 19:45:32 2015
\begin{table}[p]
\centering
\caption{Frecuencia de aparición de valores patrimoniais identificados na participación pública e frecuencia de tipos asociados Chairas, Fosas e Serras Ourensás} 
\label{vsixotpat8}
\begin{tabular}{lrrr}
  \hline
Tipo de paisaxe & F.Aparic (\%) & F.Tipo (\%) & Ratio \\ 
  \hline
Serras; Agrosistema extensivo; Mesotemperado inferior & 12.86 & 2.57 & 5.00 \\ 
  Serras; Agrosistema extensivo; Supra e orotemperado & 10.00 & 2.50 & 4.00 \\ 
  Chairas e vales interiores; Agrosistema extensivo; Mesotemperado inferior & 8.57 & 3.58 & 2.39 \\ 
  Serras; Matogueira e rochedo; Mesotemperado superior & 8.57 & 5.03 & 1.70 \\ 
  Chairas e vales interiores; Matogueira e rochedo; Mesotemperado inferior & 7.14 & 1.38 & 5.17 \\ 
  Serras; Matogueira e rochedo; Mesotemperado inferior & 7.14 & 2.73 & 2.62 \\ 
  Serras; Bosque; Mesotemperado superior & 5.71 & 1.00 & 5.71 \\ 
  Chairas e vales interiores; Bosque; Mesotemperado inferior & 4.29 & 0.77 & 5.60 \\ 
  Chairas e vales interiores; Matogueira e rochedo; Termotemperado & 4.29 & 0.66 & 6.53 \\ 
  Chairas e vales interiores; Rururbano (diseminado); Mesotemperado inferior & 4.29 & 0.51 & 8.37 \\ 
  Serras; Agrosistema extensivo; Mesotemperado superior & 4.29 & 5.76 & 0.74 \\ 
  Chairas e vales interiores; Agrosistema extensivo; Termotemperado & 2.86 & 0.61 & 4.66 \\ 
  Chairas e vales interiores; Agrosistema intensivo (superficie de cultivo); Mesotemperado inferior & 2.86 & 1.16 & 2.47 \\ 
  Chairas e vales interiores; Conxunto Historico; Mesotemperado inferior & 2.86 & 0.00 & 2071.80 \\ 
  Serras; Agrosistema intensivo (superficie de cultivo); Supra e orotemperado & 2.86 & 0.23 & 12.35 \\ 
  Chairas e vales interiores; Agrosistema intensivo (mosaico agroforestal); Mesotemperado inferior & 1.43 & 1.71 & 0.84 \\ 
  Chairas e vales interiores; Agrosistema intensivo (mosaico agroforestal); Termotemperado & 1.43 & 0.34 & 4.21 \\ 
  Chairas e vales interiores; Agrosistema intensivo (superficie de cultivo); no data & 1.43 & 0.00 & 808.18 \\ 
  Chairas e vales interiores; Urbano; Mesotemperado inferior & 1.43 & 0.06 & 25.63 \\ 
  Chairas e vales interiores; Viñedo; Termotemperado & 1.43 & 0.28 & 5.15 \\ 
  Serras; Agrosistema intensivo (superficie de cultivo); Mesotemperado inferior & 1.43 & 0.28 & 5.04 \\ 
  Serras; Bosque; Mesotemperado inferior & 1.43 & 0.69 & 2.07 \\ 
  Serras; Matogueira e rochedo; Supra e orotemperado & 1.43 & 6.12 & 0.23 \\ 
   \hline
\end{tabular}
\end{table}
% latex table generated in R 3.2.2 by xtable 1.8-0 package
% Wed Dec  2 19:45:32 2015
\begin{table}[p]
\centering
\caption{Frecuencia de aparición de valores patrimoniais identificados na participación pública e frecuencia de tipos asociados Serras Surorientais} 
\label{vsixotpat9}
\begin{tabular}{lrrr}
  \hline
Tipo de paisaxe & F.Aparic (\%) & F.Tipo (\%) & Ratio \\ 
  \hline
Serras; Agrosistema extensivo; Mesotemperado inferior & 27.59 & 2.57 & 10.74 \\ 
  Canons; Viñedo; Mesomediterráneo & 13.79 & 0.01 & 1154.74 \\ 
  Serras; Agrosistema extensivo; Supra e orotemperado & 10.34 & 2.50 & 4.14 \\ 
  Serras; Agrosistema extensivo; Mesomediterráneo & 6.90 & 0.06 & 124.22 \\ 
  Serras; Agrosistema extensivo; Mesotemperado superior & 6.90 & 5.76 & 1.20 \\ 
  Serras; Matogueira e rochedo; Supra e orotemperado & 6.90 & 6.12 & 1.13 \\ 
  Canons; Bosque; Mesotemperado inferior & 3.45 & 0.30 & 11.67 \\ 
  Canons; Matogueira e rochedo; Mesomediterráneo & 3.45 & 0.18 & 18.66 \\ 
  Canons; Viñedo; Mesotemperado inferior & 3.45 & 0.02 & 149.11 \\ 
  Serras; Agrosistema intensivo (superficie de cultivo); Mesotemperado inferior & 3.45 & 0.28 & 12.16 \\ 
  Serras; Bosque; Mesotemperado inferior & 3.45 & 0.69 & 4.99 \\ 
  Serras; Rururbano (diseminado); Mesotemperado superior & 3.45 & 0.11 & 31.04 \\ 
  Vales sublitorais; Agrosistema extensivo; Mesomediterráneo & 3.45 & 0.00 & 18346.01 \\ 
  Vales sublitorais; Viñedo; Mesomediterráneo & 3.45 & 0.00 & 28645.53 \\ 
   \hline
\end{tabular}
\end{table}
% latex table generated in R 3.2.2 by xtable 1.8-0 package
% Wed Dec  2 19:45:32 2015
\begin{table}[p]
\centering
\caption{Frecuencia de aparición de valores patrimoniais identificados na participación pública e frecuencia de tipos asociados Galicia Setentrional} 
\label{vsixotpat10}
\begin{tabular}{lrrr}
  \hline
Tipo de paisaxe & F.Aparic (\%) & F.Tipo (\%) & Ratio \\ 
  \hline
Vales sublitorais; Agrosistema intensivo (mosaico agroforestal); Mesotemperado inferior & 27.27 & 7.54 & 3.62 \\ 
  Litoral Cantabro-Atlantico; Agrosistema intensivo (mosaico agroforestal); Mesotemperado inferior & 13.64 & 0.29 & 46.55 \\ 
  Serras; Turbeira; Mesotemperado superior & 13.64 & 0.71 & 19.25 \\ 
  Serras; Agrosistema intensivo (mosaico agroforestal); Mesotemperado superior & 9.09 & 1.77 & 5.12 \\ 
  Vales sublitorais; Agrosistema intensivo (plantacion forestal); Mesotemperado inferior & 9.09 & 2.90 & 3.14 \\ 
  Vales sublitorais; Agrosistema intensivo (plantacion forestal); Mesotemperado superior & 9.09 & 0.60 & 15.21 \\ 
  Litoral Cantabro-Atlantico; Agrosistema intensivo (mosaico agroforestal); Termotemperado & 4.55 & 2.43 & 1.87 \\ 
  Litoral Cantabro-Atlantico; Rururbano (diseminado); Termotemperado & 4.55 & 3.29 & 1.38 \\ 
  Serras; Turbeira; Mesotemperado inferior & 4.55 & 0.05 & 85.31 \\ 
  Vales sublitorais; Rururbano (diseminado); Mesotemperado inferior & 4.55 & 0.98 & 4.63 \\ 
   \hline
\end{tabular}
\end{table}
% latex table generated in R 3.2.2 by xtable 1.8-0 package
% Wed Dec  2 19:45:32 2015
\begin{table}[p]
\centering
\caption{Frecuencia de aparición de valores patrimoniais identificados na participación pública e frecuencia de tipos asociados Chairas e Fosas Occidentais} 
\label{vsixotpat11}
\begin{tabular}{lrrr}
  \hline
Tipo de paisaxe & F.Aparic (\%) & F.Tipo (\%) & Ratio \\ 
  \hline
Litoral Cantabro-Atlantico; Agrosistema intensivo (mosaico agroforestal); Termotemperado & 15.38 & 2.43 & 6.33 \\ 
  Vales sublitorais; Agrosistema intensivo (plantacion forestal); Mesotemperado inferior & 15.38 & 2.90 & 5.31 \\ 
  Vales sublitorais; Agrosistema extensivo; Mesotemperado inferior & 12.82 & 2.73 & 4.69 \\ 
  Vales sublitorais; Agrosistema intensivo (mosaico agroforestal); Termotemperado & 10.26 & 2.62 & 3.92 \\ 
  Litoral Cantabro-Atlantico; Conxunto Historico; Termotemperado & 7.69 & 0.02 & 406.74 \\ 
  Vales sublitorais; Agrosistema intensivo (mosaico agroforestal); Mesotemperado inferior & 7.69 & 7.54 & 1.02 \\ 
  Vales sublitorais; Rururbano (diseminado); Mesotemperado inferior & 7.69 & 0.98 & 7.83 \\ 
  Vales sublitorais; Agrosistema intensivo (superficie de cultivo); Termotemperado & 5.13 & 0.10 & 52.32 \\ 
  Vales sublitorais; Rururbano (diseminado); Termotemperado & 5.13 & 1.68 & 3.05 \\ 
  Litoral Cantabro-Atlantico; Agrosistema extensivo; no data & 2.56 & 0.01 & 235.98 \\ 
  Litoral Cantabro-Atlantico; Rururbano (diseminado); Termotemperado & 2.56 & 3.29 & 0.78 \\ 
  Litoral Cantabro-Atlantico; Urbano; Termotemperado & 2.56 & 0.53 & 4.83 \\ 
  Vales sublitorais; Agrosistema intensivo (mosaico agroforestal); Mesotemperado superior & 2.56 & 1.52 & 1.69 \\ 
  Vales sublitorais; Agrosistema intensivo (plantacion forestal); Termotemperado & 2.56 & 1.79 & 1.43 \\ 
   \hline
\end{tabular}
\end{table}
% latex table generated in R 3.2.2 by xtable 1.8-0 package
% Wed Dec  2 19:45:32 2015
\begin{table}[p]
\centering
\caption{Frecuencia de aparición de valores patrimoniais identificados na participación pública e frecuencia de tipos asociados Rías Baixas} 
\label{vsixotpat12}
\begin{tabular}{lrrr}
  \hline
Tipo de paisaxe & F.Aparic (\%) & F.Tipo (\%) & Ratio \\ 
  \hline
Serras; Matogueira e rochedo; Mesotemperado superior & 16.13 & 5.03 & 3.21 \\ 
  Litoral Cantabro-Atlantico; Rururbano (diseminado); Termotemperado & 9.68 & 3.29 & 2.94 \\ 
  Vales sublitorais; Rururbano (diseminado); Termotemperado & 9.68 & 1.68 & 5.76 \\ 
  Litoral Cantabro-Atlantico; Agrosistema intensivo (mosaico agroforestal); Termotemperado & 6.45 & 2.43 & 2.66 \\ 
  Litoral Cantabro-Atlantico; Matogueira e rochedo; Termotemperado & 6.45 & 0.83 & 7.76 \\ 
  Serras; Matogueira e rochedo; Mesotemperado inferior & 6.45 & 2.73 & 2.37 \\ 
  Vales sublitorais; Agrosistema intensivo (plantacion forestal); Termotemperado & 6.45 & 1.79 & 3.60 \\ 
  Vales sublitorais; Matogueira e rochedo; Mesotemperado inferior & 6.45 & 1.89 & 3.41 \\ 
  Litoral Cantabro-Atlantico; Agrosistema intensivo (mosaico agroforestal); no data & 3.23 & 0.04 & 75.86 \\ 
  Litoral Cantabro-Atlantico; Agrosistema intensivo (plantacion forestal); Termotemperado & 3.23 & 1.32 & 2.45 \\ 
  Litoral Cantabro-Atlantico; Bosque; Termotemperado & 3.23 & 0.03 & 97.99 \\ 
  Litoral Cantabro-Atlantico; Urbano; no data & 3.23 & 0.04 & 72.04 \\ 
  Litoral Cantabro-Atlantico; Viñedo; Termotemperado & 3.23 & 0.14 & 23.31 \\ 
  Serras; Rururbano (diseminado); Mesotemperado inferior & 3.23 & 0.08 & 38.10 \\ 
  Vales sublitorais; Agrosistema intensivo (mosaico agroforestal); Mesotemperado inferior & 3.23 & 7.54 & 0.43 \\ 
  Vales sublitorais; Agrosistema intensivo (mosaico agroforestal); Termotemperado & 3.23 & 2.62 & 1.23 \\ 
  Vales sublitorais; Agrosistema intensivo (plantacion forestal); Mesotemperado superior & 3.23 & 0.60 & 5.40 \\ 
  Vales sublitorais; Matogueira e rochedo; Termotemperado & 3.23 & 0.93 & 3.46 \\ 
   \hline
\end{tabular}
\end{table}


\clearpage
% latex table generated in R 3.2.5 by xtable 1.8-0 package
% Mon May  9 13:23:06 2016
\begin{table}[p]
\centering
\caption{Frecuencia de aparición de valores estéticos identificados na participación pública e frecuencia de tipos asociados Golfo Ártabro} 
\label{vsixotest1}
\begin{tabular}{lrrr}
  \hline
Tipo de paisaxe & F.Aparic (\%) & F.Tipo (\%) & Ratio \\ 
  \hline
category 0; Conxunto Historico; category 0 & 13.56 & 0.08 & 164.21 \\ 
  Litoral Cantabro-Atlantico; Agrosistema intensivo (mosaico agroforestal); Termotemperado & 11.86 & 18.64 & 0.64 \\ 
  Vales sublitorais; Agrosistema intensivo (mosaico agroforestal); Mesotemperado inferior & 11.86 & 21.75 & 0.55 \\ 
  Canons; Bosque; Mesotemperado inferior & 8.47 & 1.56 & 5.43 \\ 
  Vales sublitorais; Bosque; Mesotemperado inferior & 8.47 & 0.78 & 10.81 \\ 
  Litoral Cantabro-Atlantico; Agrosistema extensivo; Termotemperado & 5.08 & 1.02 & 5.00 \\ 
  Litoral Cantabro-Atlantico; Agrosistema intensivo (plantacion forestal); Termotemperado & 5.08 & 5.79 & 0.88 \\ 
  Litoral Cantabro-Atlantico; Matogueira e rochedo; Termotemperado & 5.08 & 1.19 & 4.26 \\ 
  Litoral Cantabro-Atlantico; Urbano; Termotemperado & 5.08 & 4.93 & 1.03 \\ 
  Vales sublitorais; Agrosistema intensivo (plantacion forestal); Mesotemperado inferior & 5.08 & 13.21 & 0.38 \\ 
  Canons; Matogueira e rochedo; Mesotemperado inferior & 3.39 & 0.08 & 41.37 \\ 
  Litoral Cantabro-Atlantico; Rururbano (diseminado); Termotemperado & 3.39 & 7.02 & 0.48 \\ 
  Vales sublitorais; Matogueira e rochedo; Mesotemperado inferior & 3.39 & 0.64 & 5.32 \\ 
  category 0; Lamina de auga; category 0 & 1.69 & 0.55 & 3.09 \\ 
  Serras; Agrosistema intensivo (plantacion forestal); Mesotemperado inferior & 1.69 & 0.51 & 3.34 \\ 
  Serras; Turbeira; Mesotemperado superior & 1.69 & 1.53 & 1.11 \\ 
  Vales sublitorais; Matogueira e rochedo; Mesotemperado superior & 1.69 & 1.17 & 1.45 \\ 
  Vales sublitorais; Urbano; Termotemperado & 1.69 & 0.22 & 7.84 \\ 
   & 1.69 &  &  \\ 
   \hline
\end{tabular}
\end{table}
% latex table generated in R 3.2.5 by xtable 1.8-0 package
% Mon May  9 13:23:06 2016
\begin{table}[p]
\centering
\caption{Frecuencia de aparición de valores estéticos identificados na participación pública e frecuencia de tipos asociados A Mariña - Baixo Eo} 
\label{vsixotest2}
\begin{tabular}{lrrr}
  \hline
Tipo de paisaxe & F.Aparic (\%) & F.Tipo (\%) & Ratio \\ 
  \hline
Vales sublitorais; Agrosistema intensivo (plantacion forestal); Mesotemperado inferior & 25.58 & 23.68 & 1.08 \\ 
  Litoral Cantabro-Atlantico; Agrosistema intensivo (plantacion forestal); Mesotemperado inferior & 11.63 & 18.89 & 0.62 \\ 
  Litoral Cantabro-Atlantico; Agrosistema intensivo (mosaico agroforestal); Mesotemperado inferior & 6.98 & 3.57 & 1.95 \\ 
  Litoral Cantabro-Atlantico; Agrosistema intensivo (mosaico agroforestal); Termotemperado & 6.98 & 7.20 & 0.97 \\ 
  Litoral Cantabro-Atlantico; Matogueira e rochedo; Mesotemperado inferior & 6.98 & 0.05 & 148.71 \\ 
  category 0; Praias e cantis; category 0 & 4.65 & 0.25 & 18.96 \\ 
  Litoral Cantabro-Atlantico; Rururbano (diseminado); Termotemperado & 4.65 & 2.68 & 1.73 \\ 
  Litoral Cantabro-Atlantico; Urbano; Termotemperado & 4.65 & 1.10 & 4.21 \\ 
  Vales sublitorais; Agrosistema intensivo (mosaico agroforestal); Mesotemperado inferior & 4.65 & 12.86 & 0.36 \\ 
   & 4.65 &  &  \\ 
  category 0; Conxunto Historico; category 0 & 2.33 & 0.10 & 24.21 \\ 
  Litoral Cantabro-Atlantico; Matogueira e rochedo; Termotemperado & 2.33 & 0.43 & 5.42 \\ 
  Serras; Agrosistema intensivo (plantacion forestal); Mesotemperado superior & 2.33 & 6.80 & 0.34 \\ 
  Serras; Bosque; Mesotemperado superior & 2.33 & 0.70 & 3.30 \\ 
  Serras; Matogueira e rochedo; Mesotemperado superior & 2.33 & 1.71 & 1.36 \\ 
  Serras; Turbeira; Mesotemperado superior & 2.33 & 2.64 & 0.88 \\ 
  Vales sublitorais; Agrosistema extensivo; Mesotemperado inferior & 2.33 & 0.98 & 2.37 \\ 
  Vales sublitorais; Bosque; Mesotemperado superior & 2.33 & 1.22 & 1.91 \\ 
   \hline
\end{tabular}
\end{table}
% latex table generated in R 3.2.5 by xtable 1.8-0 package
% Mon May  9 13:23:06 2016
\begin{table}[p]
\centering
\caption{Frecuencia de aparición de valores estéticos identificados na participación pública e frecuencia de tipos asociados Costa Sur - Baixo Miño} 
\label{vsixotest3}
\begin{tabular}{lrrr}
  \hline
Tipo de paisaxe & F.Aparic (\%) & F.Tipo (\%) & Ratio \\ 
  \hline
Serras; Agrosistema intensivo (plantacion forestal); Termotemperado & 14.13 & 3.79 & 3.73 \\ 
  Serras; Matogueira e rochedo; Mesotemperado inferior & 14.13 & 3.84 & 3.68 \\ 
  Vales sublitorais; Agrosistema intensivo (plantacion forestal); Termotemperado & 13.04 & 16.82 & 0.78 \\ 
  category 0; Conxunto Historico; category 0 & 8.70 & 0.22 & 38.90 \\ 
  Litoral Cantabro-Atlantico; Agrosistema intensivo (mosaico agroforestal); Termotemperado & 5.43 & 5.83 & 0.93 \\ 
  Litoral Cantabro-Atlantico; Agrosistema intensivo (plantacion forestal); Termotemperado & 5.43 & 6.69 & 0.81 \\ 
  Serras; Agrosistema intensivo (plantacion forestal); Mesotemperado inferior & 5.43 & 3.38 & 1.61 \\ 
  Serras; Turbeira; Mesotemperado inferior & 5.43 & 0.12 & 45.08 \\ 
  Vales sublitorais; Agrosistema intensivo (mosaico agroforestal); Termotemperado & 5.43 & 7.55 & 0.72 \\ 
  Vales sublitorais; Matogueira e rochedo; Termotemperado & 3.26 & 2.72 & 1.20 \\ 
  Litoral Cantabro-Atlantico; Agrosistema intensivo (superficie de cultivo); Termotemperado & 2.17 & 0.16 & 13.25 \\ 
  Vales sublitorais; Bosque; Termotemperado & 2.17 & 1.44 & 1.51 \\ 
  category 0; Lamina de auga; category 0 & 1.09 & 0.46 & 2.38 \\ 
  category 0; Praias e cantis; category 0 & 1.09 & 0.07 & 15.79 \\ 
  Chairas e vales interiores; Agrosistema intensivo (mosaico agroforestal); Mesotemperado inferior & 1.09 & 0.84 & 1.29 \\ 
  Chairas e vales interiores; Agrosistema intensivo (plantacion forestal); Termotemperado & 1.09 & 6.71 & 0.16 \\ 
  Chairas e vales interiores; Matogueira e rochedo; Mesotemperado inferior & 1.09 & 1.96 & 0.55 \\ 
  Litoral Cantabro-Atlantico; Agrosistema extensivo; Termotemperado & 1.09 & 0.61 & 1.77 \\ 
  Litoral Cantabro-Atlantico; Matogueira e rochedo; Termotemperado & 1.09 & 0.52 & 2.09 \\ 
  Litoral Cantabro-Atlantico; Rururbano (diseminado); Termotemperado & 1.09 & 4.97 & 0.22 \\ 
  Litoral Cantabro-Atlantico; Vinedo; Termotemperado & 1.09 & 0.70 & 1.55 \\ 
  Serras; Agrosistema extensivo; Mesotemperado superior & 1.09 & 0.28 & 3.89 \\ 
  Serras; Matogueira e rochedo; Supra e orotemperado & 1.09 & 0.14 & 7.84 \\ 
  Vales sublitorais; Agrosistema intensivo (superficie de cultivo); Termotemperado & 1.09 & 0.10 & 11.01 \\ 
  Vales sublitorais; Extractivo; Termotemperado & 1.09 & 0.17 & 6.52 \\ 
  Vales sublitorais; Rururbano (diseminado); Termotemperado & 1.09 & 2.82 & 0.39 \\ 
   \hline
\end{tabular}
\end{table}
% latex table generated in R 3.2.5 by xtable 1.8-0 package
% Mon May  9 13:23:06 2016
\begin{table}[p]
\centering
\caption{Frecuencia de aparición de valores estéticos identificados na participación pública e frecuencia de tipos asociados Ribeiras Encaixadas do Miño e do Sil} 
\label{vsixotest4}
\begin{tabular}{lrrr}
  \hline
Tipo de paisaxe & F.Aparic (\%) & F.Tipo (\%) & Ratio \\ 
  \hline
Canons; Bosque; Mesotemperado inferior & 17.59 & 2.67 & 6.60 \\ 
  category 0; Lamina de auga; category 0 & 8.33 & 1.89 & 4.40 \\ 
  Chairas e vales interiores; Bosque; Termotemperado & 8.33 & 3.03 & 2.75 \\ 
  Chairas e vales interiores; Agrosistema intensivo (plantacion forestal); Termotemperado & 4.63 & 6.81 & 0.68 \\ 
  Canons; Agrosistema intensivo (plantacion forestal); category 0 & 3.70 & 1.25 & 2.97 \\ 
  category 0; Conxunto Historico; category 0 & 3.70 & 0.04 & 90.47 \\ 
  Chairas e vales interiores; Agrosistema intensivo (plantacion forestal); category 0 & 3.70 & 1.83 & 2.02 \\ 
  Serras; Matogueira e rochedo; Supra e orotemperado & 3.70 & 9.08 & 0.41 \\ 
  Canons; Vinedo; Mesotemperado inferior & 2.78 & 0.25 & 11.13 \\ 
  Chairas e vales interiores; Agrosistema intensivo (mosaico agroforestal); Mesotemperado inferior & 2.78 & 5.37 & 0.52 \\ 
  Chairas e vales interiores; Agrosistema intensivo (plantacion forestal); Mesotemperado inferior & 2.78 & 4.37 & 0.64 \\ 
  Serras; Bosque; Mesotemperado inferior & 2.78 & 2.32 & 1.20 \\ 
  Serras; Bosque; Supra e orotemperado & 2.78 & 0.47 & 5.85 \\ 
  Serras; Matogueira e rochedo; Mesotemperado superior & 2.78 & 3.10 & 0.90 \\ 
  Canons; Agrosistema intensivo (plantacion forestal); Termotemperado & 1.85 & 1.39 & 1.34 \\ 
  Canons; Bosque; category 0 & 1.85 & 0.39 & 4.70 \\ 
  Canons; Matogueira e rochedo; category 0 & 1.85 & 0.96 & 1.93 \\ 
  Chairas e vales interiores; Matogueira e rochedo; Mesotemperado inferior & 1.85 & 3.61 & 0.51 \\ 
  Chairas e vales interiores; Rururbano (diseminado); Termotemperado & 1.85 & 0.91 & 2.04 \\ 
  Chairas e vales interiores; Urbano; Termotemperado & 1.85 & 1.01 & 1.84 \\ 
  Serras; Agrosistema extensivo; Mesotemperado inferior & 1.85 & 1.98 & 0.94 \\ 
   \hline
\end{tabular}
\end{table}
% latex table generated in R 3.2.5 by xtable 1.8-0 package
% Mon May  9 13:23:06 2016
\begin{table}[p]
\centering
\caption{Frecuencia de aparición de valores estéticos identificados na participación pública e frecuencia de tipos asociados Serras Orientais} 
\label{vsixotest5}
\begin{tabular}{lrrr}
  \hline
Tipo de paisaxe & F.Aparic (\%) & F.Tipo (\%) & Ratio \\ 
  \hline
Serras; Bosque; Supra e orotemperado & 20.51 & 7.67 & 2.68 \\ 
  Serras; Matogueira e rochedo; Supra e orotemperado & 14.10 & 15.61 & 0.90 \\ 
  Serras; Agrosistema extensivo; Supra e orotemperado & 11.54 & 8.91 & 1.29 \\ 
  Vales sublitorais; Bosque; Mesotemperado inferior & 8.97 & 3.83 & 2.34 \\ 
  Serras; Agrosistema intensivo (mosaico agroforestal); Mesotemperado superior & 6.41 & 3.95 & 1.62 \\ 
  Serras; Agrosistema extensivo; Mesotemperado superior & 5.13 & 7.30 & 0.70 \\ 
  Serras; Matogueira e rochedo; Mesotemperado superior & 5.13 & 5.41 & 0.95 \\ 
  Serras; Agrosistema intensivo (mosaico agroforestal); Supra e orotemperado & 3.85 & 5.74 & 0.67 \\ 
  Vales sublitorais; Agrosistema extensivo; Mesotemperado inferior & 3.85 & 1.41 & 2.73 \\ 
  Vales sublitorais; Bosque; Mesotemperado superior & 3.85 & 4.16 & 0.92 \\ 
  Canons; Bosque; Mesotemperado inferior & 2.56 & 0.44 & 5.88 \\ 
  category 0; Lamina de auga; category 0 & 2.56 & 0.16 & 15.93 \\ 
  Serras; Agrosistema intensivo (plantacion forestal); Supra e orotemperado & 2.56 & 6.86 & 0.37 \\ 
  Serras; Bosque; Mesotemperado superior & 2.56 & 3.65 & 0.70 \\ 
  Vales sublitorais; Agrosistema extensivo; Mesotemperado superior & 2.56 & 5.11 & 0.50 \\ 
  Chairas e vales interiores; Agrosistema extensivo; Mesotemperado superior & 1.28 & 0.25 & 5.19 \\ 
  Chairas e vales interiores; Bosque; category 0 & 1.28 & 0.19 & 6.61 \\ 
  Serras; Bosque; Mesotemperado inferior & 1.28 & 0.55 & 2.34 \\ 
   \hline
\end{tabular}
\end{table}
% latex table generated in R 3.2.5 by xtable 1.8-0 package
% Mon May  9 13:23:06 2016
\begin{table}[p]
\centering
\caption{Frecuencia de aparición de valores estéticos identificados na participación pública e frecuencia de tipos asociados Chairas e Fosas Luguesas} 
\label{vsixotest6}
\begin{tabular}{lrrr}
  \hline
Tipo de paisaxe & F.Aparic (\%) & F.Tipo (\%) & Ratio \\ 
  \hline
Chairas e vales interiores; Agrosistema extensivo; Mesotemperado inferior & 11.11 & 6.80 & 1.63 \\ 
  Chairas e vales interiores; Agrosistema extensivo; Mesotemperado superior & 11.11 & 9.68 & 1.15 \\ 
  Chairas e vales interiores; Agrosistema intensivo (mosaico agroforestal); Mesotemperado superior & 11.11 & 20.11 & 0.55 \\ 
  Chairas e vales interiores; Bosque; Mesotemperado inferior & 11.11 & 1.92 & 5.79 \\ 
  category 0; Conxunto Historico; category 0 & 7.94 & 0.02 & 398.19 \\ 
  Chairas e vales interiores; Urbano; Mesotemperado superior & 7.94 & 0.45 & 17.60 \\ 
  Serras; Agrosistema intensivo (plantacion forestal); Supra e orotemperado & 6.35 & 0.59 & 10.79 \\ 
  Chairas e vales interiores; Agrosistema intensivo (superficie de cultivo); Mesotemperado superior & 4.76 & 5.78 & 0.82 \\ 
  Chairas e vales interiores; Bosque; Mesotemperado superior & 4.76 & 2.02 & 2.36 \\ 
  Chairas e vales interiores; Agrosistema intensivo (plantacion forestal); Mesotemperado inferior & 3.17 & 3.31 & 0.96 \\ 
  Serras; Agrosistema intensivo (mosaico agroforestal); Mesotemperado superior & 3.17 & 9.79 & 0.32 \\ 
  Serras; Turbeira; Mesotemperado superior & 3.17 & 1.08 & 2.93 \\ 
  category 0; Lamina de auga; category 0 & 1.59 & 0.09 & 16.71 \\ 
  Chairas e vales interiores; Agrosistema intensivo (mosaico agroforestal); Mesotemperado inferior & 1.59 & 9.44 & 0.17 \\ 
  Chairas e vales interiores; Agrosistema intensivo (plantacion forestal); Mesotemperado superior & 1.59 & 3.97 & 0.40 \\ 
  Chairas e vales interiores; Urbano; Mesotemperado inferior & 1.59 & 0.19 & 8.49 \\ 
  Serras; Agrosistema extensivo; Mesotemperado superior & 1.59 & 5.94 & 0.27 \\ 
  Serras; Agrosistema extensivo; Supra e orotemperado & 1.59 & 0.37 & 4.29 \\ 
  Serras; Agrosistema intensivo (plantacion forestal); Mesotemperado superior & 1.59 & 2.29 & 0.69 \\ 
  Serras; Agrosistema intensivo (superficie de cultivo); Mesotemperado superior & 1.59 & 3.04 & 0.52 \\ 
  Serras; Matogueira e rochedo; Supra e orotemperado & 1.59 & 0.48 & 3.31 \\ 
   \hline
\end{tabular}
\end{table}
% latex table generated in R 3.2.5 by xtable 1.8-0 package
% Mon May  9 13:23:06 2016
\begin{table}[p]
\centering
\caption{Frecuencia de aparición de valores estéticos identificados na participación pública e frecuencia de tipos asociados Galicia Central} 
\label{vsixotest7}
\begin{tabular}{lrrr}
  \hline
Tipo de paisaxe & F.Aparic (\%) & F.Tipo (\%) & Ratio \\ 
  \hline
Vales sublitorais; Agrosistema intensivo (mosaico agroforestal); Mesotemperado inferior & 26.94 & 43.06 & 0.63 \\ 
  Vales sublitorais; Agrosistema extensivo; Mesotemperado inferior & 9.84 & 5.21 & 1.89 \\ 
  Vales sublitorais; Urbano; Mesotemperado inferior & 6.74 & 0.64 & 10.49 \\ 
  Vales sublitorais; Agrosistema intensivo (superficie de cultivo); Mesotemperado inferior & 6.22 & 3.18 & 1.96 \\ 
  Serras; Agrosistema extensivo; Mesotemperado superior & 5.70 & 4.47 & 1.27 \\ 
  category 0; Lamina de auga; category 0 & 4.15 & 0.30 & 13.90 \\ 
  Serras; Matogueira e rochedo; Mesotemperado superior & 3.63 & 6.90 & 0.53 \\ 
  Vales sublitorais; Agrosistema intensivo (plantacion forestal); Mesotemperado inferior & 3.63 & 5.84 & 0.62 \\ 
  category 0; Conxunto Historico; category 0 & 3.11 & 0.02 & 176.99 \\ 
  Vales sublitorais; Bosque; Mesotemperado inferior & 2.59 & 0.80 & 3.24 \\ 
  Serras; Agrosistema intensivo (mosaico agroforestal); Mesotemperado superior & 2.07 & 3.93 & 0.53 \\ 
  Vales sublitorais; Matogueira e rochedo; Mesotemperado inferior & 2.07 & 1.65 & 1.26 \\ 
  Chairas e vales interiores; Urbano; Termotemperado & 1.55 & 0.03 & 61.16 \\ 
  Serras; Agrosistema extensivo; Mesotemperado inferior & 1.55 & 1.17 & 1.32 \\ 
  Serras; Agrosistema intensivo (plantacion forestal); Supra e orotemperado & 1.55 & 0.54 & 2.90 \\ 
  Serras; Agrosistema intensivo (superficie de cultivo); Mesotemperado superior & 1.55 & 1.44 & 1.08 \\ 
  Serras; Bosque; Mesotemperado inferior & 1.55 & 0.45 & 3.44 \\ 
  Vales sublitorais; Rururbano (diseminado); Mesotemperado inferior & 1.55 & 0.23 & 6.81 \\ 
  Chairas e vales interiores; Bosque; Mesotemperado inferior & 1.04 & 0.68 & 1.53 \\ 
  Serras; Matogueira e rochedo; Mesotemperado inferior & 1.04 & 0.62 & 1.67 \\ 
  Vales sublitorais; Agrosistema extensivo; Termotemperado & 1.04 & 0.29 & 3.60 \\ 
  Vales sublitorais; Agrosistema intensivo (mosaico agroforestal); Mesotemperado superior & 1.04 & 0.57 & 1.80 \\ 
  Vales sublitorais; Agrosistema intensivo (plantacion forestal); Termotemperado & 1.04 & 1.80 & 0.58 \\ 
  Vales sublitorais; Bosque; Termotemperado & 1.04 & 0.32 & 3.20 \\ 
  Vales sublitorais; Matogueira e rochedo; Termotemperado & 1.04 & 0.43 & 2.42 \\ 
  Vales sublitorais; Rururbano (diseminado); Termotemperado & 1.04 & 0.34 & 3.08 \\ 
   \hline
\end{tabular}
\end{table}
% latex table generated in R 3.2.5 by xtable 1.8-0 package
% Mon May  9 13:23:07 2016
\begin{table}[p]
\centering
\caption{Frecuencia de aparición de valores estéticos identificados na participación pública e frecuencia de tipos asociados Chairas, Fosas e Serras Ourensás} 
\label{vsixotest8}
\begin{tabular}{lrrr}
  \hline
Tipo de paisaxe & F.Aparic (\%) & F.Tipo (\%) & Ratio \\ 
  \hline
Serras; Matogueira e rochedo; Supra e orotemperado & 13.73 & 14.06 & 0.98 \\ 
  Serras; Matogueira e rochedo; Mesotemperado superior & 10.78 & 10.51 & 1.03 \\ 
  Serras; Agrosistema extensivo; Mesotemperado inferior & 6.86 & 5.23 & 1.31 \\ 
  Chairas e vales interiores; Agrosistema extensivo; Mesotemperado inferior & 5.88 & 7.38 & 0.80 \\ 
  Chairas e vales interiores; Matogueira e rochedo; Mesotemperado inferior & 5.88 & 4.82 & 1.22 \\ 
  Serras; Matogueira e rochedo; Mesotemperado inferior & 5.88 & 5.15 & 1.14 \\ 
  Chairas e vales interiores; Bosque; Mesotemperado inferior & 4.90 & 6.18 & 0.79 \\ 
  Serras; Bosque; Mesotemperado superior & 4.90 & 2.22 & 2.21 \\ 
  Chairas e vales interiores; Agrosistema intensivo (plantacion forestal); Termotemperado & 3.92 & 3.78 & 1.04 \\ 
  Chairas e vales interiores; Agrosistema intensivo (superficie de cultivo); Mesotemperado inferior & 3.92 & 7.53 & 0.52 \\ 
  Chairas e vales interiores; Matogueira e rochedo; Termotemperado & 2.94 & 1.80 & 1.64 \\ 
  Serras; Agrosistema intensivo (mosaico agroforestal); Supra e orotemperado & 2.94 & 0.26 & 11.15 \\ 
  category 0; Conxunto Historico; category 0 & 1.96 & 0.01 & 193.61 \\ 
  Chairas e vales interiores; Agrosistema extensivo; Termotemperado & 1.96 & 0.66 & 2.98 \\ 
  Chairas e vales interiores; Agrosistema intensivo (plantacion forestal); Mesotemperado inferior & 1.96 & 3.30 & 0.59 \\ 
  Chairas e vales interiores; Rururbano (diseminado); Mesotemperado inferior & 1.96 & 0.32 & 6.06 \\ 
  Chairas e vales interiores; Urbano; Mesotemperado inferior & 1.96 & 0.23 & 8.44 \\ 
  Serras; Agrosistema extensivo; Mesotemperado superior & 1.96 & 2.44 & 0.80 \\ 
  Serras; Agrosistema extensivo; Supra e orotemperado & 1.96 & 1.96 & 1.00 \\ 
  Serras; Agrosistema intensivo (plantacion forestal); Mesotemperado superior & 1.96 & 1.41 & 1.39 \\ 
  Serras; Agrosistema intensivo (superficie de cultivo); Supra e orotemperado & 1.96 & 0.46 & 4.23 \\ 
  Serras; Bosque; Supra e orotemperado & 1.96 & 1.18 & 1.66 \\ 
   \hline
\end{tabular}
\end{table}
% latex table generated in R 3.2.5 by xtable 1.8-0 package
% Mon May  9 13:23:07 2016
\begin{table}[p]
\centering
\caption{Frecuencia de aparición de valores estéticos identificados na participación pública e frecuencia de tipos asociados Serras Surorientais} 
\label{vsixotest9}
\begin{tabular}{lrrr}
  \hline
Tipo de paisaxe & F.Aparic (\%) & F.Tipo (\%) & Ratio \\ 
  \hline
Serras; Matogueira e rochedo; Supra e orotemperado & 32.32 & 47.28 & 0.68 \\ 
  Serras; Matogueira e rochedo; Mesotemperado inferior & 12.12 & 2.75 & 4.41 \\ 
  Serras; Agrosistema extensivo; Mesotemperado inferior & 10.10 & 4.79 & 2.11 \\ 
  Serras; Bosque; Mesotemperado superior & 8.08 & 5.61 & 1.44 \\ 
  Canons; Vinedo; category 0 & 7.07 & 0.03 & 222.75 \\ 
  Serras; Agrosistema intensivo (plantacion forestal); Supra e orotemperado & 6.06 & 10.23 & 0.59 \\ 
  category 0; Lamina de auga; category 0 & 4.04 & 1.38 & 2.92 \\ 
  Canons; Agrosistema intensivo (plantacion forestal); category 0 & 3.03 & 0.26 & 11.64 \\ 
  Serras; Agrosistema extensivo; Mesotemperado superior & 3.03 & 6.24 & 0.49 \\ 
  Serras; Bosque; Mesotemperado inferior & 3.03 & 3.06 & 0.99 \\ 
  Serras; Bosque; Supra e orotemperado & 2.02 & 1.26 & 1.61 \\ 
  Serras; Urbano; category 0 & 2.02 & 0.01 & 329.41 \\ 
  Canons; Bosque; Mesotemperado inferior & 1.01 & 1.00 & 1.01 \\ 
  Canons; Vinedo; Mesotemperado inferior & 1.01 & 0.07 & 15.35 \\ 
  Serras; Agrosistema extensivo; Supra e orotemperado & 1.01 & 2.94 & 0.34 \\ 
  Serras; Agrosistema intensivo (mosaico agroforestal); Mesotemperado inferior & 1.01 & 0.76 & 1.33 \\ 
  Serras; Agrosistema intensivo (mosaico agroforestal); Mesotemperado superior & 1.01 & 0.81 & 1.25 \\ 
  Serras; Agrosistema intensivo (superficie de cultivo); Mesotemperado superior & 1.01 & 0.72 & 1.41 \\ 
  Serras; Urbano; Mesotemperado inferior & 1.01 & 0.02 & 40.71 \\ 
   \hline
\end{tabular}
\end{table}
% latex table generated in R 3.2.5 by xtable 1.8-0 package
% Mon May  9 13:23:07 2016
\begin{table}[p]
\centering
\caption{Frecuencia de aparición de valores estéticos identificados na participación pública e frecuencia de tipos asociados Galicia Setentrional} 
\label{vsixotest10}
\begin{tabular}{lrrr}
  \hline
Tipo de paisaxe & F.Aparic (\%) & F.Tipo (\%) & Ratio \\ 
  \hline
Serras; Turbeira; Mesotemperado superior & 22.22 & 12.44 & 1.79 \\ 
  Vales sublitorais; Agrosistema intensivo (plantacion forestal); Mesotemperado inferior & 13.89 & 17.31 & 0.80 \\ 
  Vales sublitorais; Agrosistema intensivo (mosaico agroforestal); Mesotemperado inferior & 8.33 & 13.63 & 0.61 \\ 
  Litoral Cantabro-Atlantico; Agrosistema intensivo (mosaico agroforestal); Termotemperado & 6.94 & 5.53 & 1.26 \\ 
  Serras; Matogueira e rochedo; Mesotemperado superior & 5.56 & 6.79 & 0.82 \\ 
  Vales sublitorais; Bosque; Mesotemperado inferior & 5.56 & 1.49 & 3.72 \\ 
  Litoral Cantabro-Atlantico; Agrosistema intensivo (plantacion forestal); Mesotemperado inferior & 4.17 & 2.79 & 1.49 \\ 
  Litoral Cantabro-Atlantico; Agrosistema intensivo (plantacion forestal); Termotemperado & 4.17 & 7.61 & 0.55 \\ 
  Serras; Agrosistema intensivo (plantacion forestal); Mesotemperado superior & 4.17 & 4.68 & 0.89 \\ 
  category 0; Lamina de auga; category 0 & 2.78 & 0.89 & 3.12 \\ 
  category 0; Praias e cantis; category 0 & 2.78 & 0.94 & 2.94 \\ 
  Serras; Agrosistema extensivo; Mesotemperado superior & 2.78 & 2.70 & 1.03 \\ 
  Serras; Agrosistema intensivo (mosaico agroforestal); Mesotemperado superior & 2.78 & 3.70 & 0.75 \\ 
  Serras; Bosque; Mesotemperado inferior & 2.78 & 0.22 & 12.67 \\ 
  Vales sublitorais; Matogueira e rochedo; Mesotemperado inferior & 2.78 & 0.62 & 4.45 \\ 
  Litoral Cantabro-Atlantico; Agrosistema intensivo (superficie de cultivo); Termotemperado & 1.39 & 0.47 & 2.93 \\ 
  Litoral Cantabro-Atlantico; Matogueira e rochedo; Termotemperado & 1.39 & 1.77 & 0.78 \\ 
  Litoral Cantabro-Atlantico; Urbano; Mesotemperado inferior & 1.39 & 0.03 & 43.75 \\ 
  Vales sublitorais; Agrosistema intensivo (mosaico agroforestal); Mesotemperado superior & 1.39 & 0.82 & 1.69 \\ 
  Vales sublitorais; Rururbano (diseminado); Mesotemperado inferior & 1.39 & 0.08 & 16.98 \\ 
   & 1.39 &  &  \\ 
   \hline
\end{tabular}
\end{table}
% latex table generated in R 3.2.5 by xtable 1.8-0 package
% Mon May  9 13:23:07 2016
\begin{table}[p]
\centering
\caption{Frecuencia de aparición de valores estéticos identificados na participación pública e frecuencia de tipos asociados Chairas e Fosas Occidentais} 
\label{vsixotest11}
\begin{tabular}{lrrr}
  \hline
Tipo de paisaxe & F.Aparic (\%) & F.Tipo (\%) & Ratio \\ 
  \hline
Vales sublitorais; Agrosistema intensivo (plantacion forestal); Mesotemperado inferior & 27.50 & 15.20 & 1.81 \\ 
  Vales sublitorais; Agrosistema intensivo (mosaico agroforestal); Mesotemperado inferior & 20.00 & 36.05 & 0.55 \\ 
  Litoral Cantabro-Atlantico; Agrosistema intensivo (plantacion forestal); Termotemperado & 9.17 & 7.60 & 1.21 \\ 
  category 0; Praias e cantis; category 0 & 8.33 & 0.54 & 15.50 \\ 
  Litoral Cantabro-Atlantico; Matogueira e rochedo; Termotemperado & 7.50 & 4.58 & 1.64 \\ 
  category 0; Conxunto Historico; category 0 & 5.83 & 0.12 & 47.91 \\ 
  Litoral Cantabro-Atlantico; Urbano; Termotemperado & 4.17 & 0.66 & 6.33 \\ 
  Litoral Cantabro-Atlantico; Agrosistema intensivo (mosaico agroforestal); Termotemperado & 2.50 & 9.91 & 0.25 \\ 
  Vales sublitorais; Agrosistema extensivo; Mesotemperado inferior & 2.50 & 0.95 & 2.63 \\ 
  Vales sublitorais; Matogueira e rochedo; Mesotemperado inferior & 2.50 & 4.77 & 0.52 \\ 
  Vales sublitorais; Agrosistema intensivo (plantacion forestal); Termotemperado & 1.67 & 3.44 & 0.48 \\ 
  Vales sublitorais; Agrosistema intensivo (superficie de cultivo); Mesotemperado inferior & 1.67 & 4.84 & 0.34 \\ 
  Vales sublitorais; Matogueira e rochedo; Mesotemperado superior & 1.67 & 2.22 & 0.75 \\ 
   \hline
\end{tabular}
\end{table}
% latex table generated in R 3.2.5 by xtable 1.8-0 package
% Mon May  9 13:23:07 2016
\begin{table}[p]
\centering
\caption{Frecuencia de aparición de valores estéticos identificados na participación pública e frecuencia de tipos asociados Rías Baixas} 
\label{vsixotest12}
\begin{tabular}{lrrr}
  \hline
Tipo de paisaxe & F.Aparic (\%) & F.Tipo (\%) & Ratio \\ 
  \hline
Serras; Matogueira e rochedo; Mesotemperado superior & 18.26 & 6.70 & 2.73 \\ 
  Vales sublitorais; Agrosistema intensivo (plantacion forestal); Termotemperado & 18.26 & 16.41 & 1.11 \\ 
  Serras; Matogueira e rochedo; Mesotemperado inferior & 17.39 & 3.29 & 5.28 \\ 
  Vales sublitorais; Matogueira e rochedo; Mesotemperado inferior & 10.43 & 4.71 & 2.22 \\ 
  Serras; Agrosistema intensivo (plantacion forestal); Mesotemperado inferior & 5.22 & 2.82 & 1.85 \\ 
  Litoral Cantabro-Atlantico; Agrosistema intensivo (plantacion forestal); Termotemperado & 4.35 & 10.20 & 0.43 \\ 
  Litoral Cantabro-Atlantico; Matogueira e rochedo; Termotemperado & 3.48 & 1.73 & 2.01 \\ 
  Litoral Cantabro-Atlantico; Agrosistema intensivo (mosaico agroforestal); Termotemperado & 2.61 & 10.49 & 0.25 \\ 
  Litoral Cantabro-Atlantico; Rururbano (diseminado); Termotemperado & 2.61 & 7.10 & 0.37 \\ 
  Litoral Cantabro-Atlantico; Urbano; Termotemperado & 2.61 & 3.29 & 0.79 \\ 
  Vales sublitorais; Agrosistema intensivo (mosaico agroforestal); Termotemperado & 2.61 & 7.63 & 0.34 \\ 
  category 0; Praias e cantis; category 0 & 1.74 & 0.56 & 3.08 \\ 
  Vales sublitorais; Agrosistema intensivo (mosaico agroforestal); Mesotemperado inferior & 1.74 & 4.05 & 0.43 \\ 
   \hline
\end{tabular}
\end{table}


\clearpage
% latex table generated in R 3.2.2 by xtable 1.8-0 package
% Wed Dec  2 19:31:35 2015
\begin{table}[p]
\centering
\caption{Frecuencia de aparición dos Camiños de Santiago (área de influencia de 500 m a ambos lados) e frecuencia de tipos asociados Golfo Ártabro} 
\label{vcamino1}
\begin{tabular}{lrr}
  \hline
V4 & porcent & porcentT \\ 
  \hline
Litoral Cantabro-Atlantico; Rururbano (diseminado); Termotemperado & 33.30 & 3.29 \\ 
  Litoral Cantabro-Atlantico; Urbano; Termotemperado & 15.84 & 0.53 \\ 
  Litoral Cantabro-Atlantico; Agrosistema intensivo (mosaico agroforestal); Termotemperado & 13.67 & 2.43 \\ 
  Vales sublitorais; Agrosistema intensivo (mosaico agroforestal); Termotemperado & 8.34 & 2.62 \\ 
  Vales sublitorais; Rururbano (diseminado); Termotemperado & 6.23 & 1.68 \\ 
  Vales sublitorais; Agrosistema intensivo (mosaico agroforestal); Mesotemperado inferior & 4.70 & 7.54 \\ 
  Litoral Cantabro-Atlantico; Agrosistema intensivo (plantacion forestal); Termotemperado & 4.18 & 1.32 \\ 
  Vales sublitorais; Agrosistema intensivo (plantacion forestal); Termotemperado & 2.24 & 1.79 \\ 
  Vales sublitorais; Agrosistema intensivo (plantacion forestal); Mesotemperado inferior & 1.84 & 2.90 \\ 
  Vales sublitorais; Matogueira e rochedo; Mesotemperado superior & 1.56 & 0.85 \\ 
  Litoral Cantabro-Atlantico; Urbano; no data & 1.48 & 0.04 \\ 
   \hline
\end{tabular}
\end{table}
% latex table generated in R 3.2.2 by xtable 1.8-0 package
% Wed Dec  2 19:32:09 2015
\begin{table}[p]
\centering
\caption{Frecuencia de aparición dos Camiños de Santiago (área de influencia de 500 m a ambos lados) e frecuencia de tipos asociados Golfo Ártabro} 
\label{vcamino1}
\begin{tabular}{lrrr}
  \hline
Tipo de paisaxe & F.Aparic (\%) & F.Tipo (\%) & Ratio \\ 
  \hline
Litoral Cantabro-Atlantico; Rururbano (diseminado); Termotemperado & 33.30 & 3.29 & 10.12 \\ 
  Litoral Cantabro-Atlantico; Urbano; Termotemperado & 15.84 & 0.53 & 29.86 \\ 
  Litoral Cantabro-Atlantico; Agrosistema intensivo (mosaico agroforestal); Termotemperado & 13.67 & 2.43 & 5.63 \\ 
  Vales sublitorais; Agrosistema intensivo (mosaico agroforestal); Termotemperado & 8.34 & 2.62 & 3.19 \\ 
  Vales sublitorais; Rururbano (diseminado); Termotemperado & 6.23 & 1.68 & 3.71 \\ 
  Vales sublitorais; Agrosistema intensivo (mosaico agroforestal); Mesotemperado inferior & 4.70 & 7.54 & 0.62 \\ 
  Litoral Cantabro-Atlantico; Agrosistema intensivo (plantacion forestal); Termotemperado & 4.18 & 1.32 & 3.18 \\ 
  Vales sublitorais; Agrosistema intensivo (plantacion forestal); Termotemperado & 2.24 & 1.79 & 1.25 \\ 
  Vales sublitorais; Agrosistema intensivo (plantacion forestal); Mesotemperado inferior & 1.84 & 2.90 & 0.64 \\ 
  Vales sublitorais; Matogueira e rochedo; Mesotemperado superior & 1.56 & 0.85 & 1.84 \\ 
  Litoral Cantabro-Atlantico; Urbano; no data & 1.48 & 0.04 & 33.02 \\ 
   \hline
\end{tabular}
\end{table}


\clearpage
% latex table generated in R 3.2.5 by xtable 1.8-0 package
% Mon May  9 13:23:06 2016
\begin{table}[p]
\centering
\caption{Frecuencia de aparición de Lugares de Importancia Comunitaria e frecuencia de tipos asociados Golfo Ártabro} 
\label{vnatura1}
\begin{tabular}{lrrr}
  \hline
Tipo de paisaxe & F.Aparic (\%) & F.Tipo (\%) & Ratio \\ 
  \hline
Canons; Bosque; Mesotemperado inferior & 16.05 & 1.56 & 10.28 \\ 
  Serras; Matogueira e rochedo; Mesotemperado superior & 10.97 & 3.06 & 3.59 \\ 
  Serras; Turbeira; Mesotemperado superior & 9.03 & 1.53 & 5.90 \\ 
  Vales sublitorais; Agrosistema intensivo (mosaico agroforestal); Mesotemperado inferior & 6.56 & 21.75 & 0.30 \\ 
  Serras; Bosque; Mesotemperado superior & 5.92 & 0.71 & 8.36 \\ 
  category 0; Lamina de auga; category 0 & 5.59 & 0.55 & 10.19 \\ 
  Vales sublitorais; Agrosistema intensivo (plantacion forestal); Mesotemperado inferior & 4.44 & 13.21 & 0.34 \\ 
  Serras; Agrosistema intensivo (mosaico agroforestal); Mesotemperado superior & 3.65 & 2.44 & 1.50 \\ 
  Vales sublitorais; Agrosistema intensivo (mosaico agroforestal); Mesotemperado superior & 3.09 & 1.47 & 2.10 \\ 
  Vales sublitorais; Bosque; Mesotemperado inferior & 2.66 & 0.78 & 3.40 \\ 
  Litoral Cantabro-Atlantico; Agrosistema intensivo (mosaico agroforestal); Termotemperado & 2.46 & 18.64 & 0.13 \\ 
  Serras; Agrosistema intensivo (plantacion forestal); Mesotemperado superior & 2.45 & 0.88 & 2.78 \\ 
  Canons; Agrosistema intensivo (mosaico agroforestal); Mesotemperado inferior & 2.45 & 0.25 & 9.83 \\ 
  Litoral Cantabro-Atlantico; Agrosistema intensivo (plantacion forestal); Termotemperado & 2.42 & 5.79 & 0.42 \\ 
  Vales sublitorais; Turbeira; Mesotemperado superior & 2.23 & 0.50 & 4.50 \\ 
  Vales sublitorais; Agrosistema extensivo; Mesotemperado inferior & 2.08 & 0.47 & 4.41 \\ 
  Serras; Agrosistema extensivo; Mesotemperado superior & 2.04 & 0.80 & 2.56 \\ 
  Serras; Agrosistema intensivo (mosaico agroforestal); Mesotemperado inferior & 1.39 & 0.16 & 8.52 \\ 
  Vales sublitorais; Matogueira e rochedo; Mesotemperado superior & 1.35 & 1.17 & 1.15 \\ 
  Litoral Cantabro-Atlantico; Bosque; Termotemperado & 1.17 & 0.54 & 2.19 \\ 
  Canons; Agrosistema intensivo (plantacion forestal); Mesotemperado inferior & 1.13 & 0.11 & 10.23 \\ 
  Vales sublitorais; Agrosistema extensivo; Mesotemperado superior & 1.03 & 0.68 & 1.51 \\ 
   \hline
\end{tabular}
\end{table}
% latex table generated in R 3.2.5 by xtable 1.8-0 package
% Mon May  9 13:23:06 2016
\begin{table}[p]
\centering
\caption{Frecuencia de aparición de Lugares de Importancia Comunitaria e frecuencia de tipos asociados A Mariña - Baixo Eo} 
\label{vnatura2}
\begin{tabular}{lrrr}
  \hline
Tipo de paisaxe & F.Aparic (\%) & F.Tipo (\%) & Ratio \\ 
  \hline
Serras; Turbeira; Mesotemperado superior & 32.83 & 2.64 & 12.44 \\ 
  Serras; Agrosistema intensivo (plantacion forestal); Mesotemperado superior & 14.33 & 6.80 & 2.11 \\ 
  Vales sublitorais; Agrosistema intensivo (plantacion forestal); Mesotemperado inferior & 13.58 & 23.68 & 0.57 \\ 
  Litoral Cantabro-Atlantico; Rururbano (diseminado); Termotemperado & 6.78 & 2.68 & 2.53 \\ 
  Serras; Agrosistema intensivo (plantacion forestal); Mesotemperado inferior & 5.98 & 1.66 & 3.60 \\ 
  Litoral Cantabro-Atlantico; Agrosistema intensivo (mosaico agroforestal); Termotemperado & 5.95 & 7.20 & 0.83 \\ 
   & 3.42 &  &  \\ 
  Litoral Cantabro-Atlantico; Agrosistema intensivo (plantacion forestal); Mesotemperado inferior & 2.54 & 18.89 & 0.13 \\ 
  category 0; Praias e cantis; category 0 & 2.11 & 0.25 & 8.60 \\ 
  Litoral Cantabro-Atlantico; Agrosistema intensivo (superficie de cultivo); Termotemperado & 1.96 & 2.07 & 0.95 \\ 
  Vales sublitorais; Agrosistema intensivo (mosaico agroforestal); Mesotemperado inferior & 1.63 & 12.86 & 0.13 \\ 
  category 0; Lamina de auga; category 0 & 1.62 & 0.13 & 12.91 \\ 
  Litoral Cantabro-Atlantico; Agrosistema intensivo (plantacion forestal); Termotemperado & 1.37 & 0.51 & 2.68 \\ 
  Serras; Agrosistema intensivo (mosaico agroforestal); Mesotemperado inferior & 1.32 & 0.26 & 5.02 \\ 
  Serras; Matogueira e rochedo; Mesotemperado superior & 1.10 & 1.71 & 0.64 \\ 
   \hline
\end{tabular}
\end{table}
% latex table generated in R 3.2.5 by xtable 1.8-0 package
% Mon May  9 13:23:06 2016
\begin{table}[p]
\centering
\caption{Frecuencia de aparición de Lugares de Importancia Comunitaria e frecuencia de tipos asociados Costa Sur - Baixo Miño} 
\label{vnatura3}
\begin{tabular}{lrrr}
  \hline
Tipo de paisaxe & F.Aparic (\%) & F.Tipo (\%) & Ratio \\ 
  \hline
Litoral Cantabro-Atlantico; Agrosistema intensivo (plantacion forestal); Termotemperado & 11.79 & 6.69 & 1.76 \\ 
  Serras; Agrosistema intensivo (plantacion forestal); Mesotemperado inferior & 10.79 & 3.38 & 3.20 \\ 
  Litoral Cantabro-Atlantico; Agrosistema intensivo (mosaico agroforestal); Termotemperado & 10.27 & 5.83 & 1.76 \\ 
  Litoral Cantabro-Atlantico; Bosque; Termotemperado & 8.88 & 0.40 & 22.41 \\ 
  Litoral Cantabro-Atlantico; Agrosistema extensivo; Termotemperado & 8.77 & 0.61 & 14.29 \\ 
  category 0; Lamina de auga; category 0 & 8.02 & 0.46 & 17.53 \\ 
  Chairas e vales interiores; Matogueira e rochedo; Mesotemperado inferior & 4.77 & 1.96 & 2.43 \\ 
  Litoral Cantabro-Atlantico; Rururbano (diseminado); Termotemperado & 4.54 & 4.97 & 0.91 \\ 
  Vales sublitorais; Agrosistema intensivo (plantacion forestal); Termotemperado & 4.50 & 16.82 & 0.27 \\ 
  Serras; Matogueira e rochedo; Supra e orotemperado & 4.12 & 0.14 & 29.70 \\ 
  Chairas e vales interiores; Agrosistema intensivo (plantacion forestal); Termotemperado & 2.84 & 6.71 & 0.42 \\ 
  Vales sublitorais; Agrosistema intensivo (mosaico agroforestal); Termotemperado & 2.11 & 7.55 & 0.28 \\ 
  Vales sublitorais; Agrosistema extensivo; Termotemperado & 1.74 & 2.59 & 0.67 \\ 
  Litoral Cantabro-Atlantico; Urbano; Termotemperado & 1.64 & 1.10 & 1.49 \\ 
  Litoral Cantabro-Atlantico; Matogueira e rochedo; Termotemperado & 1.48 & 0.52 & 2.85 \\ 
  Chairas e vales interiores; Agrosistema intensivo (plantacion forestal); Mesotemperado inferior & 1.44 & 0.44 & 3.28 \\ 
  Litoral Cantabro-Atlantico; Vinedo; Termotemperado & 1.35 & 0.70 & 1.91 \\ 
  Vales sublitorais; Bosque; Termotemperado & 1.33 & 1.44 & 0.92 \\ 
  Chairas e vales interiores; Vinedo; Termotemperado & 1.17 & 0.95 & 1.22 \\ 
   & 1.04 &  &  \\ 
  Serras; Matogueira e rochedo; Termotemperado & 1.02 & 0.40 & 2.51 \\ 
   \hline
\end{tabular}
\end{table}
% latex table generated in R 3.2.5 by xtable 1.8-0 package
% Mon May  9 13:23:06 2016
\begin{table}[p]
\centering
\caption{Frecuencia de aparición de Lugares de Importancia Comunitaria e frecuencia de tipos asociados Ribeiras Encaixadas do Miño e do Sil} 
\label{vnatura4}
\begin{tabular}{lrrr}
  \hline
Tipo de paisaxe & F.Aparic (\%) & F.Tipo (\%) & Ratio \\ 
  \hline
Serras; Matogueira e rochedo; Supra e orotemperado & 17.79 & 9.08 & 1.96 \\ 
  Canons; Bosque; Mesotemperado inferior & 11.11 & 2.67 & 4.17 \\ 
  Canons; Agrosistema intensivo (plantacion forestal); Termotemperado & 7.40 & 1.39 & 5.34 \\ 
  Serras; Bosque; Mesotemperado inferior & 7.20 & 2.32 & 3.11 \\ 
  Chairas e vales interiores; Matogueira e rochedo; category 0 & 5.13 & 2.25 & 2.28 \\ 
  Canons; Bosque; Termotemperado & 4.84 & 1.02 & 4.74 \\ 
  Canons; Matogueira e rochedo; Mesotemperado inferior & 3.87 & 0.67 & 5.74 \\ 
  Chairas e vales interiores; Agrosistema intensivo (plantacion forestal); Mesotemperado inferior & 3.71 & 4.37 & 0.85 \\ 
  category 0; Lamina de auga; category 0 & 3.40 & 1.89 & 1.79 \\ 
  Serras; Matogueira e rochedo; Mesotemperado superior & 3.29 & 3.10 & 1.06 \\ 
  Canons; Agrosistema intensivo (plantacion forestal); category 0 & 2.70 & 1.25 & 2.17 \\ 
  Serras; Agrosistema extensivo; Mesotemperado inferior & 2.70 & 1.98 & 1.37 \\ 
  Chairas e vales interiores; Bosque; category 0 & 2.69 & 0.80 & 3.35 \\ 
  Canons; Matogueira e rochedo; category 0 & 2.52 & 0.96 & 2.63 \\ 
  Serras; Matogueira e rochedo; Mesotemperado inferior & 2.04 & 2.79 & 0.73 \\ 
  Chairas e vales interiores; Agrosistema extensivo; Mesotemperado inferior & 2.03 & 6.56 & 0.31 \\ 
  Canons; Matogueira e rochedo; Termotemperado & 2.02 & 0.41 & 4.96 \\ 
  Serras; Matogueira e rochedo; category 0 & 1.67 & 0.48 & 3.48 \\ 
  Chairas e vales interiores; Matogueira e rochedo; Mesotemperado inferior & 1.58 & 3.61 & 0.44 \\ 
  Serras; Bosque; Supra e orotemperado & 1.28 & 0.47 & 2.69 \\ 
  Chairas e vales interiores; Agrosistema intensivo (plantacion forestal); category 0 & 1.26 & 1.83 & 0.69 \\ 
  Serras; Agrosistema intensivo (plantacion forestal); Mesotemperado inferior & 1.19 & 2.65 & 0.45 \\ 
  Chairas e vales interiores; Agrosistema extensivo; category 0 & 1.16 & 0.60 & 1.94 \\ 
   \hline
\end{tabular}
\end{table}
% latex table generated in R 3.2.5 by xtable 1.8-0 package
% Mon May  9 13:23:06 2016
\begin{table}[p]
\centering
\caption{Frecuencia de aparición de Lugares de Importancia Comunitaria e frecuencia de tipos asociados Serras Orientais} 
\label{vnatura5}
\begin{tabular}{lrrr}
  \hline
Tipo de paisaxe & F.Aparic (\%) & F.Tipo (\%) & Ratio \\ 
  \hline
Serras; Matogueira e rochedo; Supra e orotemperado & 25.31 & 15.61 & 1.62 \\ 
  Serras; Bosque; Supra e orotemperado & 13.82 & 7.67 & 1.80 \\ 
  Serras; Agrosistema extensivo; Supra e orotemperado & 12.16 & 8.91 & 1.36 \\ 
  Serras; Agrosistema intensivo (plantacion forestal); Supra e orotemperado & 7.76 & 6.86 & 1.13 \\ 
  Serras; Matogueira e rochedo; Mesotemperado superior & 5.29 & 5.41 & 0.98 \\ 
  Serras; Agrosistema intensivo (mosaico agroforestal); Supra e orotemperado & 5.02 & 5.74 & 0.88 \\ 
  Vales sublitorais; Agrosistema extensivo; Mesotemperado superior & 3.68 & 5.11 & 0.72 \\ 
  Vales sublitorais; Bosque; Mesotemperado inferior & 3.01 & 3.83 & 0.79 \\ 
  Vales sublitorais; Bosque; Mesotemperado superior & 2.95 & 4.16 & 0.71 \\ 
  Serras; Bosque; Mesotemperado superior & 2.51 & 3.65 & 0.69 \\ 
  Serras; Agrosistema intensivo (plantacion forestal); Mesotemperado superior & 2.27 & 3.86 & 0.59 \\ 
  Serras; Agrosistema extensivo; Mesotemperado superior & 2.01 & 7.30 & 0.27 \\ 
  Vales sublitorais; Matogueira e rochedo; Mesotemperado inferior & 1.92 & 1.20 & 1.60 \\ 
  Vales sublitorais; Matogueira e rochedo; Mesotemperado superior & 1.69 & 2.45 & 0.69 \\ 
  Serras; Agrosistema intensivo (mosaico agroforestal); Mesotemperado superior & 1.15 & 3.95 & 0.29 \\ 
  Chairas e vales interiores; Agrosistema intensivo (plantacion forestal); Mesotemperado inferior & 1.14 & 1.09 & 1.05 \\ 
  Vales sublitorais; Agrosistema extensivo; Mesotemperado inferior & 1.12 & 1.41 & 0.79 \\ 
   \hline
\end{tabular}
\end{table}
% latex table generated in R 3.2.5 by xtable 1.8-0 package
% Mon May  9 13:23:06 2016
\begin{table}[p]
\centering
\caption{Frecuencia de aparición de Lugares de Importancia Comunitaria e frecuencia de tipos asociados Chairas e Fosas Luguesas} 
\label{vnatura6}
\begin{tabular}{lrrr}
  \hline
Tipo de paisaxe & F.Aparic (\%) & F.Tipo (\%) & Ratio \\ 
  \hline
Serras; Turbeira; Mesotemperado superior & 16.66 & 1.08 & 15.40 \\ 
  Chairas e vales interiores; Agrosistema extensivo; Mesotemperado superior & 14.67 & 9.68 & 1.51 \\ 
  Chairas e vales interiores; Agrosistema intensivo (mosaico agroforestal); Mesotemperado superior & 7.51 & 20.11 & 0.37 \\ 
  Serras; Turbeira; Supra e orotemperado & 7.04 & 0.60 & 11.79 \\ 
  Serras; Matogueira e rochedo; Supra e orotemperado & 6.72 & 0.48 & 14.02 \\ 
  Chairas e vales interiores; Bosque; Mesotemperado superior & 6.45 & 2.02 & 3.19 \\ 
  Serras; Agrosistema extensivo; Mesotemperado superior & 5.68 & 5.94 & 0.96 \\ 
  Chairas e vales interiores; Agrosistema extensivo; Mesotemperado inferior & 4.62 & 6.80 & 0.68 \\ 
  Serras; Agrosistema intensivo (plantacion forestal); Supra e orotemperado & 3.46 & 0.59 & 5.88 \\ 
  Chairas e vales interiores; Bosque; Mesotemperado inferior & 3.46 & 1.92 & 1.80 \\ 
  Chairas e vales interiores; Agrosistema intensivo (superficie de cultivo); Mesotemperado superior & 3.27 & 5.78 & 0.57 \\ 
  Serras; Bosque; Mesotemperado superior & 2.70 & 0.71 & 3.81 \\ 
  Serras; Agrosistema intensivo (plantacion forestal); Mesotemperado superior & 2.54 & 2.29 & 1.11 \\ 
  Serras; Bosque; Supra e orotemperado & 2.47 & 0.23 & 10.90 \\ 
  Serras; Agrosistema intensivo (mosaico agroforestal); Mesotemperado superior & 2.16 & 9.79 & 0.22 \\ 
  Serras; Agrosistema intensivo (mosaico agroforestal); Supra e orotemperado & 1.67 & 0.34 & 4.94 \\ 
  Serras; Matogueira e rochedo; Mesotemperado superior & 1.60 & 1.94 & 0.83 \\ 
  category 0; Lamina de auga; category 0 & 1.08 & 0.09 & 11.34 \\ 
  Chairas e vales interiores; Agrosistema intensivo (mosaico agroforestal); Mesotemperado inferior & 1.07 & 9.44 & 0.11 \\ 
   \hline
\end{tabular}
\end{table}
% latex table generated in R 3.2.5 by xtable 1.8-0 package
% Mon May  9 13:23:06 2016
\begin{table}[p]
\centering
\caption{Frecuencia de aparición de Lugares de Importancia Comunitaria e frecuencia de tipos asociados Galicia Central} 
\label{vnatura7}
\begin{tabular}{lrrr}
  \hline
Tipo de paisaxe & F.Aparic (\%) & F.Tipo (\%) & Ratio \\ 
  \hline
Serras; Matogueira e rochedo; Mesotemperado superior & 41.48 & 6.90 & 6.01 \\ 
  Vales sublitorais; Agrosistema intensivo (mosaico agroforestal); Mesotemperado inferior & 11.63 & 43.06 & 0.27 \\ 
  Vales sublitorais; Matogueira e rochedo; Mesotemperado inferior & 7.36 & 1.65 & 4.47 \\ 
  Serras; Agrosistema intensivo (plantacion forestal); Supra e orotemperado & 7.05 & 0.54 & 13.16 \\ 
  Serras; Agrosistema extensivo; Mesotemperado superior & 5.52 & 4.47 & 1.23 \\ 
  Vales sublitorais; Agrosistema extensivo; Mesotemperado inferior & 3.69 & 5.21 & 0.71 \\ 
  Serras; Agrosistema intensivo (plantacion forestal); Mesotemperado superior & 2.76 & 1.82 & 1.52 \\ 
  Vales sublitorais; Agrosistema intensivo (plantacion forestal); Mesotemperado inferior & 2.39 & 5.84 & 0.41 \\ 
  Serras; Matogueira e rochedo; Supra e orotemperado & 2.13 & 0.53 & 4.01 \\ 
  Serras; Agrosistema intensivo (mosaico agroforestal); Mesotemperado superior & 1.90 & 3.93 & 0.48 \\ 
  Vales sublitorais; Bosque; Termotemperado & 1.51 & 0.32 & 4.67 \\ 
  Vales sublitorais; Agrosistema intensivo (mosaico agroforestal); Mesotemperado superior & 1.40 & 0.57 & 2.44 \\ 
  Serras; Agrosistema intensivo (superficie de cultivo); Mesotemperado superior & 1.25 & 1.44 & 0.86 \\ 
   \hline
\end{tabular}
\end{table}
% latex table generated in R 3.2.5 by xtable 1.8-0 package
% Mon May  9 13:23:06 2016
\begin{table}[p]
\centering
\caption{Frecuencia de aparición de Lugares de Importancia Comunitaria e frecuencia de tipos asociados Chairas, Fosas e Serras Ourensás} 
\label{vnatura8}
\begin{tabular}{lrrr}
  \hline
Tipo de paisaxe & F.Aparic (\%) & F.Tipo (\%) & Ratio \\ 
  \hline
Serras; Matogueira e rochedo; Supra e orotemperado & 53.15 & 14.06 & 3.78 \\ 
  Chairas e vales interiores; Agrosistema intensivo (superficie de cultivo); Mesotemperado inferior & 12.75 & 7.53 & 1.69 \\ 
  Serras; Matogueira e rochedo; Mesotemperado inferior & 5.24 & 5.15 & 1.02 \\ 
  Serras; Bosque; Supra e orotemperado & 3.55 & 1.18 & 3.01 \\ 
  Serras; Matogueira e rochedo; Mesotemperado superior & 3.20 & 10.51 & 0.30 \\ 
  Chairas e vales interiores; Agrosistema extensivo; Mesotemperado inferior & 2.45 & 7.38 & 0.33 \\ 
  Chairas e vales interiores; Matogueira e rochedo; Termotemperado & 1.90 & 1.80 & 1.06 \\ 
  Chairas e vales interiores; Matogueira e rochedo; Mesotemperado inferior & 1.72 & 4.82 & 0.36 \\ 
  Serras; Agrosistema intensivo (plantacion forestal); Mesotemperado inferior & 1.71 & 5.68 & 0.30 \\ 
  Serras; Agrosistema intensivo (plantacion forestal); Supra e orotemperado & 1.63 & 1.55 & 1.05 \\ 
  category 0; Lamina de auga; category 0 & 1.62 & 0.71 & 2.30 \\ 
  Serras; Bosque; Mesotemperado inferior & 1.37 & 4.67 & 0.29 \\ 
  Serras; Bosque; Mesotemperado superior & 1.05 & 2.22 & 0.47 \\ 
  Chairas e vales interiores; Bosque; Mesotemperado inferior & 1.03 & 6.18 & 0.17 \\ 
   \hline
\end{tabular}
\end{table}
% latex table generated in R 3.2.5 by xtable 1.8-0 package
% Mon May  9 13:23:07 2016
\begin{table}[p]
\centering
\caption{Frecuencia de aparición de Lugares de Importancia Comunitaria e frecuencia de tipos asociados Serras Surorientais} 
\label{vnatura9}
\begin{tabular}{lrrr}
  \hline
Tipo de paisaxe & F.Aparic (\%) & F.Tipo (\%) & Ratio \\ 
  \hline
Serras; Matogueira e rochedo; Supra e orotemperado & 69.77 & 47.28 & 1.48 \\ 
  Serras; Agrosistema intensivo (plantacion forestal); Supra e orotemperado & 6.09 & 10.23 & 0.60 \\ 
  Serras; Bosque; Mesotemperado superior & 5.21 & 5.61 & 0.93 \\ 
  Serras; Agrosistema extensivo; Supra e orotemperado & 3.20 & 2.94 & 1.09 \\ 
  Canons; Matogueira e rochedo; Mesotemperado inferior & 2.76 & 1.24 & 2.23 \\ 
  Serras; Bosque; Supra e orotemperado & 2.53 & 1.26 & 2.02 \\ 
  Serras; Agrosistema extensivo; Mesotemperado superior & 2.49 & 6.24 & 0.40 \\ 
  Serras; Matogueira e rochedo; Mesotemperado superior & 1.80 & 3.34 & 0.54 \\ 
  Serras; Agrosistema extensivo; Mesotemperado inferior & 1.15 & 4.79 & 0.24 \\ 
  Serras; Bosque; Mesotemperado inferior & 1.02 & 3.06 & 0.33 \\ 
   \hline
\end{tabular}
\end{table}
% latex table generated in R 3.2.5 by xtable 1.8-0 package
% Mon May  9 13:23:07 2016
\begin{table}[p]
\centering
\caption{Frecuencia de aparición de Lugares de Importancia Comunitaria e frecuencia de tipos asociados Galicia Setentrional} 
\label{vnatura10}
\begin{tabular}{lrrr}
  \hline
Tipo de paisaxe & F.Aparic (\%) & F.Tipo (\%) & Ratio \\ 
  \hline
Serras; Turbeira; Mesotemperado superior & 45.97 & 12.44 & 3.70 \\ 
  Serras; Agrosistema intensivo (plantacion forestal); Mesotemperado superior & 8.79 & 4.68 & 1.88 \\ 
  Litoral Cantabro-Atlantico; Matogueira e rochedo; Termotemperado & 7.51 & 1.77 & 4.23 \\ 
  category 0; Praias e cantis; category 0 & 4.91 & 0.94 & 5.20 \\ 
  Litoral Cantabro-Atlantico; Agrosistema intensivo (plantacion forestal); Termotemperado & 3.89 & 7.61 & 0.51 \\ 
  Serras; Matogueira e rochedo; Mesotemperado superior & 3.36 & 6.79 & 0.50 \\ 
  Vales sublitorais; Agrosistema intensivo (plantacion forestal); Mesotemperado inferior & 3.02 & 17.31 & 0.17 \\ 
  Litoral Cantabro-Atlantico; Agrosistema intensivo (mosaico agroforestal); Termotemperado & 2.31 & 5.53 & 0.42 \\ 
  Serras; Agrosistema extensivo; Mesotemperado superior & 2.18 & 2.70 & 0.80 \\ 
  Serras; Bosque; Mesotemperado superior & 1.92 & 1.63 & 1.18 \\ 
  Serras; Agrosistema intensivo (mosaico agroforestal); Mesotemperado superior & 1.84 & 3.70 & 0.50 \\ 
  Litoral Cantabro-Atlantico; Agrosistema intensivo (plantacion forestal); Mesotemperado inferior & 1.67 & 2.79 & 0.60 \\ 
   & 1.54 &  &  \\ 
  Litoral Cantabro-Atlantico; Matogueira e rochedo; Mesotemperado inferior & 1.33 & 0.28 & 4.71 \\ 
  Serras; Matogueira e rochedo; Supra e orotemperado & 1.20 & 0.32 & 3.78 \\ 
  Litoral Cantabro-Atlantico; Rururbano (diseminado); Termotemperado & 1.15 & 1.15 & 1.01 \\ 
  category 0; Lamina de auga; category 0 & 1.09 & 0.89 & 1.23 \\ 
   \hline
\end{tabular}
\end{table}
% latex table generated in R 3.2.5 by xtable 1.8-0 package
% Mon May  9 13:23:07 2016
\begin{table}[p]
\centering
\caption{Frecuencia de aparición de Lugares de Importancia Comunitaria e frecuencia de tipos asociados Chairas e Fosas Occidentais} 
\label{vnatura11}
\begin{tabular}{lrrr}
  \hline
Tipo de paisaxe & F.Aparic (\%) & F.Tipo (\%) & Ratio \\ 
  \hline
Litoral Cantabro-Atlantico; Matogueira e rochedo; Termotemperado & 39.24 & 4.58 & 8.56 \\ 
  Vales sublitorais; Matogueira e rochedo; Mesotemperado inferior & 14.31 & 4.77 & 3.00 \\ 
  Litoral Cantabro-Atlantico; Agrosistema intensivo (plantacion forestal); Termotemperado & 9.50 & 7.60 & 1.25 \\ 
  category 0; Praias e cantis; category 0 & 7.92 & 0.54 & 14.73 \\ 
  Litoral Cantabro-Atlantico; Agrosistema intensivo (mosaico agroforestal); Termotemperado & 6.73 & 9.91 & 0.68 \\ 
  Vales sublitorais; Agrosistema intensivo (plantacion forestal); Mesotemperado inferior & 6.50 & 15.20 & 0.43 \\ 
   & 4.40 &  &  \\ 
  Litoral Cantabro-Atlantico; Agrosistema intensivo (superficie de cultivo); Termotemperado & 2.15 & 0.90 & 2.40 \\ 
  Vales sublitorais; Agrosistema intensivo (mosaico agroforestal); Mesotemperado inferior & 1.95 & 36.05 & 0.05 \\ 
  category 0; Lamina de auga; category 0 & 1.58 & 0.46 & 3.42 \\ 
  Vales sublitorais; Agrosistema intensivo (plantacion forestal); Termotemperado & 1.26 & 3.44 & 0.37 \\ 
  Vales sublitorais; Matogueira e rochedo; Termotemperado & 1.05 & 0.80 & 1.31 \\ 
  Litoral Cantabro-Atlantico; Agrosistema extensivo; Termotemperado & 1.04 & 0.13 & 8.36 \\ 
  Litoral Cantabro-Atlantico; Rururbano (diseminado); Termotemperado & 1.01 & 0.53 & 1.92 \\ 
   \hline
\end{tabular}
\end{table}
% latex table generated in R 3.2.5 by xtable 1.8-0 package
% Mon May  9 13:23:07 2016
\begin{table}[p]
\centering
\caption{Frecuencia de aparición de Lugares de Importancia Comunitaria e frecuencia de tipos asociados Rías Baixas} 
\label{vnatura12}
\begin{tabular}{lrrr}
  \hline
Tipo de paisaxe & F.Aparic (\%) & F.Tipo (\%) & Ratio \\ 
  \hline
Serras; Matogueira e rochedo; Mesotemperado superior & 34.05 & 6.70 & 5.08 \\ 
   & 13.92 &  &  \\ 
  Litoral Cantabro-Atlantico; Matogueira e rochedo; Termotemperado & 7.69 & 1.73 & 4.44 \\ 
  Litoral Cantabro-Atlantico; Agrosistema intensivo (plantacion forestal); Termotemperado & 6.85 & 10.20 & 0.67 \\ 
  category 0; Praias e cantis; category 0 & 6.40 & 0.56 & 11.34 \\ 
  Litoral Cantabro-Atlantico; Agrosistema intensivo (mosaico agroforestal); Termotemperado & 3.87 & 10.49 & 0.37 \\ 
  Vales sublitorais; Bosque; Mesotemperado inferior & 3.67 & 0.74 & 4.93 \\ 
  Litoral Cantabro-Atlantico; Rururbano (diseminado); Termotemperado & 2.41 & 7.10 & 0.34 \\ 
  Litoral Cantabro-Atlantico; Agrosistema extensivo; Termotemperado & 2.13 & 0.60 & 3.54 \\ 
  Serras; Agrosistema extensivo; Mesotemperado inferior & 2.07 & 0.57 & 3.61 \\ 
  category 0; Lamina de auga; category 0 & 2.01 & 0.19 & 10.51 \\ 
  Litoral Cantabro-Atlantico; Vinedo; Termotemperado & 1.79 & 1.45 & 1.23 \\ 
  Vales sublitorais; Agrosistema extensivo; Mesotemperado inferior & 1.76 & 0.71 & 2.50 \\ 
  Litoral Cantabro-Atlantico; Agrosistema intensivo (superficie de cultivo); Termotemperado & 1.76 & 1.03 & 1.70 \\ 
  Vales sublitorais; Matogueira e rochedo; Mesotemperado inferior & 1.60 & 4.71 & 0.34 \\ 
  Serras; Bosque; Mesotemperado superior & 1.34 & 0.29 & 4.56 \\ 
  Serras; Bosque; Mesotemperado inferior & 1.17 & 0.47 & 2.47 \\ 
   \hline
\end{tabular}
\end{table}


\clearpage
% latex table generated in R 3.2.2 by xtable 1.8-0 package
% Wed Dec  2 19:45:31 2015
\begin{table}[p]
\centering
\caption{Frecuencia de aparición de xeneradores eólicos e frecuencia de tipos asociados Golfo Ártabro} 
\label{veolico1}
\begin{tabular}{lrrr}
  \hline
Tipo de paisaxe & F.Aparic (\%) & F.Tipo (\%) & Ratio \\ 
  \hline
Serras; Turbeira; Mesotemperado superior & 35.06 & 0.71 & 49.50 \\ 
  Serras; Matogueira e rochedo; Mesotemperado superior & 24.68 & 5.03 & 4.91 \\ 
  Serras; Turbeira; Mesotemperado inferior & 7.79 & 0.05 & 146.24 \\ 
  Serras; Agrosistema extensivo; Mesotemperado superior & 5.19 & 5.76 & 0.90 \\ 
  Vales sublitorais; Agrosistema intensivo (mosaico agroforestal); Mesotemperado inferior & 5.19 & 7.54 & 0.69 \\ 
  Serras; Agrosistema extensivo; Supra e orotemperado & 3.90 & 2.50 & 1.56 \\ 
  Serras; Agrosistema intensivo (mosaico agroforestal); Supra e orotemperado & 3.90 & 0.37 & 10.48 \\ 
  Vales sublitorais; Turbeira; Mesotemperado superior & 3.90 & 0.06 & 64.63 \\ 
  Serras; Agrosistema extensivo; Mesotemperado inferior & 2.60 & 2.57 & 1.01 \\ 
  Serras; Turbeira; Supra e orotemperado & 2.60 & 0.40 & 6.53 \\ 
  Serras; Agrosistema intensivo (mosaico agroforestal); Mesotemperado inferior & 1.30 & 0.50 & 2.61 \\ 
  Serras; Agrosistema intensivo (plantacion forestal); Mesotemperado superior & 1.30 & 1.07 & 1.21 \\ 
  Serras; Agrosistema intensivo (plantacion forestal); Supra e orotemperado & 1.30 & 0.93 & 1.39 \\ 
  Serras; Matogueira e rochedo; Supra e orotemperado & 1.30 & 6.12 & 0.21 \\ 
   \hline
\end{tabular}
\end{table}
% latex table generated in R 3.2.2 by xtable 1.8-0 package
% Wed Dec  2 19:45:31 2015
\begin{table}[p]
\centering
\caption{Frecuencia de aparición de xeneradores eólicos e frecuencia de tipos asociados A Mariña - Baixo Eo} 
\label{veolico2}
\begin{tabular}{lrrr}
  \hline
Tipo de paisaxe & F.Aparic (\%) & F.Tipo (\%) & Ratio \\ 
  \hline
Serras; Turbeira; Mesotemperado superior & 58.42 & 0.71 & 82.46 \\ 
  Serras; Agrosistema intensivo (plantacion forestal); Mesotemperado superior & 21.78 & 1.07 & 20.30 \\ 
  Serras; Matogueira e rochedo; Mesotemperado superior & 8.91 & 5.03 & 1.77 \\ 
  Serras; Agrosistema intensivo (plantacion forestal); Mesotemperado inferior & 5.94 & 0.82 & 7.21 \\ 
  Serras; Turbeira; Mesotemperado inferior & 3.96 & 0.05 & 74.33 \\ 
   \hline
\end{tabular}
\end{table}
% latex table generated in R 3.2.2 by xtable 1.8-0 package
% Wed Dec  2 19:45:31 2015
\begin{table}[p]
\centering
\caption{Frecuencia de aparición de xeneradores eólicos e frecuencia de tipos asociados Costa Sur - Baixo Miño} 
\label{veolico3}
\begin{tabular}{lrrr}
  \hline
Tipo de paisaxe & F.Aparic (\%) & F.Tipo (\%) & Ratio \\ 
  \hline
Serras; Matogueira e rochedo; Supra e orotemperado & 74.07 & 6.12 & 12.11 \\ 
  Serras; Matogueira e rochedo; Mesotemperado superior & 17.28 & 5.03 & 3.44 \\ 
  Serras; Agrosistema intensivo (plantacion forestal); Supra e orotemperado & 6.79 & 0.93 & 7.28 \\ 
  Serras; Matogueira e rochedo; Mesotemperado inferior & 1.85 & 2.73 & 0.68 \\ 
   \hline
\end{tabular}
\end{table}
% latex table generated in R 3.2.2 by xtable 1.8-0 package
% Wed Dec  2 19:45:31 2015
\begin{table}[p]
\centering
\caption{Frecuencia de aparición de xeneradores eólicos e frecuencia de tipos asociados Ribeiras Encaixadas do Miño e do Sil} 
\label{veolico4}
\begin{tabular}{lrrr}
  \hline
Tipo de paisaxe & F.Aparic (\%) & F.Tipo (\%) & Ratio \\ 
  \hline
Serras; Matogueira e rochedo; Supra e orotemperado & 48.44 & 6.12 & 7.92 \\ 
  Serras; Matogueira e rochedo; Mesotemperado inferior & 15.62 & 2.73 & 5.73 \\ 
  Serras; Bosque; Supra e orotemperado & 12.50 & 0.74 & 16.97 \\ 
  Serras; Matogueira e rochedo; Mesotemperado superior & 9.38 & 5.03 & 1.86 \\ 
  Serras; Agrosistema intensivo (plantacion forestal); Mesotemperado superior & 6.25 & 1.07 & 5.82 \\ 
  Serras; Agrosistema intensivo (plantacion forestal); Supra e orotemperado & 4.69 & 0.93 & 5.03 \\ 
  Serras; Agrosistema intensivo (plantacion forestal); Mesotemperado inferior & 1.56 & 0.82 & 1.90 \\ 
  Serras; Bosque; Mesotemperado inferior & 1.56 & 0.69 & 2.26 \\ 
   \hline
\end{tabular}
\end{table}
% latex table generated in R 3.2.2 by xtable 1.8-0 package
% Wed Dec  2 19:45:31 2015
\begin{table}[p]
\centering
\caption{Frecuencia de aparición de xeneradores eólicos e frecuencia de tipos asociados Serras Orientais} 
\label{veolico5}
\begin{tabular}{lrrr}
  \hline
Tipo de paisaxe & F.Aparic (\%) & F.Tipo (\%) & Ratio \\ 
  \hline
Serras; Matogueira e rochedo; Supra e orotemperado & 45.89 & 6.12 & 7.50 \\ 
  Serras; Agrosistema extensivo; Supra e orotemperado & 25.34 & 2.50 & 10.15 \\ 
  Serras; Agrosistema intensivo (plantacion forestal); Supra e orotemperado & 9.59 & 0.93 & 10.29 \\ 
  Serras; Matogueira e rochedo; Mesotemperado superior & 8.90 & 5.03 & 1.77 \\ 
  Serras; Agrosistema intensivo (mosaico agroforestal); Supra e orotemperado & 6.16 & 0.37 & 16.59 \\ 
  Serras; Agrosistema extensivo; Mesotemperado superior & 4.11 & 5.76 & 0.71 \\ 
   \hline
\end{tabular}
\end{table}
% latex table generated in R 3.2.2 by xtable 1.8-0 package
% Wed Dec  2 19:45:31 2015
\begin{table}[p]
\centering
\caption{Frecuencia de aparición de xeneradores eólicos e frecuencia de tipos asociados Chairas e Fosas Luguesas} 
\label{veolico6}
\begin{tabular}{lrrr}
  \hline
Tipo de paisaxe & F.Aparic (\%) & F.Tipo (\%) & Ratio \\ 
  \hline
Serras; Turbeira; Supra e orotemperado & 29.97 & 0.40 & 75.31 \\ 
  Serras; Matogueira e rochedo; Supra e orotemperado & 18.96 & 6.12 & 3.10 \\ 
  Serras; Agrosistema intensivo (plantacion forestal); Supra e orotemperado & 18.04 & 0.93 & 19.35 \\ 
  Serras; Turbeira; Mesotemperado superior & 11.31 & 0.71 & 15.97 \\ 
  Serras; Matogueira e rochedo; Mesotemperado superior & 4.89 & 5.03 & 0.97 \\ 
  Chairas e vales interiores; Agrosistema intensivo (plantacion forestal); Mesotemperado superior & 3.98 & 0.42 & 9.42 \\ 
  Serras; Agrosistema intensivo (mosaico agroforestal); Mesotemperado superior & 3.06 & 1.77 & 1.72 \\ 
  Serras; Agrosistema extensivo; Mesotemperado superior & 2.75 & 5.76 & 0.48 \\ 
  Serras; Agrosistema extensivo; Supra e orotemperado & 2.75 & 2.50 & 1.10 \\ 
  Serras; Agrosistema intensivo (plantacion forestal); Mesotemperado superior & 1.53 & 1.07 & 1.42 \\ 
   \hline
\end{tabular}
\end{table}
% latex table generated in R 3.2.2 by xtable 1.8-0 package
% Wed Dec  2 19:45:31 2015
\begin{table}[p]
\centering
\caption{Frecuencia de aparición de xeneradores eólicos e frecuencia de tipos asociados Galicia Central} 
\label{veolico7}
\begin{tabular}{lrrr}
  \hline
Tipo de paisaxe & F.Aparic (\%) & F.Tipo (\%) & Ratio \\ 
  \hline
Serras; Matogueira e rochedo; Supra e orotemperado & 43.48 & 6.12 & 7.11 \\ 
  Serras; Matogueira e rochedo; Mesotemperado superior & 22.25 & 5.03 & 4.42 \\ 
  Serras; Matogueira e rochedo; Mesotemperado inferior & 12.53 & 2.73 & 4.60 \\ 
  Serras; Agrosistema extensivo; Supra e orotemperado & 7.16 & 2.50 & 2.87 \\ 
  Vales sublitorais; Matogueira e rochedo; Mesotemperado superior & 3.84 & 0.85 & 4.50 \\ 
  Serras; Agrosistema intensivo (plantacion forestal); Mesotemperado superior & 3.58 & 1.07 & 3.34 \\ 
  Serras; Agrosistema extensivo; Mesotemperado superior & 2.30 & 5.76 & 0.40 \\ 
  Serras; Agrosistema intensivo (mosaico agroforestal); Supra e orotemperado & 1.02 & 0.37 & 2.75 \\ 
   \hline
\end{tabular}
\end{table}
% latex table generated in R 3.2.2 by xtable 1.8-0 package
% Wed Dec  2 19:45:32 2015
\begin{table}[p]
\centering
\caption{Frecuencia de aparición de xeneradores eólicos e frecuencia de tipos asociados Chairas, Fosas e Serras Ourensás} 
\label{veolico8}
\begin{tabular}{lrrr}
  \hline
Tipo de paisaxe & F.Aparic (\%) & F.Tipo (\%) & Ratio \\ 
  \hline
Serras; Matogueira e rochedo; Supra e orotemperado & 63.81 & 6.12 & 10.43 \\ 
  Serras; Matogueira e rochedo; Mesotemperado superior & 27.62 & 5.03 & 5.49 \\ 
  Serras; Agrosistema intensivo (plantacion forestal); Supra e orotemperado & 4.76 & 0.93 & 5.11 \\ 
  Serras; Agrosistema intensivo (plantacion forestal); Mesotemperado superior & 1.90 & 1.07 & 1.77 \\ 
  Serras; Bosque; Supra e orotemperado & 1.90 & 0.74 & 2.59 \\ 
   \hline
\end{tabular}
\end{table}
% latex table generated in R 3.2.2 by xtable 1.8-0 package
% Wed Dec  2 19:45:32 2015
\begin{table}[p]
\centering
\caption{Frecuencia de aparición de xeneradores eólicos e frecuencia de tipos asociados Serras Surorientais} 
\label{veolico9}
\begin{tabular}{lrrr}
  \hline
Tipo de paisaxe & F.Aparic (\%) & F.Tipo (\%) & Ratio \\ 
  \hline
Serras; Matogueira e rochedo; Supra e orotemperado & 75.51 & 6.12 & 12.34 \\ 
  Serras; Agrosistema intensivo (superficie de cultivo); Supra e orotemperado & 10.20 & 0.23 & 44.12 \\ 
  Serras; Agrosistema extensivo; Supra e orotemperado & 6.12 & 2.50 & 2.45 \\ 
  Serras; Agrosistema intensivo (mosaico agroforestal); Supra e orotemperado & 6.12 & 0.37 & 16.47 \\ 
  Serras; Agrosistema intensivo (plantacion forestal); no data & 2.04 & 0.00 & 427.59 \\ 
   \hline
\end{tabular}
\end{table}
% latex table generated in R 3.2.2 by xtable 1.8-0 package
% Wed Dec  2 19:45:32 2015
\begin{table}[p]
\centering
\caption{Frecuencia de aparición de xeneradores eólicos e frecuencia de tipos asociados Galicia Setentrional} 
\label{veolico10}
\begin{tabular}{lrrr}
  \hline
Tipo de paisaxe & F.Aparic (\%) & F.Tipo (\%) & Ratio \\ 
  \hline
Serras; Matogueira e rochedo; Mesotemperado superior & 28.10 & 5.03 & 5.59 \\ 
  Serras; Turbeira; Mesotemperado superior & 24.54 & 0.71 & 34.64 \\ 
  Serras; Turbeira; Supra e orotemperado & 21.11 & 0.40 & 53.04 \\ 
  Serras; Agrosistema intensivo (plantacion forestal); Mesotemperado superior & 4.68 & 1.07 & 4.36 \\ 
  Serras; Matogueira e rochedo; Mesotemperado inferior & 3.03 & 2.73 & 1.11 \\ 
  Serras; Agrosistema intensivo (mosaico agroforestal); Mesotemperado superior & 2.77 & 1.77 & 1.56 \\ 
  Vales sublitorais; Matogueira e rochedo; Mesotemperado inferior & 2.64 & 1.89 & 1.39 \\ 
  Serras; Agrosistema intensivo (plantacion forestal); Mesotemperado inferior & 1.78 & 0.82 & 2.16 \\ 
  Serras; Turbeira; Mesotemperado inferior & 1.72 & 0.05 & 32.19 \\ 
  Vales sublitorais; Agrosistema intensivo (plantacion forestal); Mesotemperado inferior & 1.65 & 2.90 & 0.57 \\ 
  Serras; Agrosistema extensivo; Mesotemperado superior & 1.39 & 5.76 & 0.24 \\ 
   \hline
\end{tabular}
\end{table}
% latex table generated in R 3.2.2 by xtable 1.8-0 package
% Wed Dec  2 19:45:32 2015
\begin{table}[p]
\centering
\caption{Frecuencia de aparición de xeneradores eólicos e frecuencia de tipos asociados Chairas e Fosas Occidentais} 
\label{veolico11}
\begin{tabular}{lrrr}
  \hline
Tipo de paisaxe & F.Aparic (\%) & F.Tipo (\%) & Ratio \\ 
  \hline
Litoral Cantabro-Atlantico; Matogueira e rochedo; Termotemperado & 24.22 & 0.83 & 29.14 \\ 
  Vales sublitorais; Matogueira e rochedo; Mesotemperado superior & 17.49 & 0.85 & 20.54 \\ 
  Vales sublitorais; Matogueira e rochedo; Mesotemperado inferior & 17.34 & 1.89 & 9.16 \\ 
  Vales sublitorais; Agrosistema intensivo (plantacion forestal); Mesotemperado superior & 15.99 & 0.60 & 26.77 \\ 
  Vales sublitorais; Agrosistema intensivo (plantacion forestal); Mesotemperado inferior & 11.66 & 2.90 & 4.02 \\ 
  Vales sublitorais; Turbeira; Mesotemperado superior & 4.93 & 0.06 & 81.83 \\ 
  Vales sublitorais; Matogueira e rochedo; Termotemperado & 4.04 & 0.93 & 4.33 \\ 
  Vales sublitorais; Agrosistema intensivo (mosaico agroforestal); Mesotemperado superior & 1.79 & 1.52 & 1.18 \\ 
   \hline
\end{tabular}
\end{table}
% latex table generated in R 3.2.2 by xtable 1.8-0 package
% Wed Dec  2 19:45:32 2015
\begin{table}[p]
\centering
\caption{Frecuencia de aparición de xeneradores eólicos e frecuencia de tipos asociados Rías Baixas} 
\label{veolico12}
\begin{tabular}{lrrr}
  \hline
Tipo de paisaxe & F.Aparic (\%) & F.Tipo (\%) & Ratio \\ 
  \hline
Serras; Matogueira e rochedo; Supra e orotemperado & 30.05 & 6.12 & 4.91 \\ 
  Vales sublitorais; Matogueira e rochedo; Mesotemperado superior & 19.95 & 0.85 & 23.42 \\ 
  Serras; Matogueira e rochedo; Mesotemperado superior & 18.13 & 5.03 & 3.61 \\ 
  Serras; Matogueira e rochedo; Mesotemperado inferior & 8.29 & 2.73 & 3.04 \\ 
  Vales sublitorais; Matogueira e rochedo; Mesotemperado inferior & 6.99 & 1.89 & 3.69 \\ 
  Vales sublitorais; Agrosistema intensivo (plantacion forestal); Mesotemperado superior & 5.18 & 0.60 & 8.67 \\ 
  Vales sublitorais; Turbeira; Mesotemperado superior & 3.11 & 0.06 & 51.57 \\ 
  Vales sublitorais; Agrosistema intensivo (plantacion forestal); Mesotemperado inferior & 2.33 & 2.90 & 0.80 \\ 
  Serras; Agrosistema intensivo (plantacion forestal); Mesotemperado inferior & 1.30 & 0.82 & 1.57 \\ 
  Vales sublitorais; Agrosistema intensivo (mosaico agroforestal); Mesotemperado superior & 1.30 & 1.52 & 0.85 \\ 
  Vales sublitorais; Agrosistema extensivo; Mesotemperado superior & 1.04 & 1.21 & 0.86 \\ 
   \hline
\end{tabular}
\end{table}

 
 \end{footnotesize}
\end{landscape}
\end{document}
