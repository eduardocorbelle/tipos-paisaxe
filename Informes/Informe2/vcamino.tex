% latex table generated in R 3.2.2 by xtable 1.8-0 package
% Tue Nov 24 18:51:08 2015
\begin{table}[p]
\centering
\caption{Frecuencia de aparición dos Camiños de Santiago (área de influencia de 500 m a ambos lados) e frecuencia de tipos asociados} 
\label{vcamino}
\begin{tabular}{lrr}
  \hline
Tipo de paisaxe & Frec. aparición (\%) & Frecuencia do tipo (\%) \\ 
  \hline
Vales sublitorais; Agrosistema intensivo (mosaico agroforestal); Mesotemperado inferior & 11.29 & 7.55 \\ 
  Litoral Cantabro-Atlantico; Rururbano (diseminado); Termotemperado & 6.22 & 3.29 \\ 
  Serras; Agrosistema extensivo; Mesotemperado superior & 5.12 & 5.77 \\ 
  Serras; Agrosistema extensivo; Supra e orotemperado & 4.00 & 2.50 \\ 
  Chairas e vales interiores; Agrosistema extensivo; Mesotemperado inferior & 3.82 & 3.59 \\ 
  Chairas e vales interiores; Agrosistema extensivo; Mesotemperado superior & 3.80 & 2.78 \\ 
  Vales sublitorais; Agrosistema extensivo; Mesotemperado inferior & 3.79 & 2.74 \\ 
  Vales sublitorais; Rururbano (diseminado); Mesotemperado inferior & 3.68 & 0.98 \\ 
  Vales sublitorais; Agrosistema intensivo (mosaico agroforestal); Termotemperado & 3.47 & 2.62 \\ 
  Vales sublitorais; Rururbano (diseminado); Termotemperado & 3.26 & 1.68 \\ 
  Serras; Matogueira e rochedo; Supra e orotemperado & 3.22 & 6.13 \\ 
  Litoral Cantabro-Atlantico; Agrosistema intensivo (mosaico agroforestal); Termotemperado & 3.21 & 2.43 \\ 
  Serras; Agrosistema intensivo (mosaico agroforestal); Mesotemperado superior & 2.23 & 1.78 \\ 
  Chairas e vales interiores; Agrosistema intensivo (mosaico agroforestal); Mesotemperado superior & 2.11 & 2.27 \\ 
  Vales sublitorais; Agrosistema intensivo (plantacion forestal); Mesotemperado inferior & 2.11 & 2.90 \\ 
  Litoral Cantabro-Atlantico; Urbano; Termotemperado & 2.03 & 0.53 \\ 
  Vales sublitorais; Agrosistema intensivo (superficie de cultivo); Mesotemperado inferior & 1.90 & 0.81 \\ 
  Serras; Agrosistema extensivo; Mesotemperado inferior & 1.87 & 2.57 \\ 
  Chairas e vales interiores; Agrosistema intensivo (superficie de cultivo); Mesotemperado inferior & 1.69 & 1.16 \\ 
  Serras; Agrosistema intensivo (superficie de cultivo); Mesotemperado superior & 1.62 & 0.95 \\ 
  Vales sublitorais; Agrosistema intensivo (mosaico agroforestal); Mesotemperado superior & 1.59 & 1.52 \\ 
  Chairas e vales interiores; Agrosistema intensivo (mosaico agroforestal); Mesotemperado inferior & 1.55 & 1.71 \\ 
  Chairas e vales interiores; Rururbano (diseminado); Mesotemperado inferior & 1.51 & 0.51 \\ 
  Serras; Matogueira e rochedo; Mesotemperado superior & 1.44 & 5.04 \\ 
  Vales sublitorais; Agrosistema extensivo; Mesotemperado superior & 1.34 & 1.21 \\ 
  Serras; Matogueira e rochedo; Mesotemperado inferior & 1.20 & 2.73 \\ 
   & 1.13 &  \\ 
  Vales sublitorais; Matogueira e rochedo; Mesotemperado inferior & 0.99 & 1.90 \\ 
  Litoral Cantabro-Atlantico; Agrosistema intensivo (plantacion forestal); Termotemperado & 0.95 & 1.32 \\ 
  Vales sublitorais; Urbano; Mesotemperado inferior & 0.88 & 0.11 \\ 
  Vales sublitorais; Agrosistema intensivo (plantacion forestal); Termotemperado & 0.87 & 1.79 \\ 
  Serras; Agrosistema intensivo (mosaico agroforestal); Supra e orotemperado & 0.82 & 0.37 \\ 
  Chairas e vales interiores; Urbano; Termotemperado & 0.70 & 0.08 \\ 
  Chairas e vales interiores; Rururbano (diseminado); Mesotemperado superior & 0.69 & 0.30 \\ 
  Serras; Agrosistema intensivo (plantacion forestal); Supra e orotemperado & 0.69 & 0.93 \\ 
  Chairas e vales interiores; Rururbano (diseminado); Termotemperado & 0.68 & 0.42 \\ 
  Chairas e vales interiores; Matogueira e rochedo; Mesotemperado inferior & 0.60 & 1.38 \\ 
  Litoral Cantabro-Atlantico; Matogueira e rochedo; Termotemperado & 0.59 & 0.83 \\ 
  Serras; Agrosistema intensivo (mosaico agroforestal); Mesotemperado inferior & 0.55 & 0.50 \\ 
  Vales sublitorais; Matogueira e rochedo; Termotemperado & 0.54 & 0.93 \\ 
   \hline
\end{tabular}
\end{table}
