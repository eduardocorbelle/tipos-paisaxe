% latex table generated in R 3.2.2 by xtable 1.8-0 package
% Wed Dec  2 19:45:31 2015
\begin{table}[p]
\centering
\caption{Frecuencia de aparición de valores estéticos identificados na participación pública e frecuencia de tipos asociados Golfo Ártabro} 
\label{vsixotest1}
\begin{tabular}{lrrr}
  \hline
Tipo de paisaxe & F.Aparic (\%) & F.Tipo (\%) & Ratio \\ 
  \hline
Litoral Cantabro-Atlantico; Rururbano (diseminado); Termotemperado & 11.86 & 3.29 & 3.61 \\ 
  Litoral Cantabro-Atlantico; Conxunto Historico; no data & 10.17 & 0.00 & 3855.37 \\ 
  Canons; Bosque; Mesotemperado inferior & 6.78 & 0.30 & 22.95 \\ 
  Litoral Cantabro-Atlantico; Agrosistema intensivo (plantacion forestal); Termotemperado & 6.78 & 1.32 & 5.15 \\ 
  Vales sublitorais; Agrosistema intensivo (mosaico agroforestal); Mesotemperado inferior & 6.78 & 7.54 & 0.90 \\ 
  Litoral Cantabro-Atlantico; Matogueira e rochedo; Termotemperado & 5.08 & 0.83 & 6.12 \\ 
  Vales sublitorais; Agrosistema intensivo (mosaico agroforestal); Termotemperado & 5.08 & 2.62 & 1.94 \\ 
  Canons; Matogueira e rochedo; Mesotemperado inferior & 3.39 & 0.20 & 16.92 \\ 
  Litoral Cantabro-Atlantico; Agrosistema extensivo; Termotemperado & 3.39 & 0.16 & 21.51 \\ 
  Litoral Cantabro-Atlantico; Agrosistema intensivo (mosaico agroforestal); Termotemperado & 3.39 & 2.43 & 1.40 \\ 
  Litoral Cantabro-Atlantico; Conxunto Historico; Termotemperado & 3.39 & 0.02 & 179.24 \\ 
  Litoral Cantabro-Atlantico; Rururbano (diseminado); Mesotemperado inferior & 3.39 & 0.04 & 90.02 \\ 
  Litoral Cantabro-Atlantico; Urbano; Termotemperado & 3.39 & 0.53 & 6.39 \\ 
  Vales sublitorais; Agrosistema intensivo (plantacion forestal); Mesotemperado inferior & 3.39 & 2.90 & 1.17 \\ 
  Vales sublitorais; Bosque; Mesotemperado inferior & 3.39 & 0.49 & 6.85 \\ 
  Vales sublitorais; Bosque; Termotemperado & 3.39 & 0.14 & 23.90 \\ 
  Vales sublitorais; Matogueira e rochedo; Mesotemperado inferior & 3.39 & 1.89 & 1.79 \\ 
  Canons; Bosque; Termotemperado & 1.69 & 0.15 & 11.20 \\ 
  Serras; Agrosistema intensivo (mosaico agroforestal); Mesotemperado superior & 1.69 & 1.77 & 0.96 \\ 
  Serras; Turbeira; Mesotemperado superior & 1.69 & 0.71 & 2.39 \\ 
  Vales sublitorais; Agrosistema extensivo; Mesotemperado inferior & 1.69 & 2.73 & 0.62 \\ 
  Vales sublitorais; Agrosistema intensivo (plantacion forestal); Termotemperado & 1.69 & 1.79 & 0.95 \\ 
  Vales sublitorais; Matogueira e rochedo; Mesotemperado superior & 1.69 & 0.85 & 1.99 \\ 
  Vales sublitorais; Rururbano (diseminado); Termotemperado & 1.69 & 1.68 & 1.01 \\ 
   & 1.69 &  &  \\ 
   \hline
\end{tabular}
\end{table}
% latex table generated in R 3.2.2 by xtable 1.8-0 package
% Wed Dec  2 19:45:31 2015
\begin{table}[p]
\centering
\caption{Frecuencia de aparición de valores estéticos identificados na participación pública e frecuencia de tipos asociados A Mariña - Baixo Eo} 
\label{vsixotest2}
\begin{tabular}{lrrr}
  \hline
Tipo de paisaxe & F.Aparic (\%) & F.Tipo (\%) & Ratio \\ 
  \hline
Vales sublitorais; Agrosistema intensivo (plantacion forestal); Mesotemperado inferior & 18.60 & 2.90 & 6.42 \\ 
  Litoral Cantabro-Atlantico; Rururbano (diseminado); Termotemperado & 9.30 & 3.29 & 2.83 \\ 
  Vales sublitorais; Rururbano (diseminado); Mesotemperado inferior & 9.30 & 0.98 & 9.47 \\ 
  Litoral Cantabro-Atlantico; Matogueira e rochedo; Mesotemperado inferior & 6.98 & 0.10 & 70.09 \\ 
  Litoral Cantabro-Atlantico; Urbano; Termotemperado & 6.98 & 0.53 & 13.15 \\ 
  Litoral Cantabro-Atlantico; Agrosistema intensivo (mosaico agroforestal); Mesotemperado inferior & 4.65 & 0.29 & 15.88 \\ 
  Litoral Cantabro-Atlantico; Agrosistema intensivo (mosaico agroforestal); no data & 4.65 & 0.04 & 109.39 \\ 
  Litoral Cantabro-Atlantico; Agrosistema intensivo (plantacion forestal); Mesotemperado inferior & 4.65 & 0.62 & 7.53 \\ 
  Litoral Cantabro-Atlantico; Matogueira e rochedo; Termotemperado & 4.65 & 0.83 & 5.60 \\ 
  Serras; Turbeira; Mesotemperado superior & 4.65 & 0.71 & 6.57 \\ 
  Vales sublitorais; Agrosistema intensivo (mosaico agroforestal); Mesotemperado inferior & 4.65 & 7.54 & 0.62 \\ 
   & 4.65 &  &  \\ 
  Litoral Cantabro-Atlantico; Agrosistema intensivo (mosaico agroforestal); Termotemperado & 2.33 & 2.43 & 0.96 \\ 
  Litoral Cantabro-Atlantico; Conxunto Historico; Termotemperado & 2.33 & 0.02 & 122.97 \\ 
  Litoral Cantabro-Atlantico; Rururbano (diseminado); Mesotemperado inferior & 2.33 & 0.04 & 61.76 \\ 
  Litoral Cantabro-Atlantico; Urbano; no data & 2.33 & 0.04 & 51.94 \\ 
  Serras; Bosque; Mesotemperado superior & 2.33 & 1.00 & 2.33 \\ 
  Serras; Matogueira e rochedo; Supra e orotemperado & 2.33 & 6.12 & 0.38 \\ 
  Vales sublitorais; Bosque; Mesotemperado superior & 2.33 & 0.39 & 6.03 \\ 
   \hline
\end{tabular}
\end{table}
% latex table generated in R 3.2.2 by xtable 1.8-0 package
% Wed Dec  2 19:45:31 2015
\begin{table}[p]
\centering
\caption{Frecuencia de aparición de valores estéticos identificados na participación pública e frecuencia de tipos asociados Costa Sur - Baixo Miño} 
\label{vsixotest3}
\begin{tabular}{lrrr}
  \hline
Tipo de paisaxe & F.Aparic (\%) & F.Tipo (\%) & Ratio \\ 
  \hline
Serras; Matogueira e rochedo; Mesotemperado inferior & 10.87 & 2.73 & 3.99 \\ 
  Litoral Cantabro-Atlantico; Rururbano (diseminado); Termotemperado & 9.78 & 3.29 & 2.97 \\ 
  Serras; Conxunto Historico; Termotemperado & 8.70 & 0.01 & 1086.98 \\ 
  Serras; Agrosistema intensivo (plantacion forestal); Mesotemperado inferior & 6.52 & 0.82 & 7.92 \\ 
  Serras; Agrosistema intensivo (plantacion forestal); Termotemperado & 6.52 & 0.15 & 42.77 \\ 
  Vales sublitorais; Agrosistema intensivo (plantacion forestal); Termotemperado & 6.52 & 1.79 & 3.64 \\ 
  Serras; Matogueira e rochedo; Mesotemperado superior & 5.43 & 5.03 & 1.08 \\ 
  Serras; Turbeira; Mesotemperado inferior & 5.43 & 0.05 & 102.00 \\ 
  Vales sublitorais; Rururbano (diseminado); Termotemperado & 5.43 & 1.68 & 3.23 \\ 
  Litoral Cantabro-Atlantico; Agrosistema intensivo (plantacion forestal); Termotemperado & 4.35 & 1.32 & 3.30 \\ 
  Serras; Matogueira e rochedo; Termotemperado & 4.35 & 0.16 & 27.66 \\ 
  Vales sublitorais; Matogueira e rochedo; Termotemperado & 4.35 & 0.93 & 4.66 \\ 
  Vales sublitorais; Agrosistema intensivo (mosaico agroforestal); Termotemperado & 3.26 & 2.62 & 1.25 \\ 
  Litoral Cantabro-Atlantico; Matogueira e rochedo; no data & 2.17 & 0.09 & 25.54 \\ 
  Vales sublitorais; Bosque; Termotemperado & 2.17 & 0.14 & 15.33 \\ 
  Chairas e vales interiores; Agrosistema intensivo (mosaico agroforestal); Mesotemperado inferior & 1.09 & 1.71 & 0.64 \\ 
  Chairas e vales interiores; Matogueira e rochedo; Mesotemperado inferior & 1.09 & 1.38 & 0.79 \\ 
  Chairas e vales interiores; Matogueira e rochedo; Termotemperado & 1.09 & 0.66 & 1.66 \\ 
  Litoral Cantabro-Atlantico; Agrosistema extensivo; Termotemperado & 1.09 & 0.16 & 6.90 \\ 
  Litoral Cantabro-Atlantico; Agrosistema intensivo (mosaico agroforestal); Termotemperado & 1.09 & 2.43 & 0.45 \\ 
  Litoral Cantabro-Atlantico; Agrosistema intensivo (superficie de cultivo); no data & 1.09 & 0.01 & 111.55 \\ 
  Litoral Cantabro-Atlantico; Viñedo; Termotemperado & 1.09 & 0.14 & 7.85 \\ 
  no data; Agrosistema intensivo (plantacion forestal); Termotemperado & 1.09 & 0.03 & 39.38 \\ 
  no data; Rururbano (diseminado); Termotemperado & 1.09 & 0.05 & 24.01 \\ 
  Serras; Agrosistema extensivo; Mesotemperado inferior & 1.09 & 2.57 & 0.42 \\ 
  Serras; Agrosistema extensivo; Mesotemperado superior & 1.09 & 5.76 & 0.19 \\ 
  Vales sublitorais; Agrosistema intensivo (plantacion forestal); Mesotemperado inferior & 1.09 & 2.90 & 0.37 \\ 
  Vales sublitorais; Matogueira e rochedo; Mesotemperado inferior & 1.09 & 1.89 & 0.57 \\ 
   \hline
\end{tabular}
\end{table}
% latex table generated in R 3.2.2 by xtable 1.8-0 package
% Wed Dec  2 19:45:31 2015
\begin{table}[p]
\centering
\caption{Frecuencia de aparición de valores estéticos identificados na participación pública e frecuencia de tipos asociados Ribeiras Encaixadas do Miño e do Sil} 
\label{vsixotest4}
\begin{tabular}{lrrr}
  \hline
Tipo de paisaxe & F.Aparic (\%) & F.Tipo (\%) & Ratio \\ 
  \hline
Chairas e vales interiores; Bosque; Termotemperado & 9.26 & 0.29 & 31.76 \\ 
  Canons; Bosque; Mesotemperado inferior & 7.41 & 0.30 & 25.07 \\ 
  Canons; Bosque; Termotemperado & 7.41 & 0.15 & 48.94 \\ 
  Canons; Matogueira e rochedo; Mesomediterráneo & 5.56 & 0.18 & 30.07 \\ 
  Canons; Viñedo; Mesotemperado inferior & 5.56 & 0.02 & 240.24 \\ 
  Chairas e vales interiores; Rururbano (diseminado); Mesotemperado inferior & 4.63 & 0.51 & 9.04 \\ 
  Serras; Agrosistema extensivo; Mesotemperado inferior & 4.63 & 2.57 & 1.80 \\ 
  Chairas e vales interiores; Conxunto Historico; Termotemperado & 3.70 & 0.00 & 1275.45 \\ 
  Chairas e vales interiores; Matogueira e rochedo; Mesomediterráneo & 3.70 & 0.27 & 13.82 \\ 
  Chairas e vales interiores; Rururbano (diseminado); Termotemperado & 3.70 & 0.42 & 8.75 \\ 
  Serras; Bosque; Mesotemperado inferior & 3.70 & 0.69 & 5.36 \\ 
  Serras; Matogueira e rochedo; Mesotemperado inferior & 3.70 & 2.73 & 1.36 \\ 
  Canons; Matogueira e rochedo; Mesotemperado inferior & 2.78 & 0.20 & 13.87 \\ 
  Chairas e vales interiores; Agrosistema intensivo (plantacion forestal); Termotemperado & 2.78 & 0.37 & 7.54 \\ 
  Chairas e vales interiores; Matogueira e rochedo; Mesotemperado inferior & 2.78 & 1.38 & 2.01 \\ 
  Serras; Agrosistema extensivo; Mesotemperado superior & 2.78 & 5.76 & 0.48 \\ 
  Canons; Agrosistema extensivo; Mesomediterráneo & 1.85 & 0.03 & 53.67 \\ 
  Canons; Agrosistema intensivo (plantacion forestal); Termotemperado & 1.85 & 0.07 & 25.81 \\ 
  Chairas e vales interiores; Bosque; Mesomediterráneo & 1.85 & 0.06 & 32.82 \\ 
  Chairas e vales interiores; Matogueira e rochedo; Termotemperado & 1.85 & 0.66 & 2.82 \\ 
  Chairas e vales interiores; Viñedo; Termotemperado & 1.85 & 0.28 & 6.67 \\ 
   \hline
\end{tabular}
\end{table}
% latex table generated in R 3.2.2 by xtable 1.8-0 package
% Wed Dec  2 19:45:31 2015
\begin{table}[p]
\centering
\caption{Frecuencia de aparición de valores estéticos identificados na participación pública e frecuencia de tipos asociados Serras Orientais} 
\label{vsixotest5}
\begin{tabular}{lrrr}
  \hline
Tipo de paisaxe & F.Aparic (\%) & F.Tipo (\%) & Ratio \\ 
  \hline
Serras; Agrosistema extensivo; Supra e orotemperado & 16.67 & 2.50 & 6.67 \\ 
  Serras; Agrosistema extensivo; Mesotemperado superior & 12.82 & 5.76 & 2.23 \\ 
  Serras; Bosque; Supra e orotemperado & 11.54 & 0.74 & 15.66 \\ 
  Serras; Matogueira e rochedo; Supra e orotemperado & 11.54 & 6.12 & 1.89 \\ 
  Serras; Bosque; Mesotemperado superior & 7.69 & 1.00 & 7.69 \\ 
  Vales sublitorais; Bosque; Mesotemperado inferior & 6.41 & 0.49 & 12.95 \\ 
  Serras; Matogueira e rochedo; Mesotemperado superior & 5.13 & 5.03 & 1.02 \\ 
  Vales sublitorais; Agrosistema extensivo; Mesotemperado inferior & 5.13 & 2.73 & 1.88 \\ 
  Vales sublitorais; Agrosistema extensivo; Mesotemperado superior & 3.85 & 1.21 & 3.19 \\ 
  Vales sublitorais; Bosque; Mesotemperado superior & 3.85 & 0.39 & 9.97 \\ 
  Canons; Matogueira e rochedo; Termotemperado & 2.56 & 0.09 & 29.74 \\ 
  Serras; Agrosistema extensivo; Mesotemperado inferior & 2.56 & 2.57 & 1.00 \\ 
  Serras; Agrosistema intensivo (mosaico agroforestal); Supra e orotemperado & 2.56 & 0.37 & 6.90 \\ 
  Canons; Agrosistema intensivo (plantacion forestal); Termotemperado & 1.28 & 0.07 & 17.87 \\ 
  Canons; Bosque; Mesotemperado inferior & 1.28 & 0.30 & 4.34 \\ 
  Chairas e vales interiores; Agrosistema extensivo; Mesotemperado inferior & 1.28 & 3.58 & 0.36 \\ 
  Chairas e vales interiores; Agrosistema extensivo; Mesotemperado superior & 1.28 & 2.77 & 0.46 \\ 
  Serras; Agrosistema intensivo (mosaico agroforestal); Mesotemperado superior & 1.28 & 1.77 & 0.72 \\ 
  Serras; Agrosistema intensivo (plantacion forestal); Supra e orotemperado & 1.28 & 0.93 & 1.38 \\ 
   \hline
\end{tabular}
\end{table}
% latex table generated in R 3.2.2 by xtable 1.8-0 package
% Wed Dec  2 19:45:31 2015
\begin{table}[p]
\centering
\caption{Frecuencia de aparición de valores estéticos identificados na participación pública e frecuencia de tipos asociados Chairas e Fosas Luguesas} 
\label{vsixotest6}
\begin{tabular}{lrrr}
  \hline
Tipo de paisaxe & F.Aparic (\%) & F.Tipo (\%) & Ratio \\ 
  \hline
Chairas e vales interiores; Agrosistema extensivo; Mesotemperado inferior & 15.87 & 3.58 & 4.43 \\ 
  Chairas e vales interiores; Agrosistema extensivo; Mesotemperado superior & 14.29 & 2.77 & 5.15 \\ 
  Chairas e vales interiores; Bosque; Mesotemperado inferior & 9.52 & 0.77 & 12.45 \\ 
  Serras; Agrosistema intensivo (plantacion forestal); Supra e orotemperado & 7.94 & 0.93 & 8.51 \\ 
  Chairas e vales interiores; Agrosistema intensivo (mosaico agroforestal); Mesotemperado superior & 6.35 & 2.27 & 2.80 \\ 
  Chairas e vales interiores; Urbano; Mesotemperado inferior & 6.35 & 0.06 & 113.91 \\ 
  Serras; Agrosistema extensivo; Mesotemperado superior & 6.35 & 5.76 & 1.10 \\ 
  Chairas e vales interiores; Conxunto Historico; Termotemperado & 4.76 & 0.00 & 1639.86 \\ 
  Chairas e vales interiores; Rururbano (diseminado); Mesotemperado superior & 4.76 & 0.30 & 15.89 \\ 
  Chairas e vales interiores; Agrosistema intensivo (mosaico agroforestal); Mesotemperado inferior & 3.17 & 1.71 & 1.86 \\ 
  Chairas e vales interiores; Agrosistema intensivo (superficie de cultivo); Mesotemperado superior & 3.17 & 0.72 & 4.44 \\ 
  Chairas e vales interiores; Bosque; Mesotemperado superior & 3.17 & 0.26 & 11.99 \\ 
  Chairas e vales interiores; Conxunto Historico; Mesotemperado superior & 3.17 & 0.00 & 2874.20 \\ 
  Serras; Turbeira; Mesotemperado superior & 3.17 & 0.71 & 4.48 \\ 
  Chairas e vales interiores; Agrosistema intensivo (plantacion forestal); Mesotemperado superior & 1.59 & 0.42 & 3.76 \\ 
  Chairas e vales interiores; Rururbano (diseminado); Mesotemperado inferior & 1.59 & 0.51 & 3.10 \\ 
  Chairas e vales interiores; Urbano; Mesotemperado superior & 1.59 & 0.03 & 46.82 \\ 
  Serras; Agrosistema extensivo; Supra e orotemperado & 1.59 & 2.50 & 0.64 \\ 
  Serras; Matogueira e rochedo; Supra e orotemperado & 1.59 & 6.12 & 0.26 \\ 
   \hline
\end{tabular}
\end{table}
% latex table generated in R 3.2.2 by xtable 1.8-0 package
% Wed Dec  2 19:45:32 2015
\begin{table}[p]
\centering
\caption{Frecuencia de aparición de valores estéticos identificados na participación pública e frecuencia de tipos asociados Galicia Central} 
\label{vsixotest7}
\begin{tabular}{lrrr}
  \hline
Tipo de paisaxe & F.Aparic (\%) & F.Tipo (\%) & Ratio \\ 
  \hline
Vales sublitorais; Agrosistema extensivo; Mesotemperado inferior & 15.03 & 2.73 & 5.50 \\ 
  Vales sublitorais; Agrosistema intensivo (mosaico agroforestal); Termotemperado & 8.81 & 2.62 & 3.36 \\ 
  Vales sublitorais; Rururbano (diseminado); Termotemperado & 7.77 & 1.68 & 4.62 \\ 
  Vales sublitorais; Agrosistema intensivo (mosaico agroforestal); Mesotemperado inferior & 6.74 & 7.54 & 0.89 \\ 
  Vales sublitorais; Urbano; Mesotemperado inferior & 6.74 & 0.11 & 62.02 \\ 
  Vales sublitorais; Rururbano (diseminado); Mesotemperado inferior & 5.70 & 0.98 & 5.80 \\ 
  Serras; Agrosistema extensivo; Mesotemperado superior & 5.18 & 5.76 & 0.90 \\ 
  Vales sublitorais; Agrosistema intensivo (superficie de cultivo); Mesotemperado inferior & 4.66 & 0.81 & 5.76 \\ 
  Serras; Matogueira e rochedo; Mesotemperado superior & 4.15 & 5.03 & 0.82 \\ 
  Serras; Agrosistema extensivo; Mesotemperado inferior & 3.11 & 2.57 & 1.21 \\ 
  Vales sublitorais; Conxunto Historico; Mesotemperado inferior & 3.11 & 0.00 & 905.88 \\ 
  Serras; Matogueira e rochedo; Mesotemperado inferior & 2.59 & 2.73 & 0.95 \\ 
  Vales sublitorais; Bosque; Mesotemperado inferior & 2.59 & 0.49 & 5.24 \\ 
  Vales sublitorais; Agrosistema extensivo; Termotemperado & 2.07 & 0.36 & 5.78 \\ 
  Vales sublitorais; Agrosistema intensivo (plantacion forestal); Termotemperado & 2.07 & 1.79 & 1.16 \\ 
  Vales sublitorais; Matogueira e rochedo; Mesotemperado inferior & 2.07 & 1.89 & 1.09 \\ 
  Vales sublitorais; Matogueira e rochedo; Termotemperado & 2.07 & 0.93 & 2.22 \\ 
  Serras; Agrosistema intensivo (superficie de cultivo); Mesotemperado superior & 1.55 & 0.95 & 1.63 \\ 
  Vales sublitorais; Agrosistema extensivo; Mesotemperado superior & 1.55 & 1.21 & 1.29 \\ 
  Chairas e vales interiores; Agrosistema extensivo; Termotemperado & 1.04 & 0.61 & 1.69 \\ 
  Chairas e vales interiores; Urbano; Termotemperado & 1.04 & 0.08 & 12.63 \\ 
  Serras; Agrosistema intensivo (plantacion forestal); Supra e orotemperado & 1.04 & 0.93 & 1.11 \\ 
  Serras; Rururbano (diseminado); Mesotemperado superior & 1.04 & 0.11 & 9.33 \\ 
  Vales sublitorais; Agrosistema intensivo (mosaico agroforestal); Mesotemperado superior & 1.04 & 1.52 & 0.68 \\ 
  Vales sublitorais; Agrosistema intensivo (plantacion forestal); Mesotemperado inferior & 1.04 & 2.90 & 0.36 \\ 
  Vales sublitorais; Matogueira e rochedo; Mesotemperado superior & 1.04 & 0.85 & 1.22 \\ 
   \hline
\end{tabular}
\end{table}
% latex table generated in R 3.2.2 by xtable 1.8-0 package
% Wed Dec  2 19:45:32 2015
\begin{table}[p]
\centering
\caption{Frecuencia de aparición de valores estéticos identificados na participación pública e frecuencia de tipos asociados Chairas, Fosas e Serras Ourensás} 
\label{vsixotest8}
\begin{tabular}{lrrr}
  \hline
Tipo de paisaxe & F.Aparic (\%) & F.Tipo (\%) & Ratio \\ 
  \hline
Serras; Matogueira e rochedo; Mesotemperado superior & 11.76 & 5.03 & 2.34 \\ 
  Serras; Matogueira e rochedo; Supra e orotemperado & 9.80 & 6.12 & 1.60 \\ 
  Chairas e vales interiores; Agrosistema extensivo; Mesotemperado inferior & 7.84 & 3.58 & 2.19 \\ 
  Chairas e vales interiores; Matogueira e rochedo; Mesotemperado inferior & 7.84 & 1.38 & 5.67 \\ 
  Serras; Matogueira e rochedo; Mesotemperado inferior & 7.84 & 2.73 & 2.88 \\ 
  Serras; Agrosistema extensivo; Mesotemperado inferior & 6.86 & 2.57 & 2.67 \\ 
  Chairas e vales interiores; Bosque; Mesotemperado inferior & 4.90 & 0.77 & 6.41 \\ 
  Chairas e vales interiores; Rururbano (diseminado); Mesotemperado inferior & 4.90 & 0.51 & 9.57 \\ 
  Serras; Agrosistema extensivo; Mesotemperado superior & 4.90 & 5.76 & 0.85 \\ 
  Serras; Agrosistema extensivo; Supra e orotemperado & 4.90 & 2.50 & 1.96 \\ 
  Chairas e vales interiores; Agrosistema intensivo (superficie de cultivo); Mesotemperado inferior & 2.94 & 1.16 & 2.54 \\ 
  Chairas e vales interiores; Matogueira e rochedo; Termotemperado & 2.94 & 0.66 & 4.48 \\ 
  Serras; Bosque; Mesotemperado inferior & 2.94 & 0.69 & 4.26 \\ 
  Serras; Bosque; Mesotemperado superior & 2.94 & 1.00 & 2.94 \\ 
  Chairas e vales interiores; Agrosistema extensivo; Termotemperado & 1.96 & 0.61 & 3.20 \\ 
  Chairas e vales interiores; Agrosistema intensivo (mosaico agroforestal); Mesotemperado inferior & 1.96 & 1.71 & 1.15 \\ 
  Chairas e vales interiores; Conxunto Historico; Mesotemperado inferior & 1.96 & 0.00 & 1421.83 \\ 
  Chairas e vales interiores; Viñedo; Termotemperado & 1.96 & 0.28 & 7.06 \\ 
  Serras; Bosque; Supra e orotemperado & 1.96 & 0.74 & 2.66 \\ 
   \hline
\end{tabular}
\end{table}
% latex table generated in R 3.2.2 by xtable 1.8-0 package
% Wed Dec  2 19:45:32 2015
\begin{table}[p]
\centering
\caption{Frecuencia de aparición de valores estéticos identificados na participación pública e frecuencia de tipos asociados Serras Surorientais} 
\label{vsixotest9}
\begin{tabular}{lrrr}
  \hline
Tipo de paisaxe & F.Aparic (\%) & F.Tipo (\%) & Ratio \\ 
  \hline
Serras; Matogueira e rochedo; Supra e orotemperado & 29.29 & 6.12 & 4.79 \\ 
  Serras; Agrosistema extensivo; Mesotemperado inferior & 13.13 & 2.57 & 5.11 \\ 
  Serras; Agrosistema extensivo; Mesotemperado superior & 10.10 & 5.76 & 1.75 \\ 
  Serras; Agrosistema extensivo; Supra e orotemperado & 7.07 & 2.50 & 2.83 \\ 
  Serras; Matogueira e rochedo; Mesotemperado superior & 6.06 & 5.03 & 1.21 \\ 
  Canons; Viñedo; Mesomediterráneo & 4.04 & 0.01 & 338.26 \\ 
  Serras; Bosque; Mesotemperado superior & 4.04 & 1.00 & 4.04 \\ 
  Serras; Agrosistema intensivo (plantacion forestal); Supra e orotemperado & 3.03 & 0.93 & 3.25 \\ 
  Serras; Matogueira e rochedo; Mesotemperado inferior & 3.03 & 2.73 & 1.11 \\ 
  Serras; Matogueira e rochedo; no data & 3.03 & 0.07 & 41.11 \\ 
  Canons; Matogueira e rochedo; Mesomediterráneo & 2.02 & 0.18 & 10.93 \\ 
  Canons; Matogueira e rochedo; Mesotemperado inferior & 2.02 & 0.20 & 10.08 \\ 
  Serras; Agrosistema extensivo; Mesomediterráneo & 2.02 & 0.06 & 36.39 \\ 
  Serras; Bosque; Mesotemperado inferior & 2.02 & 0.69 & 2.92 \\ 
  Canons; Agrosistema extensivo; Mesomediterráneo & 1.01 & 0.03 & 29.28 \\ 
  Canons; Bosque; Mesotemperado inferior & 1.01 & 0.30 & 3.42 \\ 
  Canons; Viñedo; Mesotemperado inferior & 1.01 & 0.02 & 43.68 \\ 
  Serras; Agrosistema intensivo (superficie de cultivo); Mesotemperado superior & 1.01 & 0.95 & 1.06 \\ 
  Serras; Bosque; Supra e orotemperado & 1.01 & 0.74 & 1.37 \\ 
  Serras; Rururbano (diseminado); Mesotemperado inferior & 1.01 & 0.08 & 11.93 \\ 
  Serras; Rururbano (diseminado); Mesotemperado superior & 1.01 & 0.11 & 9.09 \\ 
  Vales sublitorais; Agrosistema extensivo; Mesomediterráneo & 1.01 & 0.00 & 5374.09 \\ 
  Vales sublitorais; Viñedo; Mesomediterráneo & 1.01 & 0.00 & 8391.12 \\ 
   \hline
\end{tabular}
\end{table}
% latex table generated in R 3.2.2 by xtable 1.8-0 package
% Wed Dec  2 19:45:32 2015
\begin{table}[p]
\centering
\caption{Frecuencia de aparición de valores estéticos identificados na participación pública e frecuencia de tipos asociados Galicia Setentrional} 
\label{vsixotest10}
\begin{tabular}{lrrr}
  \hline
Tipo de paisaxe & F.Aparic (\%) & F.Tipo (\%) & Ratio \\ 
  \hline
Serras; Turbeira; Supra e orotemperado & 13.89 & 0.40 & 34.90 \\ 
  Vales sublitorais; Agrosistema intensivo (plantacion forestal); Mesotemperado inferior & 12.50 & 2.90 & 4.31 \\ 
  Vales sublitorais; Agrosistema intensivo (mosaico agroforestal); Mesotemperado inferior & 8.33 & 7.54 & 1.10 \\ 
  Serras; Matogueira e rochedo; Mesotemperado superior & 6.94 & 5.03 & 1.38 \\ 
  Serras; Turbeira; Mesotemperado superior & 6.94 & 0.71 & 9.80 \\ 
  Vales sublitorais; Bosque; Mesotemperado inferior & 5.56 & 0.49 & 11.23 \\ 
  Litoral Cantabro-Atlantico; Agrosistema intensivo (mosaico agroforestal); Mesotemperado inferior & 4.17 & 0.29 & 14.22 \\ 
  Litoral Cantabro-Atlantico; Agrosistema intensivo (mosaico agroforestal); Termotemperado & 4.17 & 2.43 & 1.72 \\ 
  Litoral Cantabro-Atlantico; Agrosistema intensivo (plantacion forestal); Termotemperado & 4.17 & 1.32 & 3.16 \\ 
  Litoral Cantabro-Atlantico; Agrosistema intensivo (plantacion forestal); no data & 2.78 & 0.03 & 88.97 \\ 
  Litoral Cantabro-Atlantico; Matogueira e rochedo; Mesotemperado inferior & 2.78 & 0.10 & 27.90 \\ 
  Litoral Cantabro-Atlantico; Rururbano (diseminado); Termotemperado & 2.78 & 3.29 & 0.84 \\ 
  Serras; Agrosistema extensivo; Mesotemperado superior & 2.78 & 5.76 & 0.48 \\ 
  Serras; Agrosistema intensivo (mosaico agroforestal); Mesotemperado inferior & 2.78 & 0.50 & 5.59 \\ 
  Serras; Agrosistema intensivo (mosaico agroforestal); Mesotemperado superior & 2.78 & 1.77 & 1.57 \\ 
  Serras; Turbeira; Mesotemperado inferior & 2.78 & 0.05 & 52.13 \\ 
  Vales sublitorais; Matogueira e rochedo; Mesotemperado inferior & 2.78 & 1.89 & 1.47 \\ 
  Litoral Cantabro-Atlantico; Agrosistema intensivo (plantacion forestal); Mesotemperado inferior & 1.39 & 0.62 & 2.25 \\ 
  Serras; Agrosistema intensivo (plantacion forestal); Mesotemperado superior & 1.39 & 1.07 & 1.29 \\ 
  Serras; Bosque; Mesotemperado inferior & 1.39 & 0.69 & 2.01 \\ 
  Vales sublitorais; Agrosistema intensivo (mosaico agroforestal); Mesotemperado superior & 1.39 & 1.52 & 0.91 \\ 
  Vales sublitorais; Agrosistema intensivo (plantacion forestal); Mesotemperado superior & 1.39 & 0.60 & 2.32 \\ 
  Vales sublitorais; no data; Mesotemperado inferior & 1.39 & 0.01 & 188.28 \\ 
  Vales sublitorais; Rururbano (diseminado); Mesotemperado inferior & 1.39 & 0.98 & 1.41 \\ 
   & 1.39 &  &  \\ 
   \hline
\end{tabular}
\end{table}
% latex table generated in R 3.2.2 by xtable 1.8-0 package
% Wed Dec  2 19:45:32 2015
\begin{table}[p]
\centering
\caption{Frecuencia de aparición de valores estéticos identificados na participación pública e frecuencia de tipos asociados Chairas e Fosas Occidentais} 
\label{vsixotest11}
\begin{tabular}{lrrr}
  \hline
Tipo de paisaxe & F.Aparic (\%) & F.Tipo (\%) & Ratio \\ 
  \hline
Vales sublitorais; Agrosistema intensivo (plantacion forestal); Mesotemperado inferior & 14.17 & 2.90 & 4.89 \\ 
  Litoral Cantabro-Atlantico; Matogueira e rochedo; Termotemperado & 10.83 & 0.83 & 13.04 \\ 
  Vales sublitorais; Agrosistema extensivo; Mesotemperado inferior & 10.00 & 2.73 & 3.66 \\ 
  Vales sublitorais; Agrosistema intensivo (mosaico agroforestal); Mesotemperado inferior & 7.50 & 7.54 & 0.99 \\ 
  Vales sublitorais; Agrosistema intensivo (plantacion forestal); Termotemperado & 6.67 & 1.79 & 3.72 \\ 
  Litoral Cantabro-Atlantico; Conxunto Historico; Termotemperado & 5.83 & 0.02 & 308.45 \\ 
  Vales sublitorais; Agrosistema intensivo (mosaico agroforestal); Termotemperado & 5.83 & 2.62 & 2.23 \\ 
  Vales sublitorais; Matogueira e rochedo; Mesotemperado inferior & 5.00 & 1.89 & 2.64 \\ 
  Litoral Cantabro-Atlantico; Agrosistema intensivo (plantacion forestal); Termotemperado & 4.17 & 1.32 & 3.16 \\ 
  Litoral Cantabro-Atlantico; Matogueira e rochedo; no data & 4.17 & 0.09 & 48.96 \\ 
  Litoral Cantabro-Atlantico; Rururbano (diseminado); Termotemperado & 3.33 & 3.29 & 1.01 \\ 
  Litoral Cantabro-Atlantico; Urbano; Termotemperado & 3.33 & 0.53 & 6.28 \\ 
  Vales sublitorais; Rururbano (diseminado); Mesotemperado inferior & 3.33 & 0.98 & 3.40 \\ 
  Litoral Cantabro-Atlantico; Agrosistema intensivo (mosaico agroforestal); Termotemperado & 2.50 & 2.43 & 1.03 \\ 
  Litoral Cantabro-Atlantico; Agrosistema intensivo (plantacion forestal); Mesotemperado inferior & 1.67 & 0.62 & 2.70 \\ 
  Vales sublitorais; Agrosistema intensivo (mosaico agroforestal); Mesotemperado superior & 1.67 & 1.52 & 1.10 \\ 
  Vales sublitorais; Matogueira e rochedo; Mesotemperado superior & 1.67 & 0.85 & 1.96 \\ 
  Vales sublitorais; Rururbano (diseminado); Termotemperado & 1.67 & 1.68 & 0.99 \\ 
   \hline
\end{tabular}
\end{table}
% latex table generated in R 3.2.2 by xtable 1.8-0 package
% Wed Dec  2 19:45:32 2015
\begin{table}[p]
\centering
\caption{Frecuencia de aparición de valores estéticos identificados na participación pública e frecuencia de tipos asociados Rías Baixas} 
\label{vsixotest12}
\begin{tabular}{lrrr}
  \hline
Tipo de paisaxe & F.Aparic (\%) & F.Tipo (\%) & Ratio \\ 
  \hline
Serras; Matogueira e rochedo; Mesotemperado superior & 21.74 & 5.03 & 4.32 \\ 
  Serras; Matogueira e rochedo; Mesotemperado inferior & 16.52 & 2.73 & 6.06 \\ 
  Vales sublitorais; Rururbano (diseminado); Termotemperado & 7.83 & 1.68 & 4.66 \\ 
  Vales sublitorais; Agrosistema intensivo (plantacion forestal); Termotemperado & 6.96 & 1.79 & 3.89 \\ 
  Vales sublitorais; Matogueira e rochedo; Mesotemperado inferior & 6.96 & 1.89 & 3.67 \\ 
  Litoral Cantabro-Atlantico; Rururbano (diseminado); Termotemperado & 4.35 & 3.29 & 1.32 \\ 
  Vales sublitorais; Agrosistema intensivo (mosaico agroforestal); Termotemperado & 4.35 & 2.62 & 1.66 \\ 
  Vales sublitorais; Agrosistema intensivo (plantacion forestal); Mesotemperado inferior & 4.35 & 2.90 & 1.50 \\ 
  Serras; Agrosistema intensivo (plantacion forestal); Mesotemperado inferior & 3.48 & 0.82 & 4.22 \\ 
  Litoral Cantabro-Atlantico; Agrosistema intensivo (plantacion forestal); Termotemperado & 2.61 & 1.32 & 1.98 \\ 
  Litoral Cantabro-Atlantico; Matogueira e rochedo; no data & 2.61 & 0.09 & 30.65 \\ 
  Litoral Cantabro-Atlantico; Urbano; Termotemperado & 2.61 & 0.53 & 4.92 \\ 
  Vales sublitorais; Agrosistema intensivo (mosaico agroforestal); Mesotemperado inferior & 2.61 & 7.54 & 0.35 \\ 
  Litoral Cantabro-Atlantico; Matogueira e rochedo; Termotemperado & 1.74 & 0.83 & 2.09 \\ 
  Vales sublitorais; Matogueira e rochedo; Mesotemperado superior & 1.74 & 0.85 & 2.04 \\ 
  Vales sublitorais; Matogueira e rochedo; Termotemperado & 1.74 & 0.93 & 1.86 \\ 
   \hline
\end{tabular}
\end{table}
