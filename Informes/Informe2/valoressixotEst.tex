% latex table generated in R 3.2.2 by xtable 1.8-0 package
% Fri Dec  4 16:54:41 2015
\begin{table}[p]
\centering
\caption{Frecuencia de aparición de valores estéticos identificados na participación pública e frecuencia de tipos asociados Golfo Ártabro} 
\label{vsixotest1}
\begin{tabular}{lrrr}
  \hline
Tipo de paisaxe & F.Aparic (\%) & F.Tipo (\%) & Ratio \\ 
  \hline
Litoral Cantabro-Atlantico; Rururbano (diseminado); Termotemperado & 11.86 & 19.32 & 0.61 \\ 
  Litoral Cantabro-Atlantico; Conxunto Historico; no data & 10.17 & 0.02 & 551.12 \\ 
  Canons; Bosque; Mesotemperado inferior & 6.78 & 1.04 & 6.50 \\ 
  Litoral Cantabro-Atlantico; Agrosistema intensivo (plantacion forestal); Termotemperado & 6.78 & 2.83 & 2.39 \\ 
  Vales sublitorais; Agrosistema intensivo (mosaico agroforestal); Mesotemperado inferior & 6.78 & 14.40 & 0.47 \\ 
  Litoral Cantabro-Atlantico; Matogueira e rochedo; Termotemperado & 5.08 & 0.87 & 5.85 \\ 
  Vales sublitorais; Agrosistema intensivo (mosaico agroforestal); Termotemperado & 5.08 & 4.06 & 1.25 \\ 
  Canons; Matogueira e rochedo; Mesotemperado inferior & 3.39 & 0.13 & 26.13 \\ 
  Litoral Cantabro-Atlantico; Agrosistema extensivo; Termotemperado & 3.39 & 0.82 & 4.12 \\ 
  Litoral Cantabro-Atlantico; Agrosistema intensivo (mosaico agroforestal); Termotemperado & 3.39 & 10.45 & 0.32 \\ 
  Litoral Cantabro-Atlantico; Conxunto Historico; Termotemperado & 3.39 & 0.07 & 47.61 \\ 
  Litoral Cantabro-Atlantico; Rururbano (diseminado); Mesotemperado inferior & 3.39 & 0.21 & 16.24 \\ 
  Litoral Cantabro-Atlantico; Urbano; Termotemperado & 3.39 & 3.10 & 1.09 \\ 
  Vales sublitorais; Agrosistema intensivo (plantacion forestal); Mesotemperado inferior & 3.39 & 7.86 & 0.43 \\ 
  Vales sublitorais; Bosque; Mesotemperado inferior & 3.39 & 0.58 & 5.84 \\ 
  Vales sublitorais; Bosque; Termotemperado & 3.39 & 0.03 & 112.71 \\ 
  Vales sublitorais; Matogueira e rochedo; Mesotemperado inferior & 3.39 & 1.43 & 2.37 \\ 
  Canons; Bosque; Termotemperado & 1.69 & 0.34 & 5.02 \\ 
  Serras; Agrosistema intensivo (mosaico agroforestal); Mesotemperado superior & 1.69 & 1.64 & 1.03 \\ 
  Serras; Turbeira; Mesotemperado superior & 1.69 & 1.50 & 1.13 \\ 
  Vales sublitorais; Agrosistema extensivo; Mesotemperado inferior & 1.69 & 0.90 & 1.88 \\ 
  Vales sublitorais; Agrosistema intensivo (plantacion forestal); Termotemperado & 1.69 & 1.65 & 1.03 \\ 
  Vales sublitorais; Matogueira e rochedo; Mesotemperado superior & 1.69 & 1.16 & 1.46 \\ 
  Vales sublitorais; Rururbano (diseminado); Termotemperado & 1.69 & 2.11 & 0.80 \\ 
   & 1.69 &  &  \\ 
   \hline
\end{tabular}
\end{table}
% latex table generated in R 3.2.2 by xtable 1.8-0 package
% Fri Dec  4 16:54:41 2015
\begin{table}[p]
\centering
\caption{Frecuencia de aparición de valores estéticos identificados na participación pública e frecuencia de tipos asociados A Mariña - Baixo Eo} 
\label{vsixotest2}
\begin{tabular}{lrrr}
  \hline
Tipo de paisaxe & F.Aparic (\%) & F.Tipo (\%) & Ratio \\ 
  \hline
Vales sublitorais; Agrosistema intensivo (plantacion forestal); Mesotemperado inferior & 18.60 & 14.02 & 1.33 \\ 
  Litoral Cantabro-Atlantico; Rururbano (diseminado); Termotemperado & 9.30 & 3.92 & 2.37 \\ 
  Vales sublitorais; Rururbano (diseminado); Mesotemperado inferior & 9.30 & 1.16 & 8.02 \\ 
  Litoral Cantabro-Atlantico; Matogueira e rochedo; Mesotemperado inferior & 6.98 & 0.09 & 79.57 \\ 
  Litoral Cantabro-Atlantico; Urbano; Termotemperado & 6.98 & 2.18 & 3.20 \\ 
  Litoral Cantabro-Atlantico; Agrosistema intensivo (mosaico agroforestal); Mesotemperado inferior & 4.65 & 4.26 & 1.09 \\ 
  Litoral Cantabro-Atlantico; Agrosistema intensivo (mosaico agroforestal); no data & 4.65 & 0.11 & 40.71 \\ 
  Litoral Cantabro-Atlantico; Agrosistema intensivo (plantacion forestal); Mesotemperado inferior & 4.65 & 10.84 & 0.43 \\ 
  Litoral Cantabro-Atlantico; Matogueira e rochedo; Termotemperado & 4.65 & 0.31 & 14.79 \\ 
  Serras; Turbeira; Mesotemperado superior & 4.65 & 2.35 & 1.98 \\ 
  Vales sublitorais; Agrosistema intensivo (mosaico agroforestal); Mesotemperado inferior & 4.65 & 12.56 & 0.37 \\ 
   & 4.65 &  &  \\ 
  Litoral Cantabro-Atlantico; Agrosistema intensivo (mosaico agroforestal); Termotemperado & 2.33 & 8.49 & 0.27 \\ 
  Litoral Cantabro-Atlantico; Conxunto Historico; Termotemperado & 2.33 & 0.07 & 35.03 \\ 
  Litoral Cantabro-Atlantico; Rururbano (diseminado); Mesotemperado inferior & 2.33 & 0.50 & 4.67 \\ 
  Litoral Cantabro-Atlantico; Urbano; no data & 2.33 & 0.03 & 74.46 \\ 
  Serras; Bosque; Mesotemperado superior & 2.33 & 0.68 & 3.43 \\ 
  Serras; Matogueira e rochedo; Supra e orotemperado & 2.33 & 0.24 & 9.80 \\ 
  Vales sublitorais; Bosque; Mesotemperado superior & 2.33 & 0.85 & 2.73 \\ 
   \hline
\end{tabular}
\end{table}
% latex table generated in R 3.2.2 by xtable 1.8-0 package
% Fri Dec  4 16:54:42 2015
\begin{table}[p]
\centering
\caption{Frecuencia de aparición de valores estéticos identificados na participación pública e frecuencia de tipos asociados Costa Sur - Baixo Miño} 
\label{vsixotest3}
\begin{tabular}{lrrr}
  \hline
Tipo de paisaxe & F.Aparic (\%) & F.Tipo (\%) & Ratio \\ 
  \hline
Serras; Matogueira e rochedo; Mesotemperado inferior & 10.87 & 5.06 & 2.15 \\ 
  Litoral Cantabro-Atlantico; Rururbano (diseminado); Termotemperado & 9.78 & 11.41 & 0.86 \\ 
  Serras; Conxunto Historico; Termotemperado & 8.70 & 0.20 & 43.46 \\ 
  Serras; Agrosistema intensivo (plantacion forestal); Mesotemperado inferior & 6.52 & 2.04 & 3.20 \\ 
  Serras; Agrosistema intensivo (plantacion forestal); Termotemperado & 6.52 & 2.79 & 2.34 \\ 
  Vales sublitorais; Agrosistema intensivo (plantacion forestal); Termotemperado & 6.52 & 8.94 & 0.73 \\ 
  Serras; Matogueira e rochedo; Mesotemperado superior & 5.43 & 5.89 & 0.92 \\ 
  Serras; Turbeira; Mesotemperado inferior & 5.43 & 0.09 & 62.31 \\ 
  Vales sublitorais; Rururbano (diseminado); Termotemperado & 5.43 & 12.49 & 0.43 \\ 
  Litoral Cantabro-Atlantico; Agrosistema intensivo (plantacion forestal); Termotemperado & 4.35 & 2.51 & 1.73 \\ 
  Serras; Matogueira e rochedo; Termotemperado & 4.35 & 1.75 & 2.49 \\ 
  Vales sublitorais; Matogueira e rochedo; Termotemperado & 4.35 & 3.42 & 1.27 \\ 
  Vales sublitorais; Agrosistema intensivo (mosaico agroforestal); Termotemperado & 3.26 & 2.92 & 1.12 \\ 
  Litoral Cantabro-Atlantico; Matogueira e rochedo; no data & 2.17 & 0.10 & 21.61 \\ 
  Vales sublitorais; Bosque; Termotemperado & 2.17 & 1.03 & 2.11 \\ 
  Chairas e vales interiores; Agrosistema intensivo (mosaico agroforestal); Mesotemperado inferior & 1.09 & 0.98 & 1.11 \\ 
  Chairas e vales interiores; Matogueira e rochedo; Mesotemperado inferior & 1.09 & 2.49 & 0.44 \\ 
  Chairas e vales interiores; Matogueira e rochedo; Termotemperado & 1.09 & 1.45 & 0.75 \\ 
  Litoral Cantabro-Atlantico; Agrosistema extensivo; Termotemperado & 1.09 & 0.39 & 2.80 \\ 
  Litoral Cantabro-Atlantico; Agrosistema intensivo (mosaico agroforestal); Termotemperado & 1.09 & 3.27 & 0.33 \\ 
  Litoral Cantabro-Atlantico; Agrosistema intensivo (superficie de cultivo); no data & 1.09 & 0.02 & 51.18 \\ 
  Litoral Cantabro-Atlantico; Viñedo; Termotemperado & 1.09 & 0.59 & 1.83 \\ 
  no data; Agrosistema intensivo (plantacion forestal); Termotemperado & 1.09 & 0.69 & 1.57 \\ 
  no data; Rururbano (diseminado); Termotemperado & 1.09 & 1.13 & 0.96 \\ 
  Serras; Agrosistema extensivo; Mesotemperado inferior & 1.09 & 0.80 & 1.35 \\ 
  Serras; Agrosistema extensivo; Mesotemperado superior & 1.09 & 0.91 & 1.20 \\ 
  Vales sublitorais; Agrosistema intensivo (plantacion forestal); Mesotemperado inferior & 1.09 & 0.28 & 3.85 \\ 
  Vales sublitorais; Matogueira e rochedo; Mesotemperado inferior & 1.09 & 0.86 & 1.26 \\ 
   \hline
\end{tabular}
\end{table}
% latex table generated in R 3.2.2 by xtable 1.8-0 package
% Fri Dec  4 16:54:42 2015
\begin{table}[p]
\centering
\caption{Frecuencia de aparición de valores estéticos identificados na participación pública e frecuencia de tipos asociados Ribeiras Encaixadas do Miño e do Sil} 
\label{vsixotest4}
\begin{tabular}{lrrr}
  \hline
Tipo de paisaxe & F.Aparic (\%) & F.Tipo (\%) & Ratio \\ 
  \hline
Chairas e vales interiores; Bosque; Termotemperado & 9.26 & 2.38 & 3.89 \\ 
  Canons; Bosque; Mesotemperado inferior & 7.41 & 1.90 & 3.90 \\ 
  Canons; Bosque; Termotemperado & 7.41 & 1.40 & 5.27 \\ 
  Canons; Matogueira e rochedo; Mesomediterráneo & 5.56 & 1.77 & 3.14 \\ 
  Canons; Viñedo; Mesotemperado inferior & 5.56 & 0.24 & 22.80 \\ 
  Chairas e vales interiores; Rururbano (diseminado); Mesotemperado inferior & 4.63 & 1.23 & 3.77 \\ 
  Serras; Agrosistema extensivo; Mesotemperado inferior & 4.63 & 4.45 & 1.04 \\ 
  Chairas e vales interiores; Conxunto Historico; Termotemperado & 3.70 & 0.01 & 272.77 \\ 
  Chairas e vales interiores; Matogueira e rochedo; Mesomediterráneo & 3.70 & 2.99 & 1.24 \\ 
  Chairas e vales interiores; Rururbano (diseminado); Termotemperado & 3.70 & 2.59 & 1.43 \\ 
  Serras; Bosque; Mesotemperado inferior & 3.70 & 1.46 & 2.54 \\ 
  Serras; Matogueira e rochedo; Mesotemperado inferior & 3.70 & 5.11 & 0.72 \\ 
  Canons; Matogueira e rochedo; Mesotemperado inferior & 2.78 & 1.29 & 2.15 \\ 
  Chairas e vales interiores; Agrosistema intensivo (plantacion forestal); Termotemperado & 2.78 & 2.44 & 1.14 \\ 
  Chairas e vales interiores; Matogueira e rochedo; Mesotemperado inferior & 2.78 & 4.17 & 0.67 \\ 
  Serras; Agrosistema extensivo; Mesotemperado superior & 2.78 & 3.22 & 0.86 \\ 
  Canons; Agrosistema extensivo; Mesomediterráneo & 1.85 & 0.35 & 5.30 \\ 
  Canons; Agrosistema intensivo (plantacion forestal); Termotemperado & 1.85 & 0.67 & 2.74 \\ 
  Chairas e vales interiores; Bosque; Mesomediterráneo & 1.85 & 0.54 & 3.40 \\ 
  Chairas e vales interiores; Matogueira e rochedo; Termotemperado & 1.85 & 4.28 & 0.43 \\ 
  Chairas e vales interiores; Viñedo; Termotemperado & 1.85 & 2.21 & 0.84 \\ 
   \hline
\end{tabular}
\end{table}
% latex table generated in R 3.2.2 by xtable 1.8-0 package
% Fri Dec  4 16:54:42 2015
\begin{table}[p]
\centering
\caption{Frecuencia de aparición de valores estéticos identificados na participación pública e frecuencia de tipos asociados Serras Orientais} 
\label{vsixotest5}
\begin{tabular}{lrrr}
  \hline
Tipo de paisaxe & F.Aparic (\%) & F.Tipo (\%) & Ratio \\ 
  \hline
Serras; Agrosistema extensivo; Supra e orotemperado & 16.67 & 15.17 & 1.10 \\ 
  Serras; Agrosistema extensivo; Mesotemperado superior & 12.82 & 11.84 & 1.08 \\ 
  Serras; Bosque; Supra e orotemperado & 11.54 & 5.16 & 2.23 \\ 
  Serras; Matogueira e rochedo; Supra e orotemperado & 11.54 & 14.82 & 0.78 \\ 
  Serras; Bosque; Mesotemperado superior & 7.69 & 3.78 & 2.04 \\ 
  Vales sublitorais; Bosque; Mesotemperado inferior & 6.41 & 2.82 & 2.27 \\ 
  Serras; Matogueira e rochedo; Mesotemperado superior & 5.13 & 6.90 & 0.74 \\ 
  Vales sublitorais; Agrosistema extensivo; Mesotemperado inferior & 5.13 & 3.89 & 1.32 \\ 
  Vales sublitorais; Agrosistema extensivo; Mesotemperado superior & 3.85 & 5.67 & 0.68 \\ 
  Vales sublitorais; Bosque; Mesotemperado superior & 3.85 & 3.48 & 1.10 \\ 
  Canons; Matogueira e rochedo; Termotemperado & 2.56 & 0.33 & 7.85 \\ 
  Serras; Agrosistema extensivo; Mesotemperado inferior & 2.56 & 0.47 & 5.43 \\ 
  Serras; Agrosistema intensivo (mosaico agroforestal); Supra e orotemperado & 2.56 & 2.44 & 1.05 \\ 
  Canons; Agrosistema intensivo (plantacion forestal); Termotemperado & 1.28 & 0.13 & 9.51 \\ 
  Canons; Bosque; Mesotemperado inferior & 1.28 & 0.28 & 4.57 \\ 
  Chairas e vales interiores; Agrosistema extensivo; Mesotemperado inferior & 1.28 & 0.29 & 4.40 \\ 
  Chairas e vales interiores; Agrosistema extensivo; Mesotemperado superior & 1.28 & 0.46 & 2.78 \\ 
  Serras; Agrosistema intensivo (mosaico agroforestal); Mesotemperado superior & 1.28 & 1.96 & 0.65 \\ 
  Serras; Agrosistema intensivo (plantacion forestal); Supra e orotemperado & 1.28 & 3.43 & 0.37 \\ 
   \hline
\end{tabular}
\end{table}
% latex table generated in R 3.2.2 by xtable 1.8-0 package
% Fri Dec  4 16:54:42 2015
\begin{table}[p]
\centering
\caption{Frecuencia de aparición de valores estéticos identificados na participación pública e frecuencia de tipos asociados Chairas e Fosas Luguesas} 
\label{vsixotest6}
\begin{tabular}{lrrr}
  \hline
Tipo de paisaxe & F.Aparic (\%) & F.Tipo (\%) & Ratio \\ 
  \hline
Chairas e vales interiores; Agrosistema extensivo; Mesotemperado inferior & 15.87 & 9.57 & 1.66 \\ 
  Chairas e vales interiores; Agrosistema extensivo; Mesotemperado superior & 14.29 & 17.50 & 0.82 \\ 
  Chairas e vales interiores; Bosque; Mesotemperado inferior & 9.52 & 1.29 & 7.37 \\ 
  Serras; Agrosistema intensivo (plantacion forestal); Supra e orotemperado & 7.94 & 0.44 & 18.03 \\ 
  Chairas e vales interiores; Agrosistema intensivo (mosaico agroforestal); Mesotemperado superior & 6.35 & 14.58 & 0.44 \\ 
  Chairas e vales interiores; Urbano; Mesotemperado inferior & 6.35 & 0.29 & 21.84 \\ 
  Serras; Agrosistema extensivo; Mesotemperado superior & 6.35 & 10.14 & 0.63 \\ 
  Chairas e vales interiores; Conxunto Historico; Termotemperado & 4.76 & 0.01 & 414.53 \\ 
  Chairas e vales interiores; Rururbano (diseminado); Mesotemperado superior & 4.76 & 1.94 & 2.45 \\ 
  Chairas e vales interiores; Agrosistema intensivo (mosaico agroforestal); Mesotemperado inferior & 3.17 & 6.48 & 0.49 \\ 
  Chairas e vales interiores; Agrosistema intensivo (superficie de cultivo); Mesotemperado superior & 3.17 & 4.58 & 0.69 \\ 
  Chairas e vales interiores; Bosque; Mesotemperado superior & 3.17 & 1.53 & 2.07 \\ 
  Chairas e vales interiores; Conxunto Historico; Mesotemperado superior & 3.17 & 0.01 & 442.80 \\ 
  Serras; Turbeira; Mesotemperado superior & 3.17 & 1.14 & 2.79 \\ 
  Chairas e vales interiores; Agrosistema intensivo (plantacion forestal); Mesotemperado superior & 1.59 & 2.61 & 0.61 \\ 
  Chairas e vales interiores; Rururbano (diseminado); Mesotemperado inferior & 1.59 & 1.41 & 1.12 \\ 
  Chairas e vales interiores; Urbano; Mesotemperado superior & 1.59 & 0.22 & 7.21 \\ 
  Serras; Agrosistema extensivo; Supra e orotemperado & 1.59 & 1.09 & 1.45 \\ 
  Serras; Matogueira e rochedo; Supra e orotemperado & 1.59 & 0.84 & 1.89 \\ 
   \hline
\end{tabular}
\end{table}
% latex table generated in R 3.2.2 by xtable 1.8-0 package
% Fri Dec  4 16:54:42 2015
\begin{table}[p]
\centering
\caption{Frecuencia de aparición de valores estéticos identificados na participación pública e frecuencia de tipos asociados Galicia Central} 
\label{vsixotest7}
\begin{tabular}{lrrr}
  \hline
Tipo de paisaxe & F.Aparic (\%) & F.Tipo (\%) & Ratio \\ 
  \hline
Vales sublitorais; Agrosistema extensivo; Mesotemperado inferior & 15.03 & 10.39 & 1.45 \\ 
  Vales sublitorais; Agrosistema intensivo (mosaico agroforestal); Termotemperado & 8.81 & 5.79 & 1.52 \\ 
  Vales sublitorais; Rururbano (diseminado); Termotemperado & 7.77 & 1.94 & 4.01 \\ 
  Vales sublitorais; Agrosistema intensivo (mosaico agroforestal); Mesotemperado inferior & 6.74 & 23.79 & 0.28 \\ 
  Vales sublitorais; Urbano; Mesotemperado inferior & 6.74 & 0.49 & 13.64 \\ 
  Vales sublitorais; Rururbano (diseminado); Mesotemperado inferior & 5.70 & 3.66 & 1.56 \\ 
  Serras; Agrosistema extensivo; Mesotemperado superior & 5.18 & 7.06 & 0.73 \\ 
  Vales sublitorais; Agrosistema intensivo (superficie de cultivo); Mesotemperado inferior & 4.66 & 2.45 & 1.90 \\ 
  Serras; Matogueira e rochedo; Mesotemperado superior & 4.15 & 5.00 & 0.83 \\ 
  Serras; Agrosistema extensivo; Mesotemperado inferior & 3.11 & 2.67 & 1.16 \\ 
  Vales sublitorais; Conxunto Historico; Mesotemperado inferior & 3.11 & 0.02 & 176.99 \\ 
  Serras; Matogueira e rochedo; Mesotemperado inferior & 2.59 & 1.61 & 1.61 \\ 
  Vales sublitorais; Bosque; Mesotemperado inferior & 2.59 & 0.61 & 4.27 \\ 
  Vales sublitorais; Agrosistema extensivo; Termotemperado & 2.07 & 0.96 & 2.16 \\ 
  Vales sublitorais; Agrosistema intensivo (plantacion forestal); Termotemperado & 2.07 & 1.21 & 1.71 \\ 
  Vales sublitorais; Matogueira e rochedo; Mesotemperado inferior & 2.07 & 3.12 & 0.67 \\ 
  Vales sublitorais; Matogueira e rochedo; Termotemperado & 2.07 & 0.95 & 2.18 \\ 
  Serras; Agrosistema intensivo (superficie de cultivo); Mesotemperado superior & 1.55 & 0.86 & 1.80 \\ 
  Vales sublitorais; Agrosistema extensivo; Mesotemperado superior & 1.55 & 2.91 & 0.53 \\ 
  Chairas e vales interiores; Agrosistema extensivo; Termotemperado & 1.04 & 0.24 & 4.30 \\ 
  Chairas e vales interiores; Urbano; Termotemperado & 1.04 & 0.00 & 220.74 \\ 
  Serras; Agrosistema intensivo (plantacion forestal); Supra e orotemperado & 1.04 & 0.26 & 3.96 \\ 
  Serras; Rururbano (diseminado); Mesotemperado superior & 1.04 & 0.15 & 6.71 \\ 
  Vales sublitorais; Agrosistema intensivo (mosaico agroforestal); Mesotemperado superior & 1.04 & 4.40 & 0.24 \\ 
  Vales sublitorais; Agrosistema intensivo (plantacion forestal); Mesotemperado inferior & 1.04 & 2.72 & 0.38 \\ 
  Vales sublitorais; Matogueira e rochedo; Mesotemperado superior & 1.04 & 1.07 & 0.97 \\ 
   \hline
\end{tabular}
\end{table}
% latex table generated in R 3.2.2 by xtable 1.8-0 package
% Fri Dec  4 16:54:42 2015
\begin{table}[p]
\centering
\caption{Frecuencia de aparición de valores estéticos identificados na participación pública e frecuencia de tipos asociados Chairas, Fosas e Serras Ourensás} 
\label{vsixotest8}
\begin{tabular}{lrrr}
  \hline
Tipo de paisaxe & F.Aparic (\%) & F.Tipo (\%) & Ratio \\ 
  \hline
Serras; Matogueira e rochedo; Mesotemperado superior & 11.76 & 9.99 & 1.18 \\ 
  Serras; Matogueira e rochedo; Supra e orotemperado & 9.80 & 8.42 & 1.16 \\ 
  Chairas e vales interiores; Agrosistema extensivo; Mesotemperado inferior & 7.84 & 12.05 & 0.65 \\ 
  Chairas e vales interiores; Matogueira e rochedo; Mesotemperado inferior & 7.84 & 4.79 & 1.64 \\ 
  Serras; Matogueira e rochedo; Mesotemperado inferior & 7.84 & 8.98 & 0.87 \\ 
  Serras; Agrosistema extensivo; Mesotemperado inferior & 6.86 & 9.02 & 0.76 \\ 
  Chairas e vales interiores; Bosque; Mesotemperado inferior & 4.90 & 2.80 & 1.75 \\ 
  Chairas e vales interiores; Rururbano (diseminado); Mesotemperado inferior & 4.90 & 1.22 & 4.02 \\ 
  Serras; Agrosistema extensivo; Mesotemperado superior & 4.90 & 4.93 & 1.00 \\ 
  Serras; Agrosistema extensivo; Supra e orotemperado & 4.90 & 3.09 & 1.59 \\ 
  Chairas e vales interiores; Agrosistema intensivo (superficie de cultivo); Mesotemperado inferior & 2.94 & 7.47 & 0.39 \\ 
  Chairas e vales interiores; Matogueira e rochedo; Termotemperado & 2.94 & 2.16 & 1.36 \\ 
  Serras; Bosque; Mesotemperado inferior & 2.94 & 2.68 & 1.10 \\ 
  Serras; Bosque; Mesotemperado superior & 2.94 & 1.61 & 1.82 \\ 
  Chairas e vales interiores; Agrosistema extensivo; Termotemperado & 1.96 & 1.77 & 1.10 \\ 
  Chairas e vales interiores; Agrosistema intensivo (mosaico agroforestal); Mesotemperado inferior & 1.96 & 1.44 & 1.36 \\ 
  Chairas e vales interiores; Conxunto Historico; Mesotemperado inferior & 1.96 & 0.01 & 193.61 \\ 
  Chairas e vales interiores; Viñedo; Termotemperado & 1.96 & 0.65 & 3.02 \\ 
  Serras; Bosque; Supra e orotemperado & 1.96 & 0.97 & 2.01 \\ 
   \hline
\end{tabular}
\end{table}
% latex table generated in R 3.2.2 by xtable 1.8-0 package
% Fri Dec  4 16:54:42 2015
\begin{table}[p]
\centering
\caption{Frecuencia de aparición de valores estéticos identificados na participación pública e frecuencia de tipos asociados Serras Surorientais} 
\label{vsixotest9}
\begin{tabular}{lrrr}
  \hline
Tipo de paisaxe & F.Aparic (\%) & F.Tipo (\%) & Ratio \\ 
  \hline
Serras; Matogueira e rochedo; Supra e orotemperado & 29.29 & 37.48 & 0.78 \\ 
  Serras; Agrosistema extensivo; Mesotemperado inferior & 13.13 & 7.06 & 1.86 \\ 
  Serras; Agrosistema extensivo; Mesotemperado superior & 10.10 & 10.41 & 0.97 \\ 
  Serras; Agrosistema extensivo; Supra e orotemperado & 7.07 & 7.54 & 0.94 \\ 
  Serras; Matogueira e rochedo; Mesotemperado superior & 6.06 & 9.91 & 0.61 \\ 
  Canons; Viñedo; Mesomediterráneo & 4.04 & 0.02 & 170.63 \\ 
  Serras; Bosque; Mesotemperado superior & 4.04 & 2.58 & 1.57 \\ 
  Serras; Agrosistema intensivo (plantacion forestal); Supra e orotemperado & 3.03 & 5.13 & 0.59 \\ 
  Serras; Matogueira e rochedo; Mesotemperado inferior & 3.03 & 4.18 & 0.73 \\ 
  Serras; Matogueira e rochedo; no data & 3.03 & 0.31 & 9.65 \\ 
  Canons; Matogueira e rochedo; Mesomediterráneo & 2.02 & 0.50 & 4.05 \\ 
  Canons; Matogueira e rochedo; Mesotemperado inferior & 2.02 & 0.93 & 2.18 \\ 
  Serras; Agrosistema extensivo; Mesomediterráneo & 2.02 & 0.18 & 10.95 \\ 
  Serras; Bosque; Mesotemperado inferior & 2.02 & 1.96 & 1.03 \\ 
  Canons; Agrosistema extensivo; Mesomediterráneo & 1.01 & 0.07 & 14.23 \\ 
  Canons; Bosque; Mesotemperado inferior & 1.01 & 0.80 & 1.26 \\ 
  Canons; Viñedo; Mesotemperado inferior & 1.01 & 0.04 & 27.05 \\ 
  Serras; Agrosistema intensivo (superficie de cultivo); Mesotemperado superior & 1.01 & 1.15 & 0.88 \\ 
  Serras; Bosque; Supra e orotemperado & 1.01 & 2.05 & 0.49 \\ 
  Serras; Rururbano (diseminado); Mesotemperado inferior & 1.01 & 0.19 & 5.33 \\ 
  Serras; Rururbano (diseminado); Mesotemperado superior & 1.01 & 0.12 & 8.26 \\ 
  Vales sublitorais; Agrosistema extensivo; Mesomediterráneo & 1.01 & 0.00 & 399.73 \\ 
  Vales sublitorais; Viñedo; Mesomediterráneo & 1.01 & 0.00 & 624.14 \\ 
   \hline
\end{tabular}
\end{table}
% latex table generated in R 3.2.2 by xtable 1.8-0 package
% Fri Dec  4 16:54:42 2015
\begin{table}[p]
\centering
\caption{Frecuencia de aparición de valores estéticos identificados na participación pública e frecuencia de tipos asociados Galicia Setentrional} 
\label{vsixotest10}
\begin{tabular}{lrrr}
  \hline
Tipo de paisaxe & F.Aparic (\%) & F.Tipo (\%) & Ratio \\ 
  \hline
Serras; Turbeira; Supra e orotemperado & 13.89 & 5.12 & 2.71 \\ 
  Vales sublitorais; Agrosistema intensivo (plantacion forestal); Mesotemperado inferior & 12.50 & 11.84 & 1.06 \\ 
  Vales sublitorais; Agrosistema intensivo (mosaico agroforestal); Mesotemperado inferior & 8.33 & 12.50 & 0.67 \\ 
  Serras; Matogueira e rochedo; Mesotemperado superior & 6.94 & 6.61 & 1.05 \\ 
  Serras; Turbeira; Mesotemperado superior & 6.94 & 6.59 & 1.05 \\ 
  Vales sublitorais; Bosque; Mesotemperado inferior & 5.56 & 0.94 & 5.94 \\ 
  Litoral Cantabro-Atlantico; Agrosistema intensivo (mosaico agroforestal); Mesotemperado inferior & 4.17 & 1.86 & 2.24 \\ 
  Litoral Cantabro-Atlantico; Agrosistema intensivo (mosaico agroforestal); Termotemperado & 4.17 & 5.96 & 0.70 \\ 
  Litoral Cantabro-Atlantico; Agrosistema intensivo (plantacion forestal); Termotemperado & 4.17 & 3.32 & 1.26 \\ 
  Litoral Cantabro-Atlantico; Agrosistema intensivo (plantacion forestal); no data & 2.78 & 0.10 & 27.99 \\ 
  Litoral Cantabro-Atlantico; Matogueira e rochedo; Mesotemperado inferior & 2.78 & 0.60 & 4.63 \\ 
  Litoral Cantabro-Atlantico; Rururbano (diseminado); Termotemperado & 2.78 & 2.78 & 1.00 \\ 
  Serras; Agrosistema extensivo; Mesotemperado superior & 2.78 & 4.36 & 0.64 \\ 
  Serras; Agrosistema intensivo (mosaico agroforestal); Mesotemperado inferior & 2.78 & 1.66 & 1.67 \\ 
  Serras; Agrosistema intensivo (mosaico agroforestal); Mesotemperado superior & 2.78 & 3.06 & 0.91 \\ 
  Serras; Turbeira; Mesotemperado inferior & 2.78 & 0.68 & 4.06 \\ 
  Vales sublitorais; Matogueira e rochedo; Mesotemperado inferior & 2.78 & 0.99 & 2.80 \\ 
  Litoral Cantabro-Atlantico; Agrosistema intensivo (plantacion forestal); Mesotemperado inferior & 1.39 & 3.84 & 0.36 \\ 
  Serras; Agrosistema intensivo (plantacion forestal); Mesotemperado superior & 1.39 & 2.79 & 0.50 \\ 
  Serras; Bosque; Mesotemperado inferior & 1.39 & 0.24 & 5.89 \\ 
  Vales sublitorais; Agrosistema intensivo (mosaico agroforestal); Mesotemperado superior & 1.39 & 2.20 & 0.63 \\ 
  Vales sublitorais; Agrosistema intensivo (plantacion forestal); Mesotemperado superior & 1.39 & 2.01 & 0.69 \\ 
  Vales sublitorais; no data; Mesotemperado inferior & 1.39 & 0.10 & 14.01 \\ 
  Vales sublitorais; Rururbano (diseminado); Mesotemperado inferior & 1.39 & 0.53 & 2.60 \\ 
   & 1.39 &  &  \\ 
   \hline
\end{tabular}
\end{table}
% latex table generated in R 3.2.2 by xtable 1.8-0 package
% Fri Dec  4 16:54:42 2015
\begin{table}[p]
\centering
\caption{Frecuencia de aparición de valores estéticos identificados na participación pública e frecuencia de tipos asociados Chairas e Fosas Occidentais} 
\label{vsixotest11}
\begin{tabular}{lrrr}
  \hline
Tipo de paisaxe & F.Aparic (\%) & F.Tipo (\%) & Ratio \\ 
  \hline
Vales sublitorais; Agrosistema intensivo (plantacion forestal); Mesotemperado inferior & 14.17 & 8.33 & 1.70 \\ 
  Litoral Cantabro-Atlantico; Matogueira e rochedo; Termotemperado & 10.83 & 5.61 & 1.93 \\ 
  Vales sublitorais; Agrosistema extensivo; Mesotemperado inferior & 10.00 & 4.56 & 2.19 \\ 
  Vales sublitorais; Agrosistema intensivo (mosaico agroforestal); Mesotemperado inferior & 7.50 & 18.05 & 0.42 \\ 
  Vales sublitorais; Agrosistema intensivo (plantacion forestal); Termotemperado & 6.67 & 3.89 & 1.72 \\ 
  Litoral Cantabro-Atlantico; Conxunto Historico; Termotemperado & 5.83 & 0.12 & 47.79 \\ 
  Vales sublitorais; Agrosistema intensivo (mosaico agroforestal); Termotemperado & 5.83 & 8.83 & 0.66 \\ 
  Vales sublitorais; Matogueira e rochedo; Mesotemperado inferior & 5.00 & 7.15 & 0.70 \\ 
  Litoral Cantabro-Atlantico; Agrosistema intensivo (plantacion forestal); Termotemperado & 4.17 & 4.31 & 0.97 \\ 
  Litoral Cantabro-Atlantico; Matogueira e rochedo; no data & 4.17 & 0.47 & 8.94 \\ 
  Litoral Cantabro-Atlantico; Rururbano (diseminado); Termotemperado & 3.33 & 2.77 & 1.20 \\ 
  Litoral Cantabro-Atlantico; Urbano; Termotemperado & 3.33 & 0.44 & 7.50 \\ 
  Vales sublitorais; Rururbano (diseminado); Mesotemperado inferior & 3.33 & 1.88 & 1.77 \\ 
  Litoral Cantabro-Atlantico; Agrosistema intensivo (mosaico agroforestal); Termotemperado & 2.50 & 8.86 & 0.28 \\ 
  Litoral Cantabro-Atlantico; Agrosistema intensivo (plantacion forestal); Mesotemperado inferior & 1.67 & 0.31 & 5.31 \\ 
  Vales sublitorais; Agrosistema intensivo (mosaico agroforestal); Mesotemperado superior & 1.67 & 4.01 & 0.42 \\ 
  Vales sublitorais; Matogueira e rochedo; Mesotemperado superior & 1.67 & 3.30 & 0.51 \\ 
  Vales sublitorais; Rururbano (diseminado); Termotemperado & 1.67 & 2.48 & 0.67 \\ 
   \hline
\end{tabular}
\end{table}
% latex table generated in R 3.2.2 by xtable 1.8-0 package
% Fri Dec  4 16:54:42 2015
\begin{table}[p]
\centering
\caption{Frecuencia de aparición de valores estéticos identificados na participación pública e frecuencia de tipos asociados Rías Baixas} 
\label{vsixotest12}
\begin{tabular}{lrrr}
  \hline
Tipo de paisaxe & F.Aparic (\%) & F.Tipo (\%) & Ratio \\ 
  \hline
Serras; Matogueira e rochedo; Mesotemperado superior & 21.74 & 4.07 & 5.34 \\ 
  Serras; Matogueira e rochedo; Mesotemperado inferior & 16.52 & 4.31 & 3.83 \\ 
  Vales sublitorais; Rururbano (diseminado); Termotemperado & 7.83 & 6.06 & 1.29 \\ 
  Vales sublitorais; Agrosistema intensivo (plantacion forestal); Termotemperado & 6.96 & 8.25 & 0.84 \\ 
  Vales sublitorais; Matogueira e rochedo; Mesotemperado inferior & 6.96 & 5.63 & 1.24 \\ 
  Litoral Cantabro-Atlantico; Rururbano (diseminado); Termotemperado & 4.35 & 16.40 & 0.27 \\ 
  Vales sublitorais; Agrosistema intensivo (mosaico agroforestal); Termotemperado & 4.35 & 5.62 & 0.77 \\ 
  Vales sublitorais; Agrosistema intensivo (plantacion forestal); Mesotemperado inferior & 4.35 & 3.99 & 1.09 \\ 
  Serras; Agrosistema intensivo (plantacion forestal); Mesotemperado inferior & 3.48 & 1.60 & 2.18 \\ 
  Litoral Cantabro-Atlantico; Agrosistema intensivo (plantacion forestal); Termotemperado & 2.61 & 4.94 & 0.53 \\ 
  Litoral Cantabro-Atlantico; Matogueira e rochedo; no data & 2.61 & 0.23 & 11.33 \\ 
  Litoral Cantabro-Atlantico; Urbano; Termotemperado & 2.61 & 2.21 & 1.18 \\ 
  Vales sublitorais; Agrosistema intensivo (mosaico agroforestal); Mesotemperado inferior & 2.61 & 3.76 & 0.69 \\ 
  Litoral Cantabro-Atlantico; Matogueira e rochedo; Termotemperado & 1.74 & 2.61 & 0.67 \\ 
  Vales sublitorais; Matogueira e rochedo; Mesotemperado superior & 1.74 & 0.86 & 2.03 \\ 
  Vales sublitorais; Matogueira e rochedo; Termotemperado & 1.74 & 5.27 & 0.33 \\ 
   \hline
\end{tabular}
\end{table}
