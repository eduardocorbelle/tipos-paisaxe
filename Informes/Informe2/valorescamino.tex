% latex table generated in R 3.2.2 by xtable 1.8-0 package
% Fri Dec  4 16:54:41 2015
\begin{table}[p]
\centering
\caption{Frecuencia de aparición dos Camiños de Santiago (área de influencia de 500 m a ambos lados) e frecuencia de tipos asociados Golfo Ártabro} 
\label{vcamino1}
\begin{tabular}{lrrr}
  \hline
Tipo de paisaxe & F.Aparic (\%) & F.Tipo (\%) & Ratio \\ 
  \hline
Litoral Cantabro-Atlantico; Rururbano (diseminado); Termotemperado & 33.30 & 19.32 & 1.72 \\ 
  Litoral Cantabro-Atlantico; Urbano; Termotemperado & 15.84 & 3.10 & 5.10 \\ 
  Litoral Cantabro-Atlantico; Agrosistema intensivo (mosaico agroforestal); Termotemperado & 13.67 & 10.45 & 1.31 \\ 
  Vales sublitorais; Agrosistema intensivo (mosaico agroforestal); Termotemperado & 8.34 & 4.06 & 2.05 \\ 
  Vales sublitorais; Rururbano (diseminado); Termotemperado & 6.23 & 2.11 & 2.95 \\ 
  Vales sublitorais; Agrosistema intensivo (mosaico agroforestal); Mesotemperado inferior & 4.70 & 14.40 & 0.33 \\ 
  Litoral Cantabro-Atlantico; Agrosistema intensivo (plantacion forestal); Termotemperado & 4.18 & 2.83 & 1.48 \\ 
  Vales sublitorais; Agrosistema intensivo (plantacion forestal); Termotemperado & 2.24 & 1.65 & 1.36 \\ 
  Vales sublitorais; Agrosistema intensivo (plantacion forestal); Mesotemperado inferior & 1.84 & 7.86 & 0.23 \\ 
  Vales sublitorais; Matogueira e rochedo; Mesotemperado superior & 1.56 & 1.16 & 1.35 \\ 
  Litoral Cantabro-Atlantico; Urbano; no data & 1.48 & 0.21 & 7.20 \\ 
   \hline
\end{tabular}
\end{table}
% latex table generated in R 3.2.2 by xtable 1.8-0 package
% Fri Dec  4 16:54:41 2015
\begin{table}[p]
\centering
\caption{Frecuencia de aparición dos Camiños de Santiago (área de influencia de 500 m a ambos lados) e frecuencia de tipos asociados A Mariña - Baixo Eo} 
\label{vcamino2}
\begin{tabular}{lrrr}
  \hline
Tipo de paisaxe & F.Aparic (\%) & F.Tipo (\%) & Ratio \\ 
  \hline
Vales sublitorais; Agrosistema intensivo (mosaico agroforestal); Mesotemperado inferior & 33.34 & 12.56 & 2.66 \\ 
  Vales sublitorais; Agrosistema intensivo (plantacion forestal); Mesotemperado inferior & 15.13 & 14.02 & 1.08 \\ 
  Vales sublitorais; Agrosistema intensivo (mosaico agroforestal); Termotemperado & 12.98 & 3.71 & 3.50 \\ 
  Vales sublitorais; Rururbano (diseminado); Mesotemperado inferior & 7.33 & 1.16 & 6.32 \\ 
  Litoral Cantabro-Atlantico; Agrosistema intensivo (mosaico agroforestal); Termotemperado & 6.04 & 8.49 & 0.71 \\ 
  Vales sublitorais; Rururbano (diseminado); Termotemperado & 3.18 & 0.67 & 4.71 \\ 
  Vales sublitorais; Agrosistema intensivo (plantacion forestal); Termotemperado & 3.06 & 2.13 & 1.44 \\ 
  Vales sublitorais; Agrosistema intensivo (superficie de cultivo); Mesotemperado inferior & 2.85 & 0.43 & 6.62 \\ 
  Litoral Cantabro-Atlantico; Urbano; Termotemperado & 2.73 & 2.18 & 1.25 \\ 
  Vales sublitorais; Agrosistema intensivo (plantacion forestal); Mesotemperado superior & 2.65 & 3.20 & 0.83 \\ 
  Vales sublitorais; Agrosistema intensivo (mosaico agroforestal); Mesotemperado superior & 2.27 & 1.64 & 1.38 \\ 
  Vales sublitorais; Agrosistema intensivo (superficie de cultivo); Termotemperado & 1.49 & 0.27 & 5.57 \\ 
  Serras; Matogueira e rochedo; Mesotemperado superior & 1.40 & 2.07 & 0.68 \\ 
  Serras; Agrosistema intensivo (plantacion forestal); Mesotemperado superior & 1.34 & 4.01 & 0.33 \\ 
   \hline
\end{tabular}
\end{table}
% latex table generated in R 3.2.2 by xtable 1.8-0 package
% Fri Dec  4 16:54:41 2015
\begin{table}[p]
\centering
\caption{Frecuencia de aparición dos Camiños de Santiago (área de influencia de 500 m a ambos lados) e frecuencia de tipos asociados Costa Sur - Baixo Miño} 
\label{vcamino3}
\begin{tabular}{lrrr}
  \hline
Tipo de paisaxe & F.Aparic (\%) & F.Tipo (\%) & Ratio \\ 
  \hline
Litoral Cantabro-Atlantico; Rururbano (diseminado); Termotemperado & 44.74 & 11.41 & 3.92 \\ 
  Vales sublitorais; Rururbano (diseminado); Termotemperado & 22.13 & 12.49 & 1.77 \\ 
  Litoral Cantabro-Atlantico; Urbano; Termotemperado & 12.83 & 1.44 & 8.90 \\ 
  Litoral Cantabro-Atlantico; Bosque; Termotemperado & 8.34 & 0.29 & 28.74 \\ 
  Litoral Cantabro-Atlantico; Agrosistema intensivo (mosaico agroforestal); Termotemperado & 3.14 & 3.27 & 0.96 \\ 
  Litoral Cantabro-Atlantico; Agrosistema extensivo; Termotemperado & 2.51 & 0.39 & 6.46 \\ 
  Vales sublitorais; Agrosistema intensivo (plantacion forestal); Termotemperado & 1.95 & 8.94 & 0.22 \\ 
  Litoral Cantabro-Atlantico; Agrosistema intensivo (plantacion forestal); Termotemperado & 1.25 & 2.51 & 0.50 \\ 
   \hline
\end{tabular}
\end{table}
% latex table generated in R 3.2.2 by xtable 1.8-0 package
% Fri Dec  4 16:54:42 2015
\begin{table}[p]
\centering
\caption{Frecuencia de aparición dos Camiños de Santiago (área de influencia de 500 m a ambos lados) e frecuencia de tipos asociados Ribeiras Encaixadas do Miño e do Sil} 
\label{vcamino4}
\begin{tabular}{lrrr}
  \hline
Tipo de paisaxe & F.Aparic (\%) & F.Tipo (\%) & Ratio \\ 
  \hline
Chairas e vales interiores; Agrosistema intensivo (mosaico agroforestal); Mesotemperado inferior & 20.94 & 4.80 & 4.37 \\ 
  Chairas e vales interiores; Agrosistema extensivo; Mesotemperado inferior & 17.39 & 9.17 & 1.90 \\ 
  Chairas e vales interiores; Urbano; Termotemperado & 15.01 & 0.86 & 17.53 \\ 
  Chairas e vales interiores; Agrosistema extensivo; Termotemperado & 11.29 & 3.71 & 3.04 \\ 
  Chairas e vales interiores; Rururbano (diseminado); Termotemperado & 8.47 & 2.59 & 3.27 \\ 
  Chairas e vales interiores; Rururbano (diseminado); Mesotemperado inferior & 8.05 & 1.23 & 6.56 \\ 
  Chairas e vales interiores; Matogueira e rochedo; Termotemperado & 4.68 & 4.28 & 1.09 \\ 
  Chairas e vales interiores; Agrosistema intensivo (plantacion forestal); Mesotemperado inferior & 3.99 & 1.58 & 2.54 \\ 
  Chairas e vales interiores; Agrosistema intensivo (superficie de cultivo); Mesotemperado inferior & 1.91 & 0.97 & 1.97 \\ 
  Chairas e vales interiores; Matogueira e rochedo; Mesotemperado inferior & 1.19 & 4.17 & 0.29 \\ 
  Chairas e vales interiores; Bosque; Mesotemperado inferior & 1.15 & 2.35 & 0.49 \\ 
   \hline
\end{tabular}
\end{table}
% latex table generated in R 3.2.2 by xtable 1.8-0 package
% Fri Dec  4 16:54:42 2015
\begin{table}[p]
\centering
\caption{Frecuencia de aparición dos Camiños de Santiago (área de influencia de 500 m a ambos lados) e frecuencia de tipos asociados Serras Orientais} 
\label{vcamino5}
\begin{tabular}{lrrr}
  \hline
Tipo de paisaxe & F.Aparic (\%) & F.Tipo (\%) & Ratio \\ 
  \hline
Serras; Agrosistema extensivo; Supra e orotemperado & 36.06 & 15.17 & 2.38 \\ 
  Serras; Agrosistema extensivo; Mesotemperado superior & 17.21 & 11.84 & 1.45 \\ 
  Serras; Matogueira e rochedo; Supra e orotemperado & 11.51 & 14.82 & 0.78 \\ 
  Serras; Agrosistema intensivo (mosaico agroforestal); Supra e orotemperado & 9.14 & 2.44 & 3.74 \\ 
  Serras; Agrosistema intensivo (plantacion forestal); Supra e orotemperado & 5.59 & 3.43 & 1.63 \\ 
  Serras; Bosque; Supra e orotemperado & 4.28 & 5.16 & 0.83 \\ 
  Serras; Bosque; Mesotemperado superior & 3.24 & 3.78 & 0.86 \\ 
  Chairas e vales interiores; Agrosistema extensivo; Mesotemperado superior & 2.67 & 0.46 & 5.79 \\ 
  Serras; Matogueira e rochedo; Mesotemperado superior & 2.27 & 6.90 & 0.33 \\ 
  Serras; Agrosistema intensivo (plantacion forestal); Mesotemperado superior & 1.68 & 2.33 & 0.72 \\ 
  Chairas e vales interiores; Bosque; Mesotemperado superior & 1.38 & 0.24 & 5.66 \\ 
  Vales sublitorais; Agrosistema extensivo; Supra e orotemperado & 1.08 & 0.27 & 4.02 \\ 
   \hline
\end{tabular}
\end{table}
% latex table generated in R 3.2.2 by xtable 1.8-0 package
% Fri Dec  4 16:54:42 2015
\begin{table}[p]
\centering
\caption{Frecuencia de aparición dos Camiños de Santiago (área de influencia de 500 m a ambos lados) e frecuencia de tipos asociados Chairas e Fosas Luguesas} 
\label{vcamino6}
\begin{tabular}{lrrr}
  \hline
Tipo de paisaxe & F.Aparic (\%) & F.Tipo (\%) & Ratio \\ 
  \hline
Chairas e vales interiores; Agrosistema extensivo; Mesotemperado superior & 20.88 & 17.50 & 1.19 \\ 
  Serras; Agrosistema extensivo; Mesotemperado superior & 14.04 & 10.14 & 1.38 \\ 
  Chairas e vales interiores; Agrosistema intensivo (mosaico agroforestal); Mesotemperado superior & 12.18 & 14.58 & 0.84 \\ 
  Serras; Agrosistema intensivo (mosaico agroforestal); Mesotemperado superior & 10.49 & 5.41 & 1.94 \\ 
  Serras; Agrosistema intensivo (superficie de cultivo); Mesotemperado superior & 7.05 & 2.91 & 2.42 \\ 
  Chairas e vales interiores; Agrosistema extensivo; Mesotemperado inferior & 5.67 & 9.57 & 0.59 \\ 
  Chairas e vales interiores; Rururbano (diseminado); Mesotemperado superior & 4.00 & 1.94 & 2.06 \\ 
  Chairas e vales interiores; Rururbano (diseminado); Mesotemperado inferior & 3.00 & 1.41 & 2.12 \\ 
  Chairas e vales interiores; Agrosistema intensivo (superficie de cultivo); Mesotemperado superior & 2.63 & 4.58 & 0.57 \\ 
  Chairas e vales interiores; Agrosistema intensivo (mosaico agroforestal); Mesotemperado inferior & 2.16 & 6.48 & 0.33 \\ 
  Chairas e vales interiores; Matogueira e rochedo; Mesotemperado superior & 2.15 & 1.93 & 1.12 \\ 
  Chairas e vales interiores; Agrosistema intensivo (plantacion forestal); Mesotemperado superior & 1.85 & 2.61 & 0.71 \\ 
  Chairas e vales interiores; Agrosistema intensivo (superficie de cultivo); Mesotemperado inferior & 1.53 & 2.33 & 0.66 \\ 
  Chairas e vales interiores; Urbano; Mesotemperado inferior & 1.48 & 0.29 & 5.09 \\ 
   \hline
\end{tabular}
\end{table}
% latex table generated in R 3.2.2 by xtable 1.8-0 package
% Fri Dec  4 16:54:42 2015
\begin{table}[p]
\centering
\caption{Frecuencia de aparición dos Camiños de Santiago (área de influencia de 500 m a ambos lados) e frecuencia de tipos asociados Galicia Central} 
\label{vcamino7}
\begin{tabular}{lrrr}
  \hline
Tipo de paisaxe & F.Aparic (\%) & F.Tipo (\%) & Ratio \\ 
  \hline
Vales sublitorais; Agrosistema intensivo (mosaico agroforestal); Mesotemperado inferior & 27.78 & 23.79 & 1.17 \\ 
  Vales sublitorais; Agrosistema extensivo; Mesotemperado inferior & 12.31 & 10.39 & 1.18 \\ 
  Vales sublitorais; Rururbano (diseminado); Mesotemperado inferior & 11.22 & 3.66 & 3.07 \\ 
  Vales sublitorais; Rururbano (diseminado); Termotemperado & 6.00 & 1.94 & 3.09 \\ 
  Vales sublitorais; Agrosistema intensivo (mosaico agroforestal); Mesotemperado superior & 4.78 & 4.40 & 1.09 \\ 
  Vales sublitorais; Agrosistema extensivo; Mesotemperado superior & 3.97 & 2.91 & 1.36 \\ 
  Vales sublitorais; Agrosistema intensivo (superficie de cultivo); Mesotemperado inferior & 3.92 & 2.45 & 1.60 \\ 
  Serras; Agrosistema extensivo; Mesotemperado superior & 3.63 & 7.06 & 0.51 \\ 
  Vales sublitorais; Agrosistema intensivo (mosaico agroforestal); Termotemperado & 3.21 & 5.79 & 0.55 \\ 
  Vales sublitorais; Urbano; Mesotemperado inferior & 3.17 & 0.49 & 6.42 \\ 
  Vales sublitorais; Agrosistema intensivo (plantacion forestal); Mesotemperado inferior & 2.09 & 2.72 & 0.77 \\ 
  Serras; Matogueira e rochedo; Mesotemperado superior & 1.89 & 5.00 & 0.38 \\ 
  Vales sublitorais; Matogueira e rochedo; Mesotemperado inferior & 1.63 & 3.12 & 0.52 \\ 
  Serras; Agrosistema extensivo; Mesotemperado inferior & 1.44 & 2.67 & 0.54 \\ 
  Serras; Agrosistema intensivo (mosaico agroforestal); Mesotemperado inferior & 1.29 & 0.77 & 1.66 \\ 
  Serras; Agrosistema intensivo (mosaico agroforestal); Mesotemperado superior & 1.16 & 1.61 & 0.72 \\ 
   \hline
\end{tabular}
\end{table}
% latex table generated in R 3.2.2 by xtable 1.8-0 package
% Fri Dec  4 16:54:42 2015
\begin{table}[p]
\centering
\caption{Frecuencia de aparición dos Camiños de Santiago (área de influencia de 500 m a ambos lados) e frecuencia de tipos asociados Chairas, Fosas e Serras Ourensás} 
\label{vcamino8}
\begin{tabular}{lrrr}
  \hline
Tipo de paisaxe & F.Aparic (\%) & F.Tipo (\%) & Ratio \\ 
  \hline
Chairas e vales interiores; Agrosistema extensivo; Mesotemperado inferior & 23.42 & 12.05 & 1.94 \\ 
  Chairas e vales interiores; Agrosistema intensivo (superficie de cultivo); Mesotemperado inferior & 15.26 & 7.47 & 2.04 \\ 
  Serras; Agrosistema extensivo; Mesotemperado inferior & 11.57 & 9.02 & 1.28 \\ 
  Serras; Matogueira e rochedo; Mesotemperado inferior & 8.77 & 8.98 & 0.98 \\ 
  Chairas e vales interiores; Rururbano (diseminado); Mesotemperado inferior & 6.32 & 1.22 & 5.18 \\ 
  Chairas e vales interiores; Matogueira e rochedo; Mesotemperado inferior & 4.84 & 4.79 & 1.01 \\ 
  Serras; Bosque; Mesotemperado inferior & 4.07 & 2.68 & 1.52 \\ 
  Chairas e vales interiores; Rururbano (diseminado); Termotemperado & 3.58 & 0.75 & 4.79 \\ 
  Chairas e vales interiores; Bosque; Mesotemperado inferior & 3.25 & 2.80 & 1.16 \\ 
  Serras; Matogueira e rochedo; Supra e orotemperado & 2.36 & 8.42 & 0.28 \\ 
  Serras; Agrosistema extensivo; Supra e orotemperado & 2.22 & 3.09 & 0.72 \\ 
  Serras; Matogueira e rochedo; Mesotemperado superior & 1.98 & 9.99 & 0.20 \\ 
  Chairas e vales interiores; Viñedo; Termotemperado & 1.91 & 0.65 & 2.94 \\ 
  Serras; Agrosistema intensivo (superficie de cultivo); Mesotemperado inferior & 1.55 & 1.62 & 0.96 \\ 
  Chairas e vales interiores; Agrosistema intensivo (mosaico agroforestal); Mesotemperado inferior & 1.45 & 1.44 & 1.00 \\ 
  Serras; Agrosistema intensivo (plantacion forestal); Mesotemperado inferior & 1.29 & 2.00 & 0.65 \\ 
   \hline
\end{tabular}
\end{table}
% latex table generated in R 3.2.2 by xtable 1.8-0 package
% Fri Dec  4 16:54:42 2015
\begin{table}[p]
\centering
\caption{Frecuencia de aparición dos Camiños de Santiago (área de influencia de 500 m a ambos lados) e frecuencia de tipos asociados Serras Surorientais} 
\label{vcamino9}
\begin{tabular}{lrrr}
  \hline
Tipo de paisaxe & F.Aparic (\%) & F.Tipo (\%) & Ratio \\ 
  \hline
Serras; Matogueira e rochedo; Supra e orotemperado & 33.13 & 37.48 & 0.88 \\ 
  Serras; Agrosistema extensivo; Supra e orotemperado & 18.84 & 7.54 & 2.50 \\ 
  Serras; Agrosistema extensivo; Mesotemperado inferior & 6.93 & 7.06 & 0.98 \\ 
  Serras; Agrosistema extensivo; Mesotemperado superior & 6.86 & 10.41 & 0.66 \\ 
  Serras; Matogueira e rochedo; Mesotemperado inferior & 6.15 & 4.18 & 1.47 \\ 
  Serras; Matogueira e rochedo; Mesotemperado superior & 5.50 & 9.91 & 0.56 \\ 
  Serras; Agrosistema intensivo (plantacion forestal); Mesotemperado inferior & 3.72 & 0.46 & 8.09 \\ 
  Serras; Agrosistema intensivo (superficie de cultivo); Mesotemperado superior & 3.30 & 1.15 & 2.88 \\ 
  Serras; Agrosistema intensivo (superficie de cultivo); Supra e orotemperado & 2.91 & 0.88 & 3.33 \\ 
  Serras; Agrosistema intensivo (plantacion forestal); Supra e orotemperado & 2.49 & 5.13 & 0.49 \\ 
  Serras; Rururbano (diseminado); Supra e orotemperado & 2.04 & 0.17 & 11.94 \\ 
  Serras; Agrosistema intensivo (mosaico agroforestal); Supra e orotemperado & 1.47 & 0.71 & 2.08 \\ 
  Serras; Agrosistema intensivo (superficie de cultivo); Mesotemperado inferior & 1.38 & 0.47 & 2.90 \\ 
  Serras; Rururbano (diseminado); Mesotemperado superior & 1.36 & 0.12 & 11.13 \\ 
  Serras; Agrosistema intensivo (mosaico agroforestal); Mesotemperado inferior & 1.19 & 0.28 & 4.21 \\ 
   \hline
\end{tabular}
\end{table}
% latex table generated in R 3.2.2 by xtable 1.8-0 package
% Fri Dec  4 16:54:42 2015
\begin{table}[p]
\centering
\caption{Frecuencia de aparición dos Camiños de Santiago (área de influencia de 500 m a ambos lados) e frecuencia de tipos asociados Galicia Setentrional} 
\label{vcamino10}
\begin{tabular}{lrrr}
  \hline
Tipo de paisaxe & F.Aparic (\%) & F.Tipo (\%) & Ratio \\ 
  \hline
\hline
\end{tabular}
\end{table}
% latex table generated in R 3.2.2 by xtable 1.8-0 package
% Fri Dec  4 16:54:42 2015
\begin{table}[p]
\centering
\caption{Frecuencia de aparición dos Camiños de Santiago (área de influencia de 500 m a ambos lados) e frecuencia de tipos asociados Chairas e Fosas Occidentais} 
\label{vcamino11}
\begin{tabular}{lrrr}
  \hline
Tipo de paisaxe & F.Aparic (\%) & F.Tipo (\%) & Ratio \\ 
  \hline
Litoral Cantabro-Atlantico; Agrosistema intensivo (mosaico agroforestal); Termotemperado & 18.01 & 8.86 & 2.03 \\ 
  Vales sublitorais; Agrosistema intensivo (mosaico agroforestal); Mesotemperado inferior & 15.02 & 18.05 & 0.83 \\ 
  Vales sublitorais; Agrosistema intensivo (superficie de cultivo); Mesotemperado inferior & 8.27 & 4.94 & 1.67 \\ 
  Litoral Cantabro-Atlantico; Matogueira e rochedo; Termotemperado & 7.71 & 5.61 & 1.37 \\ 
  Vales sublitorais; Agrosistema intensivo (mosaico agroforestal); Termotemperado & 7.69 & 8.83 & 0.87 \\ 
  Vales sublitorais; Matogueira e rochedo; Mesotemperado inferior & 7.10 & 7.15 & 0.99 \\ 
  Litoral Cantabro-Atlantico; Agrosistema intensivo (plantacion forestal); Termotemperado & 7.02 & 4.31 & 1.63 \\ 
  Vales sublitorais; Agrosistema intensivo (plantacion forestal); Mesotemperado inferior & 6.53 & 8.33 & 0.78 \\ 
  Litoral Cantabro-Atlantico; Rururbano (diseminado); Termotemperado & 4.26 & 2.77 & 1.54 \\ 
  Vales sublitorais; Agrosistema extensivo; Mesotemperado inferior & 2.93 & 4.56 & 0.64 \\ 
  Vales sublitorais; Agrosistema intensivo (plantacion forestal); Termotemperado & 2.58 & 3.89 & 0.66 \\ 
  Vales sublitorais; Matogueira e rochedo; Termotemperado & 2.05 & 1.68 & 1.22 \\ 
  Vales sublitorais; Rururbano (diseminado); Mesotemperado inferior & 1.53 & 1.88 & 0.81 \\ 
  Vales sublitorais; Agrosistema intensivo (superficie de cultivo); Mesotemperado superior & 1.34 & 0.77 & 1.74 \\ 
  Vales sublitorais; Agrosistema extensivo; Mesotemperado superior & 1.13 & 0.72 & 1.57 \\ 
  Litoral Cantabro-Atlantico; Agrosistema intensivo (superficie de cultivo); Termotemperado & 1.08 & 0.93 & 1.17 \\ 
   \hline
\end{tabular}
\end{table}
% latex table generated in R 3.2.2 by xtable 1.8-0 package
% Fri Dec  4 16:54:42 2015
\begin{table}[p]
\centering
\caption{Frecuencia de aparición dos Camiños de Santiago (área de influencia de 500 m a ambos lados) e frecuencia de tipos asociados Rías Baixas} 
\label{vcamino12}
\begin{tabular}{lrrr}
  \hline
Tipo de paisaxe & F.Aparic (\%) & F.Tipo (\%) & Ratio \\ 
  \hline
Litoral Cantabro-Atlantico; Rururbano (diseminado); Termotemperado & 47.63 & 16.40 & 2.90 \\ 
  Vales sublitorais; Agrosistema intensivo (mosaico agroforestal); Termotemperado & 12.24 & 5.62 & 2.18 \\ 
  Litoral Cantabro-Atlantico; Agrosistema intensivo (mosaico agroforestal); Termotemperado & 9.67 & 6.56 & 1.47 \\ 
  Vales sublitorais; Rururbano (diseminado); Termotemperado & 9.62 & 6.06 & 1.59 \\ 
  Litoral Cantabro-Atlantico; Urbano; Termotemperado & 6.67 & 2.21 & 3.02 \\ 
  Vales sublitorais; Agrosistema intensivo (plantacion forestal); Termotemperado & 2.45 & 8.25 & 0.30 \\ 
  Litoral Cantabro-Atlantico; Agrosistema intensivo (plantacion forestal); Termotemperado & 1.90 & 4.94 & 0.38 \\ 
  Vales sublitorais; Matogueira e rochedo; Termotemperado & 1.89 & 5.27 & 0.36 \\ 
  Litoral Cantabro-Atlantico; Viñedo; Termotemperado & 1.71 & 1.24 & 1.37 \\ 
  Vales sublitorais; Agrosistema intensivo (mosaico agroforestal); Mesotemperado inferior & 1.06 & 3.76 & 0.28 \\ 
   \hline
\end{tabular}
\end{table}
