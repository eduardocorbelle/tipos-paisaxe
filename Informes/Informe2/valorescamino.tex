% latex table generated in R 3.2.2 by xtable 1.8-0 package
% Wed Dec  2 19:45:31 2015
\begin{table}[p]
\centering
\caption{Frecuencia de aparición dos Camiños de Santiago (área de influencia de 500 m a ambos lados) e frecuencia de tipos asociados Golfo Ártabro} 
\label{vcamino1}
\begin{tabular}{lrrr}
  \hline
Tipo de paisaxe & F.Aparic (\%) & F.Tipo (\%) & Ratio \\ 
  \hline
Litoral Cantabro-Atlantico; Rururbano (diseminado); Termotemperado & 33.30 & 3.29 & 10.12 \\ 
  Litoral Cantabro-Atlantico; Urbano; Termotemperado & 15.84 & 0.53 & 29.86 \\ 
  Litoral Cantabro-Atlantico; Agrosistema intensivo (mosaico agroforestal); Termotemperado & 13.67 & 2.43 & 5.63 \\ 
  Vales sublitorais; Agrosistema intensivo (mosaico agroforestal); Termotemperado & 8.34 & 2.62 & 3.19 \\ 
  Vales sublitorais; Rururbano (diseminado); Termotemperado & 6.23 & 1.68 & 3.71 \\ 
  Vales sublitorais; Agrosistema intensivo (mosaico agroforestal); Mesotemperado inferior & 4.70 & 7.54 & 0.62 \\ 
  Litoral Cantabro-Atlantico; Agrosistema intensivo (plantacion forestal); Termotemperado & 4.18 & 1.32 & 3.18 \\ 
  Vales sublitorais; Agrosistema intensivo (plantacion forestal); Termotemperado & 2.24 & 1.79 & 1.25 \\ 
  Vales sublitorais; Agrosistema intensivo (plantacion forestal); Mesotemperado inferior & 1.84 & 2.90 & 0.64 \\ 
  Vales sublitorais; Matogueira e rochedo; Mesotemperado superior & 1.56 & 0.85 & 1.84 \\ 
  Litoral Cantabro-Atlantico; Urbano; no data & 1.48 & 0.04 & 33.02 \\ 
   \hline
\end{tabular}
\end{table}
% latex table generated in R 3.2.2 by xtable 1.8-0 package
% Wed Dec  2 19:45:31 2015
\begin{table}[p]
\centering
\caption{Frecuencia de aparición dos Camiños de Santiago (área de influencia de 500 m a ambos lados) e frecuencia de tipos asociados A Mariña - Baixo Eo} 
\label{vcamino2}
\begin{tabular}{lrrr}
  \hline
Tipo de paisaxe & F.Aparic (\%) & F.Tipo (\%) & Ratio \\ 
  \hline
Vales sublitorais; Agrosistema intensivo (mosaico agroforestal); Mesotemperado inferior & 33.34 & 7.54 & 4.42 \\ 
  Vales sublitorais; Agrosistema intensivo (plantacion forestal); Mesotemperado inferior & 15.13 & 2.90 & 5.22 \\ 
  Vales sublitorais; Agrosistema intensivo (mosaico agroforestal); Termotemperado & 12.98 & 2.62 & 4.96 \\ 
  Vales sublitorais; Rururbano (diseminado); Mesotemperado inferior & 7.33 & 0.98 & 7.47 \\ 
  Litoral Cantabro-Atlantico; Agrosistema intensivo (mosaico agroforestal); Termotemperado & 6.04 & 2.43 & 2.49 \\ 
  Vales sublitorais; Rururbano (diseminado); Termotemperado & 3.18 & 1.68 & 1.89 \\ 
  Vales sublitorais; Agrosistema intensivo (plantacion forestal); Termotemperado & 3.06 & 1.79 & 1.71 \\ 
  Vales sublitorais; Agrosistema intensivo (superficie de cultivo); Mesotemperado inferior & 2.85 & 0.81 & 3.52 \\ 
  Litoral Cantabro-Atlantico; Urbano; Termotemperado & 2.73 & 0.53 & 5.15 \\ 
  Vales sublitorais; Agrosistema intensivo (plantacion forestal); Mesotemperado superior & 2.65 & 0.60 & 4.44 \\ 
  Vales sublitorais; Agrosistema intensivo (mosaico agroforestal); Mesotemperado superior & 2.27 & 1.52 & 1.49 \\ 
  Vales sublitorais; Agrosistema intensivo (superficie de cultivo); Termotemperado & 1.49 & 0.10 & 15.24 \\ 
  Serras; Matogueira e rochedo; Mesotemperado superior & 1.40 & 5.03 & 0.28 \\ 
  Serras; Agrosistema intensivo (plantacion forestal); Mesotemperado superior & 1.34 & 1.07 & 1.25 \\ 
   \hline
\end{tabular}
\end{table}
% latex table generated in R 3.2.2 by xtable 1.8-0 package
% Wed Dec  2 19:45:31 2015
\begin{table}[p]
\centering
\caption{Frecuencia de aparición dos Camiños de Santiago (área de influencia de 500 m a ambos lados) e frecuencia de tipos asociados Costa Sur - Baixo Miño} 
\label{vcamino3}
\begin{tabular}{lrrr}
  \hline
Tipo de paisaxe & F.Aparic (\%) & F.Tipo (\%) & Ratio \\ 
  \hline
Litoral Cantabro-Atlantico; Rururbano (diseminado); Termotemperado & 44.74 & 3.29 & 13.60 \\ 
  Vales sublitorais; Rururbano (diseminado); Termotemperado & 22.13 & 1.68 & 13.17 \\ 
  Litoral Cantabro-Atlantico; Urbano; Termotemperado & 12.83 & 0.53 & 24.19 \\ 
  Litoral Cantabro-Atlantico; Bosque; Termotemperado & 8.34 & 0.03 & 253.41 \\ 
  Litoral Cantabro-Atlantico; Agrosistema intensivo (mosaico agroforestal); Termotemperado & 3.14 & 2.43 & 1.29 \\ 
  Litoral Cantabro-Atlantico; Agrosistema extensivo; Termotemperado & 2.51 & 0.16 & 15.94 \\ 
  Vales sublitorais; Agrosistema intensivo (plantacion forestal); Termotemperado & 1.95 & 1.79 & 1.09 \\ 
  Litoral Cantabro-Atlantico; Agrosistema intensivo (plantacion forestal); Termotemperado & 1.25 & 1.32 & 0.95 \\ 
   \hline
\end{tabular}
\end{table}
% latex table generated in R 3.2.2 by xtable 1.8-0 package
% Wed Dec  2 19:45:31 2015
\begin{table}[p]
\centering
\caption{Frecuencia de aparición dos Camiños de Santiago (área de influencia de 500 m a ambos lados) e frecuencia de tipos asociados Ribeiras Encaixadas do Miño e do Sil} 
\label{vcamino4}
\begin{tabular}{lrrr}
  \hline
Tipo de paisaxe & F.Aparic (\%) & F.Tipo (\%) & Ratio \\ 
  \hline
Chairas e vales interiores; Agrosistema intensivo (mosaico agroforestal); Mesotemperado inferior & 20.94 & 1.71 & 12.27 \\ 
  Chairas e vales interiores; Agrosistema extensivo; Mesotemperado inferior & 17.39 & 3.58 & 4.85 \\ 
  Chairas e vales interiores; Urbano; Termotemperado & 15.01 & 0.08 & 183.00 \\ 
  Chairas e vales interiores; Agrosistema extensivo; Termotemperado & 11.29 & 0.61 & 18.42 \\ 
  Chairas e vales interiores; Rururbano (diseminado); Termotemperado & 8.47 & 0.42 & 20.03 \\ 
  Chairas e vales interiores; Rururbano (diseminado); Mesotemperado inferior & 8.05 & 0.51 & 15.74 \\ 
  Chairas e vales interiores; Matogueira e rochedo; Termotemperado & 4.68 & 0.66 & 7.12 \\ 
  Chairas e vales interiores; Agrosistema intensivo (plantacion forestal); Mesotemperado inferior & 3.99 & 0.55 & 7.25 \\ 
  Chairas e vales interiores; Agrosistema intensivo (superficie de cultivo); Mesotemperado inferior & 1.91 & 1.16 & 1.65 \\ 
  Chairas e vales interiores; Matogueira e rochedo; Mesotemperado inferior & 1.19 & 1.38 & 0.86 \\ 
  Chairas e vales interiores; Bosque; Mesotemperado inferior & 1.15 & 0.76 & 1.51 \\ 
   \hline
\end{tabular}
\end{table}
% latex table generated in R 3.2.2 by xtable 1.8-0 package
% Wed Dec  2 19:45:31 2015
\begin{table}[p]
\centering
\caption{Frecuencia de aparición dos Camiños de Santiago (área de influencia de 500 m a ambos lados) e frecuencia de tipos asociados Serras Orientais} 
\label{vcamino5}
\begin{tabular}{lrrr}
  \hline
Tipo de paisaxe & F.Aparic (\%) & F.Tipo (\%) & Ratio \\ 
  \hline
Serras; Agrosistema extensivo; Supra e orotemperado & 36.06 & 2.50 & 14.44 \\ 
  Serras; Agrosistema extensivo; Mesotemperado superior & 17.21 & 5.76 & 2.99 \\ 
  Serras; Matogueira e rochedo; Supra e orotemperado & 11.51 & 6.12 & 1.88 \\ 
  Serras; Agrosistema intensivo (mosaico agroforestal); Supra e orotemperado & 9.14 & 0.37 & 24.60 \\ 
  Serras; Agrosistema intensivo (plantacion forestal); Supra e orotemperado & 5.59 & 0.93 & 6.00 \\ 
  Serras; Bosque; Supra e orotemperado & 4.28 & 0.74 & 5.81 \\ 
  Serras; Bosque; Mesotemperado superior & 3.24 & 1.00 & 3.24 \\ 
  Chairas e vales interiores; Agrosistema extensivo; Mesotemperado superior & 2.67 & 2.77 & 0.96 \\ 
  Serras; Matogueira e rochedo; Mesotemperado superior & 2.27 & 5.03 & 0.45 \\ 
  Serras; Agrosistema intensivo (plantacion forestal); Mesotemperado superior & 1.68 & 1.07 & 1.56 \\ 
  Chairas e vales interiores; Bosque; Mesotemperado superior & 1.38 & 0.26 & 5.23 \\ 
  Vales sublitorais; Agrosistema extensivo; Supra e orotemperado & 1.08 & 0.02 & 47.13 \\ 
   \hline
\end{tabular}
\end{table}
% latex table generated in R 3.2.2 by xtable 1.8-0 package
% Wed Dec  2 19:45:31 2015
\begin{table}[p]
\centering
\caption{Frecuencia de aparición dos Camiños de Santiago (área de influencia de 500 m a ambos lados) e frecuencia de tipos asociados Chairas e Fosas Luguesas} 
\label{vcamino6}
\begin{tabular}{lrrr}
  \hline
Tipo de paisaxe & F.Aparic (\%) & F.Tipo (\%) & Ratio \\ 
  \hline
Chairas e vales interiores; Agrosistema extensivo; Mesotemperado superior & 20.88 & 2.77 & 7.53 \\ 
  Serras; Agrosistema extensivo; Mesotemperado superior & 14.04 & 5.76 & 2.44 \\ 
  Chairas e vales interiores; Agrosistema intensivo (mosaico agroforestal); Mesotemperado superior & 12.18 & 2.27 & 5.37 \\ 
  Serras; Agrosistema intensivo (mosaico agroforestal); Mesotemperado superior & 10.49 & 1.77 & 5.91 \\ 
  Serras; Agrosistema intensivo (superficie de cultivo); Mesotemperado superior & 7.05 & 0.95 & 7.40 \\ 
  Chairas e vales interiores; Agrosistema extensivo; Mesotemperado inferior & 5.67 & 3.58 & 1.58 \\ 
  Chairas e vales interiores; Rururbano (diseminado); Mesotemperado superior & 4.00 & 0.30 & 13.35 \\ 
  Chairas e vales interiores; Rururbano (diseminado); Mesotemperado inferior & 3.00 & 0.51 & 5.85 \\ 
  Chairas e vales interiores; Agrosistema intensivo (superficie de cultivo); Mesotemperado superior & 2.63 & 0.72 & 3.67 \\ 
  Chairas e vales interiores; Agrosistema intensivo (mosaico agroforestal); Mesotemperado inferior & 2.16 & 1.71 & 1.26 \\ 
  Chairas e vales interiores; Matogueira e rochedo; Mesotemperado superior & 2.15 & 0.34 & 6.39 \\ 
  Chairas e vales interiores; Agrosistema intensivo (plantacion forestal); Mesotemperado superior & 1.85 & 0.42 & 4.39 \\ 
  Chairas e vales interiores; Agrosistema intensivo (superficie de cultivo); Mesotemperado inferior & 1.53 & 1.16 & 1.32 \\ 
  Chairas e vales interiores; Urbano; Mesotemperado inferior & 1.48 & 0.06 & 26.55 \\ 
   \hline
\end{tabular}
\end{table}
% latex table generated in R 3.2.2 by xtable 1.8-0 package
% Wed Dec  2 19:45:31 2015
\begin{table}[p]
\centering
\caption{Frecuencia de aparición dos Camiños de Santiago (área de influencia de 500 m a ambos lados) e frecuencia de tipos asociados Galicia Central} 
\label{vcamino7}
\begin{tabular}{lrrr}
  \hline
Tipo de paisaxe & F.Aparic (\%) & F.Tipo (\%) & Ratio \\ 
  \hline
Vales sublitorais; Agrosistema intensivo (mosaico agroforestal); Mesotemperado inferior & 27.78 & 7.54 & 3.68 \\ 
  Vales sublitorais; Agrosistema extensivo; Mesotemperado inferior & 12.31 & 2.73 & 4.50 \\ 
  Vales sublitorais; Rururbano (diseminado); Mesotemperado inferior & 11.22 & 0.98 & 11.43 \\ 
  Vales sublitorais; Rururbano (diseminado); Termotemperado & 6.00 & 1.68 & 3.57 \\ 
  Vales sublitorais; Agrosistema intensivo (mosaico agroforestal); Mesotemperado superior & 4.78 & 1.52 & 3.15 \\ 
  Vales sublitorais; Agrosistema extensivo; Mesotemperado superior & 3.97 & 1.21 & 3.29 \\ 
  Vales sublitorais; Agrosistema intensivo (superficie de cultivo); Mesotemperado inferior & 3.92 & 0.81 & 4.84 \\ 
  Serras; Agrosistema extensivo; Mesotemperado superior & 3.63 & 5.76 & 0.63 \\ 
  Vales sublitorais; Agrosistema intensivo (mosaico agroforestal); Termotemperado & 3.21 & 2.62 & 1.22 \\ 
  Vales sublitorais; Urbano; Mesotemperado inferior & 3.17 & 0.11 & 29.21 \\ 
  Vales sublitorais; Agrosistema intensivo (plantacion forestal); Mesotemperado inferior & 2.09 & 2.90 & 0.72 \\ 
  Serras; Matogueira e rochedo; Mesotemperado superior & 1.89 & 5.03 & 0.38 \\ 
  Vales sublitorais; Matogueira e rochedo; Mesotemperado inferior & 1.63 & 1.89 & 0.86 \\ 
  Serras; Agrosistema extensivo; Mesotemperado inferior & 1.44 & 2.57 & 0.56 \\ 
  Serras; Agrosistema intensivo (mosaico agroforestal); Mesotemperado inferior & 1.29 & 0.50 & 2.59 \\ 
  Serras; Agrosistema intensivo (mosaico agroforestal); Mesotemperado superior & 1.16 & 1.77 & 0.65 \\ 
   \hline
\end{tabular}
\end{table}
% latex table generated in R 3.2.2 by xtable 1.8-0 package
% Wed Dec  2 19:45:32 2015
\begin{table}[p]
\centering
\caption{Frecuencia de aparición dos Camiños de Santiago (área de influencia de 500 m a ambos lados) e frecuencia de tipos asociados Chairas, Fosas e Serras Ourensás} 
\label{vcamino8}
\begin{tabular}{lrrr}
  \hline
Tipo de paisaxe & F.Aparic (\%) & F.Tipo (\%) & Ratio \\ 
  \hline
Chairas e vales interiores; Agrosistema extensivo; Mesotemperado inferior & 23.42 & 3.58 & 6.54 \\ 
  Chairas e vales interiores; Agrosistema intensivo (superficie de cultivo); Mesotemperado inferior & 15.26 & 1.16 & 13.18 \\ 
  Serras; Agrosistema extensivo; Mesotemperado inferior & 11.57 & 2.57 & 4.51 \\ 
  Serras; Matogueira e rochedo; Mesotemperado inferior & 8.77 & 2.72 & 3.22 \\ 
  Chairas e vales interiores; Rururbano (diseminado); Mesotemperado inferior & 6.32 & 0.51 & 12.35 \\ 
  Chairas e vales interiores; Matogueira e rochedo; Mesotemperado inferior & 4.84 & 1.38 & 3.51 \\ 
  Serras; Bosque; Mesotemperado inferior & 4.07 & 0.69 & 5.89 \\ 
  Chairas e vales interiores; Rururbano (diseminado); Termotemperado & 3.58 & 0.42 & 8.47 \\ 
  Chairas e vales interiores; Bosque; Mesotemperado inferior & 3.25 & 0.76 & 4.25 \\ 
  Serras; Matogueira e rochedo; Supra e orotemperado & 2.36 & 6.12 & 0.39 \\ 
  Serras; Agrosistema extensivo; Supra e orotemperado & 2.22 & 2.50 & 0.89 \\ 
  Serras; Matogueira e rochedo; Mesotemperado superior & 1.98 & 5.03 & 0.39 \\ 
  Chairas e vales interiores; Viñedo; Termotemperado & 1.91 & 0.28 & 6.88 \\ 
  Serras; Agrosistema intensivo (superficie de cultivo); Mesotemperado inferior & 1.55 & 0.28 & 5.47 \\ 
  Chairas e vales interiores; Agrosistema intensivo (mosaico agroforestal); Mesotemperado inferior & 1.45 & 1.71 & 0.85 \\ 
  Serras; Agrosistema intensivo (plantacion forestal); Mesotemperado inferior & 1.29 & 0.82 & 1.57 \\ 
   \hline
\end{tabular}
\end{table}
% latex table generated in R 3.2.2 by xtable 1.8-0 package
% Wed Dec  2 19:45:32 2015
\begin{table}[p]
\centering
\caption{Frecuencia de aparición dos Camiños de Santiago (área de influencia de 500 m a ambos lados) e frecuencia de tipos asociados Serras Surorientais} 
\label{vcamino9}
\begin{tabular}{lrrr}
  \hline
Tipo de paisaxe & F.Aparic (\%) & F.Tipo (\%) & Ratio \\ 
  \hline
Serras; Matogueira e rochedo; Supra e orotemperado & 33.13 & 6.12 & 5.42 \\ 
  Serras; Agrosistema extensivo; Supra e orotemperado & 18.84 & 2.50 & 7.54 \\ 
  Serras; Agrosistema extensivo; Mesotemperado inferior & 6.93 & 2.57 & 2.70 \\ 
  Serras; Agrosistema extensivo; Mesotemperado superior & 6.86 & 5.76 & 1.19 \\ 
  Serras; Matogueira e rochedo; Mesotemperado inferior & 6.15 & 2.72 & 2.26 \\ 
  Serras; Matogueira e rochedo; Mesotemperado superior & 5.50 & 5.03 & 1.10 \\ 
  Serras; Agrosistema intensivo (plantacion forestal); Mesotemperado inferior & 3.72 & 0.82 & 4.52 \\ 
  Serras; Agrosistema intensivo (superficie de cultivo); Mesotemperado superior & 3.30 & 0.95 & 3.46 \\ 
  Serras; Agrosistema intensivo (superficie de cultivo); Supra e orotemperado & 2.91 & 0.23 & 12.61 \\ 
  Serras; Agrosistema intensivo (plantacion forestal); Supra e orotemperado & 2.49 & 0.93 & 2.67 \\ 
  Serras; Rururbano (diseminado); Supra e orotemperado & 2.04 & 0.02 & 85.45 \\ 
  Serras; Agrosistema intensivo (mosaico agroforestal); Supra e orotemperado & 1.47 & 0.37 & 3.95 \\ 
  Serras; Agrosistema intensivo (superficie de cultivo); Mesotemperado inferior & 1.38 & 0.28 & 4.85 \\ 
  Serras; Rururbano (diseminado); Mesotemperado superior & 1.36 & 0.11 & 12.26 \\ 
  Serras; Agrosistema intensivo (mosaico agroforestal); Mesotemperado inferior & 1.19 & 0.50 & 2.40 \\ 
   \hline
\end{tabular}
\end{table}
% latex table generated in R 3.2.2 by xtable 1.8-0 package
% Wed Dec  2 19:45:32 2015
\begin{table}[p]
\centering
\caption{Frecuencia de aparición dos Camiños de Santiago (área de influencia de 500 m a ambos lados) e frecuencia de tipos asociados Galicia Setentrional} 
\label{vcamino10}
\begin{tabular}{lrrr}
  \hline
Tipo de paisaxe & F.Aparic (\%) & F.Tipo (\%) & Ratio \\ 
  \hline
\hline
\end{tabular}
\end{table}
% latex table generated in R 3.2.2 by xtable 1.8-0 package
% Wed Dec  2 19:45:32 2015
\begin{table}[p]
\centering
\caption{Frecuencia de aparición dos Camiños de Santiago (área de influencia de 500 m a ambos lados) e frecuencia de tipos asociados Chairas e Fosas Occidentais} 
\label{vcamino11}
\begin{tabular}{lrrr}
  \hline
Tipo de paisaxe & F.Aparic (\%) & F.Tipo (\%) & Ratio \\ 
  \hline
Litoral Cantabro-Atlantico; Agrosistema intensivo (mosaico agroforestal); Termotemperado & 18.01 & 2.43 & 7.42 \\ 
  Vales sublitorais; Agrosistema intensivo (mosaico agroforestal); Mesotemperado inferior & 15.02 & 7.54 & 1.99 \\ 
  Vales sublitorais; Agrosistema intensivo (superficie de cultivo); Mesotemperado inferior & 8.27 & 0.81 & 10.21 \\ 
  Litoral Cantabro-Atlantico; Matogueira e rochedo; Termotemperado & 7.71 & 0.83 & 9.28 \\ 
  Vales sublitorais; Agrosistema intensivo (mosaico agroforestal); Termotemperado & 7.69 & 2.62 & 2.94 \\ 
  Vales sublitorais; Matogueira e rochedo; Mesotemperado inferior & 7.10 & 1.89 & 3.75 \\ 
  Litoral Cantabro-Atlantico; Agrosistema intensivo (plantacion forestal); Termotemperado & 7.02 & 1.32 & 5.34 \\ 
  Vales sublitorais; Agrosistema intensivo (plantacion forestal); Mesotemperado inferior & 6.53 & 2.90 & 2.25 \\ 
  Litoral Cantabro-Atlantico; Rururbano (diseminado); Termotemperado & 4.26 & 3.29 & 1.29 \\ 
  Vales sublitorais; Agrosistema extensivo; Mesotemperado inferior & 2.93 & 2.73 & 1.07 \\ 
  Vales sublitorais; Agrosistema intensivo (plantacion forestal); Termotemperado & 2.58 & 1.79 & 1.44 \\ 
  Vales sublitorais; Matogueira e rochedo; Termotemperado & 2.05 & 0.93 & 2.20 \\ 
  Vales sublitorais; Rururbano (diseminado); Mesotemperado inferior & 1.53 & 0.98 & 1.56 \\ 
  Vales sublitorais; Agrosistema intensivo (superficie de cultivo); Mesotemperado superior & 1.34 & 0.18 & 7.50 \\ 
  Vales sublitorais; Agrosistema extensivo; Mesotemperado superior & 1.13 & 1.21 & 0.93 \\ 
  Litoral Cantabro-Atlantico; Agrosistema intensivo (superficie de cultivo); Termotemperado & 1.08 & 0.22 & 4.93 \\ 
   \hline
\end{tabular}
\end{table}
% latex table generated in R 3.2.2 by xtable 1.8-0 package
% Wed Dec  2 19:45:32 2015
\begin{table}[p]
\centering
\caption{Frecuencia de aparición dos Camiños de Santiago (área de influencia de 500 m a ambos lados) e frecuencia de tipos asociados Rías Baixas} 
\label{vcamino12}
\begin{tabular}{lrrr}
  \hline
Tipo de paisaxe & F.Aparic (\%) & F.Tipo (\%) & Ratio \\ 
  \hline
Litoral Cantabro-Atlantico; Rururbano (diseminado); Termotemperado & 47.63 & 3.29 & 14.48 \\ 
  Vales sublitorais; Agrosistema intensivo (mosaico agroforestal); Termotemperado & 12.24 & 2.62 & 4.67 \\ 
  Litoral Cantabro-Atlantico; Agrosistema intensivo (mosaico agroforestal); Termotemperado & 9.67 & 2.43 & 3.98 \\ 
  Vales sublitorais; Rururbano (diseminado); Termotemperado & 9.62 & 1.68 & 5.72 \\ 
  Litoral Cantabro-Atlantico; Urbano; Termotemperado & 6.67 & 0.53 & 12.57 \\ 
  Vales sublitorais; Agrosistema intensivo (plantacion forestal); Termotemperado & 2.45 & 1.79 & 1.37 \\ 
  Litoral Cantabro-Atlantico; Agrosistema intensivo (plantacion forestal); Termotemperado & 1.90 & 1.32 & 1.44 \\ 
  Vales sublitorais; Matogueira e rochedo; Termotemperado & 1.89 & 0.93 & 2.03 \\ 
  Litoral Cantabro-Atlantico; Viñedo; Termotemperado & 1.71 & 0.14 & 12.33 \\ 
  Vales sublitorais; Agrosistema intensivo (mosaico agroforestal); Mesotemperado inferior & 1.06 & 7.54 & 0.14 \\ 
   \hline
\end{tabular}
\end{table}
