% latex table generated in R 3.2.2 by xtable 1.8-0 package
% Wed Dec  2 19:31:35 2015
\begin{table}[p]
\centering
\caption{Frecuencia de aparición dos Camiños de Santiago (área de influencia de 500 m a ambos lados) e frecuencia de tipos asociados Golfo Ártabro} 
\label{vcamino1}
\begin{tabular}{lrr}
  \hline
V4 & porcent & porcentT \\ 
  \hline
Litoral Cantabro-Atlantico; Rururbano (diseminado); Termotemperado & 33.30 & 3.29 \\ 
  Litoral Cantabro-Atlantico; Urbano; Termotemperado & 15.84 & 0.53 \\ 
  Litoral Cantabro-Atlantico; Agrosistema intensivo (mosaico agroforestal); Termotemperado & 13.67 & 2.43 \\ 
  Vales sublitorais; Agrosistema intensivo (mosaico agroforestal); Termotemperado & 8.34 & 2.62 \\ 
  Vales sublitorais; Rururbano (diseminado); Termotemperado & 6.23 & 1.68 \\ 
  Vales sublitorais; Agrosistema intensivo (mosaico agroforestal); Mesotemperado inferior & 4.70 & 7.54 \\ 
  Litoral Cantabro-Atlantico; Agrosistema intensivo (plantacion forestal); Termotemperado & 4.18 & 1.32 \\ 
  Vales sublitorais; Agrosistema intensivo (plantacion forestal); Termotemperado & 2.24 & 1.79 \\ 
  Vales sublitorais; Agrosistema intensivo (plantacion forestal); Mesotemperado inferior & 1.84 & 2.90 \\ 
  Vales sublitorais; Matogueira e rochedo; Mesotemperado superior & 1.56 & 0.85 \\ 
  Litoral Cantabro-Atlantico; Urbano; no data & 1.48 & 0.04 \\ 
   \hline
\end{tabular}
\end{table}
% latex table generated in R 3.2.2 by xtable 1.8-0 package
% Wed Dec  2 19:32:09 2015
\begin{table}[p]
\centering
\caption{Frecuencia de aparición dos Camiños de Santiago (área de influencia de 500 m a ambos lados) e frecuencia de tipos asociados Golfo Ártabro} 
\label{vcamino1}
\begin{tabular}{lrrr}
  \hline
Tipo de paisaxe & F.Aparic (\%) & F.Tipo (\%) & Ratio \\ 
  \hline
Litoral Cantabro-Atlantico; Rururbano (diseminado); Termotemperado & 33.30 & 3.29 & 10.12 \\ 
  Litoral Cantabro-Atlantico; Urbano; Termotemperado & 15.84 & 0.53 & 29.86 \\ 
  Litoral Cantabro-Atlantico; Agrosistema intensivo (mosaico agroforestal); Termotemperado & 13.67 & 2.43 & 5.63 \\ 
  Vales sublitorais; Agrosistema intensivo (mosaico agroforestal); Termotemperado & 8.34 & 2.62 & 3.19 \\ 
  Vales sublitorais; Rururbano (diseminado); Termotemperado & 6.23 & 1.68 & 3.71 \\ 
  Vales sublitorais; Agrosistema intensivo (mosaico agroforestal); Mesotemperado inferior & 4.70 & 7.54 & 0.62 \\ 
  Litoral Cantabro-Atlantico; Agrosistema intensivo (plantacion forestal); Termotemperado & 4.18 & 1.32 & 3.18 \\ 
  Vales sublitorais; Agrosistema intensivo (plantacion forestal); Termotemperado & 2.24 & 1.79 & 1.25 \\ 
  Vales sublitorais; Agrosistema intensivo (plantacion forestal); Mesotemperado inferior & 1.84 & 2.90 & 0.64 \\ 
  Vales sublitorais; Matogueira e rochedo; Mesotemperado superior & 1.56 & 0.85 & 1.84 \\ 
  Litoral Cantabro-Atlantico; Urbano; no data & 1.48 & 0.04 & 33.02 \\ 
   \hline
\end{tabular}
\end{table}
