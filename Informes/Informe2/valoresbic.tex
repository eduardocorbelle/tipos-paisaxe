% latex table generated in R 3.2.2 by xtable 1.8-0 package
% Wed Dec  2 19:45:31 2015
\begin{table}[p]
\centering
\caption{Frecuencia de aparición de BIC e frecuencia de tipos asociados, GAP Golfo Ártabro} 
\label{vbic1}
\begin{tabular}{lrrr}
  \hline
Tipo de paisaxe & F.Aparic (\%) & F.Tipo (\%) & Ratio \\ 
  \hline
Litoral Cantabro-Atlantico; Rururbano (diseminado); Termotemperado & 29.63 & 3.29 & 9.00 \\ 
  Litoral Cantabro-Atlantico; Conxunto Historico; Termotemperado & 18.52 & 0.02 & 979.20 \\ 
  Litoral Cantabro-Atlantico; Conxunto Historico; no data & 11.11 & 0.00 & 4212.35 \\ 
  Litoral Cantabro-Atlantico; Agrosistema intensivo (mosaico agroforestal); Termotemperado & 5.56 & 2.43 & 2.29 \\ 
  Litoral Cantabro-Atlantico; Urbano; Termotemperado & 5.56 & 0.53 & 10.47 \\ 
  Litoral Cantabro-Atlantico; Agrosistema intensivo (mosaico agroforestal); no data & 3.70 & 0.04 & 87.10 \\ 
  Litoral Cantabro-Atlantico; Agrosistema intensivo (plantacion forestal); Termotemperado & 3.70 & 1.32 & 2.81 \\ 
   & 3.70 &  &  \\ 
  Canons; Bosque; Mesotemperado inferior & 1.85 & 0.30 & 6.27 \\ 
  Litoral Cantabro-Atlantico; Agrosistema intensivo (mosaico agroforestal); Mesotemperado inferior & 1.85 & 0.29 & 6.32 \\ 
  Litoral Cantabro-Atlantico; Matogueira e rochedo; no data & 1.85 & 0.09 & 21.76 \\ 
  Litoral Cantabro-Atlantico; Urbano; no data & 1.85 & 0.04 & 41.36 \\ 
  Serras; Agrosistema intensivo (mosaico agroforestal); Mesotemperado superior & 1.85 & 1.77 & 1.04 \\ 
  Vales sublitorais; Agrosistema intensivo (mosaico agroforestal); Mesotemperado inferior & 1.85 & 7.54 & 0.25 \\ 
  Vales sublitorais; Agrosistema intensivo (plantacion forestal); Termotemperado & 1.85 & 1.79 & 1.03 \\ 
  Vales sublitorais; Matogueira e rochedo; Termotemperado & 1.85 & 0.93 & 1.99 \\ 
  Vales sublitorais; Rururbano (diseminado); Mesotemperado inferior & 1.85 & 0.98 & 1.89 \\ 
  Vales sublitorais; Rururbano (diseminado); Termotemperado & 1.85 & 1.68 & 1.10 \\ 
   \hline
\end{tabular}
\end{table}
% latex table generated in R 3.2.2 by xtable 1.8-0 package
% Wed Dec  2 19:45:31 2015
\begin{table}[p]
\centering
\caption{Frecuencia de aparición de BIC e frecuencia de tipos asociados, GAP A Mariña - Baixo Eo} 
\label{vbic2}
\begin{tabular}{lrrr}
  \hline
Tipo de paisaxe & F.Aparic (\%) & F.Tipo (\%) & Ratio \\ 
  \hline
Litoral Cantabro-Atlantico; Conxunto Historico; Termotemperado & 31.25 & 0.02 & 1652.40 \\ 
  Litoral Cantabro-Atlantico; Agrosistema intensivo (mosaico agroforestal); Termotemperado & 12.50 & 2.43 & 5.15 \\ 
  Vales sublitorais; Agrosistema intensivo (superficie de cultivo); Termotemperado & 12.50 & 0.10 & 127.54 \\ 
  Vales sublitorais; Conxunto Historico; Mesotemperado inferior & 12.50 & 0.00 & 3642.39 \\ 
  Litoral Cantabro-Atlantico; Agrosistema intensivo (mosaico agroforestal); Mesotemperado inferior & 6.25 & 0.29 & 21.33 \\ 
  Litoral Cantabro-Atlantico; Matogueira e rochedo; Mesotemperado inferior & 6.25 & 0.10 & 62.79 \\ 
  Litoral Cantabro-Atlantico; Rururbano (diseminado); Termotemperado & 6.25 & 3.29 & 1.90 \\ 
  Vales sublitorais; Agrosistema intensivo (plantacion forestal); Mesotemperado inferior & 6.25 & 2.90 & 2.16 \\ 
  Vales sublitorais; Rururbano (diseminado); Termotemperado & 6.25 & 1.68 & 3.72 \\ 
   \hline
\end{tabular}
\end{table}
% latex table generated in R 3.2.2 by xtable 1.8-0 package
% Wed Dec  2 19:45:31 2015
\begin{table}[p]
\centering
\caption{Frecuencia de aparición de BIC e frecuencia de tipos asociados, GAP Costa Sur - Baixo Miño} 
\label{vbic3}
\begin{tabular}{lrrr}
  \hline
Tipo de paisaxe & F.Aparic (\%) & F.Tipo (\%) & Ratio \\ 
  \hline
Vales sublitorais; Matogueira e rochedo; Mesotemperado inferior & 38.46 & 1.89 & 20.31 \\ 
  Litoral Cantabro-Atlantico; Rururbano (diseminado); Termotemperado & 9.23 & 3.29 & 2.81 \\ 
  Vales sublitorais; Rururbano (diseminado); Termotemperado & 9.23 & 1.68 & 5.49 \\ 
  Vales sublitorais; Matogueira e rochedo; Termotemperado & 6.15 & 0.93 & 6.60 \\ 
  Litoral Cantabro-Atlantico; Conxunto Historico; Termotemperado & 4.62 & 0.02 & 244.05 \\ 
  Litoral Cantabro-Atlantico; Urbano; no data & 4.62 & 0.04 & 103.07 \\ 
  Serras; Conxunto Historico; Termotemperado & 4.62 & 0.01 & 576.94 \\ 
  Litoral Cantabro-Atlantico; Agrosistema intensivo (plantacion forestal); Termotemperado & 3.08 & 1.32 & 2.34 \\ 
  Litoral Cantabro-Atlantico; Urbano; Termotemperado & 3.08 & 0.53 & 5.80 \\ 
  Serras; Agrosistema intensivo (plantacion forestal); Mesotemperado inferior & 3.08 & 0.82 & 3.74 \\ 
  Serras; Matogueira e rochedo; Mesotemperado inferior & 3.08 & 2.73 & 1.13 \\ 
  Litoral Cantabro-Atlantico; Agrosistema intensivo (mosaico agroforestal); Termotemperado & 1.54 & 2.43 & 0.63 \\ 
  no data; Agrosistema intensivo (mosaico agroforestal); Termotemperado & 1.54 & 0.02 & 86.84 \\ 
  no data; Rururbano (diseminado); Termotemperado & 1.54 & 0.05 & 33.98 \\ 
  Serras; Agrosistema intensivo (plantacion forestal); Termotemperado & 1.54 & 0.15 & 10.09 \\ 
  Vales sublitorais; Agrosistema intensivo (mosaico agroforestal); Termotemperado & 1.54 & 2.62 & 0.59 \\ 
  Vales sublitorais; Agrosistema intensivo (plantacion forestal); Termotemperado & 1.54 & 1.79 & 0.86 \\ 
  Vales sublitorais; Urbano; Termotemperado & 1.54 & 0.07 & 21.25 \\ 
   \hline
\end{tabular}
\end{table}
% latex table generated in R 3.2.2 by xtable 1.8-0 package
% Wed Dec  2 19:45:31 2015
\begin{table}[p]
\centering
\caption{Frecuencia de aparición de BIC e frecuencia de tipos asociados, GAP Ribeiras Encaixadas do Miño e do Sil} 
\label{vbic4}
\begin{tabular}{lrrr}
  \hline
Tipo de paisaxe & F.Aparic (\%) & F.Tipo (\%) & Ratio \\ 
  \hline
Chairas e vales interiores; Conxunto Historico; Termotemperado & 20.41 & 0.00 & 7027.99 \\ 
  Chairas e vales interiores; Agrosistema extensivo; Mesotemperado inferior & 14.29 & 3.58 & 3.99 \\ 
  Canons; Bosque; Mesotemperado inferior & 10.20 & 0.30 & 34.54 \\ 
  Chairas e vales interiores; Rururbano (diseminado); Mesotemperado inferior & 8.16 & 0.51 & 15.95 \\ 
  Chairas e vales interiores; Agrosistema intensivo (mosaico agroforestal); Termotemperado & 6.12 & 0.34 & 18.04 \\ 
  Chairas e vales interiores; Urbano; Termotemperado & 6.12 & 0.08 & 74.59 \\ 
  Chairas e vales interiores; Agrosistema extensivo; Termotemperado & 4.08 & 0.61 & 6.66 \\ 
  Chairas e vales interiores; Agrosistema intensivo (mosaico agroforestal); Mesotemperado inferior & 4.08 & 1.71 & 2.39 \\ 
  Serras; Conxunto Historico; Mesotemperado inferior & 4.08 & 0.00 & 1966.12 \\ 
  Canons; Agrosistema intensivo (mosaico agroforestal); Mesotemperado inferior & 2.04 & 0.04 & 48.50 \\ 
  Canons; Agrosistema intensivo (plantacion forestal); Termotemperado & 2.04 & 0.07 & 28.45 \\ 
  Canons; Bosque; Termotemperado & 2.04 & 0.15 & 13.48 \\ 
  Canons; Matogueira e rochedo; Mesomediterráneo & 2.04 & 0.18 & 11.05 \\ 
  Canons; Matogueira e rochedo; Mesotemperado inferior & 2.04 & 0.20 & 10.19 \\ 
  Canons; Rururbano (diseminado); Mesotemperado inferior & 2.04 & 0.00 & 437.46 \\ 
  Chairas e vales interiores; Conxunto Historico; Mesotemperado inferior & 2.04 & 0.00 & 1479.86 \\ 
  Chairas e vales interiores; Rururbano (diseminado); Mesomediterráneo & 2.04 & 0.05 & 43.84 \\ 
  Chairas e vales interiores; Rururbano (diseminado); Termotemperado & 2.04 & 0.42 & 4.82 \\ 
  Chairas e vales interiores; Viñedo; Termotemperado & 2.04 & 0.28 & 7.35 \\ 
  Serras; Matogueira e rochedo; Mesotemperado inferior & 2.04 & 2.73 & 0.75 \\ 
   \hline
\end{tabular}
\end{table}
% latex table generated in R 3.2.2 by xtable 1.8-0 package
% Wed Dec  2 19:45:31 2015
\begin{table}[p]
\centering
\caption{Frecuencia de aparición de BIC e frecuencia de tipos asociados, GAP Serras Orientais} 
\label{vbic5}
\begin{tabular}{lrrr}
  \hline
Tipo de paisaxe & F.Aparic (\%) & F.Tipo (\%) & Ratio \\ 
  \hline
Serras; Agrosistema extensivo; Mesotemperado superior & 40.00 & 5.76 & 6.95 \\ 
  Chairas e vales interiores; Agrosistema extensivo; Mesotemperado superior & 10.00 & 2.77 & 3.60 \\ 
  Serras; Agrosistema extensivo; Supra e orotemperado & 10.00 & 2.50 & 4.00 \\ 
  Serras; Bosque; Supra e orotemperado & 10.00 & 0.74 & 13.58 \\ 
  Vales sublitorais; Bosque; Mesotemperado superior & 10.00 & 0.39 & 25.91 \\ 
  Chairas e vales interiores; Agrosistema extensivo; Mesomediterráneo & 5.00 & 0.11 & 45.59 \\ 
  Serras; Agrosistema intensivo (mosaico agroforestal); Mesotemperado superior & 5.00 & 1.77 & 2.82 \\ 
  Serras; Matogueira e rochedo; Mesotemperado superior & 5.00 & 5.03 & 0.99 \\ 
  Vales sublitorais; Bosque; Mesotemperado inferior & 5.00 & 0.49 & 10.10 \\ 
   \hline
\end{tabular}
\end{table}
% latex table generated in R 3.2.2 by xtable 1.8-0 package
% Wed Dec  2 19:45:31 2015
\begin{table}[p]
\centering
\caption{Frecuencia de aparición de BIC e frecuencia de tipos asociados, GAP Chairas e Fosas Luguesas} 
\label{vbic6}
\begin{tabular}{lrrr}
  \hline
Tipo de paisaxe & F.Aparic (\%) & F.Tipo (\%) & Ratio \\ 
  \hline
Chairas e vales interiores; Agrosistema extensivo; Mesotemperado inferior & 21.82 & 3.58 & 6.09 \\ 
  Chairas e vales interiores; Agrosistema extensivo; Mesotemperado superior & 10.91 & 2.77 & 3.93 \\ 
  Chairas e vales interiores; Agrosistema intensivo (mosaico agroforestal); Mesotemperado superior & 10.91 & 2.27 & 4.81 \\ 
  Chairas e vales interiores; Conxunto Historico; Mesotemperado superior & 10.91 & 0.00 & 9876.81 \\ 
  Chairas e vales interiores; Rururbano (diseminado); Mesotemperado superior & 7.27 & 0.30 & 24.26 \\ 
  Chairas e vales interiores; Rururbano (diseminado); Mesotemperado inferior & 5.45 & 0.51 & 10.65 \\ 
  Chairas e vales interiores; Agrosistema intensivo (mosaico agroforestal); Mesotemperado inferior & 3.64 & 1.71 & 2.13 \\ 
  Chairas e vales interiores; Conxunto Historico; Mesotemperado inferior & 3.64 & 0.00 & 2636.84 \\ 
  Chairas e vales interiores; Conxunto Historico; Termotemperado & 3.64 & 0.00 & 1252.26 \\ 
  Chairas e vales interiores; Urbano; Mesotemperado inferior & 3.64 & 0.06 & 65.24 \\ 
  Serras; Agrosistema extensivo; Mesotemperado superior & 3.64 & 5.76 & 0.63 \\ 
  Chairas e vales interiores; Agrosistema extensivo; Termotemperado & 1.82 & 0.61 & 2.96 \\ 
  Chairas e vales interiores; Agrosistema intensivo (plantacion forestal); Mesotemperado inferior & 1.82 & 0.55 & 3.30 \\ 
  Chairas e vales interiores; Agrosistema intensivo (plantacion forestal); Mesotemperado superior & 1.82 & 0.42 & 4.31 \\ 
  Chairas e vales interiores; Agrosistema intensivo (superficie de cultivo); Mesotemperado superior & 1.82 & 0.72 & 2.54 \\ 
  Chairas e vales interiores; Bosque; Mesotemperado inferior & 1.82 & 0.77 & 2.38 \\ 
  Serras; Agrosistema extensivo; Mesotemperado inferior & 1.82 & 2.57 & 0.71 \\ 
  Serras; Agrosistema intensivo (superficie de cultivo); Mesotemperado superior & 1.82 & 0.95 & 1.91 \\ 
  Serras; Rururbano (diseminado); Mesotemperado superior & 1.82 & 0.11 & 16.37 \\ 
   \hline
\end{tabular}
\end{table}
% latex table generated in R 3.2.2 by xtable 1.8-0 package
% Wed Dec  2 19:45:31 2015
\begin{table}[p]
\centering
\caption{Frecuencia de aparición de BIC e frecuencia de tipos asociados, GAP Galicia Central} 
\label{vbic7}
\begin{tabular}{lrrr}
  \hline
Tipo de paisaxe & F.Aparic (\%) & F.Tipo (\%) & Ratio \\ 
  \hline
Vales sublitorais; Conxunto Historico; Mesotemperado inferior & 16.95 & 0.00 & 4938.83 \\ 
  Vales sublitorais; Agrosistema extensivo; Mesotemperado inferior & 15.25 & 2.73 & 5.58 \\ 
  Vales sublitorais; Agrosistema intensivo (mosaico agroforestal); Mesotemperado inferior & 15.25 & 7.54 & 2.02 \\ 
  Vales sublitorais; Rururbano (diseminado); Mesotemperado inferior & 8.47 & 0.98 & 8.63 \\ 
  Serras; Agrosistema extensivo; Mesotemperado superior & 6.78 & 5.76 & 1.18 \\ 
  Vales sublitorais; Matogueira e rochedo; Mesotemperado inferior & 5.08 & 1.89 & 2.69 \\ 
  Serras; Agrosistema extensivo; Mesotemperado inferior & 3.39 & 2.57 & 1.32 \\ 
  Serras; Matogueira e rochedo; Supra e orotemperado & 3.39 & 6.12 & 0.55 \\ 
  Vales sublitorais; Agrosistema intensivo (mosaico agroforestal); Termotemperado & 3.39 & 2.62 & 1.29 \\ 
  Vales sublitorais; Agrosistema intensivo (superficie de cultivo); Mesotemperado inferior & 3.39 & 0.81 & 4.18 \\ 
  Chairas e vales interiores; Agrosistema extensivo; Termotemperado & 1.69 & 0.61 & 2.76 \\ 
  Chairas e vales interiores; Agrosistema intensivo (mosaico agroforestal); Mesotemperado inferior & 1.69 & 1.71 & 0.99 \\ 
  Chairas e vales interiores; Bosque; Termotemperado & 1.69 & 0.29 & 5.81 \\ 
  Chairas e vales interiores; Urbano; Termotemperado & 1.69 & 0.08 & 20.65 \\ 
  Serras; Agrosistema intensivo (mosaico agroforestal); Mesotemperado inferior & 1.69 & 0.50 & 3.41 \\ 
  Serras; Matogueira e rochedo; Mesotemperado superior & 1.69 & 5.03 & 0.34 \\ 
  Vales sublitorais; Agrosistema extensivo; Mesotemperado superior & 1.69 & 1.21 & 1.40 \\ 
  Vales sublitorais; Agrosistema extensivo; Termotemperado & 1.69 & 0.36 & 4.73 \\ 
  Vales sublitorais; Bosque; Mesotemperado inferior & 1.69 & 0.49 & 3.43 \\ 
  Vales sublitorais; Rururbano (diseminado); Termotemperado & 1.69 & 1.68 & 1.01 \\ 
  Vales sublitorais; Urbano; Mesotemperado inferior & 1.69 & 0.11 & 15.61 \\ 
   \hline
\end{tabular}
\end{table}
% latex table generated in R 3.2.2 by xtable 1.8-0 package
% Wed Dec  2 19:45:32 2015
\begin{table}[p]
\centering
\caption{Frecuencia de aparición de BIC e frecuencia de tipos asociados, GAP Chairas, Fosas e Serras Ourensás} 
\label{vbic8}
\begin{tabular}{lrrr}
  \hline
Tipo de paisaxe & F.Aparic (\%) & F.Tipo (\%) & Ratio \\ 
  \hline
Chairas e vales interiores; Agrosistema extensivo; Mesotemperado inferior & 36.00 & 3.58 & 10.04 \\ 
  Chairas e vales interiores; Rururbano (diseminado); Mesotemperado inferior & 16.00 & 0.51 & 31.25 \\ 
  Chairas e vales interiores; Agrosistema intensivo (superficie de cultivo); Mesotemperado inferior & 8.00 & 1.16 & 6.91 \\ 
  Chairas e vales interiores; Conxunto Historico; Mesotemperado inferior & 8.00 & 0.00 & 5801.05 \\ 
  Serras; Agrosistema extensivo; Mesotemperado inferior & 8.00 & 2.57 & 3.11 \\ 
  Chairas e vales interiores; Agrosistema extensivo; Termotemperado & 4.00 & 0.61 & 6.52 \\ 
  Chairas e vales interiores; Urbano; Mesotemperado inferior & 4.00 & 0.06 & 71.76 \\ 
  Serras; Agrosistema extensivo; Mesotemperado superior & 4.00 & 5.76 & 0.69 \\ 
  Serras; Agrosistema extensivo; Supra e orotemperado & 4.00 & 2.50 & 1.60 \\ 
  Serras; Matogueira e rochedo; no data & 4.00 & 0.07 & 54.26 \\ 
  Serras; Matogueira e rochedo; Supra e orotemperado & 4.00 & 6.12 & 0.65 \\ 
   \hline
\end{tabular}
\end{table}
% latex table generated in R 3.2.2 by xtable 1.8-0 package
% Wed Dec  2 19:45:32 2015
\begin{table}[p]
\centering
\caption{Frecuencia de aparición de BIC e frecuencia de tipos asociados, GAP Serras Surorientais} 
\label{vbic9}
\begin{tabular}{lrrr}
  \hline
Tipo de paisaxe & F.Aparic (\%) & F.Tipo (\%) & Ratio \\ 
  \hline
Canons; Agrosistema extensivo; Mesomediterráneo & 25.00 & 0.03 & 724.60 \\ 
  Canons; Matogueira e rochedo; Mesomediterráneo & 25.00 & 0.18 & 135.32 \\ 
  Canons; Rururbano (diseminado); Mesotemperado inferior & 25.00 & 0.00 & 5358.88 \\ 
  Serras; Agrosistema extensivo; Mesotemperado inferior & 25.00 & 2.57 & 9.73 \\ 
   \hline
\end{tabular}
\end{table}
% latex table generated in R 3.2.2 by xtable 1.8-0 package
% Wed Dec  2 19:45:32 2015
\begin{table}[p]
\centering
\caption{Frecuencia de aparición de BIC e frecuencia de tipos asociados, GAP Galicia Setentrional} 
\label{vbic10}
\begin{tabular}{lrrr}
  \hline
Tipo de paisaxe & F.Aparic (\%) & F.Tipo (\%) & Ratio \\ 
  \hline
Litoral Cantabro-Atlantico; Rururbano (diseminado); Termotemperado & 41.67 & 3.29 & 12.66 \\ 
  Litoral Cantabro-Atlantico; Agrosistema intensivo (mosaico agroforestal); Termotemperado & 16.67 & 2.43 & 6.86 \\ 
  Litoral Cantabro-Atlantico; Conxunto Historico; Termotemperado & 8.33 & 0.02 & 440.64 \\ 
  Litoral Cantabro-Atlantico; Rururbano (diseminado); no data & 8.33 & 0.10 & 83.53 \\ 
  Serras; Agrosistema intensivo (mosaico agroforestal); Mesotemperado superior & 8.33 & 1.77 & 4.70 \\ 
  Vales sublitorais; Agrosistema intensivo (mosaico agroforestal); Mesotemperado inferior & 8.33 & 7.54 & 1.10 \\ 
  Vales sublitorais; Rururbano (diseminado); Mesotemperado inferior & 8.33 & 0.98 & 8.49 \\ 
   \hline
\end{tabular}
\end{table}
% latex table generated in R 3.2.2 by xtable 1.8-0 package
% Wed Dec  2 19:45:32 2015
\begin{table}[p]
\centering
\caption{Frecuencia de aparición de BIC e frecuencia de tipos asociados, GAP Chairas e Fosas Occidentais} 
\label{vbic11}
\begin{tabular}{lrrr}
  \hline
Tipo de paisaxe & F.Aparic (\%) & F.Tipo (\%) & Ratio \\ 
  \hline
Vales sublitorais; Agrosistema intensivo (mosaico agroforestal); Mesotemperado inferior & 34.62 & 7.54 & 4.59 \\ 
  Litoral Cantabro-Atlantico; Agrosistema intensivo (mosaico agroforestal); Termotemperado & 11.54 & 2.43 & 4.75 \\ 
  Litoral Cantabro-Atlantico; Conxunto Historico; Termotemperado & 11.54 & 0.02 & 610.12 \\ 
  Vales sublitorais; Agrosistema intensivo (plantacion forestal); Mesotemperado inferior & 7.69 & 2.90 & 2.65 \\ 
  Litoral Cantabro-Atlantico; Agrosistema extensivo; no data & 3.85 & 0.01 & 353.97 \\ 
  Litoral Cantabro-Atlantico; Agrosistema intensivo (mosaico agroforestal); Mesotemperado inferior & 3.85 & 0.29 & 13.13 \\ 
  Litoral Cantabro-Atlantico; Agrosistema intensivo (superficie de cultivo); Termotemperado & 3.85 & 0.22 & 17.53 \\ 
  Litoral Cantabro-Atlantico; Rururbano (diseminado); Termotemperado & 3.85 & 3.29 & 1.17 \\ 
  Litoral Cantabro-Atlantico; Urbano; Termotemperado & 3.85 & 0.53 & 7.25 \\ 
  Vales sublitorais; Agrosistema extensivo; Mesotemperado inferior & 3.85 & 2.73 & 1.41 \\ 
  Vales sublitorais; Agrosistema intensivo (superficie de cultivo); Termotemperado & 3.85 & 0.10 & 39.24 \\ 
  Vales sublitorais; Rururbano (diseminado); Mesotemperado inferior & 3.85 & 0.98 & 3.92 \\ 
  Vales sublitorais; Rururbano (diseminado); Termotemperado & 3.85 & 1.68 & 2.29 \\ 
   \hline
\end{tabular}
\end{table}
% latex table generated in R 3.2.2 by xtable 1.8-0 package
% Wed Dec  2 19:45:32 2015
\begin{table}[p]
\centering
\caption{Frecuencia de aparición de BIC e frecuencia de tipos asociados, GAP Rías Baixas} 
\label{vbic12}
\begin{tabular}{lrrr}
  \hline
Tipo de paisaxe & F.Aparic (\%) & F.Tipo (\%) & Ratio \\ 
  \hline
Litoral Cantabro-Atlantico; Rururbano (diseminado); Termotemperado & 16.09 & 3.29 & 4.89 \\ 
  Litoral Cantabro-Atlantico; Conxunto Historico; Termotemperado & 10.92 & 0.02 & 577.39 \\ 
  Vales sublitorais; Matogueira e rochedo; Termotemperado & 10.34 & 0.93 & 11.09 \\ 
  Vales sublitorais; Rururbano (diseminado); Termotemperado & 10.34 & 1.68 & 6.15 \\ 
  Vales sublitorais; Agrosistema intensivo (plantacion forestal); Termotemperado & 6.90 & 1.79 & 3.85 \\ 
  Litoral Cantabro-Atlantico; Urbano; Termotemperado & 5.75 & 0.53 & 10.83 \\ 
  Vales sublitorais; Agrosistema intensivo (mosaico agroforestal); Mesotemperado inferior & 5.17 & 7.54 & 0.69 \\ 
  Vales sublitorais; Agrosistema intensivo (mosaico agroforestal); Termotemperado & 4.60 & 2.62 & 1.76 \\ 
  Vales sublitorais; Matogueira e rochedo; Mesotemperado inferior & 3.45 & 1.89 & 1.82 \\ 
  Litoral Cantabro-Atlantico; Urbano; no data & 2.87 & 0.04 & 64.17 \\ 
  Vales sublitorais; Agrosistema intensivo (plantacion forestal); Mesotemperado inferior & 2.87 & 2.90 & 0.99 \\ 
  Litoral Cantabro-Atlantico; Matogueira e rochedo; Termotemperado & 2.30 & 0.83 & 2.77 \\ 
  Serras; Matogueira e rochedo; Mesotemperado inferior & 2.30 & 2.73 & 0.84 \\ 
  Litoral Cantabro-Atlantico; Agrosistema intensivo (mosaico agroforestal); Termotemperado & 1.72 & 2.43 & 0.71 \\ 
  Litoral Cantabro-Atlantico; Conxunto Historico; no data & 1.72 & 0.00 & 653.64 \\ 
  Litoral Cantabro-Atlantico; Rururbano (diseminado); no data & 1.72 & 0.10 & 17.28 \\ 
  Serras; Agrosistema intensivo (plantacion forestal); Mesotemperado inferior & 1.72 & 0.82 & 2.09 \\ 
  Serras; Matogueira e rochedo; Mesotemperado superior & 1.72 & 5.03 & 0.34 \\ 
  Litoral Cantabro-Atlantico; Matogueira e rochedo; no data & 1.15 & 0.09 & 13.51 \\ 
  Serras; Agrosistema extensivo; Mesotemperado inferior & 1.15 & 2.57 & 0.45 \\ 
  Vales sublitorais; Urbano; Termotemperado & 1.15 & 0.07 & 15.88 \\ 
   \hline
\end{tabular}
\end{table}
