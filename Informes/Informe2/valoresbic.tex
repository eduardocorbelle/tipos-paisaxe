% latex table generated in R 3.2.2 by xtable 1.8-0 package
% Wed Dec  2 19:29:14 2015
\begin{table}[p]
\centering
\caption{Frecuencia de aparición de BIC e frecuencia de tipos asociados, GAP Golfo Ártabro} 
\label{vbic1}
\begin{tabular}{lrr}
  \hline
V4 & porcent & porcentT \\ 
  \hline
Litoral Cantabro-Atlantico; Rururbano (diseminado); Termotemperado & 29.63 & 3.29 \\ 
  Litoral Cantabro-Atlantico; Conxunto Historico; Termotemperado & 18.52 & 0.02 \\ 
  Litoral Cantabro-Atlantico; Conxunto Historico; no data & 11.11 & 0.00 \\ 
  Litoral Cantabro-Atlantico; Agrosistema intensivo (mosaico agroforestal); Termotemperado & 5.56 & 2.43 \\ 
  Litoral Cantabro-Atlantico; Urbano; Termotemperado & 5.56 & 0.53 \\ 
   \hline
\end{tabular}
\end{table}
% latex table generated in R 3.2.2 by xtable 1.8-0 package
% Wed Dec  2 19:31:35 2015
\begin{table}[p]
\centering
\caption{Frecuencia de aparición de BIC e frecuencia de tipos asociados, GAP Golfo Ártabro} 
\label{vbic1}
\begin{tabular}{lrr}
  \hline
V4 & porcent & porcentT \\ 
  \hline
Litoral Cantabro-Atlantico; Rururbano (diseminado); Termotemperado & 29.63 & 3.29 \\ 
  Litoral Cantabro-Atlantico; Conxunto Historico; Termotemperado & 18.52 & 0.02 \\ 
  Litoral Cantabro-Atlantico; Conxunto Historico; no data & 11.11 & 0.00 \\ 
  Litoral Cantabro-Atlantico; Agrosistema intensivo (mosaico agroforestal); Termotemperado & 5.56 & 2.43 \\ 
  Litoral Cantabro-Atlantico; Urbano; Termotemperado & 5.56 & 0.53 \\ 
  Litoral Cantabro-Atlantico; Agrosistema intensivo (mosaico agroforestal); no data & 3.70 & 0.04 \\ 
  Litoral Cantabro-Atlantico; Agrosistema intensivo (plantacion forestal); Termotemperado & 3.70 & 1.32 \\ 
   & 3.70 &  \\ 
  Canons; Bosque; Mesotemperado inferior & 1.85 & 0.30 \\ 
  Litoral Cantabro-Atlantico; Agrosistema intensivo (mosaico agroforestal); Mesotemperado inferior & 1.85 & 0.29 \\ 
  Litoral Cantabro-Atlantico; Matogueira e rochedo; no data & 1.85 & 0.09 \\ 
  Litoral Cantabro-Atlantico; Urbano; no data & 1.85 & 0.04 \\ 
  Serras; Agrosistema intensivo (mosaico agroforestal); Mesotemperado superior & 1.85 & 1.77 \\ 
  Vales sublitorais; Agrosistema intensivo (mosaico agroforestal); Mesotemperado inferior & 1.85 & 7.54 \\ 
  Vales sublitorais; Agrosistema intensivo (plantacion forestal); Termotemperado & 1.85 & 1.79 \\ 
  Vales sublitorais; Matogueira e rochedo; Termotemperado & 1.85 & 0.93 \\ 
  Vales sublitorais; Rururbano (diseminado); Mesotemperado inferior & 1.85 & 0.98 \\ 
  Vales sublitorais; Rururbano (diseminado); Termotemperado & 1.85 & 1.68 \\ 
   \hline
\end{tabular}
\end{table}
% latex table generated in R 3.2.2 by xtable 1.8-0 package
% Wed Dec  2 19:32:09 2015
\begin{table}[p]
\centering
\caption{Frecuencia de aparición de BIC e frecuencia de tipos asociados, GAP Golfo Ártabro} 
\label{vbic1}
\begin{tabular}{lrrr}
  \hline
Tipo de paisaxe & F.Aparic (\%) & F.Tipo (\%) & Ratio \\ 
  \hline
Litoral Cantabro-Atlantico; Rururbano (diseminado); Termotemperado & 29.63 & 3.29 & 9.00 \\ 
  Litoral Cantabro-Atlantico; Conxunto Historico; Termotemperado & 18.52 & 0.02 & 979.20 \\ 
  Litoral Cantabro-Atlantico; Conxunto Historico; no data & 11.11 & 0.00 & 4212.35 \\ 
  Litoral Cantabro-Atlantico; Agrosistema intensivo (mosaico agroforestal); Termotemperado & 5.56 & 2.43 & 2.29 \\ 
  Litoral Cantabro-Atlantico; Urbano; Termotemperado & 5.56 & 0.53 & 10.47 \\ 
  Litoral Cantabro-Atlantico; Agrosistema intensivo (mosaico agroforestal); no data & 3.70 & 0.04 & 87.10 \\ 
  Litoral Cantabro-Atlantico; Agrosistema intensivo (plantacion forestal); Termotemperado & 3.70 & 1.32 & 2.81 \\ 
   & 3.70 &  &  \\ 
  Canons; Bosque; Mesotemperado inferior & 1.85 & 0.30 & 6.27 \\ 
  Litoral Cantabro-Atlantico; Agrosistema intensivo (mosaico agroforestal); Mesotemperado inferior & 1.85 & 0.29 & 6.32 \\ 
  Litoral Cantabro-Atlantico; Matogueira e rochedo; no data & 1.85 & 0.09 & 21.76 \\ 
  Litoral Cantabro-Atlantico; Urbano; no data & 1.85 & 0.04 & 41.36 \\ 
  Serras; Agrosistema intensivo (mosaico agroforestal); Mesotemperado superior & 1.85 & 1.77 & 1.04 \\ 
  Vales sublitorais; Agrosistema intensivo (mosaico agroforestal); Mesotemperado inferior & 1.85 & 7.54 & 0.25 \\ 
  Vales sublitorais; Agrosistema intensivo (plantacion forestal); Termotemperado & 1.85 & 1.79 & 1.03 \\ 
  Vales sublitorais; Matogueira e rochedo; Termotemperado & 1.85 & 0.93 & 1.99 \\ 
  Vales sublitorais; Rururbano (diseminado); Mesotemperado inferior & 1.85 & 0.98 & 1.89 \\ 
  Vales sublitorais; Rururbano (diseminado); Termotemperado & 1.85 & 1.68 & 1.10 \\ 
   \hline
\end{tabular}
\end{table}
