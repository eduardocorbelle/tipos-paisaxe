% latex table generated in R 3.2.2 by xtable 1.8-0 package
% Fri Dec  4 16:54:41 2015
\begin{table}[p]
\centering
\caption{Frecuencia de aparición de BIC e frecuencia de tipos asociados, GAP Golfo Ártabro} 
\label{vbic1}
\begin{tabular}{lrrr}
  \hline
Tipo de paisaxe & F.Aparic (\%) & F.Tipo (\%) & Ratio \\ 
  \hline
Litoral Cantabro-Atlantico; Rururbano (diseminado); Termotemperado & 29.63 & 19.32 & 1.53 \\ 
  Litoral Cantabro-Atlantico; Conxunto Historico; Termotemperado & 18.52 & 0.07 & 260.11 \\ 
  Litoral Cantabro-Atlantico; Conxunto Historico; no data & 11.11 & 0.02 & 602.15 \\ 
  Litoral Cantabro-Atlantico; Agrosistema intensivo (mosaico agroforestal); Termotemperado & 5.56 & 10.45 & 0.53 \\ 
  Litoral Cantabro-Atlantico; Urbano; Termotemperado & 5.56 & 3.10 & 1.79 \\ 
  Litoral Cantabro-Atlantico; Agrosistema intensivo (mosaico agroforestal); no data & 3.70 & 0.06 & 63.10 \\ 
  Litoral Cantabro-Atlantico; Agrosistema intensivo (plantacion forestal); Termotemperado & 3.70 & 2.83 & 1.31 \\ 
   & 3.70 &  &  \\ 
  Canons; Bosque; Mesotemperado inferior & 1.85 & 1.04 & 1.77 \\ 
  Litoral Cantabro-Atlantico; Agrosistema intensivo (mosaico agroforestal); Mesotemperado inferior & 1.85 & 0.89 & 2.09 \\ 
  Litoral Cantabro-Atlantico; Matogueira e rochedo; no data & 1.85 & 0.10 & 18.50 \\ 
  Litoral Cantabro-Atlantico; Urbano; no data & 1.85 & 0.21 & 9.02 \\ 
  Serras; Agrosistema intensivo (mosaico agroforestal); Mesotemperado superior & 1.85 & 1.64 & 1.13 \\ 
  Vales sublitorais; Agrosistema intensivo (mosaico agroforestal); Mesotemperado inferior & 1.85 & 14.40 & 0.13 \\ 
  Vales sublitorais; Agrosistema intensivo (plantacion forestal); Termotemperado & 1.85 & 1.65 & 1.12 \\ 
  Vales sublitorais; Matogueira e rochedo; Termotemperado & 1.85 & 0.30 & 6.08 \\ 
  Vales sublitorais; Rururbano (diseminado); Mesotemperado inferior & 1.85 & 1.52 & 1.22 \\ 
  Vales sublitorais; Rururbano (diseminado); Termotemperado & 1.85 & 2.11 & 0.88 \\ 
   \hline
\end{tabular}
\end{table}
% latex table generated in R 3.2.2 by xtable 1.8-0 package
% Fri Dec  4 16:54:41 2015
\begin{table}[p]
\centering
\caption{Frecuencia de aparición de BIC e frecuencia de tipos asociados, GAP A Mariña - Baixo Eo} 
\label{vbic2}
\begin{tabular}{lrrr}
  \hline
Tipo de paisaxe & F.Aparic (\%) & F.Tipo (\%) & Ratio \\ 
  \hline
Litoral Cantabro-Atlantico; Conxunto Historico; Termotemperado & 31.25 & 0.07 & 470.76 \\ 
  Litoral Cantabro-Atlantico; Agrosistema intensivo (mosaico agroforestal); Termotemperado & 12.50 & 8.49 & 1.47 \\ 
  Vales sublitorais; Agrosistema intensivo (superficie de cultivo); Termotemperado & 12.50 & 0.27 & 46.60 \\ 
  Vales sublitorais; Conxunto Historico; Mesotemperado inferior & 12.50 & 0.01 & 1044.71 \\ 
  Litoral Cantabro-Atlantico; Agrosistema intensivo (mosaico agroforestal); Mesotemperado inferior & 6.25 & 4.26 & 1.47 \\ 
  Litoral Cantabro-Atlantico; Matogueira e rochedo; Mesotemperado inferior & 6.25 & 0.09 & 71.29 \\ 
  Litoral Cantabro-Atlantico; Rururbano (diseminado); Termotemperado & 6.25 & 3.92 & 1.59 \\ 
  Vales sublitorais; Agrosistema intensivo (plantacion forestal); Mesotemperado inferior & 6.25 & 14.02 & 0.45 \\ 
  Vales sublitorais; Rururbano (diseminado); Termotemperado & 6.25 & 0.67 & 9.27 \\ 
   \hline
\end{tabular}
\end{table}
% latex table generated in R 3.2.2 by xtable 1.8-0 package
% Fri Dec  4 16:54:41 2015
\begin{table}[p]
\centering
\caption{Frecuencia de aparición de BIC e frecuencia de tipos asociados, GAP Costa Sur - Baixo Miño} 
\label{vbic3}
\begin{tabular}{lrrr}
  \hline
Tipo de paisaxe & F.Aparic (\%) & F.Tipo (\%) & Ratio \\ 
  \hline
Vales sublitorais; Matogueira e rochedo; Mesotemperado inferior & 38.46 & 0.86 & 44.53 \\ 
  Litoral Cantabro-Atlantico; Rururbano (diseminado); Termotemperado & 9.23 & 11.41 & 0.81 \\ 
  Vales sublitorais; Rururbano (diseminado); Termotemperado & 9.23 & 12.49 & 0.74 \\ 
  Vales sublitorais; Matogueira e rochedo; Termotemperado & 6.15 & 3.42 & 1.80 \\ 
  Litoral Cantabro-Atlantico; Conxunto Historico; Termotemperado & 4.62 & 0.02 & 200.39 \\ 
  Litoral Cantabro-Atlantico; Urbano; no data & 4.62 & 0.05 & 86.50 \\ 
  Serras; Conxunto Historico; Termotemperado & 4.62 & 0.20 & 23.06 \\ 
  Litoral Cantabro-Atlantico; Agrosistema intensivo (plantacion forestal); Termotemperado & 3.08 & 2.51 & 1.23 \\ 
  Litoral Cantabro-Atlantico; Urbano; Termotemperado & 3.08 & 1.44 & 2.13 \\ 
  Serras; Agrosistema intensivo (plantacion forestal); Mesotemperado inferior & 3.08 & 2.04 & 1.51 \\ 
  Serras; Matogueira e rochedo; Mesotemperado inferior & 3.08 & 5.06 & 0.61 \\ 
  Litoral Cantabro-Atlantico; Agrosistema intensivo (mosaico agroforestal); Termotemperado & 1.54 & 3.27 & 0.47 \\ 
  no data; Agrosistema intensivo (mosaico agroforestal); Termotemperado & 1.54 & 0.44 & 3.48 \\ 
  no data; Rururbano (diseminado); Termotemperado & 1.54 & 1.13 & 1.36 \\ 
  Serras; Agrosistema intensivo (plantacion forestal); Termotemperado & 1.54 & 2.79 & 0.55 \\ 
  Vales sublitorais; Agrosistema intensivo (mosaico agroforestal); Termotemperado & 1.54 & 2.92 & 0.53 \\ 
  Vales sublitorais; Agrosistema intensivo (plantacion forestal); Termotemperado & 1.54 & 8.94 & 0.17 \\ 
  Vales sublitorais; Urbano; Termotemperado & 1.54 & 0.33 & 4.61 \\ 
   \hline
\end{tabular}
\end{table}
% latex table generated in R 3.2.2 by xtable 1.8-0 package
% Fri Dec  4 16:54:42 2015
\begin{table}[p]
\centering
\caption{Frecuencia de aparición de BIC e frecuencia de tipos asociados, GAP Ribeiras Encaixadas do Miño e do Sil} 
\label{vbic4}
\begin{tabular}{lrrr}
  \hline
Tipo de paisaxe & F.Aparic (\%) & F.Tipo (\%) & Ratio \\ 
  \hline
Chairas e vales interiores; Conxunto Historico; Termotemperado & 20.41 & 0.01 & 1503.02 \\ 
  Chairas e vales interiores; Agrosistema extensivo; Mesotemperado inferior & 14.29 & 9.17 & 1.56 \\ 
  Canons; Bosque; Mesotemperado inferior & 10.20 & 1.90 & 5.37 \\ 
  Chairas e vales interiores; Rururbano (diseminado); Mesotemperado inferior & 8.16 & 1.23 & 6.65 \\ 
  Chairas e vales interiores; Agrosistema intensivo (mosaico agroforestal); Termotemperado & 6.12 & 1.55 & 3.95 \\ 
  Chairas e vales interiores; Urbano; Termotemperado & 6.12 & 0.86 & 7.15 \\ 
  Chairas e vales interiores; Agrosistema extensivo; Termotemperado & 4.08 & 3.71 & 1.10 \\ 
  Chairas e vales interiores; Agrosistema intensivo (mosaico agroforestal); Mesotemperado inferior & 4.08 & 4.80 & 0.85 \\ 
  Serras; Conxunto Historico; Mesotemperado inferior & 4.08 & 0.02 & 164.22 \\ 
  Canons; Agrosistema intensivo (mosaico agroforestal); Mesotemperado inferior & 2.04 & 0.35 & 5.90 \\ 
  Canons; Agrosistema intensivo (plantacion forestal); Termotemperado & 2.04 & 0.67 & 3.02 \\ 
  Canons; Bosque; Termotemperado & 2.04 & 1.40 & 1.45 \\ 
  Canons; Matogueira e rochedo; Mesomediterráneo & 2.04 & 1.77 & 1.15 \\ 
  Canons; Matogueira e rochedo; Mesotemperado inferior & 2.04 & 1.29 & 1.58 \\ 
  Canons; Rururbano (diseminado); Mesotemperado inferior & 2.04 & 0.01 & 170.64 \\ 
  Chairas e vales interiores; Conxunto Historico; Mesotemperado inferior & 2.04 & 0.00 & 815.28 \\ 
  Chairas e vales interiores; Rururbano (diseminado); Mesomediterráneo & 2.04 & 0.56 & 3.66 \\ 
  Chairas e vales interiores; Rururbano (diseminado); Termotemperado & 2.04 & 2.59 & 0.79 \\ 
  Chairas e vales interiores; Viñedo; Termotemperado & 2.04 & 2.21 & 0.92 \\ 
  Serras; Matogueira e rochedo; Mesotemperado inferior & 2.04 & 5.11 & 0.40 \\ 
   \hline
\end{tabular}
\end{table}
% latex table generated in R 3.2.2 by xtable 1.8-0 package
% Fri Dec  4 16:54:42 2015
\begin{table}[p]
\centering
\caption{Frecuencia de aparición de BIC e frecuencia de tipos asociados, GAP Serras Orientais} 
\label{vbic5}
\begin{tabular}{lrrr}
  \hline
Tipo de paisaxe & F.Aparic (\%) & F.Tipo (\%) & Ratio \\ 
  \hline
Serras; Agrosistema extensivo; Mesotemperado superior & 40.00 & 11.84 & 3.38 \\ 
  Chairas e vales interiores; Agrosistema extensivo; Mesotemperado superior & 10.00 & 0.46 & 21.69 \\ 
  Serras; Agrosistema extensivo; Supra e orotemperado & 10.00 & 15.17 & 0.66 \\ 
  Serras; Bosque; Supra e orotemperado & 10.00 & 5.16 & 1.94 \\ 
  Vales sublitorais; Bosque; Mesotemperado superior & 10.00 & 3.48 & 2.87 \\ 
  Chairas e vales interiores; Agrosistema extensivo; Mesomediterráneo & 5.00 & 0.07 & 68.31 \\ 
  Serras; Agrosistema intensivo (mosaico agroforestal); Mesotemperado superior & 5.00 & 1.96 & 2.55 \\ 
  Serras; Matogueira e rochedo; Mesotemperado superior & 5.00 & 6.90 & 0.72 \\ 
  Vales sublitorais; Bosque; Mesotemperado inferior & 5.00 & 2.82 & 1.77 \\ 
   \hline
\end{tabular}
\end{table}
% latex table generated in R 3.2.2 by xtable 1.8-0 package
% Fri Dec  4 16:54:42 2015
\begin{table}[p]
\centering
\caption{Frecuencia de aparición de BIC e frecuencia de tipos asociados, GAP Chairas e Fosas Luguesas} 
\label{vbic6}
\begin{tabular}{lrrr}
  \hline
Tipo de paisaxe & F.Aparic (\%) & F.Tipo (\%) & Ratio \\ 
  \hline
Chairas e vales interiores; Agrosistema extensivo; Mesotemperado inferior & 21.82 & 9.57 & 2.28 \\ 
  Chairas e vales interiores; Agrosistema extensivo; Mesotemperado superior & 10.91 & 17.50 & 0.62 \\ 
  Chairas e vales interiores; Agrosistema intensivo (mosaico agroforestal); Mesotemperado superior & 10.91 & 14.58 & 0.75 \\ 
  Chairas e vales interiores; Conxunto Historico; Mesotemperado superior & 10.91 & 0.01 & 1521.63 \\ 
  Chairas e vales interiores; Rururbano (diseminado); Mesotemperado superior & 7.27 & 1.94 & 3.74 \\ 
  Chairas e vales interiores; Rururbano (diseminado); Mesotemperado inferior & 5.45 & 1.41 & 3.86 \\ 
  Chairas e vales interiores; Agrosistema intensivo (mosaico agroforestal); Mesotemperado inferior & 3.64 & 6.48 & 0.56 \\ 
  Chairas e vales interiores; Conxunto Historico; Mesotemperado inferior & 3.64 & 0.00 & 2852.38 \\ 
  Chairas e vales interiores; Conxunto Historico; Termotemperado & 3.64 & 0.01 & 316.55 \\ 
  Chairas e vales interiores; Urbano; Mesotemperado inferior & 3.64 & 0.29 & 12.51 \\ 
  Serras; Agrosistema extensivo; Mesotemperado superior & 3.64 & 10.14 & 0.36 \\ 
  Chairas e vales interiores; Agrosistema extensivo; Termotemperado & 1.82 & 0.48 & 3.77 \\ 
  Chairas e vales interiores; Agrosistema intensivo (plantacion forestal); Mesotemperado inferior & 1.82 & 0.90 & 2.02 \\ 
  Chairas e vales interiores; Agrosistema intensivo (plantacion forestal); Mesotemperado superior & 1.82 & 2.61 & 0.70 \\ 
  Chairas e vales interiores; Agrosistema intensivo (superficie de cultivo); Mesotemperado superior & 1.82 & 4.58 & 0.40 \\ 
  Chairas e vales interiores; Bosque; Mesotemperado inferior & 1.82 & 1.29 & 1.41 \\ 
  Serras; Agrosistema extensivo; Mesotemperado inferior & 1.82 & 0.65 & 2.78 \\ 
  Serras; Agrosistema intensivo (superficie de cultivo); Mesotemperado superior & 1.82 & 2.91 & 0.62 \\ 
  Serras; Rururbano (diseminado); Mesotemperado superior & 1.82 & 0.24 & 7.47 \\ 
   \hline
\end{tabular}
\end{table}
% latex table generated in R 3.2.2 by xtable 1.8-0 package
% Fri Dec  4 16:54:42 2015
\begin{table}[p]
\centering
\caption{Frecuencia de aparición de BIC e frecuencia de tipos asociados, GAP Galicia Central} 
\label{vbic7}
\begin{tabular}{lrrr}
  \hline
Tipo de paisaxe & F.Aparic (\%) & F.Tipo (\%) & Ratio \\ 
  \hline
Vales sublitorais; Conxunto Historico; Mesotemperado inferior & 16.95 & 0.02 & 964.92 \\ 
  Vales sublitorais; Agrosistema extensivo; Mesotemperado inferior & 15.25 & 10.39 & 1.47 \\ 
  Vales sublitorais; Agrosistema intensivo (mosaico agroforestal); Mesotemperado inferior & 15.25 & 23.79 & 0.64 \\ 
  Vales sublitorais; Rururbano (diseminado); Mesotemperado inferior & 8.47 & 3.66 & 2.32 \\ 
  Serras; Agrosistema extensivo; Mesotemperado superior & 6.78 & 7.06 & 0.96 \\ 
  Vales sublitorais; Matogueira e rochedo; Mesotemperado inferior & 5.08 & 3.12 & 1.63 \\ 
  Serras; Agrosistema extensivo; Mesotemperado inferior & 3.39 & 2.67 & 1.27 \\ 
  Serras; Matogueira e rochedo; Supra e orotemperado & 3.39 & 2.19 & 1.54 \\ 
  Vales sublitorais; Agrosistema intensivo (mosaico agroforestal); Termotemperado & 3.39 & 5.79 & 0.59 \\ 
  Vales sublitorais; Agrosistema intensivo (superficie de cultivo); Mesotemperado inferior & 3.39 & 2.45 & 1.38 \\ 
  Chairas e vales interiores; Agrosistema extensivo; Termotemperado & 1.69 & 0.24 & 7.04 \\ 
  Chairas e vales interiores; Agrosistema intensivo (mosaico agroforestal); Mesotemperado inferior & 1.69 & 0.74 & 2.29 \\ 
  Chairas e vales interiores; Bosque; Termotemperado & 1.69 & 0.21 & 8.19 \\ 
  Chairas e vales interiores; Urbano; Termotemperado & 1.69 & 0.00 & 361.04 \\ 
  Serras; Agrosistema intensivo (mosaico agroforestal); Mesotemperado inferior & 1.69 & 0.77 & 2.19 \\ 
  Serras; Matogueira e rochedo; Mesotemperado superior & 1.69 & 5.00 & 0.34 \\ 
  Vales sublitorais; Agrosistema extensivo; Mesotemperado superior & 1.69 & 2.91 & 0.58 \\ 
  Vales sublitorais; Agrosistema extensivo; Termotemperado & 1.69 & 0.96 & 1.76 \\ 
  Vales sublitorais; Bosque; Mesotemperado inferior & 1.69 & 0.61 & 2.80 \\ 
  Vales sublitorais; Rururbano (diseminado); Termotemperado & 1.69 & 1.94 & 0.87 \\ 
  Vales sublitorais; Urbano; Mesotemperado inferior & 1.69 & 0.49 & 3.43 \\ 
   \hline
\end{tabular}
\end{table}
% latex table generated in R 3.2.2 by xtable 1.8-0 package
% Fri Dec  4 16:54:42 2015
\begin{table}[p]
\centering
\caption{Frecuencia de aparición de BIC e frecuencia de tipos asociados, GAP Chairas, Fosas e Serras Ourensás} 
\label{vbic8}
\begin{tabular}{lrrr}
  \hline
Tipo de paisaxe & F.Aparic (\%) & F.Tipo (\%) & Ratio \\ 
  \hline
Chairas e vales interiores; Agrosistema extensivo; Mesotemperado inferior & 36.00 & 12.05 & 2.99 \\ 
  Chairas e vales interiores; Rururbano (diseminado); Mesotemperado inferior & 16.00 & 1.22 & 13.12 \\ 
  Chairas e vales interiores; Agrosistema intensivo (superficie de cultivo); Mesotemperado inferior & 8.00 & 7.47 & 1.07 \\ 
  Chairas e vales interiores; Conxunto Historico; Mesotemperado inferior & 8.00 & 0.01 & 789.91 \\ 
  Serras; Agrosistema extensivo; Mesotemperado inferior & 8.00 & 9.02 & 0.89 \\ 
  Chairas e vales interiores; Agrosistema extensivo; Termotemperado & 4.00 & 1.77 & 2.25 \\ 
  Chairas e vales interiores; Urbano; Mesotemperado inferior & 4.00 & 0.08 & 49.79 \\ 
  Serras; Agrosistema extensivo; Mesotemperado superior & 4.00 & 4.93 & 0.81 \\ 
  Serras; Agrosistema extensivo; Supra e orotemperado & 4.00 & 3.09 & 1.30 \\ 
  Serras; Matogueira e rochedo; no data & 4.00 & 0.36 & 10.99 \\ 
  Serras; Matogueira e rochedo; Supra e orotemperado & 4.00 & 8.42 & 0.47 \\ 
   \hline
\end{tabular}
\end{table}
% latex table generated in R 3.2.2 by xtable 1.8-0 package
% Fri Dec  4 16:54:42 2015
\begin{table}[p]
\centering
\caption{Frecuencia de aparición de BIC e frecuencia de tipos asociados, GAP Serras Surorientais} 
\label{vbic9}
\begin{tabular}{lrrr}
  \hline
Tipo de paisaxe & F.Aparic (\%) & F.Tipo (\%) & Ratio \\ 
  \hline
Canons; Agrosistema extensivo; Mesomediterráneo & 25.00 & 0.07 & 352.20 \\ 
  Canons; Matogueira e rochedo; Mesomediterráneo & 25.00 & 0.50 & 50.06 \\ 
  Canons; Rururbano (diseminado); Mesotemperado inferior & 25.00 & 0.05 & 530.75 \\ 
  Serras; Agrosistema extensivo; Mesotemperado inferior & 25.00 & 7.06 & 3.54 \\ 
   \hline
\end{tabular}
\end{table}
% latex table generated in R 3.2.2 by xtable 1.8-0 package
% Fri Dec  4 16:54:42 2015
\begin{table}[p]
\centering
\caption{Frecuencia de aparición de BIC e frecuencia de tipos asociados, GAP Galicia Setentrional} 
\label{vbic10}
\begin{tabular}{lrrr}
  \hline
Tipo de paisaxe & F.Aparic (\%) & F.Tipo (\%) & Ratio \\ 
  \hline
Litoral Cantabro-Atlantico; Rururbano (diseminado); Termotemperado & 41.67 & 2.78 & 15.00 \\ 
  Litoral Cantabro-Atlantico; Agrosistema intensivo (mosaico agroforestal); Termotemperado & 16.67 & 5.96 & 2.80 \\ 
  Litoral Cantabro-Atlantico; Conxunto Historico; Termotemperado & 8.33 & 0.00 & 9866.66 \\ 
  Litoral Cantabro-Atlantico; Rururbano (diseminado); no data & 8.33 & 0.09 & 95.54 \\ 
  Serras; Agrosistema intensivo (mosaico agroforestal); Mesotemperado superior & 8.33 & 3.06 & 2.72 \\ 
  Vales sublitorais; Agrosistema intensivo (mosaico agroforestal); Mesotemperado inferior & 8.33 & 12.50 & 0.67 \\ 
  Vales sublitorais; Rururbano (diseminado); Mesotemperado inferior & 8.33 & 0.53 & 15.60 \\ 
   \hline
\end{tabular}
\end{table}
% latex table generated in R 3.2.2 by xtable 1.8-0 package
% Fri Dec  4 16:54:42 2015
\begin{table}[p]
\centering
\caption{Frecuencia de aparición de BIC e frecuencia de tipos asociados, GAP Chairas e Fosas Occidentais} 
\label{vbic11}
\begin{tabular}{lrrr}
  \hline
Tipo de paisaxe & F.Aparic (\%) & F.Tipo (\%) & Ratio \\ 
  \hline
Vales sublitorais; Agrosistema intensivo (mosaico agroforestal); Mesotemperado inferior & 34.62 & 18.05 & 1.92 \\ 
  Litoral Cantabro-Atlantico; Agrosistema intensivo (mosaico agroforestal); Termotemperado & 11.54 & 8.86 & 1.30 \\ 
  Litoral Cantabro-Atlantico; Conxunto Historico; Termotemperado & 11.54 & 0.12 & 94.53 \\ 
  Vales sublitorais; Agrosistema intensivo (plantacion forestal); Mesotemperado inferior & 7.69 & 8.33 & 0.92 \\ 
  Litoral Cantabro-Atlantico; Agrosistema extensivo; no data & 3.85 & 0.01 & 501.06 \\ 
  Litoral Cantabro-Atlantico; Agrosistema intensivo (mosaico agroforestal); Mesotemperado inferior & 3.85 & 0.19 & 20.49 \\ 
  Litoral Cantabro-Atlantico; Agrosistema intensivo (superficie de cultivo); Termotemperado & 3.85 & 0.93 & 4.15 \\ 
  Litoral Cantabro-Atlantico; Rururbano (diseminado); Termotemperado & 3.85 & 2.77 & 1.39 \\ 
  Litoral Cantabro-Atlantico; Urbano; Termotemperado & 3.85 & 0.44 & 8.65 \\ 
  Vales sublitorais; Agrosistema extensivo; Mesotemperado inferior & 3.85 & 4.56 & 0.84 \\ 
  Vales sublitorais; Agrosistema intensivo (superficie de cultivo); Termotemperado & 3.85 & 0.70 & 5.51 \\ 
  Vales sublitorais; Rururbano (diseminado); Mesotemperado inferior & 3.85 & 1.88 & 2.04 \\ 
  Vales sublitorais; Rururbano (diseminado); Termotemperado & 3.85 & 2.48 & 1.55 \\ 
   \hline
\end{tabular}
\end{table}
% latex table generated in R 3.2.2 by xtable 1.8-0 package
% Fri Dec  4 16:54:42 2015
\begin{table}[p]
\centering
\caption{Frecuencia de aparición de BIC e frecuencia de tipos asociados, GAP Rías Baixas} 
\label{vbic12}
\begin{tabular}{lrrr}
  \hline
Tipo de paisaxe & F.Aparic (\%) & F.Tipo (\%) & Ratio \\ 
  \hline
Litoral Cantabro-Atlantico; Rururbano (diseminado); Termotemperado & 16.09 & 16.40 & 0.98 \\ 
  Litoral Cantabro-Atlantico; Conxunto Historico; Termotemperado & 10.92 & 0.04 & 265.95 \\ 
  Vales sublitorais; Matogueira e rochedo; Termotemperado & 10.34 & 5.27 & 1.96 \\ 
  Vales sublitorais; Rururbano (diseminado); Termotemperado & 10.34 & 6.06 & 1.71 \\ 
  Vales sublitorais; Agrosistema intensivo (plantacion forestal); Termotemperado & 6.90 & 8.25 & 0.84 \\ 
  Litoral Cantabro-Atlantico; Urbano; Termotemperado & 5.75 & 2.21 & 2.60 \\ 
  Vales sublitorais; Agrosistema intensivo (mosaico agroforestal); Mesotemperado inferior & 5.17 & 3.76 & 1.38 \\ 
  Vales sublitorais; Agrosistema intensivo (mosaico agroforestal); Termotemperado & 4.60 & 5.62 & 0.82 \\ 
  Vales sublitorais; Matogueira e rochedo; Mesotemperado inferior & 3.45 & 5.63 & 0.61 \\ 
  Litoral Cantabro-Atlantico; Urbano; no data & 2.87 & 0.20 & 14.66 \\ 
  Vales sublitorais; Agrosistema intensivo (plantacion forestal); Mesotemperado inferior & 2.87 & 3.99 & 0.72 \\ 
  Litoral Cantabro-Atlantico; Matogueira e rochedo; Termotemperado & 2.30 & 2.61 & 0.88 \\ 
  Serras; Matogueira e rochedo; Mesotemperado inferior & 2.30 & 4.31 & 0.53 \\ 
  Litoral Cantabro-Atlantico; Agrosistema intensivo (mosaico agroforestal); Termotemperado & 1.72 & 6.56 & 0.26 \\ 
  Litoral Cantabro-Atlantico; Conxunto Historico; no data & 1.72 & 0.01 & 299.18 \\ 
  Litoral Cantabro-Atlantico; Rururbano (diseminado); no data & 1.72 & 0.51 & 3.39 \\ 
  Serras; Agrosistema intensivo (plantacion forestal); Mesotemperado inferior & 1.72 & 1.60 & 1.08 \\ 
  Serras; Matogueira e rochedo; Mesotemperado superior & 1.72 & 4.07 & 0.42 \\ 
  Litoral Cantabro-Atlantico; Matogueira e rochedo; no data & 1.15 & 0.23 & 4.99 \\ 
  Serras; Agrosistema extensivo; Mesotemperado inferior & 1.15 & 1.07 & 1.07 \\ 
  Vales sublitorais; Urbano; Termotemperado & 1.15 & 0.33 & 3.50 \\ 
   \hline
\end{tabular}
\end{table}
