% latex table generated in R 3.2.2 by xtable 1.8-0 package
% Wed Dec  2 19:45:31 2015
\begin{table}[p]
\centering
\caption{Frecuencia de aparición de valores patrimoniais identificados na participación pública e frecuencia de tipos asociados Golfo Ártabro} 
\label{vsixotpat1}
\begin{tabular}{lrrr}
  \hline
Tipo de paisaxe & F.Aparic (\%) & F.Tipo (\%) & Ratio \\ 
  \hline
Litoral Cantabro-Atlantico; Rururbano (diseminado); Termotemperado & 16.67 & 3.29 & 5.06 \\ 
  Litoral Cantabro-Atlantico; Conxunto Historico; no data & 13.89 & 0.00 & 5265.44 \\ 
  Canons; Bosque; Mesotemperado inferior & 8.33 & 0.30 & 28.20 \\ 
  Litoral Cantabro-Atlantico; Agrosistema intensivo (plantacion forestal); Termotemperado & 8.33 & 1.32 & 6.33 \\ 
  Vales sublitorais; Agrosistema intensivo (mosaico agroforestal); Mesotemperado inferior & 8.33 & 7.54 & 1.10 \\ 
  Vales sublitorais; Agrosistema intensivo (mosaico agroforestal); Termotemperado & 8.33 & 2.62 & 3.18 \\ 
   & 8.33 &  &  \\ 
  Litoral Cantabro-Atlantico; Conxunto Historico; Termotemperado & 5.56 & 0.02 & 293.76 \\ 
  Litoral Cantabro-Atlantico; Rururbano (diseminado); Mesotemperado inferior & 5.56 & 0.04 & 147.54 \\ 
  Canons; Bosque; Termotemperado & 2.78 & 0.15 & 18.35 \\ 
  Litoral Cantabro-Atlantico; Agrosistema intensivo (mosaico agroforestal); no data & 2.78 & 0.04 & 65.33 \\ 
  Serras; Agrosistema intensivo (mosaico agroforestal); Mesotemperado superior & 2.78 & 1.77 & 1.57 \\ 
  Vales sublitorais; Agrosistema intensivo (plantacion forestal); Termotemperado & 2.78 & 1.79 & 1.55 \\ 
  Vales sublitorais; Matogueira e rochedo; Mesotemperado superior & 2.78 & 0.85 & 3.26 \\ 
  Vales sublitorais; Rururbano (diseminado); Termotemperado & 2.78 & 1.68 & 1.65 \\ 
   \hline
\end{tabular}
\end{table}
% latex table generated in R 3.2.2 by xtable 1.8-0 package
% Wed Dec  2 19:45:31 2015
\begin{table}[p]
\centering
\caption{Frecuencia de aparición de valores patrimoniais identificados na participación pública e frecuencia de tipos asociados A Mariña - Baixo Eo} 
\label{vsixotpat2}
\begin{tabular}{lrrr}
  \hline
Tipo de paisaxe & F.Aparic (\%) & F.Tipo (\%) & Ratio \\ 
  \hline
Vales sublitorais; Agrosistema intensivo (plantacion forestal); Mesotemperado inferior & 26.00 & 2.90 & 8.97 \\ 
  Vales sublitorais; Agrosistema intensivo (mosaico agroforestal); Mesotemperado inferior & 22.00 & 7.54 & 2.92 \\ 
  Vales sublitorais; Rururbano (diseminado); Mesotemperado inferior & 12.00 & 0.98 & 12.22 \\ 
  Litoral Cantabro-Atlantico; Rururbano (diseminado); Termotemperado & 8.00 & 3.29 & 2.43 \\ 
  Litoral Cantabro-Atlantico; Conxunto Historico; Termotemperado & 6.00 & 0.02 & 317.26 \\ 
  Litoral Cantabro-Atlantico; Matogueira e rochedo; Mesotemperado inferior & 6.00 & 0.10 & 60.27 \\ 
  Litoral Cantabro-Atlantico; Agrosistema intensivo (mosaico agroforestal); Mesotemperado inferior & 4.00 & 0.29 & 13.65 \\ 
  Serras; Turbeira; Mesotemperado superior & 4.00 & 0.71 & 5.65 \\ 
  Litoral Cantabro-Atlantico; Agrosistema intensivo (mosaico agroforestal); Termotemperado & 2.00 & 2.43 & 0.82 \\ 
  Litoral Cantabro-Atlantico; Conxunto Historico; Mesotemperado inferior & 2.00 & 0.00 & 3431.24 \\ 
  Litoral Cantabro-Atlantico; Rururbano (diseminado); Mesotemperado inferior & 2.00 & 0.04 & 53.11 \\ 
  Vales sublitorais; Agrosistema extensivo; Mesotemperado inferior & 2.00 & 2.73 & 0.73 \\ 
  Vales sublitorais; Bosque; Mesotemperado superior & 2.00 & 0.39 & 5.18 \\ 
   & 2.00 &  &  \\ 
   \hline
\end{tabular}
\end{table}
% latex table generated in R 3.2.2 by xtable 1.8-0 package
% Wed Dec  2 19:45:31 2015
\begin{table}[p]
\centering
\caption{Frecuencia de aparición de valores patrimoniais identificados na participación pública e frecuencia de tipos asociados Costa Sur - Baixo Miño} 
\label{vsixotpat3}
\begin{tabular}{lrrr}
  \hline
Tipo de paisaxe & F.Aparic (\%) & F.Tipo (\%) & Ratio \\ 
  \hline
Litoral Cantabro-Atlantico; Rururbano (diseminado); Termotemperado & 43.75 & 3.29 & 13.29 \\ 
  Serras; Conxunto Historico; Termotemperado & 12.50 & 0.01 & 1562.54 \\ 
  Vales sublitorais; Agrosistema intensivo (plantacion forestal); Termotemperado & 9.38 & 1.79 & 5.24 \\ 
  Vales sublitorais; Rururbano (diseminado); Termotemperado & 9.38 & 1.68 & 5.58 \\ 
  Chairas e vales interiores; Agrosistema intensivo (mosaico agroforestal); Mesotemperado inferior & 3.12 & 1.71 & 1.83 \\ 
  Chairas e vales interiores; Matogueira e rochedo; Mesotemperado inferior & 3.12 & 1.38 & 2.26 \\ 
  Litoral Cantabro-Atlantico; Agrosistema intensivo (plantacion forestal); Termotemperado & 3.12 & 1.32 & 2.37 \\ 
  no data; Agrosistema intensivo (mosaico agroforestal); Termotemperado & 3.12 & 0.02 & 176.39 \\ 
  Serras; Agrosistema intensivo (plantacion forestal); Mesotemperado inferior & 3.12 & 0.82 & 3.79 \\ 
  Serras; Matogueira e rochedo; Mesotemperado inferior & 3.12 & 2.73 & 1.15 \\ 
  Vales sublitorais; Agrosistema intensivo (mosaico agroforestal); Termotemperado & 3.12 & 2.62 & 1.19 \\ 
  Vales sublitorais; Agrosistema intensivo (plantacion forestal); Mesotemperado inferior & 3.12 & 2.90 & 1.08 \\ 
   \hline
\end{tabular}
\end{table}
% latex table generated in R 3.2.2 by xtable 1.8-0 package
% Wed Dec  2 19:45:31 2015
\begin{table}[p]
\centering
\caption{Frecuencia de aparición de valores patrimoniais identificados na participación pública e frecuencia de tipos asociados Ribeiras Encaixadas do Miño e do Sil} 
\label{vsixotpat4}
\begin{tabular}{lrrr}
  \hline
Tipo de paisaxe & F.Aparic (\%) & F.Tipo (\%) & Ratio \\ 
  \hline
Canons; Matogueira e rochedo; Mesomediterráneo & 9.52 & 0.18 & 51.55 \\ 
  Chairas e vales interiores; Conxunto Historico; Termotemperado & 9.52 & 0.00 & 3279.73 \\ 
  Canons; Viñedo; Mesotemperado inferior & 7.94 & 0.02 & 343.20 \\ 
  Canons; Bosque; Mesotemperado inferior & 6.35 & 0.30 & 21.49 \\ 
  Chairas e vales interiores; Rururbano (diseminado); Mesotemperado inferior & 6.35 & 0.51 & 12.40 \\ 
  Canons; Bosque; Termotemperado & 4.76 & 0.15 & 31.46 \\ 
  Chairas e vales interiores; Bosque; Termotemperado & 4.76 & 0.29 & 16.33 \\ 
  Chairas e vales interiores; Rururbano (diseminado); Termotemperado & 4.76 & 0.42 & 11.25 \\ 
  Canons; Agrosistema extensivo; Mesomediterráneo & 3.17 & 0.03 & 92.01 \\ 
  Chairas e vales interiores; Agrosistema intensivo (mosaico agroforestal); Termotemperado & 3.17 & 0.34 & 9.36 \\ 
  Chairas e vales interiores; Matogueira e rochedo; Mesomediterráneo & 3.17 & 0.27 & 11.84 \\ 
  Chairas e vales interiores; Viñedo; Mesomediterráneo & 3.17 & 0.11 & 28.35 \\ 
  Serras; Agrosistema extensivo; Mesotemperado superior & 3.17 & 5.76 & 0.55 \\ 
  Serras; Matogueira e rochedo; Mesotemperado inferior & 3.17 & 2.73 & 1.16 \\ 
  Canons; Bosque; Mesomediterráneo & 1.59 & 0.03 & 59.11 \\ 
  Canons; Matogueira e rochedo; Termotemperado & 1.59 & 0.09 & 18.41 \\ 
  Canons; Viñedo; Termotemperado & 1.59 & 0.04 & 42.70 \\ 
  Chairas e vales interiores; Agrosistema extensivo; Mesomediterráneo & 1.59 & 0.11 & 14.47 \\ 
  Chairas e vales interiores; Agrosistema intensivo (mosaico agroforestal); Mesomediterráneo & 1.59 & 0.01 & 118.98 \\ 
  Chairas e vales interiores; Agrosistema intensivo (mosaico agroforestal); Mesotemperado inferior & 1.59 & 1.71 & 0.93 \\ 
  Chairas e vales interiores; Agrosistema intensivo (plantacion forestal); Mesotemperado inferior & 1.59 & 0.55 & 2.88 \\ 
  Chairas e vales interiores; Agrosistema intensivo (plantacion forestal); Termotemperado & 1.59 & 0.37 & 4.31 \\ 
  Chairas e vales interiores; Bosque; Mesomediterráneo & 1.59 & 0.06 & 28.14 \\ 
  Chairas e vales interiores; Matogueira e rochedo; Mesotemperado inferior & 1.59 & 1.38 & 1.15 \\ 
  Chairas e vales interiores; Matogueira e rochedo; Termotemperado & 1.59 & 0.66 & 2.42 \\ 
  Chairas e vales interiores; Urbano; Termotemperado & 1.59 & 0.08 & 19.34 \\ 
  Chairas e vales interiores; Viñedo; Termotemperado & 1.59 & 0.28 & 5.72 \\ 
  Serras; Agrosistema extensivo; Mesotemperado inferior & 1.59 & 2.57 & 0.62 \\ 
  Serras; Agrosistema intensivo (superficie de cultivo); Mesotemperado inferior & 1.59 & 0.28 & 5.60 \\ 
  Serras; Conxunto Historico; Mesotemperado inferior & 1.59 & 0.00 & 764.60 \\ 
  Serras; Viñedo; Mesotemperado inferior & 1.59 & 0.02 & 82.44 \\ 
   \hline
\end{tabular}
\end{table}
% latex table generated in R 3.2.2 by xtable 1.8-0 package
% Wed Dec  2 19:45:31 2015
\begin{table}[p]
\centering
\caption{Frecuencia de aparición de valores patrimoniais identificados na participación pública e frecuencia de tipos asociados Serras Orientais} 
\label{vsixotpat5}
\begin{tabular}{lrrr}
  \hline
Tipo de paisaxe & F.Aparic (\%) & F.Tipo (\%) & Ratio \\ 
  \hline
Serras; Agrosistema extensivo; Supra e orotemperado & 30.00 & 2.50 & 12.01 \\ 
  Serras; Agrosistema extensivo; Mesotemperado superior & 16.67 & 5.76 & 2.89 \\ 
  Serras; Matogueira e rochedo; Supra e orotemperado & 10.00 & 6.12 & 1.63 \\ 
  Vales sublitorais; Bosque; Mesotemperado superior & 10.00 & 0.39 & 25.91 \\ 
  Serras; Bosque; Supra e orotemperado & 6.67 & 0.74 & 9.05 \\ 
  Vales sublitorais; Bosque; Mesotemperado inferior & 6.67 & 0.49 & 13.47 \\ 
  Canons; Bosque; Mesotemperado inferior & 3.33 & 0.30 & 11.28 \\ 
  Chairas e vales interiores; Agrosistema extensivo; Mesotemperado inferior & 3.33 & 3.58 & 0.93 \\ 
  Chairas e vales interiores; Agrosistema extensivo; Mesotemperado superior & 3.33 & 2.77 & 1.20 \\ 
  Serras; Agrosistema intensivo (mosaico agroforestal); Mesotemperado superior & 3.33 & 1.77 & 1.88 \\ 
  Serras; Agrosistema intensivo (plantacion forestal); Supra e orotemperado & 3.33 & 0.93 & 3.58 \\ 
  Serras; Bosque; Mesotemperado superior & 3.33 & 1.00 & 3.33 \\ 
   \hline
\end{tabular}
\end{table}
% latex table generated in R 3.2.2 by xtable 1.8-0 package
% Wed Dec  2 19:45:31 2015
\begin{table}[p]
\centering
\caption{Frecuencia de aparición de valores patrimoniais identificados na participación pública e frecuencia de tipos asociados Chairas e Fosas Luguesas} 
\label{vsixotpat6}
\begin{tabular}{lrrr}
  \hline
Tipo de paisaxe & F.Aparic (\%) & F.Tipo (\%) & Ratio \\ 
  \hline
Chairas e vales interiores; Agrosistema extensivo; Mesotemperado inferior & 15.79 & 3.58 & 4.41 \\ 
  Chairas e vales interiores; Agrosistema extensivo; Mesotemperado superior & 15.79 & 2.77 & 5.69 \\ 
  Chairas e vales interiores; Bosque; Mesotemperado inferior & 13.16 & 0.77 & 17.20 \\ 
  Chairas e vales interiores; Rururbano (diseminado); Mesotemperado superior & 13.16 & 0.30 & 43.90 \\ 
  Chairas e vales interiores; Agrosistema intensivo (mosaico agroforestal); Mesotemperado superior & 10.53 & 2.27 & 4.64 \\ 
  Chairas e vales interiores; Agrosistema intensivo (superficie de cultivo); Mesotemperado superior & 5.26 & 0.72 & 7.35 \\ 
  Chairas e vales interiores; Conxunto Historico; Mesotemperado superior & 5.26 & 0.00 & 4765.13 \\ 
  Chairas e vales interiores; Conxunto Historico; Termotemperado & 5.26 & 0.00 & 1812.48 \\ 
  Serras; Agrosistema extensivo; Mesotemperado superior & 5.26 & 5.76 & 0.91 \\ 
  Chairas e vales interiores; Agrosistema intensivo (mosaico agroforestal); Mesotemperado inferior & 2.63 & 1.71 & 1.54 \\ 
  Chairas e vales interiores; Agrosistema intensivo (plantacion forestal); Mesotemperado superior & 2.63 & 0.42 & 6.24 \\ 
  Chairas e vales interiores; Rururbano (diseminado); Mesotemperado inferior & 2.63 & 0.51 & 5.14 \\ 
  Serras; Agrosistema extensivo; Supra e orotemperado & 2.63 & 2.50 & 1.05 \\ 
   \hline
\end{tabular}
\end{table}
% latex table generated in R 3.2.2 by xtable 1.8-0 package
% Wed Dec  2 19:45:32 2015
\begin{table}[p]
\centering
\caption{Frecuencia de aparición de valores patrimoniais identificados na participación pública e frecuencia de tipos asociados Galicia Central} 
\label{vsixotpat7}
\begin{tabular}{lrrr}
  \hline
Tipo de paisaxe & F.Aparic (\%) & F.Tipo (\%) & Ratio \\ 
  \hline
Vales sublitorais; Rururbano (diseminado); Termotemperado & 12.31 & 1.68 & 7.32 \\ 
  Vales sublitorais; Agrosistema intensivo (mosaico agroforestal); Mesotemperado inferior & 11.54 & 7.54 & 1.53 \\ 
  Vales sublitorais; Agrosistema extensivo; Mesotemperado inferior & 10.77 & 2.73 & 3.94 \\ 
  Vales sublitorais; Rururbano (diseminado); Mesotemperado inferior & 9.23 & 0.98 & 9.40 \\ 
  Serras; Agrosistema extensivo; Mesotemperado superior & 6.92 & 5.76 & 1.20 \\ 
  Vales sublitorais; Agrosistema intensivo (mosaico agroforestal); Termotemperado & 6.92 & 2.62 & 2.64 \\ 
  Vales sublitorais; Agrosistema intensivo (superficie de cultivo); Mesotemperado inferior & 6.92 & 0.81 & 8.54 \\ 
  Vales sublitorais; Urbano; Mesotemperado inferior & 5.38 & 0.11 & 49.58 \\ 
  Serras; Matogueira e rochedo; Mesotemperado superior & 4.62 & 5.03 & 0.92 \\ 
  Vales sublitorais; Conxunto Historico; Mesotemperado inferior & 3.85 & 0.00 & 1120.74 \\ 
  Serras; Agrosistema extensivo; Mesotemperado inferior & 3.08 & 2.57 & 1.20 \\ 
  Vales sublitorais; Bosque; Mesotemperado inferior & 2.31 & 0.49 & 4.66 \\ 
  Chairas e vales interiores; Urbano; Termotemperado & 1.54 & 0.08 & 18.74 \\ 
  Vales sublitorais; Agrosistema extensivo; Termotemperado & 1.54 & 0.36 & 4.29 \\ 
  Vales sublitorais; Agrosistema intensivo (plantacion forestal); Termotemperado & 1.54 & 1.79 & 0.86 \\ 
  Vales sublitorais; Matogueira e rochedo; Termotemperado & 1.54 & 0.93 & 1.65 \\ 
   \hline
\end{tabular}
\end{table}
% latex table generated in R 3.2.2 by xtable 1.8-0 package
% Wed Dec  2 19:45:32 2015
\begin{table}[p]
\centering
\caption{Frecuencia de aparición de valores patrimoniais identificados na participación pública e frecuencia de tipos asociados Chairas, Fosas e Serras Ourensás} 
\label{vsixotpat8}
\begin{tabular}{lrrr}
  \hline
Tipo de paisaxe & F.Aparic (\%) & F.Tipo (\%) & Ratio \\ 
  \hline
Serras; Agrosistema extensivo; Mesotemperado inferior & 12.86 & 2.57 & 5.00 \\ 
  Serras; Agrosistema extensivo; Supra e orotemperado & 10.00 & 2.50 & 4.00 \\ 
  Chairas e vales interiores; Agrosistema extensivo; Mesotemperado inferior & 8.57 & 3.58 & 2.39 \\ 
  Serras; Matogueira e rochedo; Mesotemperado superior & 8.57 & 5.03 & 1.70 \\ 
  Chairas e vales interiores; Matogueira e rochedo; Mesotemperado inferior & 7.14 & 1.38 & 5.17 \\ 
  Serras; Matogueira e rochedo; Mesotemperado inferior & 7.14 & 2.73 & 2.62 \\ 
  Serras; Bosque; Mesotemperado superior & 5.71 & 1.00 & 5.71 \\ 
  Chairas e vales interiores; Bosque; Mesotemperado inferior & 4.29 & 0.77 & 5.60 \\ 
  Chairas e vales interiores; Matogueira e rochedo; Termotemperado & 4.29 & 0.66 & 6.53 \\ 
  Chairas e vales interiores; Rururbano (diseminado); Mesotemperado inferior & 4.29 & 0.51 & 8.37 \\ 
  Serras; Agrosistema extensivo; Mesotemperado superior & 4.29 & 5.76 & 0.74 \\ 
  Chairas e vales interiores; Agrosistema extensivo; Termotemperado & 2.86 & 0.61 & 4.66 \\ 
  Chairas e vales interiores; Agrosistema intensivo (superficie de cultivo); Mesotemperado inferior & 2.86 & 1.16 & 2.47 \\ 
  Chairas e vales interiores; Conxunto Historico; Mesotemperado inferior & 2.86 & 0.00 & 2071.80 \\ 
  Serras; Agrosistema intensivo (superficie de cultivo); Supra e orotemperado & 2.86 & 0.23 & 12.35 \\ 
  Chairas e vales interiores; Agrosistema intensivo (mosaico agroforestal); Mesotemperado inferior & 1.43 & 1.71 & 0.84 \\ 
  Chairas e vales interiores; Agrosistema intensivo (mosaico agroforestal); Termotemperado & 1.43 & 0.34 & 4.21 \\ 
  Chairas e vales interiores; Agrosistema intensivo (superficie de cultivo); no data & 1.43 & 0.00 & 808.18 \\ 
  Chairas e vales interiores; Urbano; Mesotemperado inferior & 1.43 & 0.06 & 25.63 \\ 
  Chairas e vales interiores; Viñedo; Termotemperado & 1.43 & 0.28 & 5.15 \\ 
  Serras; Agrosistema intensivo (superficie de cultivo); Mesotemperado inferior & 1.43 & 0.28 & 5.04 \\ 
  Serras; Bosque; Mesotemperado inferior & 1.43 & 0.69 & 2.07 \\ 
  Serras; Matogueira e rochedo; Supra e orotemperado & 1.43 & 6.12 & 0.23 \\ 
   \hline
\end{tabular}
\end{table}
% latex table generated in R 3.2.2 by xtable 1.8-0 package
% Wed Dec  2 19:45:32 2015
\begin{table}[p]
\centering
\caption{Frecuencia de aparición de valores patrimoniais identificados na participación pública e frecuencia de tipos asociados Serras Surorientais} 
\label{vsixotpat9}
\begin{tabular}{lrrr}
  \hline
Tipo de paisaxe & F.Aparic (\%) & F.Tipo (\%) & Ratio \\ 
  \hline
Serras; Agrosistema extensivo; Mesotemperado inferior & 27.59 & 2.57 & 10.74 \\ 
  Canons; Viñedo; Mesomediterráneo & 13.79 & 0.01 & 1154.74 \\ 
  Serras; Agrosistema extensivo; Supra e orotemperado & 10.34 & 2.50 & 4.14 \\ 
  Serras; Agrosistema extensivo; Mesomediterráneo & 6.90 & 0.06 & 124.22 \\ 
  Serras; Agrosistema extensivo; Mesotemperado superior & 6.90 & 5.76 & 1.20 \\ 
  Serras; Matogueira e rochedo; Supra e orotemperado & 6.90 & 6.12 & 1.13 \\ 
  Canons; Bosque; Mesotemperado inferior & 3.45 & 0.30 & 11.67 \\ 
  Canons; Matogueira e rochedo; Mesomediterráneo & 3.45 & 0.18 & 18.66 \\ 
  Canons; Viñedo; Mesotemperado inferior & 3.45 & 0.02 & 149.11 \\ 
  Serras; Agrosistema intensivo (superficie de cultivo); Mesotemperado inferior & 3.45 & 0.28 & 12.16 \\ 
  Serras; Bosque; Mesotemperado inferior & 3.45 & 0.69 & 4.99 \\ 
  Serras; Rururbano (diseminado); Mesotemperado superior & 3.45 & 0.11 & 31.04 \\ 
  Vales sublitorais; Agrosistema extensivo; Mesomediterráneo & 3.45 & 0.00 & 18346.01 \\ 
  Vales sublitorais; Viñedo; Mesomediterráneo & 3.45 & 0.00 & 28645.53 \\ 
   \hline
\end{tabular}
\end{table}
% latex table generated in R 3.2.2 by xtable 1.8-0 package
% Wed Dec  2 19:45:32 2015
\begin{table}[p]
\centering
\caption{Frecuencia de aparición de valores patrimoniais identificados na participación pública e frecuencia de tipos asociados Galicia Setentrional} 
\label{vsixotpat10}
\begin{tabular}{lrrr}
  \hline
Tipo de paisaxe & F.Aparic (\%) & F.Tipo (\%) & Ratio \\ 
  \hline
Vales sublitorais; Agrosistema intensivo (mosaico agroforestal); Mesotemperado inferior & 27.27 & 7.54 & 3.62 \\ 
  Litoral Cantabro-Atlantico; Agrosistema intensivo (mosaico agroforestal); Mesotemperado inferior & 13.64 & 0.29 & 46.55 \\ 
  Serras; Turbeira; Mesotemperado superior & 13.64 & 0.71 & 19.25 \\ 
  Serras; Agrosistema intensivo (mosaico agroforestal); Mesotemperado superior & 9.09 & 1.77 & 5.12 \\ 
  Vales sublitorais; Agrosistema intensivo (plantacion forestal); Mesotemperado inferior & 9.09 & 2.90 & 3.14 \\ 
  Vales sublitorais; Agrosistema intensivo (plantacion forestal); Mesotemperado superior & 9.09 & 0.60 & 15.21 \\ 
  Litoral Cantabro-Atlantico; Agrosistema intensivo (mosaico agroforestal); Termotemperado & 4.55 & 2.43 & 1.87 \\ 
  Litoral Cantabro-Atlantico; Rururbano (diseminado); Termotemperado & 4.55 & 3.29 & 1.38 \\ 
  Serras; Turbeira; Mesotemperado inferior & 4.55 & 0.05 & 85.31 \\ 
  Vales sublitorais; Rururbano (diseminado); Mesotemperado inferior & 4.55 & 0.98 & 4.63 \\ 
   \hline
\end{tabular}
\end{table}
% latex table generated in R 3.2.2 by xtable 1.8-0 package
% Wed Dec  2 19:45:32 2015
\begin{table}[p]
\centering
\caption{Frecuencia de aparición de valores patrimoniais identificados na participación pública e frecuencia de tipos asociados Chairas e Fosas Occidentais} 
\label{vsixotpat11}
\begin{tabular}{lrrr}
  \hline
Tipo de paisaxe & F.Aparic (\%) & F.Tipo (\%) & Ratio \\ 
  \hline
Litoral Cantabro-Atlantico; Agrosistema intensivo (mosaico agroforestal); Termotemperado & 15.38 & 2.43 & 6.33 \\ 
  Vales sublitorais; Agrosistema intensivo (plantacion forestal); Mesotemperado inferior & 15.38 & 2.90 & 5.31 \\ 
  Vales sublitorais; Agrosistema extensivo; Mesotemperado inferior & 12.82 & 2.73 & 4.69 \\ 
  Vales sublitorais; Agrosistema intensivo (mosaico agroforestal); Termotemperado & 10.26 & 2.62 & 3.92 \\ 
  Litoral Cantabro-Atlantico; Conxunto Historico; Termotemperado & 7.69 & 0.02 & 406.74 \\ 
  Vales sublitorais; Agrosistema intensivo (mosaico agroforestal); Mesotemperado inferior & 7.69 & 7.54 & 1.02 \\ 
  Vales sublitorais; Rururbano (diseminado); Mesotemperado inferior & 7.69 & 0.98 & 7.83 \\ 
  Vales sublitorais; Agrosistema intensivo (superficie de cultivo); Termotemperado & 5.13 & 0.10 & 52.32 \\ 
  Vales sublitorais; Rururbano (diseminado); Termotemperado & 5.13 & 1.68 & 3.05 \\ 
  Litoral Cantabro-Atlantico; Agrosistema extensivo; no data & 2.56 & 0.01 & 235.98 \\ 
  Litoral Cantabro-Atlantico; Rururbano (diseminado); Termotemperado & 2.56 & 3.29 & 0.78 \\ 
  Litoral Cantabro-Atlantico; Urbano; Termotemperado & 2.56 & 0.53 & 4.83 \\ 
  Vales sublitorais; Agrosistema intensivo (mosaico agroforestal); Mesotemperado superior & 2.56 & 1.52 & 1.69 \\ 
  Vales sublitorais; Agrosistema intensivo (plantacion forestal); Termotemperado & 2.56 & 1.79 & 1.43 \\ 
   \hline
\end{tabular}
\end{table}
% latex table generated in R 3.2.2 by xtable 1.8-0 package
% Wed Dec  2 19:45:32 2015
\begin{table}[p]
\centering
\caption{Frecuencia de aparición de valores patrimoniais identificados na participación pública e frecuencia de tipos asociados Rías Baixas} 
\label{vsixotpat12}
\begin{tabular}{lrrr}
  \hline
Tipo de paisaxe & F.Aparic (\%) & F.Tipo (\%) & Ratio \\ 
  \hline
Serras; Matogueira e rochedo; Mesotemperado superior & 16.13 & 5.03 & 3.21 \\ 
  Litoral Cantabro-Atlantico; Rururbano (diseminado); Termotemperado & 9.68 & 3.29 & 2.94 \\ 
  Vales sublitorais; Rururbano (diseminado); Termotemperado & 9.68 & 1.68 & 5.76 \\ 
  Litoral Cantabro-Atlantico; Agrosistema intensivo (mosaico agroforestal); Termotemperado & 6.45 & 2.43 & 2.66 \\ 
  Litoral Cantabro-Atlantico; Matogueira e rochedo; Termotemperado & 6.45 & 0.83 & 7.76 \\ 
  Serras; Matogueira e rochedo; Mesotemperado inferior & 6.45 & 2.73 & 2.37 \\ 
  Vales sublitorais; Agrosistema intensivo (plantacion forestal); Termotemperado & 6.45 & 1.79 & 3.60 \\ 
  Vales sublitorais; Matogueira e rochedo; Mesotemperado inferior & 6.45 & 1.89 & 3.41 \\ 
  Litoral Cantabro-Atlantico; Agrosistema intensivo (mosaico agroforestal); no data & 3.23 & 0.04 & 75.86 \\ 
  Litoral Cantabro-Atlantico; Agrosistema intensivo (plantacion forestal); Termotemperado & 3.23 & 1.32 & 2.45 \\ 
  Litoral Cantabro-Atlantico; Bosque; Termotemperado & 3.23 & 0.03 & 97.99 \\ 
  Litoral Cantabro-Atlantico; Urbano; no data & 3.23 & 0.04 & 72.04 \\ 
  Litoral Cantabro-Atlantico; Viñedo; Termotemperado & 3.23 & 0.14 & 23.31 \\ 
  Serras; Rururbano (diseminado); Mesotemperado inferior & 3.23 & 0.08 & 38.10 \\ 
  Vales sublitorais; Agrosistema intensivo (mosaico agroforestal); Mesotemperado inferior & 3.23 & 7.54 & 0.43 \\ 
  Vales sublitorais; Agrosistema intensivo (mosaico agroforestal); Termotemperado & 3.23 & 2.62 & 1.23 \\ 
  Vales sublitorais; Agrosistema intensivo (plantacion forestal); Mesotemperado superior & 3.23 & 0.60 & 5.40 \\ 
  Vales sublitorais; Matogueira e rochedo; Termotemperado & 3.23 & 0.93 & 3.46 \\ 
   \hline
\end{tabular}
\end{table}
