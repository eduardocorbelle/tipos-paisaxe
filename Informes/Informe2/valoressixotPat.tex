% latex table generated in R 3.2.2 by xtable 1.8-0 package
% Fri Dec  4 16:54:41 2015
\begin{table}[p]
\centering
\caption{Frecuencia de aparición de valores patrimoniais identificados na participación pública e frecuencia de tipos asociados Golfo Ártabro} 
\label{vsixotpat1}
\begin{tabular}{lrrr}
  \hline
Tipo de paisaxe & F.Aparic (\%) & F.Tipo (\%) & Ratio \\ 
  \hline
Litoral Cantabro-Atlantico; Rururbano (diseminado); Termotemperado & 16.67 & 19.32 & 0.86 \\ 
  Litoral Cantabro-Atlantico; Conxunto Historico; no data & 13.89 & 0.02 & 752.69 \\ 
  Canons; Bosque; Mesotemperado inferior & 8.33 & 1.04 & 7.98 \\ 
  Litoral Cantabro-Atlantico; Agrosistema intensivo (plantacion forestal); Termotemperado & 8.33 & 2.83 & 2.94 \\ 
  Vales sublitorais; Agrosistema intensivo (mosaico agroforestal); Mesotemperado inferior & 8.33 & 14.40 & 0.58 \\ 
  Vales sublitorais; Agrosistema intensivo (mosaico agroforestal); Termotemperado & 8.33 & 4.06 & 2.05 \\ 
   & 8.33 &  &  \\ 
  Litoral Cantabro-Atlantico; Conxunto Historico; Termotemperado & 5.56 & 0.07 & 78.03 \\ 
  Litoral Cantabro-Atlantico; Rururbano (diseminado); Mesotemperado inferior & 5.56 & 0.21 & 26.61 \\ 
  Canons; Bosque; Termotemperado & 2.78 & 0.34 & 8.23 \\ 
  Litoral Cantabro-Atlantico; Agrosistema intensivo (mosaico agroforestal); no data & 2.78 & 0.06 & 47.32 \\ 
  Serras; Agrosistema intensivo (mosaico agroforestal); Mesotemperado superior & 2.78 & 1.64 & 1.69 \\ 
  Vales sublitorais; Agrosistema intensivo (plantacion forestal); Termotemperado & 2.78 & 1.65 & 1.69 \\ 
  Vales sublitorais; Matogueira e rochedo; Mesotemperado superior & 2.78 & 1.16 & 2.39 \\ 
  Vales sublitorais; Rururbano (diseminado); Termotemperado & 2.78 & 2.11 & 1.31 \\ 
   \hline
\end{tabular}
\end{table}
% latex table generated in R 3.2.2 by xtable 1.8-0 package
% Fri Dec  4 16:54:41 2015
\begin{table}[p]
\centering
\caption{Frecuencia de aparición de valores patrimoniais identificados na participación pública e frecuencia de tipos asociados A Mariña - Baixo Eo} 
\label{vsixotpat2}
\begin{tabular}{lrrr}
  \hline
Tipo de paisaxe & F.Aparic (\%) & F.Tipo (\%) & Ratio \\ 
  \hline
Vales sublitorais; Agrosistema intensivo (plantacion forestal); Mesotemperado inferior & 26.00 & 14.02 & 1.85 \\ 
  Vales sublitorais; Agrosistema intensivo (mosaico agroforestal); Mesotemperado inferior & 22.00 & 12.56 & 1.75 \\ 
  Vales sublitorais; Rururbano (diseminado); Mesotemperado inferior & 12.00 & 1.16 & 10.34 \\ 
  Litoral Cantabro-Atlantico; Rururbano (diseminado); Termotemperado & 8.00 & 3.92 & 2.04 \\ 
  Litoral Cantabro-Atlantico; Conxunto Historico; Termotemperado & 6.00 & 0.07 & 90.39 \\ 
  Litoral Cantabro-Atlantico; Matogueira e rochedo; Mesotemperado inferior & 6.00 & 0.09 & 68.43 \\ 
  Litoral Cantabro-Atlantico; Agrosistema intensivo (mosaico agroforestal); Mesotemperado inferior & 4.00 & 4.26 & 0.94 \\ 
  Serras; Turbeira; Mesotemperado superior & 4.00 & 2.35 & 1.70 \\ 
  Litoral Cantabro-Atlantico; Agrosistema intensivo (mosaico agroforestal); Termotemperado & 2.00 & 8.49 & 0.24 \\ 
  Litoral Cantabro-Atlantico; Conxunto Historico; Mesotemperado inferior & 2.00 & 0.02 & 107.20 \\ 
  Litoral Cantabro-Atlantico; Rururbano (diseminado); Mesotemperado inferior & 2.00 & 0.50 & 4.01 \\ 
  Vales sublitorais; Agrosistema extensivo; Mesotemperado inferior & 2.00 & 1.32 & 1.51 \\ 
  Vales sublitorais; Bosque; Mesotemperado superior & 2.00 & 0.85 & 2.35 \\ 
   & 2.00 &  &  \\ 
   \hline
\end{tabular}
\end{table}
% latex table generated in R 3.2.2 by xtable 1.8-0 package
% Fri Dec  4 16:54:42 2015
\begin{table}[p]
\centering
\caption{Frecuencia de aparición de valores patrimoniais identificados na participación pública e frecuencia de tipos asociados Costa Sur - Baixo Miño} 
\label{vsixotpat3}
\begin{tabular}{lrrr}
  \hline
Tipo de paisaxe & F.Aparic (\%) & F.Tipo (\%) & Ratio \\ 
  \hline
Litoral Cantabro-Atlantico; Rururbano (diseminado); Termotemperado & 43.75 & 11.41 & 3.83 \\ 
  Serras; Conxunto Historico; Termotemperado & 12.50 & 0.20 & 62.47 \\ 
  Vales sublitorais; Agrosistema intensivo (plantacion forestal); Termotemperado & 9.38 & 8.94 & 1.05 \\ 
  Vales sublitorais; Rururbano (diseminado); Termotemperado & 9.38 & 12.49 & 0.75 \\ 
  Chairas e vales interiores; Agrosistema intensivo (mosaico agroforestal); Mesotemperado inferior & 3.12 & 0.98 & 3.19 \\ 
  Chairas e vales interiores; Matogueira e rochedo; Mesotemperado inferior & 3.12 & 2.49 & 1.26 \\ 
  Litoral Cantabro-Atlantico; Agrosistema intensivo (plantacion forestal); Termotemperado & 3.12 & 2.51 & 1.25 \\ 
  no data; Agrosistema intensivo (mosaico agroforestal); Termotemperado & 3.12 & 0.44 & 7.06 \\ 
  Serras; Agrosistema intensivo (plantacion forestal); Mesotemperado inferior & 3.12 & 2.04 & 1.53 \\ 
  Serras; Matogueira e rochedo; Mesotemperado inferior & 3.12 & 5.06 & 0.62 \\ 
  Vales sublitorais; Agrosistema intensivo (mosaico agroforestal); Termotemperado & 3.12 & 2.92 & 1.07 \\ 
  Vales sublitorais; Agrosistema intensivo (plantacion forestal); Mesotemperado inferior & 3.12 & 0.28 & 11.07 \\ 
   \hline
\end{tabular}
\end{table}
% latex table generated in R 3.2.2 by xtable 1.8-0 package
% Fri Dec  4 16:54:42 2015
\begin{table}[p]
\centering
\caption{Frecuencia de aparición de valores patrimoniais identificados na participación pública e frecuencia de tipos asociados Ribeiras Encaixadas do Miño e do Sil} 
\label{vsixotpat4}
\begin{tabular}{lrrr}
  \hline
Tipo de paisaxe & F.Aparic (\%) & F.Tipo (\%) & Ratio \\ 
  \hline
Canons; Matogueira e rochedo; Mesomediterráneo & 9.52 & 1.77 & 5.39 \\ 
  Chairas e vales interiores; Conxunto Historico; Termotemperado & 9.52 & 0.01 & 701.41 \\ 
  Canons; Viñedo; Mesotemperado inferior & 7.94 & 0.24 & 32.58 \\ 
  Canons; Bosque; Mesotemperado inferior & 6.35 & 1.90 & 3.34 \\ 
  Chairas e vales interiores; Rururbano (diseminado); Mesotemperado inferior & 6.35 & 1.23 & 5.17 \\ 
  Canons; Bosque; Termotemperado & 4.76 & 1.40 & 3.39 \\ 
  Chairas e vales interiores; Bosque; Termotemperado & 4.76 & 2.38 & 2.00 \\ 
  Chairas e vales interiores; Rururbano (diseminado); Termotemperado & 4.76 & 2.59 & 1.84 \\ 
  Canons; Agrosistema extensivo; Mesomediterráneo & 3.17 & 0.35 & 9.08 \\ 
  Chairas e vales interiores; Agrosistema intensivo (mosaico agroforestal); Termotemperado & 3.17 & 1.55 & 2.05 \\ 
  Chairas e vales interiores; Matogueira e rochedo; Mesomediterráneo & 3.17 & 2.99 & 1.06 \\ 
  Chairas e vales interiores; Viñedo; Mesomediterráneo & 3.17 & 1.34 & 2.37 \\ 
  Serras; Agrosistema extensivo; Mesotemperado superior & 3.17 & 3.22 & 0.99 \\ 
  Serras; Matogueira e rochedo; Mesotemperado inferior & 3.17 & 5.11 & 0.62 \\ 
  Canons; Bosque; Mesomediterráneo & 1.59 & 0.28 & 5.74 \\ 
  Canons; Matogueira e rochedo; Termotemperado & 1.59 & 0.70 & 2.27 \\ 
  Canons; Viñedo; Termotemperado & 1.59 & 0.45 & 3.57 \\ 
  Chairas e vales interiores; Agrosistema extensivo; Mesomediterráneo & 1.59 & 1.24 & 1.28 \\ 
  Chairas e vales interiores; Agrosistema intensivo (mosaico agroforestal); Mesomediterráneo & 1.59 & 0.15 & 10.38 \\ 
  Chairas e vales interiores; Agrosistema intensivo (mosaico agroforestal); Mesotemperado inferior & 1.59 & 4.80 & 0.33 \\ 
  Chairas e vales interiores; Agrosistema intensivo (plantacion forestal); Mesotemperado inferior & 1.59 & 1.58 & 1.01 \\ 
  Chairas e vales interiores; Agrosistema intensivo (plantacion forestal); Termotemperado & 1.59 & 2.44 & 0.65 \\ 
  Chairas e vales interiores; Bosque; Mesomediterráneo & 1.59 & 0.54 & 2.92 \\ 
  Chairas e vales interiores; Matogueira e rochedo; Mesotemperado inferior & 1.59 & 4.17 & 0.38 \\ 
  Chairas e vales interiores; Matogueira e rochedo; Termotemperado & 1.59 & 4.28 & 0.37 \\ 
  Chairas e vales interiores; Urbano; Termotemperado & 1.59 & 0.86 & 1.85 \\ 
  Chairas e vales interiores; Viñedo; Termotemperado & 1.59 & 2.21 & 0.72 \\ 
  Serras; Agrosistema extensivo; Mesotemperado inferior & 1.59 & 4.45 & 0.36 \\ 
  Serras; Agrosistema intensivo (superficie de cultivo); Mesotemperado inferior & 1.59 & 0.33 & 4.83 \\ 
  Serras; Conxunto Historico; Mesotemperado inferior & 1.59 & 0.02 & 63.86 \\ 
  Serras; Viñedo; Mesotemperado inferior & 1.59 & 0.08 & 19.13 \\ 
   \hline
\end{tabular}
\end{table}
% latex table generated in R 3.2.2 by xtable 1.8-0 package
% Fri Dec  4 16:54:42 2015
\begin{table}[p]
\centering
\caption{Frecuencia de aparición de valores patrimoniais identificados na participación pública e frecuencia de tipos asociados Serras Orientais} 
\label{vsixotpat5}
\begin{tabular}{lrrr}
  \hline
Tipo de paisaxe & F.Aparic (\%) & F.Tipo (\%) & Ratio \\ 
  \hline
Serras; Agrosistema extensivo; Supra e orotemperado & 30.00 & 15.17 & 1.98 \\ 
  Serras; Agrosistema extensivo; Mesotemperado superior & 16.67 & 11.84 & 1.41 \\ 
  Serras; Matogueira e rochedo; Supra e orotemperado & 10.00 & 14.82 & 0.67 \\ 
  Vales sublitorais; Bosque; Mesotemperado superior & 10.00 & 3.48 & 2.87 \\ 
  Serras; Bosque; Supra e orotemperado & 6.67 & 5.16 & 1.29 \\ 
  Vales sublitorais; Bosque; Mesotemperado inferior & 6.67 & 2.82 & 2.36 \\ 
  Canons; Bosque; Mesotemperado inferior & 3.33 & 0.28 & 11.88 \\ 
  Chairas e vales interiores; Agrosistema extensivo; Mesotemperado inferior & 3.33 & 0.29 & 11.43 \\ 
  Chairas e vales interiores; Agrosistema extensivo; Mesotemperado superior & 3.33 & 0.46 & 7.23 \\ 
  Serras; Agrosistema intensivo (mosaico agroforestal); Mesotemperado superior & 3.33 & 1.96 & 1.70 \\ 
  Serras; Agrosistema intensivo (plantacion forestal); Supra e orotemperado & 3.33 & 3.43 & 0.97 \\ 
  Serras; Bosque; Mesotemperado superior & 3.33 & 3.78 & 0.88 \\ 
   \hline
\end{tabular}
\end{table}
% latex table generated in R 3.2.2 by xtable 1.8-0 package
% Fri Dec  4 16:54:42 2015
\begin{table}[p]
\centering
\caption{Frecuencia de aparición de valores patrimoniais identificados na participación pública e frecuencia de tipos asociados Chairas e Fosas Luguesas} 
\label{vsixotpat6}
\begin{tabular}{lrrr}
  \hline
Tipo de paisaxe & F.Aparic (\%) & F.Tipo (\%) & Ratio \\ 
  \hline
Chairas e vales interiores; Agrosistema extensivo; Mesotemperado inferior & 15.79 & 9.57 & 1.65 \\ 
  Chairas e vales interiores; Agrosistema extensivo; Mesotemperado superior & 15.79 & 17.50 & 0.90 \\ 
  Chairas e vales interiores; Bosque; Mesotemperado inferior & 13.16 & 1.29 & 10.18 \\ 
  Chairas e vales interiores; Rururbano (diseminado); Mesotemperado superior & 13.16 & 1.94 & 6.77 \\ 
  Chairas e vales interiores; Agrosistema intensivo (mosaico agroforestal); Mesotemperado superior & 10.53 & 14.58 & 0.72 \\ 
  Chairas e vales interiores; Agrosistema intensivo (superficie de cultivo); Mesotemperado superior & 5.26 & 4.58 & 1.15 \\ 
  Chairas e vales interiores; Conxunto Historico; Mesotemperado superior & 5.26 & 0.01 & 734.12 \\ 
  Chairas e vales interiores; Conxunto Historico; Termotemperado & 5.26 & 0.01 & 458.17 \\ 
  Serras; Agrosistema extensivo; Mesotemperado superior & 5.26 & 10.14 & 0.52 \\ 
  Chairas e vales interiores; Agrosistema intensivo (mosaico agroforestal); Mesotemperado inferior & 2.63 & 6.48 & 0.41 \\ 
  Chairas e vales interiores; Agrosistema intensivo (plantacion forestal); Mesotemperado superior & 2.63 & 2.61 & 1.01 \\ 
  Chairas e vales interiores; Rururbano (diseminado); Mesotemperado inferior & 2.63 & 1.41 & 1.86 \\ 
  Serras; Agrosistema extensivo; Supra e orotemperado & 2.63 & 1.09 & 2.41 \\ 
   \hline
\end{tabular}
\end{table}
% latex table generated in R 3.2.2 by xtable 1.8-0 package
% Fri Dec  4 16:54:42 2015
\begin{table}[p]
\centering
\caption{Frecuencia de aparición de valores patrimoniais identificados na participación pública e frecuencia de tipos asociados Galicia Central} 
\label{vsixotpat7}
\begin{tabular}{lrrr}
  \hline
Tipo de paisaxe & F.Aparic (\%) & F.Tipo (\%) & Ratio \\ 
  \hline
Vales sublitorais; Rururbano (diseminado); Termotemperado & 12.31 & 1.94 & 6.34 \\ 
  Vales sublitorais; Agrosistema intensivo (mosaico agroforestal); Mesotemperado inferior & 11.54 & 23.79 & 0.49 \\ 
  Vales sublitorais; Agrosistema extensivo; Mesotemperado inferior & 10.77 & 10.39 & 1.04 \\ 
  Vales sublitorais; Rururbano (diseminado); Mesotemperado inferior & 9.23 & 3.66 & 2.52 \\ 
  Serras; Agrosistema extensivo; Mesotemperado superior & 6.92 & 7.06 & 0.98 \\ 
  Vales sublitorais; Agrosistema intensivo (mosaico agroforestal); Termotemperado & 6.92 & 5.79 & 1.20 \\ 
  Vales sublitorais; Agrosistema intensivo (superficie de cultivo); Mesotemperado inferior & 6.92 & 2.45 & 2.83 \\ 
  Vales sublitorais; Urbano; Mesotemperado inferior & 5.38 & 0.49 & 10.91 \\ 
  Serras; Matogueira e rochedo; Mesotemperado superior & 4.62 & 5.00 & 0.92 \\ 
  Vales sublitorais; Conxunto Historico; Mesotemperado inferior & 3.85 & 0.02 & 218.96 \\ 
  Serras; Agrosistema extensivo; Mesotemperado inferior & 3.08 & 2.67 & 1.15 \\ 
  Vales sublitorais; Bosque; Mesotemperado inferior & 2.31 & 0.61 & 3.81 \\ 
  Chairas e vales interiores; Urbano; Termotemperado & 1.54 & 0.00 & 327.71 \\ 
  Vales sublitorais; Agrosistema extensivo; Termotemperado & 1.54 & 0.96 & 1.60 \\ 
  Vales sublitorais; Agrosistema intensivo (plantacion forestal); Termotemperado & 1.54 & 1.21 & 1.27 \\ 
  Vales sublitorais; Matogueira e rochedo; Termotemperado & 1.54 & 0.95 & 1.62 \\ 
   \hline
\end{tabular}
\end{table}
% latex table generated in R 3.2.2 by xtable 1.8-0 package
% Fri Dec  4 16:54:42 2015
\begin{table}[p]
\centering
\caption{Frecuencia de aparición de valores patrimoniais identificados na participación pública e frecuencia de tipos asociados Chairas, Fosas e Serras Ourensás} 
\label{vsixotpat8}
\begin{tabular}{lrrr}
  \hline
Tipo de paisaxe & F.Aparic (\%) & F.Tipo (\%) & Ratio \\ 
  \hline
Serras; Agrosistema extensivo; Mesotemperado inferior & 12.86 & 9.02 & 1.42 \\ 
  Serras; Agrosistema extensivo; Supra e orotemperado & 10.00 & 3.09 & 3.24 \\ 
  Chairas e vales interiores; Agrosistema extensivo; Mesotemperado inferior & 8.57 & 12.05 & 0.71 \\ 
  Serras; Matogueira e rochedo; Mesotemperado superior & 8.57 & 9.99 & 0.86 \\ 
  Chairas e vales interiores; Matogueira e rochedo; Mesotemperado inferior & 7.14 & 4.79 & 1.49 \\ 
  Serras; Matogueira e rochedo; Mesotemperado inferior & 7.14 & 8.98 & 0.80 \\ 
  Serras; Bosque; Mesotemperado superior & 5.71 & 1.61 & 3.55 \\ 
  Chairas e vales interiores; Bosque; Mesotemperado inferior & 4.29 & 2.80 & 1.53 \\ 
  Chairas e vales interiores; Matogueira e rochedo; Termotemperado & 4.29 & 2.16 & 1.98 \\ 
  Chairas e vales interiores; Rururbano (diseminado); Mesotemperado inferior & 4.29 & 1.22 & 3.51 \\ 
  Serras; Agrosistema extensivo; Mesotemperado superior & 4.29 & 4.93 & 0.87 \\ 
  Chairas e vales interiores; Agrosistema extensivo; Termotemperado & 2.86 & 1.77 & 1.61 \\ 
  Chairas e vales interiores; Agrosistema intensivo (superficie de cultivo); Mesotemperado inferior & 2.86 & 7.47 & 0.38 \\ 
  Chairas e vales interiores; Conxunto Historico; Mesotemperado inferior & 2.86 & 0.01 & 282.11 \\ 
  Serras; Agrosistema intensivo (superficie de cultivo); Supra e orotemperado & 2.86 & 0.88 & 3.26 \\ 
  Chairas e vales interiores; Agrosistema intensivo (mosaico agroforestal); Mesotemperado inferior & 1.43 & 1.44 & 0.99 \\ 
  Chairas e vales interiores; Agrosistema intensivo (mosaico agroforestal); Termotemperado & 1.43 & 1.20 & 1.19 \\ 
  Chairas e vales interiores; Agrosistema intensivo (superficie de cultivo); no data & 1.43 & 0.02 & 79.59 \\ 
  Chairas e vales interiores; Urbano; Mesotemperado inferior & 1.43 & 0.08 & 17.78 \\ 
  Chairas e vales interiores; Viñedo; Termotemperado & 1.43 & 0.65 & 2.20 \\ 
  Serras; Agrosistema intensivo (superficie de cultivo); Mesotemperado inferior & 1.43 & 1.62 & 0.88 \\ 
  Serras; Bosque; Mesotemperado inferior & 1.43 & 2.68 & 0.53 \\ 
  Serras; Matogueira e rochedo; Supra e orotemperado & 1.43 & 8.42 & 0.17 \\ 
   \hline
\end{tabular}
\end{table}
% latex table generated in R 3.2.2 by xtable 1.8-0 package
% Fri Dec  4 16:54:42 2015
\begin{table}[p]
\centering
\caption{Frecuencia de aparición de valores patrimoniais identificados na participación pública e frecuencia de tipos asociados Serras Surorientais} 
\label{vsixotpat9}
\begin{tabular}{lrrr}
  \hline
Tipo de paisaxe & F.Aparic (\%) & F.Tipo (\%) & Ratio \\ 
  \hline
Serras; Agrosistema extensivo; Mesotemperado inferior & 27.59 & 7.06 & 3.91 \\ 
  Canons; Viñedo; Mesomediterráneo & 13.79 & 0.02 & 582.49 \\ 
  Serras; Agrosistema extensivo; Supra e orotemperado & 10.34 & 7.54 & 1.37 \\ 
  Serras; Agrosistema extensivo; Mesomediterráneo & 6.90 & 0.18 & 37.39 \\ 
  Serras; Agrosistema extensivo; Mesotemperado superior & 6.90 & 10.41 & 0.66 \\ 
  Serras; Matogueira e rochedo; Supra e orotemperado & 6.90 & 37.48 & 0.18 \\ 
  Canons; Bosque; Mesotemperado inferior & 3.45 & 0.80 & 4.29 \\ 
  Canons; Matogueira e rochedo; Mesomediterráneo & 3.45 & 0.50 & 6.91 \\ 
  Canons; Viñedo; Mesotemperado inferior & 3.45 & 0.04 & 92.36 \\ 
  Serras; Agrosistema intensivo (superficie de cultivo); Mesotemperado inferior & 3.45 & 0.47 & 7.27 \\ 
  Serras; Bosque; Mesotemperado inferior & 3.45 & 1.96 & 1.76 \\ 
  Serras; Rururbano (diseminado); Mesotemperado superior & 3.45 & 0.12 & 28.20 \\ 
  Vales sublitorais; Agrosistema extensivo; Mesomediterráneo & 3.45 & 0.00 & 1364.60 \\ 
  Vales sublitorais; Viñedo; Mesomediterráneo & 3.45 & 0.00 & 2130.70 \\ 
   \hline
\end{tabular}
\end{table}
% latex table generated in R 3.2.2 by xtable 1.8-0 package
% Fri Dec  4 16:54:42 2015
\begin{table}[p]
\centering
\caption{Frecuencia de aparición de valores patrimoniais identificados na participación pública e frecuencia de tipos asociados Galicia Setentrional} 
\label{vsixotpat10}
\begin{tabular}{lrrr}
  \hline
Tipo de paisaxe & F.Aparic (\%) & F.Tipo (\%) & Ratio \\ 
  \hline
Vales sublitorais; Agrosistema intensivo (mosaico agroforestal); Mesotemperado inferior & 27.27 & 12.50 & 2.18 \\ 
  Litoral Cantabro-Atlantico; Agrosistema intensivo (mosaico agroforestal); Mesotemperado inferior & 13.64 & 1.86 & 7.34 \\ 
  Serras; Turbeira; Mesotemperado superior & 13.64 & 6.59 & 2.07 \\ 
  Serras; Agrosistema intensivo (mosaico agroforestal); Mesotemperado superior & 9.09 & 3.06 & 2.97 \\ 
  Vales sublitorais; Agrosistema intensivo (plantacion forestal); Mesotemperado inferior & 9.09 & 11.84 & 0.77 \\ 
  Vales sublitorais; Agrosistema intensivo (plantacion forestal); Mesotemperado superior & 9.09 & 2.01 & 4.53 \\ 
  Litoral Cantabro-Atlantico; Agrosistema intensivo (mosaico agroforestal); Termotemperado & 4.55 & 5.96 & 0.76 \\ 
  Litoral Cantabro-Atlantico; Rururbano (diseminado); Termotemperado & 4.55 & 2.78 & 1.64 \\ 
  Serras; Turbeira; Mesotemperado inferior & 4.55 & 0.68 & 6.65 \\ 
  Vales sublitorais; Rururbano (diseminado); Mesotemperado inferior & 4.55 & 0.53 & 8.51 \\ 
   \hline
\end{tabular}
\end{table}
% latex table generated in R 3.2.2 by xtable 1.8-0 package
% Fri Dec  4 16:54:42 2015
\begin{table}[p]
\centering
\caption{Frecuencia de aparición de valores patrimoniais identificados na participación pública e frecuencia de tipos asociados Chairas e Fosas Occidentais} 
\label{vsixotpat11}
\begin{tabular}{lrrr}
  \hline
Tipo de paisaxe & F.Aparic (\%) & F.Tipo (\%) & Ratio \\ 
  \hline
Litoral Cantabro-Atlantico; Agrosistema intensivo (mosaico agroforestal); Termotemperado & 15.38 & 8.86 & 1.74 \\ 
  Vales sublitorais; Agrosistema intensivo (plantacion forestal); Mesotemperado inferior & 15.38 & 8.33 & 1.85 \\ 
  Vales sublitorais; Agrosistema extensivo; Mesotemperado inferior & 12.82 & 4.56 & 2.81 \\ 
  Vales sublitorais; Agrosistema intensivo (mosaico agroforestal); Termotemperado & 10.26 & 8.83 & 1.16 \\ 
  Litoral Cantabro-Atlantico; Conxunto Historico; Termotemperado & 7.69 & 0.12 & 63.02 \\ 
  Vales sublitorais; Agrosistema intensivo (mosaico agroforestal); Mesotemperado inferior & 7.69 & 18.05 & 0.43 \\ 
  Vales sublitorais; Rururbano (diseminado); Mesotemperado inferior & 7.69 & 1.88 & 4.08 \\ 
  Vales sublitorais; Agrosistema intensivo (superficie de cultivo); Termotemperado & 5.13 & 0.70 & 7.35 \\ 
  Vales sublitorais; Rururbano (diseminado); Termotemperado & 5.13 & 2.48 & 2.07 \\ 
  Litoral Cantabro-Atlantico; Agrosistema extensivo; no data & 2.56 & 0.01 & 334.04 \\ 
  Litoral Cantabro-Atlantico; Rururbano (diseminado); Termotemperado & 2.56 & 2.77 & 0.92 \\ 
  Litoral Cantabro-Atlantico; Urbano; Termotemperado & 2.56 & 0.44 & 5.77 \\ 
  Vales sublitorais; Agrosistema intensivo (mosaico agroforestal); Mesotemperado superior & 2.56 & 4.01 & 0.64 \\ 
  Vales sublitorais; Agrosistema intensivo (plantacion forestal); Termotemperado & 2.56 & 3.89 & 0.66 \\ 
   \hline
\end{tabular}
\end{table}
% latex table generated in R 3.2.2 by xtable 1.8-0 package
% Fri Dec  4 16:54:42 2015
\begin{table}[p]
\centering
\caption{Frecuencia de aparición de valores patrimoniais identificados na participación pública e frecuencia de tipos asociados Rías Baixas} 
\label{vsixotpat12}
\begin{tabular}{lrrr}
  \hline
Tipo de paisaxe & F.Aparic (\%) & F.Tipo (\%) & Ratio \\ 
  \hline
Serras; Matogueira e rochedo; Mesotemperado superior & 16.13 & 4.07 & 3.96 \\ 
  Litoral Cantabro-Atlantico; Rururbano (diseminado); Termotemperado & 9.68 & 16.40 & 0.59 \\ 
  Vales sublitorais; Rururbano (diseminado); Termotemperado & 9.68 & 6.06 & 1.60 \\ 
  Litoral Cantabro-Atlantico; Agrosistema intensivo (mosaico agroforestal); Termotemperado & 6.45 & 6.56 & 0.98 \\ 
  Litoral Cantabro-Atlantico; Matogueira e rochedo; Termotemperado & 6.45 & 2.61 & 2.47 \\ 
  Serras; Matogueira e rochedo; Mesotemperado inferior & 6.45 & 4.31 & 1.50 \\ 
  Vales sublitorais; Agrosistema intensivo (plantacion forestal); Termotemperado & 6.45 & 8.25 & 0.78 \\ 
  Vales sublitorais; Matogueira e rochedo; Mesotemperado inferior & 6.45 & 5.63 & 1.15 \\ 
  Litoral Cantabro-Atlantico; Agrosistema intensivo (mosaico agroforestal); no data & 3.23 & 0.11 & 30.40 \\ 
  Litoral Cantabro-Atlantico; Agrosistema intensivo (plantacion forestal); Termotemperado & 3.23 & 4.94 & 0.65 \\ 
  Litoral Cantabro-Atlantico; Bosque; Termotemperado & 3.23 & 0.04 & 74.53 \\ 
  Litoral Cantabro-Atlantico; Urbano; no data & 3.23 & 0.20 & 16.46 \\ 
  Litoral Cantabro-Atlantico; Viñedo; Termotemperado & 3.23 & 1.24 & 2.60 \\ 
  Serras; Rururbano (diseminado); Mesotemperado inferior & 3.23 & 0.18 & 18.10 \\ 
  Vales sublitorais; Agrosistema intensivo (mosaico agroforestal); Mesotemperado inferior & 3.23 & 3.76 & 0.86 \\ 
  Vales sublitorais; Agrosistema intensivo (mosaico agroforestal); Termotemperado & 3.23 & 5.62 & 0.57 \\ 
  Vales sublitorais; Agrosistema intensivo (plantacion forestal); Mesotemperado superior & 3.23 & 0.62 & 5.21 \\ 
  Vales sublitorais; Matogueira e rochedo; Termotemperado & 3.23 & 5.27 & 0.61 \\ 
   \hline
\end{tabular}
\end{table}
