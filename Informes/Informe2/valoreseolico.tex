% latex table generated in R 3.2.2 by xtable 1.8-0 package
% Fri Dec  4 16:54:41 2015
\begin{table}[p]
\centering
\caption{Frecuencia de aparición de xeneradores eólicos e frecuencia de tipos asociados Golfo Ártabro} 
\label{veolico1}
\begin{tabular}{lrrr}
  \hline
Tipo de paisaxe & F.Aparic (\%) & F.Tipo (\%) & Ratio \\ 
  \hline
Serras; Turbeira; Mesotemperado superior & 35.06 & 1.50 & 23.37 \\ 
  Serras; Matogueira e rochedo; Mesotemperado superior & 24.68 & 2.97 & 8.31 \\ 
  Serras; Turbeira; Mesotemperado inferior & 7.79 & 0.03 & 300.17 \\ 
  Serras; Agrosistema extensivo; Mesotemperado superior & 5.19 & 1.59 & 3.27 \\ 
  Vales sublitorais; Agrosistema intensivo (mosaico agroforestal); Mesotemperado inferior & 5.19 & 14.40 & 0.36 \\ 
  Serras; Agrosistema extensivo; Supra e orotemperado & 3.90 & 0.11 & 36.67 \\ 
  Serras; Agrosistema intensivo (mosaico agroforestal); Supra e orotemperado & 3.90 & 0.04 & 91.62 \\ 
  Vales sublitorais; Turbeira; Mesotemperado superior & 3.90 & 0.38 & 10.28 \\ 
  Serras; Agrosistema extensivo; Mesotemperado inferior & 2.60 & 0.18 & 14.17 \\ 
  Serras; Turbeira; Supra e orotemperado & 2.60 & 0.02 & 118.39 \\ 
  Serras; Agrosistema intensivo (mosaico agroforestal); Mesotemperado inferior & 1.30 & 0.36 & 3.66 \\ 
  Serras; Agrosistema intensivo (plantacion forestal); Mesotemperado superior & 1.30 & 0.68 & 1.92 \\ 
  Serras; Agrosistema intensivo (plantacion forestal); Supra e orotemperado & 1.30 & 0.04 & 33.73 \\ 
  Serras; Matogueira e rochedo; Supra e orotemperado & 1.30 & 0.09 & 15.18 \\ 
   \hline
\end{tabular}
\end{table}
% latex table generated in R 3.2.2 by xtable 1.8-0 package
% Fri Dec  4 16:54:41 2015
\begin{table}[p]
\centering
\caption{Frecuencia de aparición de xeneradores eólicos e frecuencia de tipos asociados A Mariña - Baixo Eo} 
\label{veolico2}
\begin{tabular}{lrrr}
  \hline
Tipo de paisaxe & F.Aparic (\%) & F.Tipo (\%) & Ratio \\ 
  \hline
Serras; Turbeira; Mesotemperado superior & 58.42 & 2.35 & 24.88 \\ 
  Serras; Agrosistema intensivo (plantacion forestal); Mesotemperado superior & 21.78 & 4.01 & 5.44 \\ 
  Serras; Matogueira e rochedo; Mesotemperado superior & 8.91 & 2.07 & 4.30 \\ 
  Serras; Agrosistema intensivo (plantacion forestal); Mesotemperado inferior & 5.94 & 2.24 & 2.65 \\ 
  Serras; Turbeira; Mesotemperado inferior & 3.96 & 0.28 & 14.09 \\ 
   \hline
\end{tabular}
\end{table}
% latex table generated in R 3.2.2 by xtable 1.8-0 package
% Fri Dec  4 16:54:41 2015
\begin{table}[p]
\centering
\caption{Frecuencia de aparición de xeneradores eólicos e frecuencia de tipos asociados Costa Sur - Baixo Miño} 
\label{veolico3}
\begin{tabular}{lrrr}
  \hline
Tipo de paisaxe & F.Aparic (\%) & F.Tipo (\%) & Ratio \\ 
  \hline
Serras; Matogueira e rochedo; Supra e orotemperado & 74.07 & 4.26 & 17.41 \\ 
  Serras; Matogueira e rochedo; Mesotemperado superior & 17.28 & 5.89 & 2.93 \\ 
  Serras; Agrosistema intensivo (plantacion forestal); Supra e orotemperado & 6.79 & 0.26 & 25.68 \\ 
  Serras; Matogueira e rochedo; Mesotemperado inferior & 1.85 & 5.06 & 0.37 \\ 
   \hline
\end{tabular}
\end{table}
% latex table generated in R 3.2.2 by xtable 1.8-0 package
% Fri Dec  4 16:54:42 2015
\begin{table}[p]
\centering
\caption{Frecuencia de aparición de xeneradores eólicos e frecuencia de tipos asociados Ribeiras Encaixadas do Miño e do Sil} 
\label{veolico4}
\begin{tabular}{lrrr}
  \hline
Tipo de paisaxe & F.Aparic (\%) & F.Tipo (\%) & Ratio \\ 
  \hline
Serras; Matogueira e rochedo; Supra e orotemperado & 48.44 & 4.06 & 11.94 \\ 
  Serras; Matogueira e rochedo; Mesotemperado inferior & 15.62 & 5.11 & 3.06 \\ 
  Serras; Bosque; Supra e orotemperado & 12.50 & 0.18 & 69.63 \\ 
  Serras; Matogueira e rochedo; Mesotemperado superior & 9.38 & 5.27 & 1.78 \\ 
  Serras; Agrosistema intensivo (plantacion forestal); Mesotemperado superior & 6.25 & 0.66 & 9.44 \\ 
  Serras; Agrosistema intensivo (plantacion forestal); Supra e orotemperado & 4.69 & 0.53 & 8.76 \\ 
  Serras; Agrosistema intensivo (plantacion forestal); Mesotemperado inferior & 1.56 & 0.92 & 1.69 \\ 
  Serras; Bosque; Mesotemperado inferior & 1.56 & 1.46 & 1.07 \\ 
   \hline
\end{tabular}
\end{table}
% latex table generated in R 3.2.2 by xtable 1.8-0 package
% Fri Dec  4 16:54:42 2015
\begin{table}[p]
\centering
\caption{Frecuencia de aparición de xeneradores eólicos e frecuencia de tipos asociados Serras Orientais} 
\label{veolico5}
\begin{tabular}{lrrr}
  \hline
Tipo de paisaxe & F.Aparic (\%) & F.Tipo (\%) & Ratio \\ 
  \hline
Serras; Matogueira e rochedo; Supra e orotemperado & 45.89 & 14.82 & 3.10 \\ 
  Serras; Agrosistema extensivo; Supra e orotemperado & 25.34 & 15.17 & 1.67 \\ 
  Serras; Agrosistema intensivo (plantacion forestal); Supra e orotemperado & 9.59 & 3.43 & 2.80 \\ 
  Serras; Matogueira e rochedo; Mesotemperado superior & 8.90 & 6.90 & 1.29 \\ 
  Serras; Agrosistema intensivo (mosaico agroforestal); Supra e orotemperado & 6.16 & 2.44 & 2.52 \\ 
  Serras; Agrosistema extensivo; Mesotemperado superior & 4.11 & 11.84 & 0.35 \\ 
   \hline
\end{tabular}
\end{table}
% latex table generated in R 3.2.2 by xtable 1.8-0 package
% Fri Dec  4 16:54:42 2015
\begin{table}[p]
\centering
\caption{Frecuencia de aparición de xeneradores eólicos e frecuencia de tipos asociados Chairas e Fosas Luguesas} 
\label{veolico6}
\begin{tabular}{lrrr}
  \hline
Tipo de paisaxe & F.Aparic (\%) & F.Tipo (\%) & Ratio \\ 
  \hline
Serras; Turbeira; Supra e orotemperado & 29.97 & 0.48 & 62.50 \\ 
  Serras; Matogueira e rochedo; Supra e orotemperado & 18.96 & 0.84 & 22.63 \\ 
  Serras; Agrosistema intensivo (plantacion forestal); Supra e orotemperado & 18.04 & 0.44 & 41.00 \\ 
  Serras; Turbeira; Mesotemperado superior & 11.31 & 1.14 & 9.96 \\ 
  Serras; Matogueira e rochedo; Mesotemperado superior & 4.89 & 1.78 & 2.75 \\ 
  Chairas e vales interiores; Agrosistema intensivo (plantacion forestal); Mesotemperado superior & 3.98 & 2.61 & 1.52 \\ 
  Serras; Agrosistema intensivo (mosaico agroforestal); Mesotemperado superior & 3.06 & 5.41 & 0.56 \\ 
  Serras; Agrosistema extensivo; Mesotemperado superior & 2.75 & 10.14 & 0.27 \\ 
  Serras; Agrosistema extensivo; Supra e orotemperado & 2.75 & 1.09 & 2.52 \\ 
  Serras; Agrosistema intensivo (plantacion forestal); Mesotemperado superior & 1.53 & 1.33 & 1.15 \\ 
   \hline
\end{tabular}
\end{table}
% latex table generated in R 3.2.2 by xtable 1.8-0 package
% Fri Dec  4 16:54:42 2015
\begin{table}[p]
\centering
\caption{Frecuencia de aparición de xeneradores eólicos e frecuencia de tipos asociados Galicia Central} 
\label{veolico7}
\begin{tabular}{lrrr}
  \hline
Tipo de paisaxe & F.Aparic (\%) & F.Tipo (\%) & Ratio \\ 
  \hline
Serras; Matogueira e rochedo; Supra e orotemperado & 43.48 & 2.19 & 19.81 \\ 
  Serras; Matogueira e rochedo; Mesotemperado superior & 22.25 & 5.00 & 4.45 \\ 
  Serras; Matogueira e rochedo; Mesotemperado inferior & 12.53 & 1.61 & 7.77 \\ 
  Serras; Agrosistema extensivo; Supra e orotemperado & 7.16 & 0.78 & 9.18 \\ 
  Vales sublitorais; Matogueira e rochedo; Mesotemperado superior & 3.84 & 1.07 & 3.59 \\ 
  Serras; Agrosistema intensivo (plantacion forestal); Mesotemperado superior & 3.58 & 0.62 & 5.81 \\ 
  Serras; Agrosistema extensivo; Mesotemperado superior & 2.30 & 7.06 & 0.33 \\ 
  Serras; Agrosistema intensivo (mosaico agroforestal); Supra e orotemperado & 1.02 & 0.14 & 7.27 \\ 
   \hline
\end{tabular}
\end{table}
% latex table generated in R 3.2.2 by xtable 1.8-0 package
% Fri Dec  4 16:54:42 2015
\begin{table}[p]
\centering
\caption{Frecuencia de aparición de xeneradores eólicos e frecuencia de tipos asociados Chairas, Fosas e Serras Ourensás} 
\label{veolico8}
\begin{tabular}{lrrr}
  \hline
Tipo de paisaxe & F.Aparic (\%) & F.Tipo (\%) & Ratio \\ 
  \hline
Serras; Matogueira e rochedo; Supra e orotemperado & 63.81 & 8.42 & 7.58 \\ 
  Serras; Matogueira e rochedo; Mesotemperado superior & 27.62 & 9.99 & 2.77 \\ 
  Serras; Agrosistema intensivo (plantacion forestal); Supra e orotemperado & 4.76 & 0.79 & 6.00 \\ 
  Serras; Agrosistema intensivo (plantacion forestal); Mesotemperado superior & 1.90 & 1.11 & 1.71 \\ 
  Serras; Bosque; Supra e orotemperado & 1.90 & 0.97 & 1.95 \\ 
   \hline
\end{tabular}
\end{table}
% latex table generated in R 3.2.2 by xtable 1.8-0 package
% Fri Dec  4 16:54:42 2015
\begin{table}[p]
\centering
\caption{Frecuencia de aparición de xeneradores eólicos e frecuencia de tipos asociados Serras Surorientais} 
\label{veolico9}
\begin{tabular}{lrrr}
  \hline
Tipo de paisaxe & F.Aparic (\%) & F.Tipo (\%) & Ratio \\ 
  \hline
Serras; Matogueira e rochedo; Supra e orotemperado & 75.51 & 37.48 & 2.01 \\ 
  Serras; Agrosistema intensivo (superficie de cultivo); Supra e orotemperado & 10.20 & 0.88 & 11.65 \\ 
  Serras; Agrosistema extensivo; Supra e orotemperado & 6.12 & 7.54 & 0.81 \\ 
  Serras; Agrosistema intensivo (mosaico agroforestal); Supra e orotemperado & 6.12 & 0.71 & 8.68 \\ 
  Serras; Agrosistema intensivo (plantacion forestal); no data & 2.04 & 0.02 & 121.42 \\ 
   \hline
\end{tabular}
\end{table}
% latex table generated in R 3.2.2 by xtable 1.8-0 package
% Fri Dec  4 16:54:42 2015
\begin{table}[p]
\centering
\caption{Frecuencia de aparición de xeneradores eólicos e frecuencia de tipos asociados Galicia Setentrional} 
\label{veolico10}
\begin{tabular}{lrrr}
  \hline
Tipo de paisaxe & F.Aparic (\%) & F.Tipo (\%) & Ratio \\ 
  \hline
Serras; Matogueira e rochedo; Mesotemperado superior & 28.10 & 6.61 & 4.25 \\ 
  Serras; Turbeira; Mesotemperado superior & 24.54 & 6.59 & 3.72 \\ 
  Serras; Turbeira; Supra e orotemperado & 21.11 & 5.12 & 4.12 \\ 
  Serras; Agrosistema intensivo (plantacion forestal); Mesotemperado superior & 4.68 & 2.79 & 1.68 \\ 
  Serras; Matogueira e rochedo; Mesotemperado inferior & 3.03 & 1.53 & 1.98 \\ 
  Serras; Agrosistema intensivo (mosaico agroforestal); Mesotemperado superior & 2.77 & 3.06 & 0.91 \\ 
  Vales sublitorais; Matogueira e rochedo; Mesotemperado inferior & 2.64 & 0.99 & 2.66 \\ 
  Serras; Agrosistema intensivo (plantacion forestal); Mesotemperado inferior & 1.78 & 1.96 & 0.91 \\ 
  Serras; Turbeira; Mesotemperado inferior & 1.72 & 0.68 & 2.51 \\ 
  Vales sublitorais; Agrosistema intensivo (plantacion forestal); Mesotemperado inferior & 1.65 & 11.84 & 0.14 \\ 
  Serras; Agrosistema extensivo; Mesotemperado superior & 1.39 & 4.36 & 0.32 \\ 
   \hline
\end{tabular}
\end{table}
% latex table generated in R 3.2.2 by xtable 1.8-0 package
% Fri Dec  4 16:54:42 2015
\begin{table}[p]
\centering
\caption{Frecuencia de aparición de xeneradores eólicos e frecuencia de tipos asociados Chairas e Fosas Occidentais} 
\label{veolico11}
\begin{tabular}{lrrr}
  \hline
Tipo de paisaxe & F.Aparic (\%) & F.Tipo (\%) & Ratio \\ 
  \hline
Litoral Cantabro-Atlantico; Matogueira e rochedo; Termotemperado & 24.22 & 5.61 & 4.31 \\ 
  Vales sublitorais; Matogueira e rochedo; Mesotemperado superior & 17.49 & 3.30 & 5.30 \\ 
  Vales sublitorais; Matogueira e rochedo; Mesotemperado inferior & 17.34 & 7.15 & 2.42 \\ 
  Vales sublitorais; Agrosistema intensivo (plantacion forestal); Mesotemperado superior & 15.99 & 1.59 & 10.09 \\ 
  Vales sublitorais; Agrosistema intensivo (plantacion forestal); Mesotemperado inferior & 11.66 & 8.33 & 1.40 \\ 
  Vales sublitorais; Turbeira; Mesotemperado superior & 4.93 & 0.26 & 18.79 \\ 
  Vales sublitorais; Matogueira e rochedo; Termotemperado & 4.04 & 1.68 & 2.40 \\ 
  Vales sublitorais; Agrosistema intensivo (mosaico agroforestal); Mesotemperado superior & 1.79 & 4.01 & 0.45 \\ 
   \hline
\end{tabular}
\end{table}
% latex table generated in R 3.2.2 by xtable 1.8-0 package
% Fri Dec  4 16:54:42 2015
\begin{table}[p]
\centering
\caption{Frecuencia de aparición de xeneradores eólicos e frecuencia de tipos asociados Rías Baixas} 
\label{veolico12}
\begin{tabular}{lrrr}
  \hline
Tipo de paisaxe & F.Aparic (\%) & F.Tipo (\%) & Ratio \\ 
  \hline
Serras; Matogueira e rochedo; Supra e orotemperado & 30.05 & 2.49 & 12.09 \\ 
  Vales sublitorais; Matogueira e rochedo; Mesotemperado superior & 19.95 & 0.86 & 23.24 \\ 
  Serras; Matogueira e rochedo; Mesotemperado superior & 18.13 & 4.07 & 4.46 \\ 
  Serras; Matogueira e rochedo; Mesotemperado inferior & 8.29 & 4.31 & 1.92 \\ 
  Vales sublitorais; Matogueira e rochedo; Mesotemperado inferior & 6.99 & 5.63 & 1.24 \\ 
  Vales sublitorais; Agrosistema intensivo (plantacion forestal); Mesotemperado superior & 5.18 & 0.62 & 8.36 \\ 
  Vales sublitorais; Turbeira; Mesotemperado superior & 3.11 & 0.04 & 73.68 \\ 
  Vales sublitorais; Agrosistema intensivo (plantacion forestal); Mesotemperado inferior & 2.33 & 3.99 & 0.58 \\ 
  Serras; Agrosistema intensivo (plantacion forestal); Mesotemperado inferior & 1.30 & 1.60 & 0.81 \\ 
  Vales sublitorais; Agrosistema intensivo (mosaico agroforestal); Mesotemperado superior & 1.30 & 0.41 & 3.15 \\ 
  Vales sublitorais; Agrosistema extensivo; Mesotemperado superior & 1.04 & 0.09 & 11.22 \\ 
   \hline
\end{tabular}
\end{table}
