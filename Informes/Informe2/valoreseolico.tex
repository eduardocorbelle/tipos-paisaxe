% latex table generated in R 3.2.2 by xtable 1.8-0 package
% Wed Dec  2 19:45:31 2015
\begin{table}[p]
\centering
\caption{Frecuencia de aparición de xeneradores eólicos e frecuencia de tipos asociados Golfo Ártabro} 
\label{veolico1}
\begin{tabular}{lrrr}
  \hline
Tipo de paisaxe & F.Aparic (\%) & F.Tipo (\%) & Ratio \\ 
  \hline
Serras; Turbeira; Mesotemperado superior & 35.06 & 0.71 & 49.50 \\ 
  Serras; Matogueira e rochedo; Mesotemperado superior & 24.68 & 5.03 & 4.91 \\ 
  Serras; Turbeira; Mesotemperado inferior & 7.79 & 0.05 & 146.24 \\ 
  Serras; Agrosistema extensivo; Mesotemperado superior & 5.19 & 5.76 & 0.90 \\ 
  Vales sublitorais; Agrosistema intensivo (mosaico agroforestal); Mesotemperado inferior & 5.19 & 7.54 & 0.69 \\ 
  Serras; Agrosistema extensivo; Supra e orotemperado & 3.90 & 2.50 & 1.56 \\ 
  Serras; Agrosistema intensivo (mosaico agroforestal); Supra e orotemperado & 3.90 & 0.37 & 10.48 \\ 
  Vales sublitorais; Turbeira; Mesotemperado superior & 3.90 & 0.06 & 64.63 \\ 
  Serras; Agrosistema extensivo; Mesotemperado inferior & 2.60 & 2.57 & 1.01 \\ 
  Serras; Turbeira; Supra e orotemperado & 2.60 & 0.40 & 6.53 \\ 
  Serras; Agrosistema intensivo (mosaico agroforestal); Mesotemperado inferior & 1.30 & 0.50 & 2.61 \\ 
  Serras; Agrosistema intensivo (plantacion forestal); Mesotemperado superior & 1.30 & 1.07 & 1.21 \\ 
  Serras; Agrosistema intensivo (plantacion forestal); Supra e orotemperado & 1.30 & 0.93 & 1.39 \\ 
  Serras; Matogueira e rochedo; Supra e orotemperado & 1.30 & 6.12 & 0.21 \\ 
   \hline
\end{tabular}
\end{table}
% latex table generated in R 3.2.2 by xtable 1.8-0 package
% Wed Dec  2 19:45:31 2015
\begin{table}[p]
\centering
\caption{Frecuencia de aparición de xeneradores eólicos e frecuencia de tipos asociados A Mariña - Baixo Eo} 
\label{veolico2}
\begin{tabular}{lrrr}
  \hline
Tipo de paisaxe & F.Aparic (\%) & F.Tipo (\%) & Ratio \\ 
  \hline
Serras; Turbeira; Mesotemperado superior & 58.42 & 0.71 & 82.46 \\ 
  Serras; Agrosistema intensivo (plantacion forestal); Mesotemperado superior & 21.78 & 1.07 & 20.30 \\ 
  Serras; Matogueira e rochedo; Mesotemperado superior & 8.91 & 5.03 & 1.77 \\ 
  Serras; Agrosistema intensivo (plantacion forestal); Mesotemperado inferior & 5.94 & 0.82 & 7.21 \\ 
  Serras; Turbeira; Mesotemperado inferior & 3.96 & 0.05 & 74.33 \\ 
   \hline
\end{tabular}
\end{table}
% latex table generated in R 3.2.2 by xtable 1.8-0 package
% Wed Dec  2 19:45:31 2015
\begin{table}[p]
\centering
\caption{Frecuencia de aparición de xeneradores eólicos e frecuencia de tipos asociados Costa Sur - Baixo Miño} 
\label{veolico3}
\begin{tabular}{lrrr}
  \hline
Tipo de paisaxe & F.Aparic (\%) & F.Tipo (\%) & Ratio \\ 
  \hline
Serras; Matogueira e rochedo; Supra e orotemperado & 74.07 & 6.12 & 12.11 \\ 
  Serras; Matogueira e rochedo; Mesotemperado superior & 17.28 & 5.03 & 3.44 \\ 
  Serras; Agrosistema intensivo (plantacion forestal); Supra e orotemperado & 6.79 & 0.93 & 7.28 \\ 
  Serras; Matogueira e rochedo; Mesotemperado inferior & 1.85 & 2.73 & 0.68 \\ 
   \hline
\end{tabular}
\end{table}
% latex table generated in R 3.2.2 by xtable 1.8-0 package
% Wed Dec  2 19:45:31 2015
\begin{table}[p]
\centering
\caption{Frecuencia de aparición de xeneradores eólicos e frecuencia de tipos asociados Ribeiras Encaixadas do Miño e do Sil} 
\label{veolico4}
\begin{tabular}{lrrr}
  \hline
Tipo de paisaxe & F.Aparic (\%) & F.Tipo (\%) & Ratio \\ 
  \hline
Serras; Matogueira e rochedo; Supra e orotemperado & 48.44 & 6.12 & 7.92 \\ 
  Serras; Matogueira e rochedo; Mesotemperado inferior & 15.62 & 2.73 & 5.73 \\ 
  Serras; Bosque; Supra e orotemperado & 12.50 & 0.74 & 16.97 \\ 
  Serras; Matogueira e rochedo; Mesotemperado superior & 9.38 & 5.03 & 1.86 \\ 
  Serras; Agrosistema intensivo (plantacion forestal); Mesotemperado superior & 6.25 & 1.07 & 5.82 \\ 
  Serras; Agrosistema intensivo (plantacion forestal); Supra e orotemperado & 4.69 & 0.93 & 5.03 \\ 
  Serras; Agrosistema intensivo (plantacion forestal); Mesotemperado inferior & 1.56 & 0.82 & 1.90 \\ 
  Serras; Bosque; Mesotemperado inferior & 1.56 & 0.69 & 2.26 \\ 
   \hline
\end{tabular}
\end{table}
% latex table generated in R 3.2.2 by xtable 1.8-0 package
% Wed Dec  2 19:45:31 2015
\begin{table}[p]
\centering
\caption{Frecuencia de aparición de xeneradores eólicos e frecuencia de tipos asociados Serras Orientais} 
\label{veolico5}
\begin{tabular}{lrrr}
  \hline
Tipo de paisaxe & F.Aparic (\%) & F.Tipo (\%) & Ratio \\ 
  \hline
Serras; Matogueira e rochedo; Supra e orotemperado & 45.89 & 6.12 & 7.50 \\ 
  Serras; Agrosistema extensivo; Supra e orotemperado & 25.34 & 2.50 & 10.15 \\ 
  Serras; Agrosistema intensivo (plantacion forestal); Supra e orotemperado & 9.59 & 0.93 & 10.29 \\ 
  Serras; Matogueira e rochedo; Mesotemperado superior & 8.90 & 5.03 & 1.77 \\ 
  Serras; Agrosistema intensivo (mosaico agroforestal); Supra e orotemperado & 6.16 & 0.37 & 16.59 \\ 
  Serras; Agrosistema extensivo; Mesotemperado superior & 4.11 & 5.76 & 0.71 \\ 
   \hline
\end{tabular}
\end{table}
% latex table generated in R 3.2.2 by xtable 1.8-0 package
% Wed Dec  2 19:45:31 2015
\begin{table}[p]
\centering
\caption{Frecuencia de aparición de xeneradores eólicos e frecuencia de tipos asociados Chairas e Fosas Luguesas} 
\label{veolico6}
\begin{tabular}{lrrr}
  \hline
Tipo de paisaxe & F.Aparic (\%) & F.Tipo (\%) & Ratio \\ 
  \hline
Serras; Turbeira; Supra e orotemperado & 29.97 & 0.40 & 75.31 \\ 
  Serras; Matogueira e rochedo; Supra e orotemperado & 18.96 & 6.12 & 3.10 \\ 
  Serras; Agrosistema intensivo (plantacion forestal); Supra e orotemperado & 18.04 & 0.93 & 19.35 \\ 
  Serras; Turbeira; Mesotemperado superior & 11.31 & 0.71 & 15.97 \\ 
  Serras; Matogueira e rochedo; Mesotemperado superior & 4.89 & 5.03 & 0.97 \\ 
  Chairas e vales interiores; Agrosistema intensivo (plantacion forestal); Mesotemperado superior & 3.98 & 0.42 & 9.42 \\ 
  Serras; Agrosistema intensivo (mosaico agroforestal); Mesotemperado superior & 3.06 & 1.77 & 1.72 \\ 
  Serras; Agrosistema extensivo; Mesotemperado superior & 2.75 & 5.76 & 0.48 \\ 
  Serras; Agrosistema extensivo; Supra e orotemperado & 2.75 & 2.50 & 1.10 \\ 
  Serras; Agrosistema intensivo (plantacion forestal); Mesotemperado superior & 1.53 & 1.07 & 1.42 \\ 
   \hline
\end{tabular}
\end{table}
% latex table generated in R 3.2.2 by xtable 1.8-0 package
% Wed Dec  2 19:45:31 2015
\begin{table}[p]
\centering
\caption{Frecuencia de aparición de xeneradores eólicos e frecuencia de tipos asociados Galicia Central} 
\label{veolico7}
\begin{tabular}{lrrr}
  \hline
Tipo de paisaxe & F.Aparic (\%) & F.Tipo (\%) & Ratio \\ 
  \hline
Serras; Matogueira e rochedo; Supra e orotemperado & 43.48 & 6.12 & 7.11 \\ 
  Serras; Matogueira e rochedo; Mesotemperado superior & 22.25 & 5.03 & 4.42 \\ 
  Serras; Matogueira e rochedo; Mesotemperado inferior & 12.53 & 2.73 & 4.60 \\ 
  Serras; Agrosistema extensivo; Supra e orotemperado & 7.16 & 2.50 & 2.87 \\ 
  Vales sublitorais; Matogueira e rochedo; Mesotemperado superior & 3.84 & 0.85 & 4.50 \\ 
  Serras; Agrosistema intensivo (plantacion forestal); Mesotemperado superior & 3.58 & 1.07 & 3.34 \\ 
  Serras; Agrosistema extensivo; Mesotemperado superior & 2.30 & 5.76 & 0.40 \\ 
  Serras; Agrosistema intensivo (mosaico agroforestal); Supra e orotemperado & 1.02 & 0.37 & 2.75 \\ 
   \hline
\end{tabular}
\end{table}
% latex table generated in R 3.2.2 by xtable 1.8-0 package
% Wed Dec  2 19:45:32 2015
\begin{table}[p]
\centering
\caption{Frecuencia de aparición de xeneradores eólicos e frecuencia de tipos asociados Chairas, Fosas e Serras Ourensás} 
\label{veolico8}
\begin{tabular}{lrrr}
  \hline
Tipo de paisaxe & F.Aparic (\%) & F.Tipo (\%) & Ratio \\ 
  \hline
Serras; Matogueira e rochedo; Supra e orotemperado & 63.81 & 6.12 & 10.43 \\ 
  Serras; Matogueira e rochedo; Mesotemperado superior & 27.62 & 5.03 & 5.49 \\ 
  Serras; Agrosistema intensivo (plantacion forestal); Supra e orotemperado & 4.76 & 0.93 & 5.11 \\ 
  Serras; Agrosistema intensivo (plantacion forestal); Mesotemperado superior & 1.90 & 1.07 & 1.77 \\ 
  Serras; Bosque; Supra e orotemperado & 1.90 & 0.74 & 2.59 \\ 
   \hline
\end{tabular}
\end{table}
% latex table generated in R 3.2.2 by xtable 1.8-0 package
% Wed Dec  2 19:45:32 2015
\begin{table}[p]
\centering
\caption{Frecuencia de aparición de xeneradores eólicos e frecuencia de tipos asociados Serras Surorientais} 
\label{veolico9}
\begin{tabular}{lrrr}
  \hline
Tipo de paisaxe & F.Aparic (\%) & F.Tipo (\%) & Ratio \\ 
  \hline
Serras; Matogueira e rochedo; Supra e orotemperado & 75.51 & 6.12 & 12.34 \\ 
  Serras; Agrosistema intensivo (superficie de cultivo); Supra e orotemperado & 10.20 & 0.23 & 44.12 \\ 
  Serras; Agrosistema extensivo; Supra e orotemperado & 6.12 & 2.50 & 2.45 \\ 
  Serras; Agrosistema intensivo (mosaico agroforestal); Supra e orotemperado & 6.12 & 0.37 & 16.47 \\ 
  Serras; Agrosistema intensivo (plantacion forestal); no data & 2.04 & 0.00 & 427.59 \\ 
   \hline
\end{tabular}
\end{table}
% latex table generated in R 3.2.2 by xtable 1.8-0 package
% Wed Dec  2 19:45:32 2015
\begin{table}[p]
\centering
\caption{Frecuencia de aparición de xeneradores eólicos e frecuencia de tipos asociados Galicia Setentrional} 
\label{veolico10}
\begin{tabular}{lrrr}
  \hline
Tipo de paisaxe & F.Aparic (\%) & F.Tipo (\%) & Ratio \\ 
  \hline
Serras; Matogueira e rochedo; Mesotemperado superior & 28.10 & 5.03 & 5.59 \\ 
  Serras; Turbeira; Mesotemperado superior & 24.54 & 0.71 & 34.64 \\ 
  Serras; Turbeira; Supra e orotemperado & 21.11 & 0.40 & 53.04 \\ 
  Serras; Agrosistema intensivo (plantacion forestal); Mesotemperado superior & 4.68 & 1.07 & 4.36 \\ 
  Serras; Matogueira e rochedo; Mesotemperado inferior & 3.03 & 2.73 & 1.11 \\ 
  Serras; Agrosistema intensivo (mosaico agroforestal); Mesotemperado superior & 2.77 & 1.77 & 1.56 \\ 
  Vales sublitorais; Matogueira e rochedo; Mesotemperado inferior & 2.64 & 1.89 & 1.39 \\ 
  Serras; Agrosistema intensivo (plantacion forestal); Mesotemperado inferior & 1.78 & 0.82 & 2.16 \\ 
  Serras; Turbeira; Mesotemperado inferior & 1.72 & 0.05 & 32.19 \\ 
  Vales sublitorais; Agrosistema intensivo (plantacion forestal); Mesotemperado inferior & 1.65 & 2.90 & 0.57 \\ 
  Serras; Agrosistema extensivo; Mesotemperado superior & 1.39 & 5.76 & 0.24 \\ 
   \hline
\end{tabular}
\end{table}
% latex table generated in R 3.2.2 by xtable 1.8-0 package
% Wed Dec  2 19:45:32 2015
\begin{table}[p]
\centering
\caption{Frecuencia de aparición de xeneradores eólicos e frecuencia de tipos asociados Chairas e Fosas Occidentais} 
\label{veolico11}
\begin{tabular}{lrrr}
  \hline
Tipo de paisaxe & F.Aparic (\%) & F.Tipo (\%) & Ratio \\ 
  \hline
Litoral Cantabro-Atlantico; Matogueira e rochedo; Termotemperado & 24.22 & 0.83 & 29.14 \\ 
  Vales sublitorais; Matogueira e rochedo; Mesotemperado superior & 17.49 & 0.85 & 20.54 \\ 
  Vales sublitorais; Matogueira e rochedo; Mesotemperado inferior & 17.34 & 1.89 & 9.16 \\ 
  Vales sublitorais; Agrosistema intensivo (plantacion forestal); Mesotemperado superior & 15.99 & 0.60 & 26.77 \\ 
  Vales sublitorais; Agrosistema intensivo (plantacion forestal); Mesotemperado inferior & 11.66 & 2.90 & 4.02 \\ 
  Vales sublitorais; Turbeira; Mesotemperado superior & 4.93 & 0.06 & 81.83 \\ 
  Vales sublitorais; Matogueira e rochedo; Termotemperado & 4.04 & 0.93 & 4.33 \\ 
  Vales sublitorais; Agrosistema intensivo (mosaico agroforestal); Mesotemperado superior & 1.79 & 1.52 & 1.18 \\ 
   \hline
\end{tabular}
\end{table}
% latex table generated in R 3.2.2 by xtable 1.8-0 package
% Wed Dec  2 19:45:32 2015
\begin{table}[p]
\centering
\caption{Frecuencia de aparición de xeneradores eólicos e frecuencia de tipos asociados Rías Baixas} 
\label{veolico12}
\begin{tabular}{lrrr}
  \hline
Tipo de paisaxe & F.Aparic (\%) & F.Tipo (\%) & Ratio \\ 
  \hline
Serras; Matogueira e rochedo; Supra e orotemperado & 30.05 & 6.12 & 4.91 \\ 
  Vales sublitorais; Matogueira e rochedo; Mesotemperado superior & 19.95 & 0.85 & 23.42 \\ 
  Serras; Matogueira e rochedo; Mesotemperado superior & 18.13 & 5.03 & 3.61 \\ 
  Serras; Matogueira e rochedo; Mesotemperado inferior & 8.29 & 2.73 & 3.04 \\ 
  Vales sublitorais; Matogueira e rochedo; Mesotemperado inferior & 6.99 & 1.89 & 3.69 \\ 
  Vales sublitorais; Agrosistema intensivo (plantacion forestal); Mesotemperado superior & 5.18 & 0.60 & 8.67 \\ 
  Vales sublitorais; Turbeira; Mesotemperado superior & 3.11 & 0.06 & 51.57 \\ 
  Vales sublitorais; Agrosistema intensivo (plantacion forestal); Mesotemperado inferior & 2.33 & 2.90 & 0.80 \\ 
  Serras; Agrosistema intensivo (plantacion forestal); Mesotemperado inferior & 1.30 & 0.82 & 1.57 \\ 
  Vales sublitorais; Agrosistema intensivo (mosaico agroforestal); Mesotemperado superior & 1.30 & 1.52 & 0.85 \\ 
  Vales sublitorais; Agrosistema extensivo; Mesotemperado superior & 1.04 & 1.21 & 0.86 \\ 
   \hline
\end{tabular}
\end{table}
