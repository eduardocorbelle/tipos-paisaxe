% latex table generated in R 3.2.2 by xtable 1.8-0 package
% Wed Dec  2 19:45:31 2015
\begin{table}[p]
\centering
\caption{Frecuencia de aparición de valores naturais identificados na participación pública e frecuencia de tipos asociados Golfo Ártabro} 
\label{vsixotnat1}
\begin{tabular}{lrrr}
  \hline
Tipo de paisaxe & F.Aparic (\%) & F.Tipo (\%) & Ratio \\ 
  \hline
Litoral Cantabro-Atlantico; Rururbano (diseminado); Termotemperado & 19.44 & 3.29 & 5.91 \\ 
  Canons; Bosque; Mesotemperado inferior & 11.11 & 0.30 & 37.60 \\ 
  Canons; Matogueira e rochedo; Mesotemperado inferior & 5.56 & 0.20 & 27.73 \\ 
  Litoral Cantabro-Atlantico; Agrosistema extensivo; Termotemperado & 5.56 & 0.16 & 35.25 \\ 
  Litoral Cantabro-Atlantico; Agrosistema intensivo (mosaico agroforestal); Termotemperado & 5.56 & 2.43 & 2.29 \\ 
  Litoral Cantabro-Atlantico; Matogueira e rochedo; Termotemperado & 5.56 & 0.83 & 6.69 \\ 
  Vales sublitorais; Agrosistema intensivo (mosaico agroforestal); Mesotemperado inferior & 5.56 & 7.54 & 0.74 \\ 
  Vales sublitorais; Bosque; Mesotemperado inferior & 5.56 & 0.49 & 11.23 \\ 
  Vales sublitorais; Bosque; Termotemperado & 5.56 & 0.14 & 39.17 \\ 
   & 5.56 &  &  \\ 
  Canons; Bosque; Termotemperado & 2.78 & 0.15 & 18.35 \\ 
  Litoral Cantabro-Atlantico; Agrosistema intensivo (plantacion forestal); Termotemperado & 2.78 & 1.32 & 2.11 \\ 
  Litoral Cantabro-Atlantico; Agrosistema intensivo (superficie de cultivo); Termotemperado & 2.78 & 0.22 & 12.66 \\ 
  Litoral Cantabro-Atlantico; Conxunto Historico; Termotemperado & 2.78 & 0.02 & 146.88 \\ 
  Serras; Agrosistema extensivo; Mesotemperado superior & 2.78 & 5.76 & 0.48 \\ 
  Serras; Turbeira; Mesotemperado superior & 2.78 & 0.71 & 3.92 \\ 
  Vales sublitorais; Agrosistema extensivo; Mesotemperado inferior & 2.78 & 2.73 & 1.02 \\ 
  Vales sublitorais; Agrosistema intensivo (plantacion forestal); Mesotemperado inferior & 2.78 & 2.90 & 0.96 \\ 
  Vales sublitorais; Matogueira e rochedo; Mesotemperado inferior & 2.78 & 1.89 & 1.47 \\ 
   \hline
\end{tabular}
\end{table}
% latex table generated in R 3.2.2 by xtable 1.8-0 package
% Wed Dec  2 19:45:31 2015
\begin{table}[p]
\centering
\caption{Frecuencia de aparición de valores naturais identificados na participación pública e frecuencia de tipos asociados A Mariña - Baixo Eo} 
\label{vsixotnat2}
\begin{tabular}{lrrr}
  \hline
Tipo de paisaxe & F.Aparic (\%) & F.Tipo (\%) & Ratio \\ 
  \hline
Vales sublitorais; Agrosistema intensivo (plantacion forestal); Mesotemperado inferior & 12.90 & 2.90 & 4.45 \\ 
   & 12.90 &  &  \\ 
  Litoral Cantabro-Atlantico; Rururbano (diseminado); Termotemperado & 9.68 & 3.29 & 2.94 \\ 
  Litoral Cantabro-Atlantico; Urbano; Termotemperado & 9.68 & 0.53 & 18.24 \\ 
  Litoral Cantabro-Atlantico; Agrosistema intensivo (mosaico agroforestal); no data & 6.45 & 0.04 & 151.73 \\ 
  Litoral Cantabro-Atlantico; Agrosistema intensivo (plantacion forestal); Mesotemperado inferior & 6.45 & 0.62 & 10.45 \\ 
  Litoral Cantabro-Atlantico; Conxunto Historico; Termotemperado & 6.45 & 0.02 & 341.14 \\ 
  Litoral Cantabro-Atlantico; Matogueira e rochedo; Mesotemperado inferior & 6.45 & 0.10 & 64.81 \\ 
  Serras; Turbeira; Mesotemperado superior & 6.45 & 0.71 & 9.11 \\ 
  Litoral Cantabro-Atlantico; Rururbano (diseminado); Mesotemperado inferior & 3.23 & 0.04 & 85.67 \\ 
  Litoral Cantabro-Atlantico; Urbano; no data & 3.23 & 0.04 & 72.04 \\ 
  Serras; Bosque; Mesotemperado superior & 3.23 & 1.00 & 3.23 \\ 
  Serras; Matogueira e rochedo; Supra e orotemperado & 3.23 & 6.12 & 0.53 \\ 
  Vales sublitorais; Agrosistema intensivo (mosaico agroforestal); Mesotemperado inferior & 3.23 & 7.54 & 0.43 \\ 
  Vales sublitorais; Bosque; Mesotemperado superior & 3.23 & 0.39 & 8.36 \\ 
  Vales sublitorais; Rururbano (diseminado); Mesotemperado inferior & 3.23 & 0.98 & 3.29 \\ 
   \hline
\end{tabular}
\end{table}
% latex table generated in R 3.2.2 by xtable 1.8-0 package
% Wed Dec  2 19:45:31 2015
\begin{table}[p]
\centering
\caption{Frecuencia de aparición de valores naturais identificados na participación pública e frecuencia de tipos asociados Costa Sur - Baixo Miño} 
\label{vsixotnat3}
\begin{tabular}{lrrr}
  \hline
Tipo de paisaxe & F.Aparic (\%) & F.Tipo (\%) & Ratio \\ 
  \hline
Litoral Cantabro-Atlantico; Rururbano (diseminado); Termotemperado & 11.11 & 3.29 & 3.38 \\ 
  Serras; Matogueira e rochedo; Mesotemperado inferior & 11.11 & 2.73 & 4.08 \\ 
  Serras; Agrosistema intensivo (plantacion forestal); Mesotemperado inferior & 9.52 & 0.82 & 11.56 \\ 
  Serras; Agrosistema intensivo (plantacion forestal); Termotemperado & 9.52 & 0.15 & 62.46 \\ 
  Serras; Matogueira e rochedo; Mesotemperado superior & 7.94 & 5.03 & 1.58 \\ 
  Serras; Turbeira; Mesotemperado inferior & 7.94 & 0.05 & 148.95 \\ 
  Litoral Cantabro-Atlantico; Agrosistema intensivo (plantacion forestal); Termotemperado & 4.76 & 1.32 & 3.62 \\ 
  Serras; Matogueira e rochedo; Termotemperado & 4.76 & 0.16 & 30.29 \\ 
  Vales sublitorais; Agrosistema intensivo (plantacion forestal); Termotemperado & 4.76 & 1.79 & 2.66 \\ 
  Vales sublitorais; Agrosistema intensivo (mosaico agroforestal); Termotemperado & 3.17 & 2.62 & 1.21 \\ 
  Vales sublitorais; Bosque; Termotemperado & 3.17 & 0.14 & 22.38 \\ 
  Vales sublitorais; Rururbano (diseminado); Termotemperado & 3.17 & 1.68 & 1.89 \\ 
  Chairas e vales interiores; Agrosistema intensivo (mosaico agroforestal); Mesotemperado inferior & 1.59 & 1.71 & 0.93 \\ 
  Chairas e vales interiores; Matogueira e rochedo; Mesotemperado inferior & 1.59 & 1.38 & 1.15 \\ 
  Litoral Cantabro-Atlantico; Agrosistema extensivo; Termotemperado & 1.59 & 0.16 & 10.07 \\ 
  Litoral Cantabro-Atlantico; Agrosistema intensivo (mosaico agroforestal); Termotemperado & 1.59 & 2.43 & 0.65 \\ 
  Litoral Cantabro-Atlantico; Agrosistema intensivo (superficie de cultivo); no data & 1.59 & 0.01 & 162.90 \\ 
  Litoral Cantabro-Atlantico; no data; Termotemperado & 1.59 & 0.00 & 420.36 \\ 
  Litoral Cantabro-Atlantico; Viñedo; Termotemperado & 1.59 & 0.14 & 11.47 \\ 
  no data; Agrosistema intensivo (plantacion forestal); Termotemperado & 1.59 & 0.03 & 57.51 \\ 
  no data; Rururbano (diseminado); Termotemperado & 1.59 & 0.05 & 35.06 \\ 
  Serras; Agrosistema extensivo; Mesotemperado inferior & 1.59 & 2.57 & 0.62 \\ 
  Serras; Agrosistema extensivo; Mesotemperado superior & 1.59 & 5.76 & 0.28 \\ 
  Vales sublitorais; Matogueira e rochedo; Termotemperado & 1.59 & 0.93 & 1.70 \\ 
   \hline
\end{tabular}
\end{table}
% latex table generated in R 3.2.2 by xtable 1.8-0 package
% Wed Dec  2 19:45:31 2015
\begin{table}[p]
\centering
\caption{Frecuencia de aparición de valores naturais identificados na participación pública e frecuencia de tipos asociados Ribeiras Encaixadas do Miño e do Sil} 
\label{vsixotnat4}
\begin{tabular}{lrrr}
  \hline
Tipo de paisaxe & F.Aparic (\%) & F.Tipo (\%) & Ratio \\ 
  \hline
Chairas e vales interiores; Bosque; Termotemperado & 13.33 & 0.29 & 45.74 \\ 
  Canons; Bosque; Mesotemperado inferior & 10.00 & 0.30 & 33.84 \\ 
  Canons; Bosque; Termotemperado & 10.00 & 0.15 & 66.07 \\ 
  Serras; Agrosistema extensivo; Mesotemperado inferior & 8.33 & 2.57 & 3.24 \\ 
  Serras; Matogueira e rochedo; Mesotemperado inferior & 6.67 & 2.73 & 2.45 \\ 
  Canons; Matogueira e rochedo; Mesotemperado inferior & 5.00 & 0.20 & 24.96 \\ 
  Chairas e vales interiores; Matogueira e rochedo; Mesomediterráneo & 5.00 & 0.27 & 18.65 \\ 
  Chairas e vales interiores; Matogueira e rochedo; Mesotemperado inferior & 5.00 & 1.38 & 3.62 \\ 
  Serras; Bosque; Mesotemperado inferior & 5.00 & 0.69 & 7.23 \\ 
  Canons; Matogueira e rochedo; Mesomediterráneo & 3.33 & 0.18 & 18.04 \\ 
  Chairas e vales interiores; Bosque; Mesomediterráneo & 3.33 & 0.06 & 59.08 \\ 
  Chairas e vales interiores; Viñedo; Termotemperado & 3.33 & 0.28 & 12.01 \\ 
  Serras; Agrosistema extensivo; Mesotemperado superior & 3.33 & 5.76 & 0.58 \\ 
  Canons; Agrosistema extensivo; Mesomediterráneo & 1.67 & 0.03 & 48.31 \\ 
  Canons; Agrosistema intensivo (plantacion forestal); Termotemperado & 1.67 & 0.07 & 23.23 \\ 
  Chairas e vales interiores; Agrosistema extensivo; Mesotemperado inferior & 1.67 & 3.58 & 0.47 \\ 
  Chairas e vales interiores; Agrosistema extensivo; Termotemperado & 1.67 & 0.61 & 2.72 \\ 
  Chairas e vales interiores; Agrosistema intensivo (mosaico agroforestal); Termotemperado & 1.67 & 0.34 & 4.91 \\ 
  Chairas e vales interiores; Agrosistema intensivo (plantacion forestal); Termotemperado & 1.67 & 0.37 & 4.52 \\ 
  Chairas e vales interiores; Viñedo; Mesomediterráneo & 1.67 & 0.11 & 14.89 \\ 
  Serras; Bosque; Termotemperado & 1.67 & 0.00 & 364.52 \\ 
  Serras; Matogueira e rochedo; Mesomediterráneo & 1.67 & 0.19 & 8.76 \\ 
  Serras; Matogueira e rochedo; Mesotemperado superior & 1.67 & 5.03 & 0.33 \\ 
  Serras; Matogueira e rochedo; Supra e orotemperado & 1.67 & 6.12 & 0.27 \\ 
   \hline
\end{tabular}
\end{table}
% latex table generated in R 3.2.2 by xtable 1.8-0 package
% Wed Dec  2 19:45:31 2015
\begin{table}[p]
\centering
\caption{Frecuencia de aparición de valores naturais identificados na participación pública e frecuencia de tipos asociados Serras Orientais} 
\label{vsixotnat5}
\begin{tabular}{lrrr}
  \hline
Tipo de paisaxe & F.Aparic (\%) & F.Tipo (\%) & Ratio \\ 
  \hline
Serras; Agrosistema extensivo; Supra e orotemperado & 17.57 & 2.50 & 7.03 \\ 
  Serras; Bosque; Supra e orotemperado & 13.51 & 0.74 & 18.35 \\ 
  Serras; Matogueira e rochedo; Supra e orotemperado & 13.51 & 6.12 & 2.21 \\ 
  Serras; Agrosistema extensivo; Mesotemperado superior & 10.81 & 5.76 & 1.88 \\ 
  Serras; Bosque; Mesotemperado superior & 9.46 & 1.00 & 9.46 \\ 
  Serras; Matogueira e rochedo; Mesotemperado superior & 5.41 & 5.03 & 1.07 \\ 
  Vales sublitorais; Bosque; Mesotemperado inferior & 5.41 & 0.49 & 10.92 \\ 
  Vales sublitorais; Agrosistema extensivo; Mesotemperado inferior & 4.05 & 2.73 & 1.48 \\ 
  Vales sublitorais; Agrosistema extensivo; Mesotemperado superior & 4.05 & 1.21 & 3.36 \\ 
  Vales sublitorais; Bosque; Mesotemperado superior & 4.05 & 0.39 & 10.51 \\ 
  Serras; Agrosistema extensivo; Mesotemperado inferior & 2.70 & 2.57 & 1.05 \\ 
  Serras; Agrosistema intensivo (mosaico agroforestal); Supra e orotemperado & 2.70 & 0.37 & 7.27 \\ 
  Canons; Agrosistema intensivo (plantacion forestal); Termotemperado & 1.35 & 0.07 & 18.84 \\ 
  Canons; Bosque; Mesotemperado inferior & 1.35 & 0.30 & 4.57 \\ 
  Canons; Matogueira e rochedo; Termotemperado & 1.35 & 0.09 & 15.67 \\ 
  Chairas e vales interiores; Agrosistema extensivo; Mesotemperado inferior & 1.35 & 3.58 & 0.38 \\ 
  Serras; Agrosistema intensivo (plantacion forestal); Supra e orotemperado & 1.35 & 0.93 & 1.45 \\ 
   \hline
\end{tabular}
\end{table}
% latex table generated in R 3.2.2 by xtable 1.8-0 package
% Wed Dec  2 19:45:31 2015
\begin{table}[p]
\centering
\caption{Frecuencia de aparición de valores naturais identificados na participación pública e frecuencia de tipos asociados Chairas e Fosas Luguesas} 
\label{vsixotnat6}
\begin{tabular}{lrrr}
  \hline
Tipo de paisaxe & F.Aparic (\%) & F.Tipo (\%) & Ratio \\ 
  \hline
Chairas e vales interiores; Agrosistema extensivo; Mesotemperado inferior & 21.62 & 3.58 & 6.03 \\ 
  Chairas e vales interiores; Agrosistema intensivo (mosaico agroforestal); Mesotemperado inferior & 13.51 & 1.71 & 7.91 \\ 
  Chairas e vales interiores; Bosque; Mesotemperado inferior & 13.51 & 0.77 & 17.66 \\ 
  Serras; Agrosistema intensivo (plantacion forestal); Supra e orotemperado & 13.51 & 0.93 & 14.50 \\ 
  Chairas e vales interiores; Agrosistema extensivo; Mesotemperado superior & 10.81 & 2.77 & 3.90 \\ 
  Chairas e vales interiores; Rururbano (diseminado); Mesotemperado superior & 5.41 & 0.30 & 18.03 \\ 
  Chairas e vales interiores; Urbano; Mesotemperado inferior & 5.41 & 0.06 & 96.97 \\ 
  Serras; Agrosistema extensivo; Mesotemperado superior & 5.41 & 5.76 & 0.94 \\ 
  Chairas e vales interiores; Bosque; Mesotemperado superior & 2.70 & 0.26 & 10.21 \\ 
  Serras; Agrosistema extensivo; Supra e orotemperado & 2.70 & 2.50 & 1.08 \\ 
  Serras; Matogueira e rochedo; Supra e orotemperado & 2.70 & 6.12 & 0.44 \\ 
  Serras; Turbeira; Mesotemperado superior & 2.70 & 0.71 & 3.82 \\ 
   \hline
\end{tabular}
\end{table}
% latex table generated in R 3.2.2 by xtable 1.8-0 package
% Wed Dec  2 19:45:31 2015
\begin{table}[p]
\centering
\caption{Frecuencia de aparición de valores naturais identificados na participación pública e frecuencia de tipos asociados Galicia Central} 
\label{vsixotnat7}
\begin{tabular}{lrrr}
  \hline
Tipo de paisaxe & F.Aparic (\%) & F.Tipo (\%) & Ratio \\ 
  \hline
Vales sublitorais; Agrosistema extensivo; Mesotemperado inferior & 19.09 & 2.73 & 6.98 \\ 
  Vales sublitorais; Agrosistema intensivo (mosaico agroforestal); Termotemperado & 9.09 & 2.62 & 3.47 \\ 
  Vales sublitorais; Rururbano (diseminado); Termotemperado & 9.09 & 1.68 & 5.41 \\ 
  Serras; Agrosistema extensivo; Mesotemperado superior & 6.36 & 5.76 & 1.11 \\ 
  Serras; Matogueira e rochedo; Mesotemperado superior & 6.36 & 5.03 & 1.27 \\ 
  Serras; Matogueira e rochedo; Mesotemperado inferior & 5.45 & 2.73 & 2.00 \\ 
  Vales sublitorais; Urbano; Mesotemperado inferior & 5.45 & 0.11 & 50.22 \\ 
  Vales sublitorais; Agrosistema extensivo; Termotemperado & 3.64 & 0.36 & 10.15 \\ 
  Vales sublitorais; Agrosistema intensivo (mosaico agroforestal); Mesotemperado inferior & 3.64 & 7.54 & 0.48 \\ 
  Serras; Agrosistema extensivo; Mesotemperado inferior & 2.73 & 2.57 & 1.06 \\ 
  Vales sublitorais; Agrosistema extensivo; Mesotemperado superior & 2.73 & 1.21 & 2.26 \\ 
  Vales sublitorais; Matogueira e rochedo; Mesotemperado inferior & 2.73 & 1.89 & 1.44 \\ 
  Serras; Agrosistema intensivo (plantacion forestal); Supra e orotemperado & 1.82 & 0.93 & 1.95 \\ 
  Serras; Agrosistema intensivo (superficie de cultivo); Mesotemperado superior & 1.82 & 0.95 & 1.91 \\ 
  Serras; Matogueira e rochedo; Supra e orotemperado & 1.82 & 6.12 & 0.30 \\ 
  Vales sublitorais; Agrosistema intensivo (plantacion forestal); Mesotemperado inferior & 1.82 & 2.90 & 0.63 \\ 
  Vales sublitorais; Agrosistema intensivo (plantacion forestal); Termotemperado & 1.82 & 1.79 & 1.02 \\ 
  Vales sublitorais; Bosque; Mesotemperado inferior & 1.82 & 0.49 & 3.67 \\ 
  Vales sublitorais; Matogueira e rochedo; Termotemperado & 1.82 & 0.93 & 1.95 \\ 
  Vales sublitorais; Rururbano (diseminado); Mesotemperado inferior & 1.82 & 0.98 & 1.85 \\ 
   \hline
\end{tabular}
\end{table}
% latex table generated in R 3.2.2 by xtable 1.8-0 package
% Wed Dec  2 19:45:32 2015
\begin{table}[p]
\centering
\caption{Frecuencia de aparición de valores naturais identificados na participación pública e frecuencia de tipos asociados Chairas, Fosas e Serras Ourensás} 
\label{vsixotnat8}
\begin{tabular}{lrrr}
  \hline
Tipo de paisaxe & F.Aparic (\%) & F.Tipo (\%) & Ratio \\ 
  \hline
Serras; Matogueira e rochedo; Supra e orotemperado & 13.85 & 6.12 & 2.26 \\ 
  Serras; Matogueira e rochedo; Mesotemperado superior & 12.31 & 5.03 & 2.45 \\ 
  Chairas e vales interiores; Matogueira e rochedo; Mesotemperado inferior & 10.77 & 1.38 & 7.79 \\ 
  Chairas e vales interiores; Agrosistema extensivo; Mesotemperado inferior & 9.23 & 3.58 & 2.58 \\ 
  Serras; Matogueira e rochedo; Mesotemperado inferior & 9.23 & 2.73 & 3.39 \\ 
  Chairas e vales interiores; Agrosistema intensivo (superficie de cultivo); Mesotemperado inferior & 6.15 & 1.16 & 5.31 \\ 
  Chairas e vales interiores; Bosque; Mesotemperado inferior & 6.15 & 0.77 & 8.04 \\ 
  Serras; Agrosistema extensivo; Mesotemperado inferior & 6.15 & 2.57 & 2.40 \\ 
  Chairas e vales interiores; Viñedo; Termotemperado & 4.62 & 0.28 & 16.63 \\ 
  Chairas e vales interiores; Rururbano (diseminado); Mesotemperado inferior & 3.08 & 0.51 & 6.01 \\ 
  Serras; Bosque; Supra e orotemperado & 3.08 & 0.74 & 4.18 \\ 
  Chairas e vales interiores; Agrosistema extensivo; Termotemperado & 1.54 & 0.61 & 2.51 \\ 
  Chairas e vales interiores; Agrosistema intensivo (mosaico agroforestal); Mesotemperado inferior & 1.54 & 1.71 & 0.90 \\ 
  Chairas e vales interiores; Agrosistema intensivo (mosaico agroforestal); Termotemperado & 1.54 & 0.34 & 4.53 \\ 
  Chairas e vales interiores; Agrosistema intensivo (superficie de cultivo); no data & 1.54 & 0.00 & 870.34 \\ 
  Chairas e vales interiores; Matogueira e rochedo; Termotemperado & 1.54 & 0.66 & 2.34 \\ 
  Chairas e vales interiores; Rururbano (diseminado); Termotemperado & 1.54 & 0.42 & 3.64 \\ 
  Serras; Agrosistema extensivo; Mesotemperado superior & 1.54 & 5.76 & 0.27 \\ 
  Serras; Bosque; Mesotemperado inferior & 1.54 & 0.69 & 2.23 \\ 
  Serras; Bosque; Mesotemperado superior & 1.54 & 1.00 & 1.54 \\ 
  Serras; Matogueira e rochedo; Termotemperado & 1.54 & 0.16 & 9.79 \\ 
   \hline
\end{tabular}
\end{table}
% latex table generated in R 3.2.2 by xtable 1.8-0 package
% Wed Dec  2 19:45:32 2015
\begin{table}[p]
\centering
\caption{Frecuencia de aparición de valores naturais identificados na participación pública e frecuencia de tipos asociados Serras Surorientais} 
\label{vsixotnat9}
\begin{tabular}{lrrr}
  \hline
Tipo de paisaxe & F.Aparic (\%) & F.Tipo (\%) & Ratio \\ 
  \hline
Serras; Matogueira e rochedo; Supra e orotemperado & 39.73 & 6.12 & 6.49 \\ 
  Serras; Agrosistema extensivo; Supra e orotemperado & 9.59 & 2.50 & 3.84 \\ 
  Serras; Agrosistema extensivo; Mesotemperado inferior & 6.85 & 2.57 & 2.67 \\ 
  Canons; Viñedo; Mesomediterráneo & 5.48 & 0.01 & 458.73 \\ 
  Serras; Agrosistema extensivo; Mesotemperado superior & 5.48 & 5.76 & 0.95 \\ 
  Serras; Bosque; Mesotemperado superior & 4.11 & 1.00 & 4.11 \\ 
  Serras; Matogueira e rochedo; Mesotemperado superior & 4.11 & 5.03 & 0.82 \\ 
  Canons; Matogueira e rochedo; Mesotemperado inferior & 2.74 & 0.20 & 13.68 \\ 
  Serras; Agrosistema intensivo (plantacion forestal); Supra e orotemperado & 2.74 & 0.93 & 2.94 \\ 
  Serras; Bosque; Mesotemperado inferior & 2.74 & 0.69 & 3.96 \\ 
  Serras; Matogueira e rochedo; Mesotemperado inferior & 2.74 & 2.73 & 1.01 \\ 
  Serras; Matogueira e rochedo; no data & 2.74 & 0.07 & 37.16 \\ 
  Canons; Agrosistema extensivo; Mesomediterráneo & 1.37 & 0.03 & 39.70 \\ 
  Canons; Matogueira e rochedo; Mesomediterráneo & 1.37 & 0.18 & 7.41 \\ 
  Canons; Viñedo; Mesotemperado inferior & 1.37 & 0.02 & 59.24 \\ 
  Serras; Agrosistema intensivo (superficie de cultivo); Mesotemperado superior & 1.37 & 0.95 & 1.44 \\ 
  Serras; Bosque; Supra e orotemperado & 1.37 & 0.74 & 1.86 \\ 
  Serras; Rururbano (diseminado); Mesotemperado superior & 1.37 & 0.11 & 12.33 \\ 
  Vales sublitorais; Agrosistema extensivo; Mesomediterráneo & 1.37 & 0.00 & 7288.15 \\ 
  Vales sublitorais; Viñedo; Mesomediterráneo & 1.37 & 0.00 & 11379.75 \\ 
   \hline
\end{tabular}
\end{table}
% latex table generated in R 3.2.2 by xtable 1.8-0 package
% Wed Dec  2 19:45:32 2015
\begin{table}[p]
\centering
\caption{Frecuencia de aparición de valores naturais identificados na participación pública e frecuencia de tipos asociados Galicia Setentrional} 
\label{vsixotnat10}
\begin{tabular}{lrrr}
  \hline
Tipo de paisaxe & F.Aparic (\%) & F.Tipo (\%) & Ratio \\ 
  \hline
Serras; Turbeira; Supra e orotemperado & 14.47 & 0.40 & 36.37 \\ 
  Serras; Turbeira; Mesotemperado superior & 10.53 & 0.71 & 14.86 \\ 
   & 9.21 &  &  \\ 
  Serras; Matogueira e rochedo; Mesotemperado superior & 7.89 & 5.03 & 1.57 \\ 
  Litoral Cantabro-Atlantico; Matogueira e rochedo; Termotemperado & 6.58 & 0.83 & 7.92 \\ 
  Vales sublitorais; Bosque; Mesotemperado inferior & 6.58 & 0.49 & 13.30 \\ 
  Litoral Cantabro-Atlantico; Agrosistema intensivo (mosaico agroforestal); Termotemperado & 5.26 & 2.43 & 2.17 \\ 
  Litoral Cantabro-Atlantico; Agrosistema intensivo (plantacion forestal); Termotemperado & 5.26 & 1.32 & 4.00 \\ 
  Litoral Cantabro-Atlantico; Matogueira e rochedo; no data & 3.95 & 0.09 & 46.38 \\ 
  Litoral Cantabro-Atlantico; Rururbano (diseminado); Termotemperado & 3.95 & 3.29 & 1.20 \\ 
  Litoral Cantabro-Atlantico; Agrosistema intensivo (plantacion forestal); no data & 2.63 & 0.03 & 84.29 \\ 
  Litoral Cantabro-Atlantico; Matogueira e rochedo; Mesotemperado inferior & 2.63 & 0.10 & 26.44 \\ 
  Vales sublitorais; Agrosistema intensivo (mosaico agroforestal); Mesotemperado inferior & 2.63 & 7.54 & 0.35 \\ 
  Litoral Cantabro-Atlantico; Agrosistema intensivo (mosaico agroforestal); Mesotemperado inferior & 1.32 & 0.29 & 4.49 \\ 
  Litoral Cantabro-Atlantico; Agrosistema intensivo (mosaico agroforestal); no data & 1.32 & 0.04 & 30.94 \\ 
  Litoral Cantabro-Atlantico; Agrosistema intensivo (plantacion forestal); Mesotemperado inferior & 1.32 & 0.62 & 2.13 \\ 
  Litoral Cantabro-Atlantico; Rururbano (diseminado); no data & 1.32 & 0.10 & 13.19 \\ 
  Serras; Agrosistema extensivo; Mesotemperado superior & 1.32 & 5.76 & 0.23 \\ 
  Serras; Agrosistema intensivo (mosaico agroforestal); Mesotemperado inferior & 1.32 & 0.50 & 2.65 \\ 
  Serras; Agrosistema intensivo (mosaico agroforestal); Mesotemperado superior & 1.32 & 1.77 & 0.74 \\ 
  Serras; Agrosistema intensivo (plantacion forestal); Mesotemperado superior & 1.32 & 1.07 & 1.23 \\ 
  Serras; Bosque; Mesotemperado inferior & 1.32 & 0.69 & 1.90 \\ 
  Serras; Turbeira; Mesotemperado inferior & 1.32 & 0.05 & 24.69 \\ 
  Vales sublitorais; Agrosistema intensivo (mosaico agroforestal); Mesotemperado superior & 1.32 & 1.52 & 0.87 \\ 
  Vales sublitorais; Agrosistema intensivo (plantacion forestal); Mesotemperado inferior & 1.32 & 2.90 & 0.45 \\ 
  Vales sublitorais; Agrosistema intensivo (plantacion forestal); Mesotemperado superior & 1.32 & 0.60 & 2.20 \\ 
  Vales sublitorais; Rururbano (diseminado); Mesotemperado inferior & 1.32 & 0.98 & 1.34 \\ 
   \hline
\end{tabular}
\end{table}
% latex table generated in R 3.2.2 by xtable 1.8-0 package
% Wed Dec  2 19:45:32 2015
\begin{table}[p]
\centering
\caption{Frecuencia de aparición de valores naturais identificados na participación pública e frecuencia de tipos asociados Chairas e Fosas Occidentais} 
\label{vsixotnat11}
\begin{tabular}{lrrr}
  \hline
Tipo de paisaxe & F.Aparic (\%) & F.Tipo (\%) & Ratio \\ 
  \hline
Litoral Cantabro-Atlantico; Matogueira e rochedo; Termotemperado & 28.77 & 0.83 & 34.62 \\ 
  Vales sublitorais; Agrosistema intensivo (plantacion forestal); Mesotemperado inferior & 10.96 & 2.90 & 3.78 \\ 
   & 10.96 &  &  \\ 
  Litoral Cantabro-Atlantico; Agrosistema intensivo (plantacion forestal); Termotemperado & 6.16 & 1.32 & 4.68 \\ 
  Vales sublitorais; Agrosistema intensivo (plantacion forestal); Termotemperado & 5.48 & 1.79 & 3.06 \\ 
  Vales sublitorais; Agrosistema intensivo (mosaico agroforestal); Termotemperado & 4.79 & 2.62 & 1.83 \\ 
  Litoral Cantabro-Atlantico; Matogueira e rochedo; no data & 4.11 & 0.09 & 48.29 \\ 
  Litoral Cantabro-Atlantico; Rururbano (diseminado); Termotemperado & 4.11 & 3.29 & 1.25 \\ 
  Vales sublitorais; Agrosistema extensivo; Mesotemperado inferior & 4.11 & 2.73 & 1.50 \\ 
  Vales sublitorais; Agrosistema intensivo (mosaico agroforestal); Mesotemperado inferior & 4.11 & 7.54 & 0.54 \\ 
  Vales sublitorais; Matogueira e rochedo; Mesotemperado inferior & 4.11 & 1.89 & 2.17 \\ 
  Litoral Cantabro-Atlantico; Conxunto Historico; Termotemperado & 2.05 & 0.02 & 108.65 \\ 
  Litoral Cantabro-Atlantico; Agrosistema intensivo (mosaico agroforestal); Termotemperado & 1.37 & 2.43 & 0.56 \\ 
  Litoral Cantabro-Atlantico; Agrosistema intensivo (plantacion forestal); Mesotemperado inferior & 1.37 & 0.62 & 2.22 \\ 
  Vales sublitorais; Matogueira e rochedo; Mesotemperado superior & 1.37 & 0.85 & 1.61 \\ 
   \hline
\end{tabular}
\end{table}
% latex table generated in R 3.2.2 by xtable 1.8-0 package
% Wed Dec  2 19:45:32 2015
\begin{table}[p]
\centering
\caption{Frecuencia de aparición de valores naturais identificados na participación pública e frecuencia de tipos asociados Rías Baixas} 
\label{vsixotnat12}
\begin{tabular}{lrrr}
  \hline
Tipo de paisaxe & F.Aparic (\%) & F.Tipo (\%) & Ratio \\ 
  \hline
Serras; Matogueira e rochedo; Mesotemperado superior & 18.85 & 5.03 & 3.75 \\ 
   & 11.48 &  &  \\ 
  Serras; Matogueira e rochedo; Mesotemperado inferior & 10.66 & 2.73 & 3.91 \\ 
  Litoral Cantabro-Atlantico; Matogueira e rochedo; Termotemperado & 7.38 & 0.83 & 8.88 \\ 
  Vales sublitorais; Rururbano (diseminado); Termotemperado & 7.38 & 1.68 & 4.39 \\ 
  Vales sublitorais; Matogueira e rochedo; Mesotemperado inferior & 6.56 & 1.89 & 3.46 \\ 
  Litoral Cantabro-Atlantico; Rururbano (diseminado); Termotemperado & 4.92 & 3.29 & 1.49 \\ 
  Litoral Cantabro-Atlantico; Agrosistema intensivo (plantacion forestal); Termotemperado & 4.10 & 1.32 & 3.11 \\ 
  Litoral Cantabro-Atlantico; Matogueira e rochedo; no data & 3.28 & 0.09 & 38.53 \\ 
  Vales sublitorais; Agrosistema intensivo (mosaico agroforestal); Termotemperado & 3.28 & 2.62 & 1.25 \\ 
  Vales sublitorais; Agrosistema intensivo (plantacion forestal); Termotemperado & 3.28 & 1.79 & 1.83 \\ 
  Serras; Agrosistema intensivo (plantacion forestal); Mesotemperado inferior & 2.46 & 0.82 & 2.99 \\ 
  Vales sublitorais; Agrosistema intensivo (plantacion forestal); Mesotemperado inferior & 2.46 & 2.90 & 0.85 \\ 
  Litoral Cantabro-Atlantico; Agrosistema intensivo (mosaico agroforestal); no data & 1.64 & 0.04 & 38.55 \\ 
  Vales sublitorais; Matogueira e rochedo; Mesotemperado superior & 1.64 & 0.85 & 1.93 \\ 
  Vales sublitorais; Matogueira e rochedo; Termotemperado & 1.64 & 0.93 & 1.76 \\ 
   \hline
\end{tabular}
\end{table}
