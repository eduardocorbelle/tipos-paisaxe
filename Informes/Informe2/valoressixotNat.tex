% latex table generated in R 3.2.2 by xtable 1.8-0 package
% Fri Dec  4 16:54:41 2015
\begin{table}[p]
\centering
\caption{Frecuencia de aparición de valores naturais identificados na participación pública e frecuencia de tipos asociados Golfo Ártabro} 
\label{vsixotnat1}
\begin{tabular}{lrrr}
  \hline
Tipo de paisaxe & F.Aparic (\%) & F.Tipo (\%) & Ratio \\ 
  \hline
Litoral Cantabro-Atlantico; Rururbano (diseminado); Termotemperado & 19.44 & 19.32 & 1.01 \\ 
  Canons; Bosque; Mesotemperado inferior & 11.11 & 1.04 & 10.64 \\ 
  Canons; Matogueira e rochedo; Mesotemperado inferior & 5.56 & 0.13 & 42.82 \\ 
  Litoral Cantabro-Atlantico; Agrosistema extensivo; Termotemperado & 5.56 & 0.82 & 6.76 \\ 
  Litoral Cantabro-Atlantico; Agrosistema intensivo (mosaico agroforestal); Termotemperado & 5.56 & 10.45 & 0.53 \\ 
  Litoral Cantabro-Atlantico; Matogueira e rochedo; Termotemperado & 5.56 & 0.87 & 6.39 \\ 
  Vales sublitorais; Agrosistema intensivo (mosaico agroforestal); Mesotemperado inferior & 5.56 & 14.40 & 0.39 \\ 
  Vales sublitorais; Bosque; Mesotemperado inferior & 5.56 & 0.58 & 9.57 \\ 
  Vales sublitorais; Bosque; Termotemperado & 5.56 & 0.03 & 184.72 \\ 
   & 5.56 &  &  \\ 
  Canons; Bosque; Termotemperado & 2.78 & 0.34 & 8.23 \\ 
  Litoral Cantabro-Atlantico; Agrosistema intensivo (plantacion forestal); Termotemperado & 2.78 & 2.83 & 0.98 \\ 
  Litoral Cantabro-Atlantico; Agrosistema intensivo (superficie de cultivo); Termotemperado & 2.78 & 0.43 & 6.46 \\ 
  Litoral Cantabro-Atlantico; Conxunto Historico; Termotemperado & 2.78 & 0.07 & 39.02 \\ 
  Serras; Agrosistema extensivo; Mesotemperado superior & 2.78 & 1.59 & 1.75 \\ 
  Serras; Turbeira; Mesotemperado superior & 2.78 & 1.50 & 1.85 \\ 
  Vales sublitorais; Agrosistema extensivo; Mesotemperado inferior & 2.78 & 0.90 & 3.08 \\ 
  Vales sublitorais; Agrosistema intensivo (plantacion forestal); Mesotemperado inferior & 2.78 & 7.86 & 0.35 \\ 
  Vales sublitorais; Matogueira e rochedo; Mesotemperado inferior & 2.78 & 1.43 & 1.94 \\ 
   \hline
\end{tabular}
\end{table}
% latex table generated in R 3.2.2 by xtable 1.8-0 package
% Fri Dec  4 16:54:41 2015
\begin{table}[p]
\centering
\caption{Frecuencia de aparición de valores naturais identificados na participación pública e frecuencia de tipos asociados A Mariña - Baixo Eo} 
\label{vsixotnat2}
\begin{tabular}{lrrr}
  \hline
Tipo de paisaxe & F.Aparic (\%) & F.Tipo (\%) & Ratio \\ 
  \hline
Vales sublitorais; Agrosistema intensivo (plantacion forestal); Mesotemperado inferior & 12.90 & 14.02 & 0.92 \\ 
   & 12.90 &  &  \\ 
  Litoral Cantabro-Atlantico; Rururbano (diseminado); Termotemperado & 9.68 & 3.92 & 2.47 \\ 
  Litoral Cantabro-Atlantico; Urbano; Termotemperado & 9.68 & 2.18 & 4.44 \\ 
  Litoral Cantabro-Atlantico; Agrosistema intensivo (mosaico agroforestal); no data & 6.45 & 0.11 & 56.47 \\ 
  Litoral Cantabro-Atlantico; Agrosistema intensivo (plantacion forestal); Mesotemperado inferior & 6.45 & 10.84 & 0.59 \\ 
  Litoral Cantabro-Atlantico; Conxunto Historico; Termotemperado & 6.45 & 0.07 & 97.19 \\ 
  Litoral Cantabro-Atlantico; Matogueira e rochedo; Mesotemperado inferior & 6.45 & 0.09 & 73.58 \\ 
  Serras; Turbeira; Mesotemperado superior & 6.45 & 2.35 & 2.75 \\ 
  Litoral Cantabro-Atlantico; Rururbano (diseminado); Mesotemperado inferior & 3.23 & 0.50 & 6.47 \\ 
  Litoral Cantabro-Atlantico; Urbano; no data & 3.23 & 0.03 & 103.29 \\ 
  Serras; Bosque; Mesotemperado superior & 3.23 & 0.68 & 4.75 \\ 
  Serras; Matogueira e rochedo; Supra e orotemperado & 3.23 & 0.24 & 13.59 \\ 
  Vales sublitorais; Agrosistema intensivo (mosaico agroforestal); Mesotemperado inferior & 3.23 & 12.56 & 0.26 \\ 
  Vales sublitorais; Bosque; Mesotemperado superior & 3.23 & 0.85 & 3.78 \\ 
  Vales sublitorais; Rururbano (diseminado); Mesotemperado inferior & 3.23 & 1.16 & 2.78 \\ 
   \hline
\end{tabular}
\end{table}
% latex table generated in R 3.2.2 by xtable 1.8-0 package
% Fri Dec  4 16:54:42 2015
\begin{table}[p]
\centering
\caption{Frecuencia de aparición de valores naturais identificados na participación pública e frecuencia de tipos asociados Costa Sur - Baixo Miño} 
\label{vsixotnat3}
\begin{tabular}{lrrr}
  \hline
Tipo de paisaxe & F.Aparic (\%) & F.Tipo (\%) & Ratio \\ 
  \hline
Litoral Cantabro-Atlantico; Rururbano (diseminado); Termotemperado & 11.11 & 11.41 & 0.97 \\ 
  Serras; Matogueira e rochedo; Mesotemperado inferior & 11.11 & 5.06 & 2.20 \\ 
  Serras; Agrosistema intensivo (plantacion forestal); Mesotemperado inferior & 9.52 & 2.04 & 4.68 \\ 
  Serras; Agrosistema intensivo (plantacion forestal); Termotemperado & 9.52 & 2.79 & 3.42 \\ 
  Serras; Matogueira e rochedo; Mesotemperado superior & 7.94 & 5.89 & 1.35 \\ 
  Serras; Turbeira; Mesotemperado inferior & 7.94 & 0.09 & 91.00 \\ 
  Litoral Cantabro-Atlantico; Agrosistema intensivo (plantacion forestal); Termotemperado & 4.76 & 2.51 & 1.90 \\ 
  Serras; Matogueira e rochedo; Termotemperado & 4.76 & 1.75 & 2.72 \\ 
  Vales sublitorais; Agrosistema intensivo (plantacion forestal); Termotemperado & 4.76 & 8.94 & 0.53 \\ 
  Vales sublitorais; Agrosistema intensivo (mosaico agroforestal); Termotemperado & 3.17 & 2.92 & 1.09 \\ 
  Vales sublitorais; Bosque; Termotemperado & 3.17 & 1.03 & 3.08 \\ 
  Vales sublitorais; Rururbano (diseminado); Termotemperado & 3.17 & 12.49 & 0.25 \\ 
  Chairas e vales interiores; Agrosistema intensivo (mosaico agroforestal); Mesotemperado inferior & 1.59 & 0.98 & 1.62 \\ 
  Chairas e vales interiores; Matogueira e rochedo; Mesotemperado inferior & 1.59 & 2.49 & 0.64 \\ 
  Litoral Cantabro-Atlantico; Agrosistema extensivo; Termotemperado & 1.59 & 0.39 & 4.08 \\ 
  Litoral Cantabro-Atlantico; Agrosistema intensivo (mosaico agroforestal); Termotemperado & 1.59 & 3.27 & 0.49 \\ 
  Litoral Cantabro-Atlantico; Agrosistema intensivo (superficie de cultivo); no data & 1.59 & 0.02 & 74.75 \\ 
  Litoral Cantabro-Atlantico; no data; Termotemperado & 1.59 & 0.03 & 51.10 \\ 
  Litoral Cantabro-Atlantico; Viñedo; Termotemperado & 1.59 & 0.59 & 2.67 \\ 
  no data; Agrosistema intensivo (plantacion forestal); Termotemperado & 1.59 & 0.69 & 2.30 \\ 
  no data; Rururbano (diseminado); Termotemperado & 1.59 & 1.13 & 1.40 \\ 
  Serras; Agrosistema extensivo; Mesotemperado inferior & 1.59 & 0.80 & 1.98 \\ 
  Serras; Agrosistema extensivo; Mesotemperado superior & 1.59 & 0.91 & 1.75 \\ 
  Vales sublitorais; Matogueira e rochedo; Termotemperado & 1.59 & 3.42 & 0.46 \\ 
   \hline
\end{tabular}
\end{table}
% latex table generated in R 3.2.2 by xtable 1.8-0 package
% Fri Dec  4 16:54:42 2015
\begin{table}[p]
\centering
\caption{Frecuencia de aparición de valores naturais identificados na participación pública e frecuencia de tipos asociados Ribeiras Encaixadas do Miño e do Sil} 
\label{vsixotnat4}
\begin{tabular}{lrrr}
  \hline
Tipo de paisaxe & F.Aparic (\%) & F.Tipo (\%) & Ratio \\ 
  \hline
Chairas e vales interiores; Bosque; Termotemperado & 13.33 & 2.38 & 5.60 \\ 
  Canons; Bosque; Mesotemperado inferior & 10.00 & 1.90 & 5.26 \\ 
  Canons; Bosque; Termotemperado & 10.00 & 1.40 & 7.12 \\ 
  Serras; Agrosistema extensivo; Mesotemperado inferior & 8.33 & 4.45 & 1.87 \\ 
  Serras; Matogueira e rochedo; Mesotemperado inferior & 6.67 & 5.11 & 1.30 \\ 
  Canons; Matogueira e rochedo; Mesotemperado inferior & 5.00 & 1.29 & 3.86 \\ 
  Chairas e vales interiores; Matogueira e rochedo; Mesomediterráneo & 5.00 & 2.99 & 1.67 \\ 
  Chairas e vales interiores; Matogueira e rochedo; Mesotemperado inferior & 5.00 & 4.17 & 1.20 \\ 
  Serras; Bosque; Mesotemperado inferior & 5.00 & 1.46 & 3.43 \\ 
  Canons; Matogueira e rochedo; Mesomediterráneo & 3.33 & 1.77 & 1.89 \\ 
  Chairas e vales interiores; Bosque; Mesomediterráneo & 3.33 & 0.54 & 6.13 \\ 
  Chairas e vales interiores; Viñedo; Termotemperado & 3.33 & 2.21 & 1.51 \\ 
  Serras; Agrosistema extensivo; Mesotemperado superior & 3.33 & 3.22 & 1.04 \\ 
  Canons; Agrosistema extensivo; Mesomediterráneo & 1.67 & 0.35 & 4.77 \\ 
  Canons; Agrosistema intensivo (plantacion forestal); Termotemperado & 1.67 & 0.67 & 2.47 \\ 
  Chairas e vales interiores; Agrosistema extensivo; Mesotemperado inferior & 1.67 & 9.17 & 0.18 \\ 
  Chairas e vales interiores; Agrosistema extensivo; Termotemperado & 1.67 & 3.71 & 0.45 \\ 
  Chairas e vales interiores; Agrosistema intensivo (mosaico agroforestal); Termotemperado & 1.67 & 1.55 & 1.07 \\ 
  Chairas e vales interiores; Agrosistema intensivo (plantacion forestal); Termotemperado & 1.67 & 2.44 & 0.68 \\ 
  Chairas e vales interiores; Viñedo; Mesomediterráneo & 1.67 & 1.34 & 1.24 \\ 
  Serras; Bosque; Termotemperado & 1.67 & 0.01 & 167.30 \\ 
  Serras; Matogueira e rochedo; Mesomediterráneo & 1.67 & 1.82 & 0.91 \\ 
  Serras; Matogueira e rochedo; Mesotemperado superior & 1.67 & 5.27 & 0.32 \\ 
  Serras; Matogueira e rochedo; Supra e orotemperado & 1.67 & 4.06 & 0.41 \\ 
   \hline
\end{tabular}
\end{table}
% latex table generated in R 3.2.2 by xtable 1.8-0 package
% Fri Dec  4 16:54:42 2015
\begin{table}[p]
\centering
\caption{Frecuencia de aparición de valores naturais identificados na participación pública e frecuencia de tipos asociados Serras Orientais} 
\label{vsixotnat5}
\begin{tabular}{lrrr}
  \hline
Tipo de paisaxe & F.Aparic (\%) & F.Tipo (\%) & Ratio \\ 
  \hline
Serras; Agrosistema extensivo; Supra e orotemperado & 17.57 & 15.17 & 1.16 \\ 
  Serras; Bosque; Supra e orotemperado & 13.51 & 5.16 & 2.62 \\ 
  Serras; Matogueira e rochedo; Supra e orotemperado & 13.51 & 14.82 & 0.91 \\ 
  Serras; Agrosistema extensivo; Mesotemperado superior & 10.81 & 11.84 & 0.91 \\ 
  Serras; Bosque; Mesotemperado superior & 9.46 & 3.78 & 2.50 \\ 
  Serras; Matogueira e rochedo; Mesotemperado superior & 5.41 & 6.90 & 0.78 \\ 
  Vales sublitorais; Bosque; Mesotemperado inferior & 5.41 & 2.82 & 1.92 \\ 
  Vales sublitorais; Agrosistema extensivo; Mesotemperado inferior & 4.05 & 3.89 & 1.04 \\ 
  Vales sublitorais; Agrosistema extensivo; Mesotemperado superior & 4.05 & 5.67 & 0.72 \\ 
  Vales sublitorais; Bosque; Mesotemperado superior & 4.05 & 3.48 & 1.16 \\ 
  Serras; Agrosistema extensivo; Mesotemperado inferior & 2.70 & 0.47 & 5.73 \\ 
  Serras; Agrosistema intensivo (mosaico agroforestal); Supra e orotemperado & 2.70 & 2.44 & 1.11 \\ 
  Canons; Agrosistema intensivo (plantacion forestal); Termotemperado & 1.35 & 0.13 & 10.03 \\ 
  Canons; Bosque; Mesotemperado inferior & 1.35 & 0.28 & 4.81 \\ 
  Canons; Matogueira e rochedo; Termotemperado & 1.35 & 0.33 & 4.14 \\ 
  Chairas e vales interiores; Agrosistema extensivo; Mesotemperado inferior & 1.35 & 0.29 & 4.63 \\ 
  Serras; Agrosistema intensivo (plantacion forestal); Supra e orotemperado & 1.35 & 3.43 & 0.39 \\ 
   \hline
\end{tabular}
\end{table}
% latex table generated in R 3.2.2 by xtable 1.8-0 package
% Fri Dec  4 16:54:42 2015
\begin{table}[p]
\centering
\caption{Frecuencia de aparición de valores naturais identificados na participación pública e frecuencia de tipos asociados Chairas e Fosas Luguesas} 
\label{vsixotnat6}
\begin{tabular}{lrrr}
  \hline
Tipo de paisaxe & F.Aparic (\%) & F.Tipo (\%) & Ratio \\ 
  \hline
Chairas e vales interiores; Agrosistema extensivo; Mesotemperado inferior & 21.62 & 9.57 & 2.26 \\ 
  Chairas e vales interiores; Agrosistema intensivo (mosaico agroforestal); Mesotemperado inferior & 13.51 & 6.48 & 2.08 \\ 
  Chairas e vales interiores; Bosque; Mesotemperado inferior & 13.51 & 1.29 & 10.46 \\ 
  Serras; Agrosistema intensivo (plantacion forestal); Supra e orotemperado & 13.51 & 0.44 & 30.71 \\ 
  Chairas e vales interiores; Agrosistema extensivo; Mesotemperado superior & 10.81 & 17.50 & 0.62 \\ 
  Chairas e vales interiores; Rururbano (diseminado); Mesotemperado superior & 5.41 & 1.94 & 2.78 \\ 
  Chairas e vales interiores; Urbano; Mesotemperado inferior & 5.41 & 0.29 & 18.59 \\ 
  Serras; Agrosistema extensivo; Mesotemperado superior & 5.41 & 10.14 & 0.53 \\ 
  Chairas e vales interiores; Bosque; Mesotemperado superior & 2.70 & 1.53 & 1.76 \\ 
  Serras; Agrosistema extensivo; Supra e orotemperado & 2.70 & 1.09 & 2.47 \\ 
  Serras; Matogueira e rochedo; Supra e orotemperado & 2.70 & 0.84 & 3.23 \\ 
  Serras; Turbeira; Mesotemperado superior & 2.70 & 1.14 & 2.38 \\ 
   \hline
\end{tabular}
\end{table}
% latex table generated in R 3.2.2 by xtable 1.8-0 package
% Fri Dec  4 16:54:42 2015
\begin{table}[p]
\centering
\caption{Frecuencia de aparición de valores naturais identificados na participación pública e frecuencia de tipos asociados Galicia Central} 
\label{vsixotnat7}
\begin{tabular}{lrrr}
  \hline
Tipo de paisaxe & F.Aparic (\%) & F.Tipo (\%) & Ratio \\ 
  \hline
Vales sublitorais; Agrosistema extensivo; Mesotemperado inferior & 19.09 & 10.39 & 1.84 \\ 
  Vales sublitorais; Agrosistema intensivo (mosaico agroforestal); Termotemperado & 9.09 & 5.79 & 1.57 \\ 
  Vales sublitorais; Rururbano (diseminado); Termotemperado & 9.09 & 1.94 & 4.69 \\ 
  Serras; Agrosistema extensivo; Mesotemperado superior & 6.36 & 7.06 & 0.90 \\ 
  Serras; Matogueira e rochedo; Mesotemperado superior & 6.36 & 5.00 & 1.27 \\ 
  Serras; Matogueira e rochedo; Mesotemperado inferior & 5.45 & 1.61 & 3.38 \\ 
  Vales sublitorais; Urbano; Mesotemperado inferior & 5.45 & 0.49 & 11.05 \\ 
  Vales sublitorais; Agrosistema extensivo; Termotemperado & 3.64 & 0.96 & 3.78 \\ 
  Vales sublitorais; Agrosistema intensivo (mosaico agroforestal); Mesotemperado inferior & 3.64 & 23.79 & 0.15 \\ 
  Serras; Agrosistema extensivo; Mesotemperado inferior & 2.73 & 2.67 & 1.02 \\ 
  Vales sublitorais; Agrosistema extensivo; Mesotemperado superior & 2.73 & 2.91 & 0.94 \\ 
  Vales sublitorais; Matogueira e rochedo; Mesotemperado inferior & 2.73 & 3.12 & 0.88 \\ 
  Serras; Agrosistema intensivo (plantacion forestal); Supra e orotemperado & 1.82 & 0.26 & 6.95 \\ 
  Serras; Agrosistema intensivo (superficie de cultivo); Mesotemperado superior & 1.82 & 0.86 & 2.11 \\ 
  Serras; Matogueira e rochedo; Supra e orotemperado & 1.82 & 2.19 & 0.83 \\ 
  Vales sublitorais; Agrosistema intensivo (plantacion forestal); Mesotemperado inferior & 1.82 & 2.72 & 0.67 \\ 
  Vales sublitorais; Agrosistema intensivo (plantacion forestal); Termotemperado & 1.82 & 1.21 & 1.50 \\ 
  Vales sublitorais; Bosque; Mesotemperado inferior & 1.82 & 0.61 & 3.00 \\ 
  Vales sublitorais; Matogueira e rochedo; Termotemperado & 1.82 & 0.95 & 1.92 \\ 
  Vales sublitorais; Rururbano (diseminado); Mesotemperado inferior & 1.82 & 3.66 & 0.50 \\ 
   \hline
\end{tabular}
\end{table}
% latex table generated in R 3.2.2 by xtable 1.8-0 package
% Fri Dec  4 16:54:42 2015
\begin{table}[p]
\centering
\caption{Frecuencia de aparición de valores naturais identificados na participación pública e frecuencia de tipos asociados Chairas, Fosas e Serras Ourensás} 
\label{vsixotnat8}
\begin{tabular}{lrrr}
  \hline
Tipo de paisaxe & F.Aparic (\%) & F.Tipo (\%) & Ratio \\ 
  \hline
Serras; Matogueira e rochedo; Supra e orotemperado & 13.85 & 8.42 & 1.64 \\ 
  Serras; Matogueira e rochedo; Mesotemperado superior & 12.31 & 9.99 & 1.23 \\ 
  Chairas e vales interiores; Matogueira e rochedo; Mesotemperado inferior & 10.77 & 4.79 & 2.25 \\ 
  Chairas e vales interiores; Agrosistema extensivo; Mesotemperado inferior & 9.23 & 12.05 & 0.77 \\ 
  Serras; Matogueira e rochedo; Mesotemperado inferior & 9.23 & 8.98 & 1.03 \\ 
  Chairas e vales interiores; Agrosistema intensivo (superficie de cultivo); Mesotemperado inferior & 6.15 & 7.47 & 0.82 \\ 
  Chairas e vales interiores; Bosque; Mesotemperado inferior & 6.15 & 2.80 & 2.19 \\ 
  Serras; Agrosistema extensivo; Mesotemperado inferior & 6.15 & 9.02 & 0.68 \\ 
  Chairas e vales interiores; Viñedo; Termotemperado & 4.62 & 0.65 & 7.12 \\ 
  Chairas e vales interiores; Rururbano (diseminado); Mesotemperado inferior & 3.08 & 1.22 & 2.52 \\ 
  Serras; Bosque; Supra e orotemperado & 3.08 & 0.97 & 3.16 \\ 
  Chairas e vales interiores; Agrosistema extensivo; Termotemperado & 1.54 & 1.77 & 0.87 \\ 
  Chairas e vales interiores; Agrosistema intensivo (mosaico agroforestal); Mesotemperado inferior & 1.54 & 1.44 & 1.07 \\ 
  Chairas e vales interiores; Agrosistema intensivo (mosaico agroforestal); Termotemperado & 1.54 & 1.20 & 1.29 \\ 
  Chairas e vales interiores; Agrosistema intensivo (superficie de cultivo); no data & 1.54 & 0.02 & 85.71 \\ 
  Chairas e vales interiores; Matogueira e rochedo; Termotemperado & 1.54 & 2.16 & 0.71 \\ 
  Chairas e vales interiores; Rururbano (diseminado); Termotemperado & 1.54 & 0.75 & 2.06 \\ 
  Serras; Agrosistema extensivo; Mesotemperado superior & 1.54 & 4.93 & 0.31 \\ 
  Serras; Bosque; Mesotemperado inferior & 1.54 & 2.68 & 0.57 \\ 
  Serras; Bosque; Mesotemperado superior & 1.54 & 1.61 & 0.95 \\ 
  Serras; Matogueira e rochedo; Termotemperado & 1.54 & 0.44 & 3.47 \\ 
   \hline
\end{tabular}
\end{table}
% latex table generated in R 3.2.2 by xtable 1.8-0 package
% Fri Dec  4 16:54:42 2015
\begin{table}[p]
\centering
\caption{Frecuencia de aparición de valores naturais identificados na participación pública e frecuencia de tipos asociados Serras Surorientais} 
\label{vsixotnat9}
\begin{tabular}{lrrr}
  \hline
Tipo de paisaxe & F.Aparic (\%) & F.Tipo (\%) & Ratio \\ 
  \hline
Serras; Matogueira e rochedo; Supra e orotemperado & 39.73 & 37.48 & 1.06 \\ 
  Serras; Agrosistema extensivo; Supra e orotemperado & 9.59 & 7.54 & 1.27 \\ 
  Serras; Agrosistema extensivo; Mesotemperado inferior & 6.85 & 7.06 & 0.97 \\ 
  Canons; Viñedo; Mesomediterráneo & 5.48 & 0.02 & 231.40 \\ 
  Serras; Agrosistema extensivo; Mesotemperado superior & 5.48 & 10.41 & 0.53 \\ 
  Serras; Bosque; Mesotemperado superior & 4.11 & 2.58 & 1.59 \\ 
  Serras; Matogueira e rochedo; Mesotemperado superior & 4.11 & 9.91 & 0.41 \\ 
  Canons; Matogueira e rochedo; Mesotemperado inferior & 2.74 & 0.93 & 2.95 \\ 
  Serras; Agrosistema intensivo (plantacion forestal); Supra e orotemperado & 2.74 & 5.13 & 0.53 \\ 
  Serras; Bosque; Mesotemperado inferior & 2.74 & 1.96 & 1.40 \\ 
  Serras; Matogueira e rochedo; Mesotemperado inferior & 2.74 & 4.18 & 0.66 \\ 
  Serras; Matogueira e rochedo; no data & 2.74 & 0.31 & 8.72 \\ 
  Canons; Agrosistema extensivo; Mesomediterráneo & 1.37 & 0.07 & 19.30 \\ 
  Canons; Matogueira e rochedo; Mesomediterráneo & 1.37 & 0.50 & 2.74 \\ 
  Canons; Viñedo; Mesotemperado inferior & 1.37 & 0.04 & 36.69 \\ 
  Serras; Agrosistema intensivo (superficie de cultivo); Mesotemperado superior & 1.37 & 1.15 & 1.19 \\ 
  Serras; Bosque; Supra e orotemperado & 1.37 & 2.05 & 0.67 \\ 
  Serras; Rururbano (diseminado); Mesotemperado superior & 1.37 & 0.12 & 11.20 \\ 
  Vales sublitorais; Agrosistema extensivo; Mesomediterráneo & 1.37 & 0.00 & 542.10 \\ 
  Vales sublitorais; Viñedo; Mesomediterráneo & 1.37 & 0.00 & 846.44 \\ 
   \hline
\end{tabular}
\end{table}
% latex table generated in R 3.2.2 by xtable 1.8-0 package
% Fri Dec  4 16:54:42 2015
\begin{table}[p]
\centering
\caption{Frecuencia de aparición de valores naturais identificados na participación pública e frecuencia de tipos asociados Galicia Setentrional} 
\label{vsixotnat10}
\begin{tabular}{lrrr}
  \hline
Tipo de paisaxe & F.Aparic (\%) & F.Tipo (\%) & Ratio \\ 
  \hline
Serras; Turbeira; Supra e orotemperado & 14.47 & 5.12 & 2.83 \\ 
  Serras; Turbeira; Mesotemperado superior & 10.53 & 6.59 & 1.60 \\ 
   & 9.21 &  &  \\ 
  Serras; Matogueira e rochedo; Mesotemperado superior & 7.89 & 6.61 & 1.19 \\ 
  Litoral Cantabro-Atlantico; Matogueira e rochedo; Termotemperado & 6.58 & 1.91 & 3.44 \\ 
  Vales sublitorais; Bosque; Mesotemperado inferior & 6.58 & 0.94 & 7.04 \\ 
  Litoral Cantabro-Atlantico; Agrosistema intensivo (mosaico agroforestal); Termotemperado & 5.26 & 5.96 & 0.88 \\ 
  Litoral Cantabro-Atlantico; Agrosistema intensivo (plantacion forestal); Termotemperado & 5.26 & 3.32 & 1.59 \\ 
  Litoral Cantabro-Atlantico; Matogueira e rochedo; no data & 3.95 & 0.19 & 21.23 \\ 
  Litoral Cantabro-Atlantico; Rururbano (diseminado); Termotemperado & 3.95 & 2.78 & 1.42 \\ 
  Litoral Cantabro-Atlantico; Agrosistema intensivo (plantacion forestal); no data & 2.63 & 0.10 & 26.52 \\ 
  Litoral Cantabro-Atlantico; Matogueira e rochedo; Mesotemperado inferior & 2.63 & 0.60 & 4.39 \\ 
  Vales sublitorais; Agrosistema intensivo (mosaico agroforestal); Mesotemperado inferior & 2.63 & 12.50 & 0.21 \\ 
  Litoral Cantabro-Atlantico; Agrosistema intensivo (mosaico agroforestal); Mesotemperado inferior & 1.32 & 1.86 & 0.71 \\ 
  Litoral Cantabro-Atlantico; Agrosistema intensivo (mosaico agroforestal); no data & 1.32 & 0.09 & 14.88 \\ 
  Litoral Cantabro-Atlantico; Agrosistema intensivo (plantacion forestal); Mesotemperado inferior & 1.32 & 3.84 & 0.34 \\ 
  Litoral Cantabro-Atlantico; Rururbano (diseminado); no data & 1.32 & 0.09 & 15.09 \\ 
  Serras; Agrosistema extensivo; Mesotemperado superior & 1.32 & 4.36 & 0.30 \\ 
  Serras; Agrosistema intensivo (mosaico agroforestal); Mesotemperado inferior & 1.32 & 1.66 & 0.79 \\ 
  Serras; Agrosistema intensivo (mosaico agroforestal); Mesotemperado superior & 1.32 & 3.06 & 0.43 \\ 
  Serras; Agrosistema intensivo (plantacion forestal); Mesotemperado superior & 1.32 & 2.79 & 0.47 \\ 
  Serras; Bosque; Mesotemperado inferior & 1.32 & 0.24 & 5.58 \\ 
  Serras; Turbeira; Mesotemperado inferior & 1.32 & 0.68 & 1.93 \\ 
  Vales sublitorais; Agrosistema intensivo (mosaico agroforestal); Mesotemperado superior & 1.32 & 2.20 & 0.60 \\ 
  Vales sublitorais; Agrosistema intensivo (plantacion forestal); Mesotemperado inferior & 1.32 & 11.84 & 0.11 \\ 
  Vales sublitorais; Agrosistema intensivo (plantacion forestal); Mesotemperado superior & 1.32 & 2.01 & 0.66 \\ 
  Vales sublitorais; Rururbano (diseminado); Mesotemperado inferior & 1.32 & 0.53 & 2.46 \\ 
   \hline
\end{tabular}
\end{table}
% latex table generated in R 3.2.2 by xtable 1.8-0 package
% Fri Dec  4 16:54:42 2015
\begin{table}[p]
\centering
\caption{Frecuencia de aparición de valores naturais identificados na participación pública e frecuencia de tipos asociados Chairas e Fosas Occidentais} 
\label{vsixotnat11}
\begin{tabular}{lrrr}
  \hline
Tipo de paisaxe & F.Aparic (\%) & F.Tipo (\%) & Ratio \\ 
  \hline
Litoral Cantabro-Atlantico; Matogueira e rochedo; Termotemperado & 28.77 & 5.61 & 5.13 \\ 
  Vales sublitorais; Agrosistema intensivo (plantacion forestal); Mesotemperado inferior & 10.96 & 8.33 & 1.32 \\ 
   & 10.96 &  &  \\ 
  Litoral Cantabro-Atlantico; Agrosistema intensivo (plantacion forestal); Termotemperado & 6.16 & 4.31 & 1.43 \\ 
  Vales sublitorais; Agrosistema intensivo (plantacion forestal); Termotemperado & 5.48 & 3.89 & 1.41 \\ 
  Vales sublitorais; Agrosistema intensivo (mosaico agroforestal); Termotemperado & 4.79 & 8.83 & 0.54 \\ 
  Litoral Cantabro-Atlantico; Matogueira e rochedo; no data & 4.11 & 0.47 & 8.82 \\ 
  Litoral Cantabro-Atlantico; Rururbano (diseminado); Termotemperado & 4.11 & 2.77 & 1.48 \\ 
  Vales sublitorais; Agrosistema extensivo; Mesotemperado inferior & 4.11 & 4.56 & 0.90 \\ 
  Vales sublitorais; Agrosistema intensivo (mosaico agroforestal); Mesotemperado inferior & 4.11 & 18.05 & 0.23 \\ 
  Vales sublitorais; Matogueira e rochedo; Mesotemperado inferior & 4.11 & 7.15 & 0.57 \\ 
  Litoral Cantabro-Atlantico; Conxunto Historico; Termotemperado & 2.05 & 0.12 & 16.83 \\ 
  Litoral Cantabro-Atlantico; Agrosistema intensivo (mosaico agroforestal); Termotemperado & 1.37 & 8.86 & 0.15 \\ 
  Litoral Cantabro-Atlantico; Agrosistema intensivo (plantacion forestal); Mesotemperado inferior & 1.37 & 0.31 & 4.37 \\ 
  Vales sublitorais; Matogueira e rochedo; Mesotemperado superior & 1.37 & 3.30 & 0.42 \\ 
   \hline
\end{tabular}
\end{table}
% latex table generated in R 3.2.2 by xtable 1.8-0 package
% Fri Dec  4 16:54:42 2015
\begin{table}[p]
\centering
\caption{Frecuencia de aparición de valores naturais identificados na participación pública e frecuencia de tipos asociados Rías Baixas} 
\label{vsixotnat12}
\begin{tabular}{lrrr}
  \hline
Tipo de paisaxe & F.Aparic (\%) & F.Tipo (\%) & Ratio \\ 
  \hline
Serras; Matogueira e rochedo; Mesotemperado superior & 18.85 & 4.07 & 4.63 \\ 
   & 11.48 &  &  \\ 
  Serras; Matogueira e rochedo; Mesotemperado inferior & 10.66 & 4.31 & 2.47 \\ 
  Litoral Cantabro-Atlantico; Matogueira e rochedo; Termotemperado & 7.38 & 2.61 & 2.83 \\ 
  Vales sublitorais; Rururbano (diseminado); Termotemperado & 7.38 & 6.06 & 1.22 \\ 
  Vales sublitorais; Matogueira e rochedo; Mesotemperado inferior & 6.56 & 5.63 & 1.17 \\ 
  Litoral Cantabro-Atlantico; Rururbano (diseminado); Termotemperado & 4.92 & 16.40 & 0.30 \\ 
  Litoral Cantabro-Atlantico; Agrosistema intensivo (plantacion forestal); Termotemperado & 4.10 & 4.94 & 0.83 \\ 
  Litoral Cantabro-Atlantico; Matogueira e rochedo; no data & 3.28 & 0.23 & 14.24 \\ 
  Vales sublitorais; Agrosistema intensivo (mosaico agroforestal); Termotemperado & 3.28 & 5.62 & 0.58 \\ 
  Vales sublitorais; Agrosistema intensivo (plantacion forestal); Termotemperado & 3.28 & 8.25 & 0.40 \\ 
  Serras; Agrosistema intensivo (plantacion forestal); Mesotemperado inferior & 2.46 & 1.60 & 1.54 \\ 
  Vales sublitorais; Agrosistema intensivo (plantacion forestal); Mesotemperado inferior & 2.46 & 3.99 & 0.62 \\ 
  Litoral Cantabro-Atlantico; Agrosistema intensivo (mosaico agroforestal); no data & 1.64 & 0.11 & 15.45 \\ 
  Vales sublitorais; Matogueira e rochedo; Mesotemperado superior & 1.64 & 0.86 & 1.91 \\ 
  Vales sublitorais; Matogueira e rochedo; Termotemperado & 1.64 & 5.27 & 0.31 \\ 
   \hline
\end{tabular}
\end{table}
