% latex table generated in R 3.2.2 by xtable 1.8-0 package
% Wed Dec  2 19:45:31 2015
\begin{table}[p]
\centering
\caption{Frecuencia de aparición de Lugares de Importancia Comunitaria e frecuencia de tipos asociados Golfo Ártabro} 
\label{vnatura1}
\begin{tabular}{lrrr}
  \hline
Tipo de paisaxe & F.Aparic (\%) & F.Tipo (\%) & Ratio \\ 
  \hline
Serras; Matogueira e rochedo; Mesotemperado superior & 10.93 & 4.97 & 2.20 \\ 
  Canons; Bosque; Mesotemperado inferior & 10.73 & 0.29 & 36.76 \\ 
  Serras; Turbeira; Mesotemperado superior & 8.68 & 0.70 & 12.40 \\ 
  Vales sublitorais; Agrosistema intensivo (mosaico agroforestal); Mesotemperado inferior & 6.01 & 7.45 & 0.81 \\ 
  Serras; Bosque; Mesotemperado superior & 5.24 & 0.99 & 5.31 \\ 
  Vales sublitorais; Agrosistema extensivo; Mesotemperado inferior & 3.70 & 2.70 & 1.37 \\ 
  Canons; Bosque; Termotemperado & 3.46 & 0.15 & 23.16 \\ 
  Serras; Agrosistema extensivo; Mesotemperado superior & 3.21 & 5.69 & 0.56 \\ 
  Litoral Cantabro-Atlantico; Rururbano (diseminado); Termotemperado & 3.10 & 3.25 & 0.95 \\ 
  Vales sublitorais; Agrosistema intensivo (plantacion forestal); Mesotemperado inferior & 2.87 & 2.86 & 1.00 \\ 
  Litoral Cantabro-Atlantico; Agrosistema intensivo (mosaico agroforestal); Termotemperado & 2.86 & 2.40 & 1.19 \\ 
  Canons; Bosque; Mesotemperado superior & 2.81 & 0.03 & 108.15 \\ 
  Serras; Agrosistema intensivo (mosaico agroforestal); Mesotemperado superior & 2.71 & 1.75 & 1.54 \\ 
  Vales sublitorais; Bosque; Mesotemperado inferior & 2.54 & 0.49 & 5.19 \\ 
  Canons; Agrosistema intensivo (mosaico agroforestal); Mesotemperado inferior & 2.42 & 0.04 & 58.19 \\ 
  Canons; Agrosistema extensivo; Mesotemperado inferior & 2.01 & 0.17 & 11.92 \\ 
  Vales sublitorais; Turbeira; Mesotemperado superior & 1.84 & 0.06 & 30.90 \\ 
  Vales sublitorais; Agrosistema intensivo (mosaico agroforestal); Mesotemperado superior & 1.76 & 1.50 & 1.18 \\ 
  Canons; Agrosistema intensivo (plantacion forestal); Mesotemperado inferior & 1.68 & 0.05 & 36.13 \\ 
  Vales sublitorais; Agrosistema extensivo; Mesotemperado superior & 1.61 & 1.19 & 1.35 \\ 
  Serras; Agrosistema intensivo (mosaico agroforestal); Mesotemperado inferior & 1.43 & 0.49 & 2.92 \\ 
  Canons; Matogueira e rochedo; Mesotemperado inferior & 1.33 & 0.20 & 6.74 \\ 
  Vales sublitorais; Matogueira e rochedo; Mesotemperado superior & 1.23 & 0.84 & 1.46 \\ 
  Litoral Cantabro-Atlantico; Bosque; Termotemperado & 1.15 & 0.03 & 35.47 \\ 
  Serras; Agrosistema intensivo (plantacion forestal); Mesotemperado superior & 1.11 & 1.06 & 1.05 \\ 
  Litoral Cantabro-Atlantico; Agrosistema intensivo (plantacion forestal); Termotemperado & 1.01 & 1.30 & 0.78 \\ 
   \hline
\end{tabular}
\end{table}
% latex table generated in R 3.2.2 by xtable 1.8-0 package
% Wed Dec  2 19:45:31 2015
\begin{table}[p]
\centering
\caption{Frecuencia de aparición de Lugares de Importancia Comunitaria e frecuencia de tipos asociados A Mariña - Baixo Eo} 
\label{vnatura2}
\begin{tabular}{lrrr}
  \hline
Tipo de paisaxe & F.Aparic (\%) & F.Tipo (\%) & Ratio \\ 
  \hline
Serras; Turbeira; Mesotemperado superior & 32.65 & 0.70 & 46.65 \\ 
  Serras; Agrosistema intensivo (plantacion forestal); Mesotemperado superior & 12.42 & 1.06 & 11.71 \\ 
  Litoral Cantabro-Atlantico; Urbano; Termotemperado & 7.24 & 0.52 & 13.81 \\ 
  Serras; Matogueira e rochedo; Mesotemperado superior & 6.08 & 4.97 & 1.22 \\ 
  Litoral Cantabro-Atlantico; Agrosistema intensivo (mosaico agroforestal); Termotemperado & 5.29 & 2.40 & 2.20 \\ 
  Litoral Cantabro-Atlantico; Rururbano (diseminado); Termotemperado & 5.06 & 3.25 & 1.56 \\ 
  Vales sublitorais; Agrosistema intensivo (plantacion forestal); Mesotemperado inferior & 4.19 & 2.86 & 1.46 \\ 
  Vales sublitorais; Agrosistema intensivo (plantacion forestal); Mesotemperado superior & 3.75 & 0.59 & 6.35 \\ 
   & 3.38 &  &  \\ 
  Litoral Cantabro-Atlantico; Agrosistema intensivo (plantacion forestal); Termotemperado & 3.02 & 1.30 & 2.32 \\ 
  Vales sublitorais; Matogueira e rochedo; Mesotemperado inferior & 2.36 & 1.87 & 1.26 \\ 
  Vales sublitorais; Agrosistema intensivo (mosaico agroforestal); Mesotemperado inferior & 2.27 & 7.45 & 0.30 \\ 
  Serras; Agrosistema intensivo (mosaico agroforestal); Mesotemperado superior & 2.13 & 1.75 & 1.21 \\ 
  Litoral Cantabro-Atlantico; Agrosistema intensivo (mosaico agroforestal); no data & 1.09 & 0.04 & 25.89 \\ 
  Serras; Agrosistema intensivo (plantacion forestal); Mesotemperado inferior & 1.05 & 0.81 & 1.29 \\ 
   \hline
\end{tabular}
\end{table}
% latex table generated in R 3.2.2 by xtable 1.8-0 package
% Wed Dec  2 19:45:31 2015
\begin{table}[p]
\centering
\caption{Frecuencia de aparición de Lugares de Importancia Comunitaria e frecuencia de tipos asociados Costa Sur - Baixo Miño} 
\label{vnatura3}
\begin{tabular}{lrrr}
  \hline
Tipo de paisaxe & F.Aparic (\%) & F.Tipo (\%) & Ratio \\ 
  \hline
Litoral Cantabro-Atlantico; Rururbano (diseminado); Termotemperado & 17.27 & 3.25 & 5.31 \\ 
  Litoral Cantabro-Atlantico; Bosque; Termotemperado & 7.15 & 0.03 & 219.94 \\ 
  Serras; Agrosistema intensivo (plantacion forestal); Mesotemperado inferior & 6.41 & 0.81 & 7.87 \\ 
  Litoral Cantabro-Atlantico; Rururbano (diseminado); no data & 5.69 & 0.10 & 57.76 \\ 
  Litoral Cantabro-Atlantico; Agrosistema extensivo; Termotemperado & 5.46 & 0.16 & 35.04 \\ 
  Litoral Cantabro-Atlantico; Agrosistema intensivo (mosaico agroforestal); Termotemperado & 5.09 & 2.40 & 2.12 \\ 
  Chairas e vales interiores; Matogueira e rochedo; Mesotemperado inferior & 5.05 & 1.37 & 3.69 \\ 
  Litoral Cantabro-Atlantico; Agrosistema intensivo (plantacion forestal); Termotemperado & 4.48 & 1.30 & 3.44 \\ 
  Vales sublitorais; Rururbano (diseminado); Termotemperado & 3.00 & 1.66 & 1.81 \\ 
  Vales sublitorais; Agrosistema intensivo (plantacion forestal); Termotemperado & 2.97 & 1.77 & 1.68 \\ 
  Serras; Agrosistema intensivo (plantacion forestal); Termotemperado & 2.95 & 0.15 & 19.59 \\ 
  Litoral Cantabro-Atlantico; Matogueira e rochedo; Termotemperado & 2.11 & 0.82 & 2.57 \\ 
  Serras; Matogueira e rochedo; Mesotemperado superior & 1.92 & 4.97 & 0.39 \\ 
  Litoral Cantabro-Atlantico; Urbano; Termotemperado & 1.62 & 0.52 & 3.08 \\ 
  Litoral Cantabro-Atlantico; Agrosistema extensivo; no data & 1.50 & 0.01 & 140.00 \\ 
  Serras; Matogueira e rochedo; Mesotemperado inferior & 1.45 & 2.69 & 0.54 \\ 
  Vales sublitorais; Agrosistema extensivo; Termotemperado & 1.32 & 0.35 & 3.74 \\ 
  Litoral Cantabro-Atlantico; Agrosistema intensivo (mosaico agroforestal); no data & 1.24 & 0.04 & 29.50 \\ 
  Litoral Cantabro-Atlantico; Agrosistema intensivo (plantacion forestal); no data & 1.22 & 0.03 & 39.57 \\ 
  Serras; Matogueira e rochedo; Termotemperado & 1.19 & 0.16 & 7.65 \\ 
  Vales sublitorais; Bosque; Termotemperado & 1.18 & 0.14 & 8.43 \\ 
  Chairas e vales interiores; Rururbano (diseminado); no data & 1.07 & 0.00 & 217.40 \\ 
  Chairas e vales interiores; Agrosistema intensivo (plantacion forestal); no data & 1.07 & 0.00 & 338.56 \\ 
   \hline
\end{tabular}
\end{table}
% latex table generated in R 3.2.2 by xtable 1.8-0 package
% Wed Dec  2 19:45:31 2015
\begin{table}[p]
\centering
\caption{Frecuencia de aparición de Lugares de Importancia Comunitaria e frecuencia de tipos asociados Ribeiras Encaixadas do Miño e do Sil} 
\label{vnatura4}
\begin{tabular}{lrrr}
  \hline
Tipo de paisaxe & F.Aparic (\%) & F.Tipo (\%) & Ratio \\ 
  \hline
Serras; Matogueira e rochedo; Supra e orotemperado & 8.34 & 6.05 & 1.38 \\ 
  Canons; Bosque; Termotemperado & 7.57 & 0.15 & 50.61 \\ 
  Serras; Matogueira e rochedo; Mesotemperado inferior & 7.09 & 2.69 & 2.63 \\ 
  Canons; Bosque; Mesotemperado inferior & 6.77 & 0.29 & 23.20 \\ 
  Serras; Matogueira e rochedo; Mesotemperado superior & 5.85 & 4.97 & 1.18 \\ 
  Canons; Matogueira e rochedo; Mesomediterráneo & 5.14 & 0.18 & 28.13 \\ 
  Serras; Agrosistema extensivo; Mesotemperado inferior & 5.05 & 2.54 & 1.99 \\ 
  Serras; Bosque; Mesotemperado inferior & 4.38 & 0.68 & 6.41 \\ 
  Chairas e vales interiores; Matogueira e rochedo; Mesomediterráneo & 4.27 & 0.26 & 16.12 \\ 
  Canons; Agrosistema intensivo (plantacion forestal); Termotemperado & 4.13 & 0.07 & 58.30 \\ 
  Canons; Matogueira e rochedo; Termotemperado & 3.95 & 0.09 & 46.41 \\ 
  Serras; Matogueira e rochedo; Mesomediterráneo & 3.45 & 0.19 & 18.36 \\ 
  Chairas e vales interiores; Bosque; Mesomediterráneo & 2.87 & 0.06 & 51.44 \\ 
  Canons; Matogueira e rochedo; Mesotemperado inferior & 2.74 & 0.20 & 13.85 \\ 
  Chairas e vales interiores; Agrosistema extensivo; Mesotemperado inferior & 2.74 & 3.54 & 0.77 \\ 
  Chairas e vales interiores; Matogueira e rochedo; Mesotemperado inferior & 2.09 & 1.37 & 1.53 \\ 
  Chairas e vales interiores; Agrosistema extensivo; Mesomediterráneo & 1.84 & 0.11 & 17.01 \\ 
  Chairas e vales interiores; Bosque; Termotemperado & 1.76 & 0.29 & 6.10 \\ 
  Chairas e vales interiores; Agrosistema extensivo; Termotemperado & 1.47 & 0.61 & 2.43 \\ 
  Canons; Agrosistema extensivo; Mesotemperado inferior & 1.23 & 0.17 & 7.27 \\ 
  Chairas e vales interiores; Agrosistema intensivo (mosaico agroforestal); Mesotemperado inferior & 1.21 & 1.69 & 0.71 \\ 
  Chairas e vales interiores; Matogueira e rochedo; Termotemperado & 1.13 & 0.65 & 1.74 \\ 
  Canons; Agrosistema extensivo; Mesomediterráneo & 1.12 & 0.03 & 32.75 \\ 
  Canons; Bosque; Mesomediterráneo & 1.11 & 0.03 & 41.76 \\ 
  Canons; Agrosistema extensivo; Termotemperado & 1.05 & 0.04 & 23.32 \\ 
  Canons; Agrosistema intensivo (plantacion forestal); Mesomediterráneo & 1.01 & 0.03 & 40.32 \\ 
   \hline
\end{tabular}
\end{table}
% latex table generated in R 3.2.2 by xtable 1.8-0 package
% Wed Dec  2 19:45:31 2015
\begin{table}[p]
\centering
\caption{Frecuencia de aparición de Lugares de Importancia Comunitaria e frecuencia de tipos asociados Serras Orientais} 
\label{vnatura5}
\begin{tabular}{lrrr}
  \hline
Tipo de paisaxe & F.Aparic (\%) & F.Tipo (\%) & Ratio \\ 
  \hline
Serras; Matogueira e rochedo; Supra e orotemperado & 23.02 & 6.05 & 3.81 \\ 
  Serras; Agrosistema extensivo; Supra e orotemperado & 18.56 & 2.47 & 7.52 \\ 
  Serras; Bosque; Supra e orotemperado & 9.10 & 0.73 & 12.50 \\ 
  Serras; Matogueira e rochedo; Mesotemperado superior & 7.24 & 4.97 & 1.46 \\ 
  Serras; Agrosistema extensivo; Mesotemperado superior & 6.71 & 5.69 & 1.18 \\ 
  Serras; Bosque; Mesotemperado superior & 3.74 & 0.99 & 3.79 \\ 
  Vales sublitorais; Agrosistema extensivo; Mesotemperado superior & 3.62 & 1.19 & 3.03 \\ 
  Serras; Agrosistema intensivo (plantacion forestal); Supra e orotemperado & 3.37 & 0.92 & 3.65 \\ 
  Vales sublitorais; Agrosistema extensivo; Mesotemperado inferior & 3.27 & 2.70 & 1.21 \\ 
  Vales sublitorais; Bosque; Mesotemperado superior & 2.50 & 0.38 & 6.55 \\ 
  Vales sublitorais; Bosque; Mesotemperado inferior & 2.23 & 0.49 & 4.56 \\ 
  Serras; Matogueira e rochedo; Mesotemperado inferior & 1.90 & 2.69 & 0.71 \\ 
  Serras; Agrosistema intensivo (mosaico agroforestal); Supra e orotemperado & 1.52 & 0.37 & 4.13 \\ 
  Vales sublitorais; Matogueira e rochedo; Mesotemperado superior & 1.48 & 0.84 & 1.76 \\ 
  Vales sublitorais; Matogueira e rochedo; Mesotemperado inferior & 1.37 & 1.87 & 0.73 \\ 
  Serras; Agrosistema intensivo (plantacion forestal); Mesotemperado superior & 1.06 & 1.06 & 1.00 \\ 
   \hline
\end{tabular}
\end{table}
% latex table generated in R 3.2.2 by xtable 1.8-0 package
% Wed Dec  2 19:45:31 2015
\begin{table}[p]
\centering
\caption{Frecuencia de aparición de Lugares de Importancia Comunitaria e frecuencia de tipos asociados Chairas e Fosas Luguesas} 
\label{vnatura6}
\begin{tabular}{lrrr}
  \hline
Tipo de paisaxe & F.Aparic (\%) & F.Tipo (\%) & Ratio \\ 
  \hline
Chairas e vales interiores; Agrosistema extensivo; Mesotemperado superior & 21.07 & 2.74 & 7.69 \\ 
  Serras; Turbeira; Mesotemperado superior & 12.32 & 0.70 & 17.60 \\ 
  Serras; Turbeira; Supra e orotemperado & 11.27 & 0.39 & 28.67 \\ 
  Serras; Matogueira e rochedo; Supra e orotemperado & 9.34 & 6.05 & 1.55 \\ 
  Serras; Agrosistema extensivo; Supra e orotemperado & 7.17 & 2.47 & 2.91 \\ 
  Chairas e vales interiores; Agrosistema extensivo; Mesotemperado inferior & 6.41 & 3.54 & 1.81 \\ 
  Chairas e vales interiores; Bosque; Mesotemperado superior & 5.96 & 0.26 & 22.78 \\ 
  Serras; Agrosistema extensivo; Mesotemperado superior & 5.43 & 5.69 & 0.95 \\ 
  Chairas e vales interiores; Agrosistema intensivo (mosaico agroforestal); Mesotemperado superior & 3.32 & 2.24 & 1.48 \\ 
  Serras; Agrosistema intensivo (plantacion forestal); Supra e orotemperado & 2.75 & 0.92 & 2.98 \\ 
  Chairas e vales interiores; Bosque; Mesotemperado inferior & 2.11 & 0.76 & 2.80 \\ 
  Serras; Bosque; Supra e orotemperado & 1.91 & 0.73 & 2.62 \\ 
  Chairas e vales interiores; Agrosistema intensivo (superficie de cultivo); Mesotemperado superior & 1.88 & 0.71 & 2.65 \\ 
  Serras; Bosque; Mesotemperado superior & 1.08 & 0.99 & 1.09 \\ 
  Chairas e vales interiores; Agrosistema intensivo (mosaico agroforestal); Mesotemperado inferior & 1.07 & 1.69 & 0.63 \\ 
   \hline
\end{tabular}
\end{table}
% latex table generated in R 3.2.2 by xtable 1.8-0 package
% Wed Dec  2 19:45:31 2015
\begin{table}[p]
\centering
\caption{Frecuencia de aparición de Lugares de Importancia Comunitaria e frecuencia de tipos asociados Galicia Central} 
\label{vnatura7}
\begin{tabular}{lrrr}
  \hline
Tipo de paisaxe & F.Aparic (\%) & F.Tipo (\%) & Ratio \\ 
  \hline
Serras; Matogueira e rochedo; Mesotemperado superior & 22.90 & 4.97 & 4.61 \\ 
  Serras; Matogueira e rochedo; Supra e orotemperado & 19.98 & 6.05 & 3.30 \\ 
  Serras; Agrosistema extensivo; Mesotemperado superior & 8.69 & 5.69 & 1.53 \\ 
  Vales sublitorais; Agrosistema extensivo; Mesotemperado inferior & 7.56 & 2.70 & 2.80 \\ 
  Vales sublitorais; Matogueira e rochedo; Mesotemperado inferior & 6.91 & 1.87 & 3.69 \\ 
  Vales sublitorais; Agrosistema intensivo (mosaico agroforestal); Mesotemperado inferior & 5.55 & 7.45 & 0.74 \\ 
  Serras; Matogueira e rochedo; Mesotemperado inferior & 4.90 & 2.69 & 1.82 \\ 
  Serras; Agrosistema extensivo; Supra e orotemperado & 3.90 & 2.47 & 1.58 \\ 
  Vales sublitorais; Agrosistema intensivo (mosaico agroforestal); Termotemperado & 2.00 & 2.59 & 0.77 \\ 
  Vales sublitorais; Agrosistema extensivo; Termotemperado & 1.72 & 0.35 & 4.86 \\ 
  Vales sublitorais; Agrosistema extensivo; Mesotemperado superior & 1.50 & 1.19 & 1.26 \\ 
  Serras; Agrosistema extensivo; Mesotemperado inferior & 1.33 & 2.54 & 0.52 \\ 
  Vales sublitorais; Bosque; Termotemperado & 1.09 & 0.14 & 7.80 \\ 
  Serras; Agrosistema intensivo (plantacion forestal); Mesotemperado superior & 1.03 & 1.06 & 0.97 \\ 
   \hline
\end{tabular}
\end{table}
% latex table generated in R 3.2.2 by xtable 1.8-0 package
% Wed Dec  2 19:45:32 2015
\begin{table}[p]
\centering
\caption{Frecuencia de aparición de Lugares de Importancia Comunitaria e frecuencia de tipos asociados Chairas, Fosas e Serras Ourensás} 
\label{vnatura8}
\begin{tabular}{lrrr}
  \hline
Tipo de paisaxe & F.Aparic (\%) & F.Tipo (\%) & Ratio \\ 
  \hline
Serras; Matogueira e rochedo; Supra e orotemperado & 28.94 & 6.05 & 4.79 \\ 
  Serras; Matogueira e rochedo; Mesotemperado superior & 19.76 & 4.97 & 3.98 \\ 
  Chairas e vales interiores; Agrosistema intensivo (superficie de cultivo); Mesotemperado inferior & 13.22 & 1.14 & 11.55 \\ 
  Serras; Matogueira e rochedo; Mesotemperado inferior & 9.35 & 2.69 & 3.47 \\ 
  Chairas e vales interiores; Agrosistema extensivo; Mesotemperado inferior & 3.93 & 3.54 & 1.11 \\ 
  Serras; Bosque; Supra e orotemperado & 2.96 & 0.73 & 4.07 \\ 
  Serras; Agrosistema extensivo; Mesotemperado superior & 2.56 & 5.69 & 0.45 \\ 
  Chairas e vales interiores; Matogueira e rochedo; Termotemperado & 2.31 & 0.65 & 3.56 \\ 
  Serras; Matogueira e rochedo; Termotemperado & 2.13 & 0.16 & 13.74 \\ 
  Serras; Matogueira e rochedo; no data & 1.64 & 0.07 & 22.56 \\ 
  Chairas e vales interiores; Matogueira e rochedo; Mesotemperado inferior & 1.37 & 1.37 & 1.01 \\ 
  Chairas e vales interiores; Agrosistema extensivo; Termotemperado & 1.19 & 0.61 & 1.96 \\ 
  Serras; Agrosistema extensivo; Supra e orotemperado & 1.15 & 2.47 & 0.46 \\ 
  Serras; Agrosistema intensivo (plantacion forestal); Supra e orotemperado & 1.04 & 0.92 & 1.12 \\ 
   \hline
\end{tabular}
\end{table}
% latex table generated in R 3.2.2 by xtable 1.8-0 package
% Wed Dec  2 19:45:32 2015
\begin{table}[p]
\centering
\caption{Frecuencia de aparición de Lugares de Importancia Comunitaria e frecuencia de tipos asociados Serras Surorientais} 
\label{vnatura9}
\begin{tabular}{lrrr}
  \hline
Tipo de paisaxe & F.Aparic (\%) & F.Tipo (\%) & Ratio \\ 
  \hline
Serras; Matogueira e rochedo; Supra e orotemperado & 62.82 & 6.05 & 10.39 \\ 
  Serras; Agrosistema extensivo; Supra e orotemperado & 9.50 & 2.47 & 3.85 \\ 
  Serras; Matogueira e rochedo; Mesotemperado superior & 4.22 & 4.97 & 0.85 \\ 
  Serras; Bosque; Supra e orotemperado & 4.03 & 0.73 & 5.53 \\ 
  Serras; Agrosistema intensivo (plantacion forestal); Supra e orotemperado & 3.89 & 0.92 & 4.23 \\ 
  Serras; Agrosistema extensivo; Mesotemperado superior & 3.72 & 5.69 & 0.65 \\ 
  Serras; Agrosistema extensivo; Mesotemperado inferior & 1.55 & 2.54 & 0.61 \\ 
  Canons; Matogueira e rochedo; Mesotemperado inferior & 1.37 & 0.20 & 6.90 \\ 
  Serras; Bosque; Mesotemperado superior & 1.18 & 0.99 & 1.19 \\ 
   \hline
\end{tabular}
\end{table}
% latex table generated in R 3.2.2 by xtable 1.8-0 package
% Wed Dec  2 19:45:32 2015
\begin{table}[p]
\centering
\caption{Frecuencia de aparición de Lugares de Importancia Comunitaria e frecuencia de tipos asociados Galicia Setentrional} 
\label{vnatura10}
\begin{tabular}{lrrr}
  \hline
Tipo de paisaxe & F.Aparic (\%) & F.Tipo (\%) & Ratio \\ 
  \hline
Serras; Turbeira; Supra e orotemperado & 23.77 & 0.39 & 60.44 \\ 
  Serras; Turbeira; Mesotemperado superior & 20.06 & 0.70 & 28.66 \\ 
  Litoral Cantabro-Atlantico; Matogueira e rochedo; Termotemperado & 8.99 & 0.82 & 10.95 \\ 
  Serras; Agrosistema intensivo (plantacion forestal); Mesotemperado superior & 6.92 & 1.06 & 6.52 \\ 
  Serras; Matogueira e rochedo; Mesotemperado superior & 3.81 & 4.97 & 0.77 \\ 
  Serras; Agrosistema extensivo; Mesotemperado superior & 3.71 & 5.69 & 0.65 \\ 
  Litoral Cantabro-Atlantico; Agrosistema intensivo (mosaico agroforestal); Termotemperado & 3.09 & 2.40 & 1.29 \\ 
  Litoral Cantabro-Atlantico; Matogueira e rochedo; Mesotemperado inferior & 2.66 & 0.10 & 27.04 \\ 
  Litoral Cantabro-Atlantico; Agrosistema intensivo (plantacion forestal); Termotemperado & 2.50 & 1.30 & 1.92 \\ 
  Serras; Agrosistema intensivo (mosaico agroforestal); Mesotemperado superior & 2.31 & 1.75 & 1.32 \\ 
  Litoral Cantabro-Atlantico; Rururbano (diseminado); Termotemperado & 1.96 & 3.25 & 0.60 \\ 
  Serras; Bosque; Mesotemperado superior & 1.80 & 0.99 & 1.82 \\ 
  Serras; Turbeira; Mesotemperado inferior & 1.63 & 0.05 & 30.95 \\ 
  Serras; Agrosistema extensivo; Supra e orotemperado & 1.62 & 2.47 & 0.65 \\ 
   & 1.41 &  &  \\ 
  Vales sublitorais; Agrosistema intensivo (mosaico agroforestal); Mesotemperado superior & 1.35 & 1.50 & 0.90 \\ 
  Litoral Cantabro-Atlantico; Matogueira e rochedo; no data & 1.00 & 0.08 & 11.93 \\ 
   \hline
\end{tabular}
\end{table}
% latex table generated in R 3.2.2 by xtable 1.8-0 package
% Wed Dec  2 19:45:32 2015
\begin{table}[p]
\centering
\caption{Frecuencia de aparición de Lugares de Importancia Comunitaria e frecuencia de tipos asociados Chairas e Fosas Occidentais} 
\label{vnatura11}
\begin{tabular}{lrrr}
  \hline
Tipo de paisaxe & F.Aparic (\%) & F.Tipo (\%) & Ratio \\ 
  \hline
Litoral Cantabro-Atlantico; Matogueira e rochedo; Termotemperado & 41.83 & 0.82 & 50.95 \\ 
  Vales sublitorais; Matogueira e rochedo; Mesotemperado inferior & 12.19 & 1.87 & 6.51 \\ 
  Litoral Cantabro-Atlantico; Matogueira e rochedo; no data & 6.51 & 0.08 & 77.46 \\ 
  Vales sublitorais; Agrosistema intensivo (plantacion forestal); Mesotemperado inferior & 5.58 & 2.86 & 1.95 \\ 
  Litoral Cantabro-Atlantico; Agrosistema intensivo (plantacion forestal); Termotemperado & 5.27 & 1.30 & 4.05 \\ 
  Vales sublitorais; Matogueira e rochedo; Termotemperado & 4.60 & 0.92 & 4.99 \\ 
  Litoral Cantabro-Atlantico; Agrosistema intensivo (mosaico agroforestal); Termotemperado & 4.45 & 2.40 & 1.85 \\ 
   & 4.36 &  &  \\ 
  Litoral Cantabro-Atlantico; Rururbano (diseminado); Termotemperado & 3.25 & 3.25 & 1.00 \\ 
  Litoral Cantabro-Atlantico; Agrosistema intensivo (superficie de cultivo); Termotemperado & 2.73 & 0.22 & 12.57 \\ 
  Litoral Cantabro-Atlantico; Agrosistema extensivo; Termotemperado & 2.72 & 0.16 & 17.46 \\ 
  Litoral Cantabro-Atlantico; Urbano; Termotemperado & 1.19 & 0.52 & 2.27 \\ 
  Vales sublitorais; Agrosistema intensivo (mosaico agroforestal); Mesotemperado inferior & 1.14 & 7.45 & 0.15 \\ 
   \hline
\end{tabular}
\end{table}
% latex table generated in R 3.2.2 by xtable 1.8-0 package
% Wed Dec  2 19:45:32 2015
\begin{table}[p]
\centering
\caption{Frecuencia de aparición de Lugares de Importancia Comunitaria e frecuencia de tipos asociados Rías Baixas} 
\label{vnatura12}
\begin{tabular}{lrrr}
  \hline
Tipo de paisaxe & F.Aparic (\%) & F.Tipo (\%) & Ratio \\ 
  \hline
Serras; Matogueira e rochedo; Mesotemperado superior & 15.40 & 4.97 & 3.10 \\ 
  Serras; Matogueira e rochedo; Supra e orotemperado & 14.26 & 6.05 & 2.36 \\ 
   & 13.73 &  &  \\ 
  Litoral Cantabro-Atlantico; Matogueira e rochedo; Termotemperado & 10.40 & 0.82 & 12.67 \\ 
  Litoral Cantabro-Atlantico; Rururbano (diseminado); Termotemperado & 5.38 & 3.25 & 1.65 \\ 
  Litoral Cantabro-Atlantico; Agrosistema intensivo (mosaico agroforestal); Termotemperado & 4.84 & 2.40 & 2.02 \\ 
  Litoral Cantabro-Atlantico; Agrosistema extensivo; Termotemperado & 3.63 & 0.16 & 23.31 \\ 
  Litoral Cantabro-Atlantico; Matogueira e rochedo; no data & 3.22 & 0.08 & 38.31 \\ 
  Serras; Agrosistema extensivo; Mesotemperado superior & 3.11 & 5.69 & 0.55 \\ 
  Serras; Agrosistema extensivo; Mesotemperado inferior & 3.09 & 2.54 & 1.22 \\ 
  Vales sublitorais; Agrosistema extensivo; Mesotemperado inferior & 2.88 & 2.70 & 1.07 \\ 
  Serras; Matogueira e rochedo; Mesotemperado inferior & 2.77 & 2.69 & 1.03 \\ 
  Vales sublitorais; Bosque; Mesotemperado inferior & 2.41 & 0.49 & 4.94 \\ 
  Litoral Cantabro-Atlantico; Agrosistema intensivo (plantacion forestal); Termotemperado & 2.06 & 1.30 & 1.58 \\ 
  Litoral Cantabro-Atlantico; Agrosistema intensivo (superficie de cultivo); Termotemperado & 1.97 & 0.22 & 9.09 \\ 
  Vales sublitorais; Matogueira e rochedo; Mesotemperado inferior & 1.44 & 1.87 & 0.77 \\ 
  Litoral Cantabro-Atlantico; Viñedo; Termotemperado & 1.34 & 0.14 & 9.77 \\ 
  Litoral Cantabro-Atlantico; Rururbano (diseminado); no data & 1.24 & 0.10 & 12.53 \\ 
   \hline
\end{tabular}
\end{table}
