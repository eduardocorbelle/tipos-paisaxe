% latex table generated in R 3.2.2 by xtable 1.8-0 package
% Fri Dec  4 16:54:41 2015
\begin{table}[p]
\centering
\caption{Frecuencia de aparición de Lugares de Importancia Comunitaria e frecuencia de tipos asociados Golfo Ártabro} 
\label{vnatura1}
\begin{tabular}{lrrr}
  \hline
Tipo de paisaxe & F.Aparic (\%) & F.Tipo (\%) & Ratio \\ 
  \hline
Serras; Matogueira e rochedo; Mesotemperado superior & 10.93 & 2.97 & 3.68 \\ 
  Canons; Bosque; Mesotemperado inferior & 10.73 & 1.04 & 10.28 \\ 
  Serras; Turbeira; Mesotemperado superior & 8.68 & 1.50 & 5.78 \\ 
  Vales sublitorais; Agrosistema intensivo (mosaico agroforestal); Mesotemperado inferior & 6.01 & 14.40 & 0.42 \\ 
  Serras; Bosque; Mesotemperado superior & 5.24 & 0.62 & 8.52 \\ 
  Vales sublitorais; Agrosistema extensivo; Mesotemperado inferior & 3.70 & 0.90 & 4.10 \\ 
  Canons; Bosque; Termotemperado & 3.46 & 0.34 & 10.26 \\ 
  Serras; Agrosistema extensivo; Mesotemperado superior & 3.21 & 1.59 & 2.02 \\ 
  Litoral Cantabro-Atlantico; Rururbano (diseminado); Termotemperado & 3.10 & 19.32 & 0.16 \\ 
  Vales sublitorais; Agrosistema intensivo (plantacion forestal); Mesotemperado inferior & 2.87 & 7.86 & 0.37 \\ 
  Litoral Cantabro-Atlantico; Agrosistema intensivo (mosaico agroforestal); Termotemperado & 2.86 & 10.45 & 0.27 \\ 
  Canons; Bosque; Mesotemperado superior & 2.81 & 0.27 & 10.33 \\ 
  Serras; Agrosistema intensivo (mosaico agroforestal); Mesotemperado superior & 2.71 & 1.64 & 1.65 \\ 
  Vales sublitorais; Bosque; Mesotemperado inferior & 2.54 & 0.58 & 4.37 \\ 
  Canons; Agrosistema intensivo (mosaico agroforestal); Mesotemperado inferior & 2.42 & 0.25 & 9.76 \\ 
  Canons; Agrosistema extensivo; Mesotemperado inferior & 2.01 & 0.20 & 10.16 \\ 
  Vales sublitorais; Turbeira; Mesotemperado superior & 1.84 & 0.38 & 4.86 \\ 
  Vales sublitorais; Agrosistema intensivo (mosaico agroforestal); Mesotemperado superior & 1.76 & 4.37 & 0.40 \\ 
  Canons; Agrosistema intensivo (plantacion forestal); Mesotemperado inferior & 1.68 & 0.17 & 10.07 \\ 
  Vales sublitorais; Agrosistema extensivo; Mesotemperado superior & 1.61 & 1.35 & 1.19 \\ 
  Serras; Agrosistema intensivo (mosaico agroforestal); Mesotemperado inferior & 1.43 & 0.36 & 4.03 \\ 
  Canons; Matogueira e rochedo; Mesotemperado inferior & 1.33 & 0.13 & 10.28 \\ 
  Vales sublitorais; Matogueira e rochedo; Mesotemperado superior & 1.23 & 1.16 & 1.06 \\ 
  Litoral Cantabro-Atlantico; Bosque; Termotemperado & 1.15 & 0.40 & 2.92 \\ 
  Serras; Agrosistema intensivo (plantacion forestal); Mesotemperado superior & 1.11 & 0.68 & 1.65 \\ 
  Litoral Cantabro-Atlantico; Agrosistema intensivo (plantacion forestal); Termotemperado & 1.01 & 2.83 & 0.36 \\ 
   \hline
\end{tabular}
\end{table}
% latex table generated in R 3.2.2 by xtable 1.8-0 package
% Fri Dec  4 16:54:41 2015
\begin{table}[p]
\centering
\caption{Frecuencia de aparición de Lugares de Importancia Comunitaria e frecuencia de tipos asociados A Mariña - Baixo Eo} 
\label{vnatura2}
\begin{tabular}{lrrr}
  \hline
Tipo de paisaxe & F.Aparic (\%) & F.Tipo (\%) & Ratio \\ 
  \hline
Serras; Turbeira; Mesotemperado superior & 32.65 & 2.35 & 13.91 \\ 
  Serras; Agrosistema intensivo (plantacion forestal); Mesotemperado superior & 12.42 & 4.01 & 3.10 \\ 
  Litoral Cantabro-Atlantico; Urbano; Termotemperado & 7.24 & 2.18 & 3.32 \\ 
  Serras; Matogueira e rochedo; Mesotemperado superior & 6.08 & 2.07 & 2.94 \\ 
  Litoral Cantabro-Atlantico; Agrosistema intensivo (mosaico agroforestal); Termotemperado & 5.29 & 8.49 & 0.62 \\ 
  Litoral Cantabro-Atlantico; Rururbano (diseminado); Termotemperado & 5.06 & 3.92 & 1.29 \\ 
  Vales sublitorais; Agrosistema intensivo (plantacion forestal); Mesotemperado inferior & 4.19 & 14.02 & 0.30 \\ 
  Vales sublitorais; Agrosistema intensivo (plantacion forestal); Mesotemperado superior & 3.75 & 3.20 & 1.17 \\ 
   & 3.38 &  &  \\ 
  Litoral Cantabro-Atlantico; Agrosistema intensivo (plantacion forestal); Termotemperado & 3.02 & 4.62 & 0.65 \\ 
  Vales sublitorais; Matogueira e rochedo; Mesotemperado inferior & 2.36 & 1.48 & 1.59 \\ 
  Vales sublitorais; Agrosistema intensivo (mosaico agroforestal); Mesotemperado inferior & 2.27 & 12.56 & 0.18 \\ 
  Serras; Agrosistema intensivo (mosaico agroforestal); Mesotemperado superior & 2.13 & 2.21 & 0.96 \\ 
  Litoral Cantabro-Atlantico; Agrosistema intensivo (mosaico agroforestal); no data & 1.09 & 0.11 & 9.52 \\ 
  Serras; Agrosistema intensivo (plantacion forestal); Mesotemperado inferior & 1.05 & 2.24 & 0.47 \\ 
   \hline
\end{tabular}
\end{table}
% latex table generated in R 3.2.2 by xtable 1.8-0 package
% Fri Dec  4 16:54:41 2015
\begin{table}[p]
\centering
\caption{Frecuencia de aparición de Lugares de Importancia Comunitaria e frecuencia de tipos asociados Costa Sur - Baixo Miño} 
\label{vnatura3}
\begin{tabular}{lrrr}
  \hline
Tipo de paisaxe & F.Aparic (\%) & F.Tipo (\%) & Ratio \\ 
  \hline
Litoral Cantabro-Atlantico; Rururbano (diseminado); Termotemperado & 17.27 & 11.41 & 1.51 \\ 
  Litoral Cantabro-Atlantico; Bosque; Termotemperado & 7.15 & 0.29 & 24.66 \\ 
  Serras; Agrosistema intensivo (plantacion forestal); Mesotemperado inferior & 6.41 & 2.04 & 3.15 \\ 
  Litoral Cantabro-Atlantico; Rururbano (diseminado); no data & 5.69 & 0.34 & 16.66 \\ 
  Litoral Cantabro-Atlantico; Agrosistema extensivo; Termotemperado & 5.46 & 0.39 & 14.04 \\ 
  Litoral Cantabro-Atlantico; Agrosistema intensivo (mosaico agroforestal); Termotemperado & 5.09 & 3.27 & 1.56 \\ 
  Chairas e vales interiores; Matogueira e rochedo; Mesotemperado inferior & 5.05 & 2.49 & 2.03 \\ 
  Litoral Cantabro-Atlantico; Agrosistema intensivo (plantacion forestal); Termotemperado & 4.48 & 2.51 & 1.78 \\ 
  Vales sublitorais; Rururbano (diseminado); Termotemperado & 3.00 & 12.49 & 0.24 \\ 
  Vales sublitorais; Agrosistema intensivo (plantacion forestal); Termotemperado & 2.97 & 8.94 & 0.33 \\ 
  Serras; Agrosistema intensivo (plantacion forestal); Termotemperado & 2.95 & 2.79 & 1.06 \\ 
  Litoral Cantabro-Atlantico; Matogueira e rochedo; Termotemperado & 2.11 & 0.60 & 3.54 \\ 
  Serras; Matogueira e rochedo; Mesotemperado superior & 1.92 & 5.89 & 0.33 \\ 
  Litoral Cantabro-Atlantico; Urbano; Termotemperado & 1.62 & 1.44 & 1.12 \\ 
  Litoral Cantabro-Atlantico; Agrosistema extensivo; no data & 1.50 & 0.05 & 27.63 \\ 
  Serras; Matogueira e rochedo; Mesotemperado inferior & 1.45 & 5.06 & 0.29 \\ 
  Vales sublitorais; Agrosistema extensivo; Termotemperado & 1.32 & 1.52 & 0.87 \\ 
  Litoral Cantabro-Atlantico; Agrosistema intensivo (mosaico agroforestal); no data & 1.24 & 0.07 & 17.60 \\ 
  Litoral Cantabro-Atlantico; Agrosistema intensivo (plantacion forestal); no data & 1.22 & 0.06 & 21.74 \\ 
  Serras; Matogueira e rochedo; Termotemperado & 1.19 & 1.75 & 0.68 \\ 
  Vales sublitorais; Bosque; Termotemperado & 1.18 & 1.03 & 1.15 \\ 
  Chairas e vales interiores; Rururbano (diseminado); no data & 1.07 & 0.10 & 10.64 \\ 
  Chairas e vales interiores; Agrosistema intensivo (plantacion forestal); no data & 1.07 & 0.06 & 16.76 \\ 
   \hline
\end{tabular}
\end{table}
% latex table generated in R 3.2.2 by xtable 1.8-0 package
% Fri Dec  4 16:54:42 2015
\begin{table}[p]
\centering
\caption{Frecuencia de aparición de Lugares de Importancia Comunitaria e frecuencia de tipos asociados Ribeiras Encaixadas do Miño e do Sil} 
\label{vnatura4}
\begin{tabular}{lrrr}
  \hline
Tipo de paisaxe & F.Aparic (\%) & F.Tipo (\%) & Ratio \\ 
  \hline
Serras; Matogueira e rochedo; Supra e orotemperado & 8.34 & 4.06 & 2.06 \\ 
  Canons; Bosque; Termotemperado & 7.57 & 1.40 & 5.39 \\ 
  Serras; Matogueira e rochedo; Mesotemperado inferior & 7.09 & 5.11 & 1.39 \\ 
  Canons; Bosque; Mesotemperado inferior & 6.77 & 1.90 & 3.56 \\ 
  Serras; Matogueira e rochedo; Mesotemperado superior & 5.85 & 5.27 & 1.11 \\ 
  Canons; Matogueira e rochedo; Mesomediterráneo & 5.14 & 1.77 & 2.91 \\ 
  Serras; Agrosistema extensivo; Mesotemperado inferior & 5.05 & 4.45 & 1.13 \\ 
  Serras; Bosque; Mesotemperado inferior & 4.38 & 1.46 & 3.00 \\ 
  Chairas e vales interiores; Matogueira e rochedo; Mesomediterráneo & 4.27 & 2.99 & 1.43 \\ 
  Canons; Agrosistema intensivo (plantacion forestal); Termotemperado & 4.13 & 0.67 & 6.13 \\ 
  Canons; Matogueira e rochedo; Termotemperado & 3.95 & 0.70 & 5.65 \\ 
  Serras; Matogueira e rochedo; Mesomediterráneo & 3.45 & 1.82 & 1.89 \\ 
  Chairas e vales interiores; Bosque; Mesomediterráneo & 2.87 & 0.54 & 5.27 \\ 
  Canons; Matogueira e rochedo; Mesotemperado inferior & 2.74 & 1.29 & 2.12 \\ 
  Chairas e vales interiores; Agrosistema extensivo; Mesotemperado inferior & 2.74 & 9.17 & 0.30 \\ 
  Chairas e vales interiores; Matogueira e rochedo; Mesotemperado inferior & 2.09 & 4.17 & 0.50 \\ 
  Chairas e vales interiores; Agrosistema extensivo; Mesomediterráneo & 1.84 & 1.24 & 1.49 \\ 
  Chairas e vales interiores; Bosque; Termotemperado & 1.76 & 2.38 & 0.74 \\ 
  Chairas e vales interiores; Agrosistema extensivo; Termotemperado & 1.47 & 3.71 & 0.40 \\ 
  Canons; Agrosistema extensivo; Mesotemperado inferior & 1.23 & 1.18 & 1.03 \\ 
  Chairas e vales interiores; Agrosistema intensivo (mosaico agroforestal); Mesotemperado inferior & 1.21 & 4.80 & 0.25 \\ 
  Chairas e vales interiores; Matogueira e rochedo; Termotemperado & 1.13 & 4.28 & 0.26 \\ 
  Canons; Agrosistema extensivo; Mesomediterráneo & 1.12 & 0.35 & 3.19 \\ 
  Canons; Bosque; Mesomediterráneo & 1.11 & 0.28 & 4.01 \\ 
  Canons; Agrosistema extensivo; Termotemperado & 1.05 & 0.46 & 2.29 \\ 
  Canons; Agrosistema intensivo (plantacion forestal); Mesomediterráneo & 1.01 & 0.30 & 3.35 \\ 
   \hline
\end{tabular}
\end{table}
% latex table generated in R 3.2.2 by xtable 1.8-0 package
% Fri Dec  4 16:54:42 2015
\begin{table}[p]
\centering
\caption{Frecuencia de aparición de Lugares de Importancia Comunitaria e frecuencia de tipos asociados Serras Orientais} 
\label{vnatura5}
\begin{tabular}{lrrr}
  \hline
Tipo de paisaxe & F.Aparic (\%) & F.Tipo (\%) & Ratio \\ 
  \hline
Serras; Matogueira e rochedo; Supra e orotemperado & 23.02 & 14.82 & 1.55 \\ 
  Serras; Agrosistema extensivo; Supra e orotemperado & 18.56 & 15.17 & 1.22 \\ 
  Serras; Bosque; Supra e orotemperado & 9.10 & 5.16 & 1.76 \\ 
  Serras; Matogueira e rochedo; Mesotemperado superior & 7.24 & 6.90 & 1.05 \\ 
  Serras; Agrosistema extensivo; Mesotemperado superior & 6.71 & 11.84 & 0.57 \\ 
  Serras; Bosque; Mesotemperado superior & 3.74 & 3.78 & 0.99 \\ 
  Vales sublitorais; Agrosistema extensivo; Mesotemperado superior & 3.62 & 5.67 & 0.64 \\ 
  Serras; Agrosistema intensivo (plantacion forestal); Supra e orotemperado & 3.37 & 3.43 & 0.98 \\ 
  Vales sublitorais; Agrosistema extensivo; Mesotemperado inferior & 3.27 & 3.89 & 0.84 \\ 
  Vales sublitorais; Bosque; Mesotemperado superior & 2.50 & 3.48 & 0.72 \\ 
  Vales sublitorais; Bosque; Mesotemperado inferior & 2.23 & 2.82 & 0.79 \\ 
  Serras; Matogueira e rochedo; Mesotemperado inferior & 1.90 & 1.10 & 1.73 \\ 
  Serras; Agrosistema intensivo (mosaico agroforestal); Supra e orotemperado & 1.52 & 2.44 & 0.62 \\ 
  Vales sublitorais; Matogueira e rochedo; Mesotemperado superior & 1.48 & 2.47 & 0.60 \\ 
  Vales sublitorais; Matogueira e rochedo; Mesotemperado inferior & 1.37 & 1.60 & 0.86 \\ 
  Serras; Agrosistema intensivo (plantacion forestal); Mesotemperado superior & 1.06 & 2.33 & 0.45 \\ 
   \hline
\end{tabular}
\end{table}
% latex table generated in R 3.2.2 by xtable 1.8-0 package
% Fri Dec  4 16:54:42 2015
\begin{table}[p]
\centering
\caption{Frecuencia de aparición de Lugares de Importancia Comunitaria e frecuencia de tipos asociados Chairas e Fosas Luguesas} 
\label{vnatura6}
\begin{tabular}{lrrr}
  \hline
Tipo de paisaxe & F.Aparic (\%) & F.Tipo (\%) & Ratio \\ 
  \hline
Chairas e vales interiores; Agrosistema extensivo; Mesotemperado superior & 21.07 & 17.50 & 1.20 \\ 
  Serras; Turbeira; Mesotemperado superior & 12.32 & 1.14 & 10.84 \\ 
  Serras; Turbeira; Supra e orotemperado & 11.27 & 0.48 & 23.51 \\ 
  Serras; Matogueira e rochedo; Supra e orotemperado & 9.34 & 0.84 & 11.15 \\ 
  Serras; Agrosistema extensivo; Supra e orotemperado & 7.17 & 1.09 & 6.55 \\ 
  Chairas e vales interiores; Agrosistema extensivo; Mesotemperado inferior & 6.41 & 9.57 & 0.67 \\ 
  Chairas e vales interiores; Bosque; Mesotemperado superior & 5.96 & 1.53 & 3.89 \\ 
  Serras; Agrosistema extensivo; Mesotemperado superior & 5.43 & 10.14 & 0.54 \\ 
  Chairas e vales interiores; Agrosistema intensivo (mosaico agroforestal); Mesotemperado superior & 3.32 & 14.58 & 0.23 \\ 
  Serras; Agrosistema intensivo (plantacion forestal); Supra e orotemperado & 2.75 & 0.44 & 6.25 \\ 
  Chairas e vales interiores; Bosque; Mesotemperado inferior & 2.11 & 1.29 & 1.64 \\ 
  Serras; Bosque; Supra e orotemperado & 1.91 & 0.14 & 13.48 \\ 
  Chairas e vales interiores; Agrosistema intensivo (superficie de cultivo); Mesotemperado superior & 1.88 & 4.58 & 0.41 \\ 
  Serras; Bosque; Mesotemperado superior & 1.08 & 0.50 & 2.14 \\ 
  Chairas e vales interiores; Agrosistema intensivo (mosaico agroforestal); Mesotemperado inferior & 1.07 & 6.48 & 0.16 \\ 
   \hline
\end{tabular}
\end{table}
% latex table generated in R 3.2.2 by xtable 1.8-0 package
% Fri Dec  4 16:54:42 2015
\begin{table}[p]
\centering
\caption{Frecuencia de aparición de Lugares de Importancia Comunitaria e frecuencia de tipos asociados Galicia Central} 
\label{vnatura7}
\begin{tabular}{lrrr}
  \hline
Tipo de paisaxe & F.Aparic (\%) & F.Tipo (\%) & Ratio \\ 
  \hline
Serras; Matogueira e rochedo; Mesotemperado superior & 22.90 & 5.00 & 4.58 \\ 
  Serras; Matogueira e rochedo; Supra e orotemperado & 19.98 & 2.19 & 9.10 \\ 
  Serras; Agrosistema extensivo; Mesotemperado superior & 8.69 & 7.06 & 1.23 \\ 
  Vales sublitorais; Agrosistema extensivo; Mesotemperado inferior & 7.56 & 10.39 & 0.73 \\ 
  Vales sublitorais; Matogueira e rochedo; Mesotemperado inferior & 6.91 & 3.12 & 2.22 \\ 
  Vales sublitorais; Agrosistema intensivo (mosaico agroforestal); Mesotemperado inferior & 5.55 & 23.79 & 0.23 \\ 
  Serras; Matogueira e rochedo; Mesotemperado inferior & 4.90 & 1.61 & 3.04 \\ 
  Serras; Agrosistema extensivo; Supra e orotemperado & 3.90 & 0.78 & 5.00 \\ 
  Vales sublitorais; Agrosistema intensivo (mosaico agroforestal); Termotemperado & 2.00 & 5.79 & 0.35 \\ 
  Vales sublitorais; Agrosistema extensivo; Termotemperado & 1.72 & 0.96 & 1.79 \\ 
  Vales sublitorais; Agrosistema extensivo; Mesotemperado superior & 1.50 & 2.91 & 0.51 \\ 
  Serras; Agrosistema extensivo; Mesotemperado inferior & 1.33 & 2.67 & 0.50 \\ 
  Vales sublitorais; Bosque; Termotemperado & 1.09 & 0.32 & 3.38 \\ 
  Serras; Agrosistema intensivo (plantacion forestal); Mesotemperado superior & 1.03 & 0.62 & 1.67 \\ 
   \hline
\end{tabular}
\end{table}
% latex table generated in R 3.2.2 by xtable 1.8-0 package
% Fri Dec  4 16:54:42 2015
\begin{table}[p]
\centering
\caption{Frecuencia de aparición de Lugares de Importancia Comunitaria e frecuencia de tipos asociados Chairas, Fosas e Serras Ourensás} 
\label{vnatura8}
\begin{tabular}{lrrr}
  \hline
Tipo de paisaxe & F.Aparic (\%) & F.Tipo (\%) & Ratio \\ 
  \hline
Serras; Matogueira e rochedo; Supra e orotemperado & 28.94 & 8.42 & 3.44 \\ 
  Serras; Matogueira e rochedo; Mesotemperado superior & 19.76 & 9.99 & 1.98 \\ 
  Chairas e vales interiores; Agrosistema intensivo (superficie de cultivo); Mesotemperado inferior & 13.22 & 7.47 & 1.77 \\ 
  Serras; Matogueira e rochedo; Mesotemperado inferior & 9.35 & 8.98 & 1.04 \\ 
  Chairas e vales interiores; Agrosistema extensivo; Mesotemperado inferior & 3.93 & 12.05 & 0.33 \\ 
  Serras; Bosque; Supra e orotemperado & 2.96 & 0.97 & 3.04 \\ 
  Serras; Agrosistema extensivo; Mesotemperado superior & 2.56 & 4.93 & 0.52 \\ 
  Chairas e vales interiores; Matogueira e rochedo; Termotemperado & 2.31 & 2.16 & 1.07 \\ 
  Serras; Matogueira e rochedo; Termotemperado & 2.13 & 0.44 & 4.81 \\ 
  Serras; Matogueira e rochedo; no data & 1.64 & 0.36 & 4.52 \\ 
  Chairas e vales interiores; Matogueira e rochedo; Mesotemperado inferior & 1.37 & 4.79 & 0.29 \\ 
  Chairas e vales interiores; Agrosistema extensivo; Termotemperado & 1.19 & 1.77 & 0.67 \\ 
  Serras; Agrosistema extensivo; Supra e orotemperado & 1.15 & 3.09 & 0.37 \\ 
  Serras; Agrosistema intensivo (plantacion forestal); Supra e orotemperado & 1.04 & 0.79 & 1.30 \\ 
   \hline
\end{tabular}
\end{table}
% latex table generated in R 3.2.2 by xtable 1.8-0 package
% Fri Dec  4 16:54:42 2015
\begin{table}[p]
\centering
\caption{Frecuencia de aparición de Lugares de Importancia Comunitaria e frecuencia de tipos asociados Serras Surorientais} 
\label{vnatura9}
\begin{tabular}{lrrr}
  \hline
Tipo de paisaxe & F.Aparic (\%) & F.Tipo (\%) & Ratio \\ 
  \hline
Serras; Matogueira e rochedo; Supra e orotemperado & 62.82 & 37.48 & 1.68 \\ 
  Serras; Agrosistema extensivo; Supra e orotemperado & 9.50 & 7.54 & 1.26 \\ 
  Serras; Matogueira e rochedo; Mesotemperado superior & 4.22 & 9.91 & 0.43 \\ 
  Serras; Bosque; Supra e orotemperado & 4.03 & 2.05 & 1.96 \\ 
  Serras; Agrosistema intensivo (plantacion forestal); Supra e orotemperado & 3.89 & 5.13 & 0.76 \\ 
  Serras; Agrosistema extensivo; Mesotemperado superior & 3.72 & 10.41 & 0.36 \\ 
  Serras; Agrosistema extensivo; Mesotemperado inferior & 1.55 & 7.06 & 0.22 \\ 
  Canons; Matogueira e rochedo; Mesotemperado inferior & 1.37 & 0.93 & 1.47 \\ 
  Serras; Bosque; Mesotemperado superior & 1.18 & 2.58 & 0.46 \\ 
   \hline
\end{tabular}
\end{table}
% latex table generated in R 3.2.2 by xtable 1.8-0 package
% Fri Dec  4 16:54:42 2015
\begin{table}[p]
\centering
\caption{Frecuencia de aparición de Lugares de Importancia Comunitaria e frecuencia de tipos asociados Galicia Setentrional} 
\label{vnatura10}
\begin{tabular}{lrrr}
  \hline
Tipo de paisaxe & F.Aparic (\%) & F.Tipo (\%) & Ratio \\ 
  \hline
Serras; Turbeira; Supra e orotemperado & 23.77 & 5.12 & 4.64 \\ 
  Serras; Turbeira; Mesotemperado superior & 20.06 & 6.59 & 3.04 \\ 
  Litoral Cantabro-Atlantico; Matogueira e rochedo; Termotemperado & 8.99 & 1.91 & 4.70 \\ 
  Serras; Agrosistema intensivo (plantacion forestal); Mesotemperado superior & 6.92 & 2.79 & 2.48 \\ 
  Serras; Matogueira e rochedo; Mesotemperado superior & 3.81 & 6.61 & 0.58 \\ 
  Serras; Agrosistema extensivo; Mesotemperado superior & 3.71 & 4.36 & 0.85 \\ 
  Litoral Cantabro-Atlantico; Agrosistema intensivo (mosaico agroforestal); Termotemperado & 3.09 & 5.96 & 0.52 \\ 
  Litoral Cantabro-Atlantico; Matogueira e rochedo; Mesotemperado inferior & 2.66 & 0.60 & 4.44 \\ 
  Litoral Cantabro-Atlantico; Agrosistema intensivo (plantacion forestal); Termotemperado & 2.50 & 3.32 & 0.75 \\ 
  Serras; Agrosistema intensivo (mosaico agroforestal); Mesotemperado superior & 2.31 & 3.06 & 0.76 \\ 
  Litoral Cantabro-Atlantico; Rururbano (diseminado); Termotemperado & 1.96 & 2.78 & 0.70 \\ 
  Serras; Bosque; Mesotemperado superior & 1.80 & 1.35 & 1.33 \\ 
  Serras; Turbeira; Mesotemperado inferior & 1.63 & 0.68 & 2.38 \\ 
  Serras; Agrosistema extensivo; Supra e orotemperado & 1.62 & 0.64 & 2.53 \\ 
   & 1.41 &  &  \\ 
  Vales sublitorais; Agrosistema intensivo (mosaico agroforestal); Mesotemperado superior & 1.35 & 2.20 & 0.61 \\ 
  Litoral Cantabro-Atlantico; Matogueira e rochedo; no data & 1.00 & 0.19 & 5.40 \\ 
   \hline
\end{tabular}
\end{table}
% latex table generated in R 3.2.2 by xtable 1.8-0 package
% Fri Dec  4 16:54:42 2015
\begin{table}[p]
\centering
\caption{Frecuencia de aparición de Lugares de Importancia Comunitaria e frecuencia de tipos asociados Chairas e Fosas Occidentais} 
\label{vnatura11}
\begin{tabular}{lrrr}
  \hline
Tipo de paisaxe & F.Aparic (\%) & F.Tipo (\%) & Ratio \\ 
  \hline
Litoral Cantabro-Atlantico; Matogueira e rochedo; Termotemperado & 41.83 & 5.61 & 7.45 \\ 
  Vales sublitorais; Matogueira e rochedo; Mesotemperado inferior & 12.19 & 7.15 & 1.70 \\ 
  Litoral Cantabro-Atlantico; Matogueira e rochedo; no data & 6.51 & 0.47 & 13.98 \\ 
  Vales sublitorais; Agrosistema intensivo (plantacion forestal); Mesotemperado inferior & 5.58 & 8.33 & 0.67 \\ 
  Litoral Cantabro-Atlantico; Agrosistema intensivo (plantacion forestal); Termotemperado & 5.27 & 4.31 & 1.22 \\ 
  Vales sublitorais; Matogueira e rochedo; Termotemperado & 4.60 & 1.68 & 2.73 \\ 
  Litoral Cantabro-Atlantico; Agrosistema intensivo (mosaico agroforestal); Termotemperado & 4.45 & 8.86 & 0.50 \\ 
   & 4.36 &  &  \\ 
  Litoral Cantabro-Atlantico; Rururbano (diseminado); Termotemperado & 3.25 & 2.77 & 1.17 \\ 
  Litoral Cantabro-Atlantico; Agrosistema intensivo (superficie de cultivo); Termotemperado & 2.73 & 0.93 & 2.94 \\ 
  Litoral Cantabro-Atlantico; Agrosistema extensivo; Termotemperado & 2.72 & 0.42 & 6.52 \\ 
  Litoral Cantabro-Atlantico; Urbano; Termotemperado & 1.19 & 0.44 & 2.68 \\ 
  Vales sublitorais; Agrosistema intensivo (mosaico agroforestal); Mesotemperado inferior & 1.14 & 18.05 & 0.06 \\ 
   \hline
\end{tabular}
\end{table}
% latex table generated in R 3.2.2 by xtable 1.8-0 package
% Fri Dec  4 16:54:42 2015
\begin{table}[p]
\centering
\caption{Frecuencia de aparición de Lugares de Importancia Comunitaria e frecuencia de tipos asociados Rías Baixas} 
\label{vnatura12}
\begin{tabular}{lrrr}
  \hline
Tipo de paisaxe & F.Aparic (\%) & F.Tipo (\%) & Ratio \\ 
  \hline
Serras; Matogueira e rochedo; Mesotemperado superior & 15.40 & 4.07 & 3.78 \\ 
  Serras; Matogueira e rochedo; Supra e orotemperado & 14.26 & 2.49 & 5.74 \\ 
   & 13.73 &  &  \\ 
  Litoral Cantabro-Atlantico; Matogueira e rochedo; Termotemperado & 10.40 & 2.61 & 3.99 \\ 
  Litoral Cantabro-Atlantico; Rururbano (diseminado); Termotemperado & 5.38 & 16.40 & 0.33 \\ 
  Litoral Cantabro-Atlantico; Agrosistema intensivo (mosaico agroforestal); Termotemperado & 4.84 & 6.56 & 0.74 \\ 
  Litoral Cantabro-Atlantico; Agrosistema extensivo; Termotemperado & 3.63 & 0.59 & 6.13 \\ 
  Litoral Cantabro-Atlantico; Matogueira e rochedo; no data & 3.22 & 0.23 & 14.00 \\ 
  Serras; Agrosistema extensivo; Mesotemperado superior & 3.11 & 0.95 & 3.28 \\ 
  Serras; Agrosistema extensivo; Mesotemperado inferior & 3.09 & 1.07 & 2.89 \\ 
  Vales sublitorais; Agrosistema extensivo; Mesotemperado inferior & 2.88 & 1.29 & 2.23 \\ 
  Serras; Matogueira e rochedo; Mesotemperado inferior & 2.77 & 4.31 & 0.64 \\ 
  Vales sublitorais; Bosque; Mesotemperado inferior & 2.41 & 0.47 & 5.12 \\ 
  Litoral Cantabro-Atlantico; Agrosistema intensivo (plantacion forestal); Termotemperado & 2.06 & 4.94 & 0.42 \\ 
  Litoral Cantabro-Atlantico; Agrosistema intensivo (superficie de cultivo); Termotemperado & 1.97 & 0.59 & 3.36 \\ 
  Vales sublitorais; Matogueira e rochedo; Mesotemperado inferior & 1.44 & 5.63 & 0.26 \\ 
  Litoral Cantabro-Atlantico; Viñedo; Termotemperado & 1.34 & 1.24 & 1.08 \\ 
  Litoral Cantabro-Atlantico; Rururbano (diseminado); no data & 1.24 & 0.51 & 2.43 \\ 
   \hline
\end{tabular}
\end{table}
